
%%%%%%%%%%%%%%%%%%%%%%%%%%%%%%%%%%%%%%%%%%%%%%%%%%%%%%%%%%%%
\chapter*{Preface}
\label{chapter:preface}
\addcontentsline{toc}{chapter}{Preface}
%%%%%%%%%%%%%%%%%%%%%%%%%%%%%%%%%%%%%%%%%%%%%%%%%%%%%%%%%%%%

This thesis is based on research conducted between 20014 and 2017 at
the Institute for Theoretical Physics at Heidelberg
University. Chapter~\ref{chapter:validity} is based on two articles,
which later became part of a CERN report:
%
\begin{itemize}
  \item[\cite{Brehmer:2015rna}] J.~Brehmer, A.~Freitas, D.~L\'opez-Val, and T.~Plehn:\newline
	\textbf{Pushing Higgs Effective Theory to its Limits}.\newline
	Phys.~Rev.~D 93, 075014 (2016). \arxivlink{1510.03443}.
  \item[\cite{Biekotter:2016ecg}] A.~Biek\"otter, J.~Brehmer, and T.~Plehn:\newline
	\textbf{Extending the Limits of Higgs Effective Theory}.\newline
	Phys.~Rev.~D 94, 055032 (2016). \arxivlink{1602.05202}. 
  \item[\cite{deFlorian:2016spz}] D.~de Florian, C.~Grojean, F.~Maltoni, et al.:\newline
        \textbf{Handbook of LHC Higgs Cross Sections: 4.~Deciphering the Nature of the Higgs Sector}.\newline
        LHC Higgs Cross Section Working Group Yellow Report. \arxivlink{1610.07922}.
\end{itemize}
%
Chapter~\ref{chapter:information} is based on the following publication:
%
\begin{itemize}
  \item[\cite{Brehmer:2016nyr}] J.~Brehmer, K.~Cranmer, F.~Kling, and T.~Plehn:\newline
	\textbf{Better Higgs Measurements Through Information Geometry}.\newline
       Phys.~Rev.~D 95, 073002 (2017). \arxivlink{1612.05261}.
\end{itemize}
%
In addition, it includes some previously unpublished results.
% %
% \begin{itemize}
%   \item[\cite{Brehmer:CPV_information}] J.~Brehmer, F.~Kling, and T.~Plehn:\newline
%     \textbf{Understanding CP violation with information geometry} (preliminary title).\newline
%     Work in progress.\comment{Discuss this with Felix and Tilman. Add Tim Tate as author? Title?}
% \end{itemize}

Chapter~\ref{chapter:foundations} consists of introductory material
that can be found in many textbooks and review articles, as well as on
a lecture on effective field theories given by the author to fellow
PhD students in Heidelberg:
%
\begin{itemize}
  \item[\cite{Brehmer:EFTlecture}]  J.~Brehmer:\newline
	\textbf{Higgs Effective Field Theory}.\newline
        Student lecture, research training group ``Particle Physics Beyond the Standard Model''.
\end{itemize}

Finally, some of the work done during my PhD is not included in this thesis:
%
\begin{itemize}
  \item[\cite{Brehmer:2015cia}] J.~Brehmer, J.~Hewett, J.~Kopp, T.~Rizzo, and J.~Tattersall:\newline
	\textbf{Symmetry Restored in Dibosons at the LHC?} \newline
	JHEP 1510, 182 (2015). \arxivlink{1507.00013}.
  \item[\cite{Brehmer:2015dan,Brooijmans:2016vro}] G.~Brooijmans, C.~Delaunay, A.~Delgado, et al.:\newline
         \textbf{Les Houches 2015: Physics at TeV Colliders~--~New Physics Working Group Report}.\newline
        \arxivlink{1605.02684}.\newline
         Part of these proceedings were published separately as\newline
         J.~Brehmer, G.~Brooijmans,  G.~Cacciapaglia, et al.:\newline
	\textbf{The Diboson Excess: Experimental Situation and Classification of Explanations; A~Les Houches Pre-Proceeding}.\newline
	\arxivlink{1512.04357}.
\end{itemize}
