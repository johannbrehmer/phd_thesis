
%%%%%%%%%%%%%%%%%%%%%%%%%%%%%%%%%%%%%%%%%%%%%%%%%%%%%%%%%%%%
% Header
%%%%%%%%%%%%%%%%%%%%%%%%%%%%%%%%%%%%%%%%%%%%%%%%%%%%%%%%%%%%

\documentclass[a4paper,
	twoside,
	captions=nooneline, % captions of floats are not centered if only one line
	fleqn, % equations to the left, not centered
	%parskip=half, %some vertical space between paragraphs
	%appendixprefix, %call appendices "appendix A" and so on, not just "A"
	BCOR=1mm, %binding correction
	headsepline, %line below heading
	bibliography=totoc,
	abstracton,
        openright,
	cleardoublepage=cleardoublepage,
        numbers=noenddot,
        DIV=10,
	11pt]{scrbook}




%%%%%%%%%%%%%%%%%%%%%%%%%%%%%%%%%%%%%%%%%%%%%%%%%%%%%%%%%%%%
% Fonts
%%%%%%%%%%%%%%%%%%%%%%%%%%%%%%%%%%%%%%%%%%%%%%%%%%%%%%%%%%%%

\usepackage[utf8]{inputenc}
\usepackage[ngerman,british]{babel}

% Option 1: Minion Pro + Myriad Pro + Inconsolata
% % maybe using semibold instead of bold with the [medfamily] option
% 1A: text figures in text modes, lining tabular figures in math mode
\usepackage[textosf,mathlf,mathtabular,minionint,roundv]{MinionPro} 
\usepackage[mathlf,mathtabular,onlytext,medfamily]{MyriadPro}
% % 1B: lining figures everywhere
% \usepackage[lf,mathtabular,minionint]{MinionPro} 
% \usepackage[lf,mathtabular,onlytext,medfamily]{MyriadPro}
\usepackage[toc,eqno,enum,bib]{tabfigures} % tabular figures in certain environments 
\linespread{1.06}

% % Option 2: Only Minion Pro
% % maybe using semibold instead of bold with the [medfamily] option?
% % \usepackage[textosf,mathlf,mathtabular,minionint]{MinionPro} % 2A: text figures in text modes, lining tabular figures in math mode
% \usepackage[lf,mathtabular,minionint]{MinionPro} % 2B: lining figures everywhere
% \usepackage[toc,eqno,enum,bib]{tabfigures} % tabular figures in certain environments
% \usepackage[mathlf,mathtabular,onlytext]{MyriadPro}
% \addtokomafont{disposition}{\rmfamily}
% \linespread{1.06}

% % Option 3: EB Garamond
% \usepackage[cmintegrals,cmbraces]{newtxmath}
% \usepackage{ebgaramond-maths}
% \addtokomafont{disposition}{\rmfamily \mdseries}
% \makeatletter
%   \DeclareSymbolFont{ntxletters}{OML}{ntxmi}{m}{it}
%   \SetSymbolFont{ntxletters}{bold}{OML}{ntxmi}{b}{it}
%   \re@DeclareMathSymbol{\leftharpoonup}{\mathrel}{ntxletters}{"28}
%   \re@DeclareMathSymbol{\leftharpoondown}{\mathrel}{ntxletters}{"29}
%   \re@DeclareMathSymbol{\rightharpoonup}{\mathrel}{ntxletters}{"2A}
%   \re@DeclareMathSymbol{\rightharpoondown}{\mathrel}{ntxletters}{"2B}
%   \re@DeclareMathSymbol{\triangleleft}{\mathbin}{ntxletters}{"2F}
%   \re@DeclareMathSymbol{\triangleright}{\mathbin}{ntxletters}{"2E}
%   \re@DeclareMathSymbol{\partial}{\mathord}{ntxletters}{"40}
%   \re@DeclareMathSymbol{\flat}{\mathord}{ntxletters}{"5B}
%   \re@DeclareMathSymbol{\natural}{\mathord}{ntxletters}{"5C}
%   \re@DeclareMathSymbol{\star}{\mathbin}{ntxletters}{"3F}
%   \re@DeclareMathSymbol{\smile}{\mathrel}{ntxletters}{"5E}
%   \re@DeclareMathSymbol{\frown}{\mathrel}{ntxletters}{"5F}
%   \re@DeclareMathSymbol{\sharp}{\mathord}{ntxletters}{"5D}
%   \re@DeclareMathAccent{\vec}{\mathord}{ntxletters}{"7E}
% \makeatother 

% % Option 4: Libertine
% \usepackage{libertine}
% \usepackage[libertine]{newtxmath}
% \addtokomafont{disposition}{\rmfamily}

% % Option 5: Palatino
% \usepackage{mathpazo} % add possibly `sc` and `osf` options
% \usepackage{eulervm}



%%%%%%%%%%%%%%%%%%%%%%%%%%%%%%%%%%%%%%%%%%%%%%%%%%%%%%%%%%%%
% Other packages
%%%%%%%%%%%%%%%%%%%%%%%%%%%%%%%%%%%%%%%%%%%%%%%%%%%%%%%%%%%%

\usepackage{color,xcolor}
\definecolor{dark-grey}{rgb}{0.5,0.5,0.5}
\definecolor{dark-red}{rgb}{0.8,0.0,0.0}
\definecolor{dark-blue}{rgb}{0.0,0.0,0.8}
\definecolor{dark-green}{rgb}{0.,0.65,0.}
\definecolor{highlight-color}{rgb}{0.8,0.0,0.0}

\usepackage{amsmath,dsfont,mathtools}
\usepackage{siunitx}
\usepackage{graphicx}
\usepackage{multirow,booktabs,tabularx,float}
\usepackage[font=normalsize]{subfig}
\usepackage{slashed,braket}
\usepackage{feynmp-auto}
\usepackage[sort&compress]{natbib}
%\usepackage[pdftex,colorlinks,linkcolor={dark-blue}, citecolor={dark-green}, urlcolor={dark-red},bookmarksopen,linktoc=all]{hyperref}
\PassOptionsToPackage{hyphens}{url}\usepackage[pdftex,hidelinks,linkcolor={dark-blue}, citecolor={dark-green}, urlcolor={dark-red},bookmarksopen,linktoc=all]{hyperref}
\usepackage{enumitem} % for user-defined bullets 
\usepackage{pdfpages} % allows to load cover page
\usepackage{microtype}
\usepackage{calc} % arithmetic expressions for lengths
\usepackage{mparhack} % avoids marginpars appearing on the wrong side, remove for final version
\usepackage{lettrine} % Initials 
\usepackage{tabto}
\usepackage{tikz}
\usepackage{censor} % just for the draft...



%%%%%%%%%%%%%%%%%%%%%%%%%%%%%%%%%%%%%%%%%%%%%%%%%%%%%%%%%%%%
% Chapter titles
%%%%%%%%%%%%%%%%%%%%%%%%%%%%%%%%%%%%%%%%%%%%%%%%%%%%%%%%%%%%

% % Option 1: Formatting chapter titles with lines, red
% \renewcommand*{\chapterformat}{%
%   \enskip\mbox{\scalebox{5}{\thechapter\autodot}}}
% \renewcommand\chapterlinesformat[3]{%
%   {\color{highlight-color}
%   \parbox[b]{\textwidth}{\hrulefill#2}\par%
%   {\color{black}#3}\par\bigskip
%   \hrule}}
% \RedeclareSectionCommand[beforeskip=2\baselineskip,
% afterskip=3\baselineskip]{chapter}

% % Option 2: Formatting chapter titles without lines, red
% \renewcommand*{\chapterformat}{%
%   \enskip\mbox{\scalebox{5}{\thechapter\autodot}}}
% \renewcommand\chapterlinesformat[3]{%
%   {\color{highlight-color}
%   \parbox[b]{\textwidth}{\hfill#2}\par%
%   {\color{black}#3}\par\bigskip
%   }}
% \RedeclareSectionCommand[beforeskip=2\baselineskip,
% afterskip=3\baselineskip]{chapter}

% % Option 3: Formatting chapter titles with "chapter"
% \newlength{\numberheight}
% \newlength{\chapterheight}
% \newlength{\effectiveheight}
% \renewcommand*{\chapterformat}{%
%   \settoheight{\numberheight}{{\figureversion{lf}\thechapter\autodot}}%
%   \settoheight{\chapterheight}{Chapter}%
%   \setlength{\effectiveheight}{4\numberheight-\chapterheight}%
%   \enskip\mbox{\raisebox{\effectiveheight}{Chapter}~~\scalebox{4}{{\figureversion{lf}\thechapter\autodot}}}}
% \renewcommand\chapterlinesformat[3]{%
%   {\color{highlight-color}
%     \parbox[b]{\textwidth}{\hfill#2}\par%
%   {\color{black}#3}\par
%   }}
% \RedeclareSectionCommand[beforeskip=1cm,
% afterskip=2cm]{chapter}

% Option 3b: Formatting chapter titles with "chapter", except appendix
\newlength{\numberheight}
\newlength{\chapterheight}
\newlength{\effectiveheight}

% Requires new toggle to check whether we're in the appendix
\newif\ifinappendix% Default is \inappendixfalse
\let\oldappendix\appendix% Store \appendix
\renewcommand{\appendix}{% Update \appendix
  \oldappendix% Default \appendix
  \inappendixtrue% Set switch to true
}

\renewcommand*{\chapterformat}{%
  \makeatletter
  \ifinappendix
  \else
  \settoheight{\numberheight}{{\figureversion{lf}\thechapter\autodot}}%
  \settoheight{\chapterheight}{Chapter}%
  \setlength{\effectiveheight}{4\numberheight-\chapterheight}%
  \enskip\mbox{\raisebox{\effectiveheight}{Chapter}~~\scalebox{4}{{\figureversion{lf}\thechapter\autodot}}}
  \fi
  \makeatother}

\renewcommand\chapterlinesformat[3]{%
  {\color{highlight-color}
    \parbox[b]{\textwidth}{\hfill#2}\par%
    \vspace*{1cm}
  {\color{black}#3}\par
  }}

\RedeclareSectionCommand[beforeskip=1cm,afterskip=2cm]{chapter}

% % Option 4: Formatting chapter titles centred
% \renewcommand*{\chapterformat}{%
%   \centering \color{highlight-color}%
%   \enskip\mbox{\scalebox{3}{\thechapter\autodot}}}
% \renewcommand\chapterlinesformat[3]{%
%   {#2\par\vspace{0.2cm}%
%    #3\par
%   }}
% \RedeclareSectionCommand[beforeskip=1.5cm,
% afterskip=1.5cm plus 0.5cm minus 0cm]{chapter}



%%%%%%%%%%%%%%%%%%%%%%%%%%%%%%%%%%%%%%%%%%%%%%%%%%%%%%%%%%%%
% Dictum
%%%%%%%%%%%%%%%%%%%%%%%%%%%%%%%%%%%%%%%%%%%%%%%%%%%%%%%%%%%%

% % customize dictum format, option A
% \setkomafont{dictumtext}{\itshape\small}
% \setkomafont{dictumauthor}{\normalfont}
% \renewcommand*\dictumwidth{\linewidth}
% \renewcommand*\dictumauthorformat[1]{--- #1}
% \renewcommand*\dictumrule{}

% customize dictum format, option B
\setkomafont{dictumtext}{\itshape \rmfamily}
\setkomafont{dictumauthor}{\rmfamily\normalfont}
\renewcommand*\dictumwidth{0.5 \linewidth}
\renewcommand*\dictumauthorformat[1]{--- #1}
\renewcommand*\dictumrule{}

% wider dictum
\newcommand{\widedictum}[2][1]{{\renewcommand*\dictumwidth{0.7 \linewidth}\dictum[#1]{#2}}}



%%%%%%%%%%%%%%%%%%%%%%%%%%%%%%%%%%%%%%%%%%%%%%%%%%%%%%%%%%%%
% Larger section etc titles
%%%%%%%%%%%%%%%%%%%%%%%%%%%%%%%%%%%%%%%%%%%%%%%%%%%%%%%%%%%%

% Here: everything one size larger than Koma-Script default
\setkomafont{chapter}{\Huge}
% \setkomafont{section}{\LARGE}
% \setkomafont{subsection}{\Large}
% \setkomafont{subsubsection}{\large}



%%%%%%%%%%%%%%%%%%%%%%%%%%%%%%%%%%%%%%%%%%%%%%%%%%%%%%%%%%%%
% Emphasis
%%%%%%%%%%%%%%%%%%%%%%%%%%%%%%%%%%%%%%%%%%%%%%%%%%%%%%%%%%%%

% \emph for bold instead of italic
\makeatletter
\DeclareRobustCommand{\em}{%
  \@nomath\em \if b\expandafter\@car\f@series\@nil
  \normalfont \else \bfseries \fi}
\makeatother 

 

%%%%%%%%%%%%%%%%%%%%%%%%%%%%%%%%%%%%%%%%%%%%%%%%%%%%%%%%%%%%
% Itemize and enumerate
%%%%%%%%%%%%%%%%%%%%%%%%%%%%%%%%%%%%%%%%%%%%%%%%%%%%%%%%%%%%

% Itemize with smaller bullets
\setlist[itemize,1]{label=\raisebox{0.1ex}{\scriptsize$\bullet$}}

% less space around items
% \setlist{noitemsep} % no space at all
% \setlist{itemsep=4.5pt plus 2.0pt minus 1.0pt,
% parsep=4.5pt plus 2.0pt minus 1.0pt,
% partopsep=3.0pt plus 1.0pt minus 1.0pt,
% topsep=9.0pt plus 3.0pt minus 5.0pt} % defaults
\setlist{itemsep=1.5pt plus 1.0pt minus 0.5pt,
parsep=1.5pt plus 1.0pt minus 0.5pt,
partopsep=3.0pt plus 1.0pt minus 1.0pt,
topsep=3.0pt plus 1.5pt minus 2.5pt} % mine 



%%%%%%%%%%%%%%%%%%%%%%%%%%%%%%%%%%%%%%%%%%%%%%%%%%%%%%%%%%%%
% Other design elements
%%%%%%%%%%%%%%%%%%%%%%%%%%%%%%%%%%%%%%%%%%%%%%%%%%%%%%%%%%%%

% always center floats, also smaller font for tables
\makeatletter
\g@addto@macro\@floatboxreset\centering\small
\makeatother

% Running header
\setkomafont{pageheadfoot}{\sffamily \upshape \small}



%%%%%%%%%%%%%%%%%%%%%%%%%%%%%%%%%%%%%%%%%%%%%%%%%%%%%%%%%%%%
% Technical stuff
%%%%%%%%%%%%%%%%%%%%%%%%%%%%%%%%%%%%%%%%%%%%%%%%%%%%%%%%%%%%

% feynmf stuff
\DeclareGraphicsRule{*}{mps}{*}{}
\unitlength = 1pt 

% pdf file info
\makeatletter
\hypersetup{pdftitle=\@title, pdfauthor=\@author}
\makeatother

%\bibliographystyle{johannstyle}
\bibliographystyle{jthesis2}
\bibpunct{[}{]}{,}{n}{,}{,}

\renewcommand\bibname{References}
\addto\captionsbritish{\renewcommand{\bibname}{References}}

% don't number subsubsections, but potentially include in toc
\setcounter{secnumdepth}{2}
\setcounter{tocdepth}{2}



%%%%%%%%%%%%%%%%%%%%%%%%%%%%%%%%%%%%%%%%%%%%%%%%%%%%%%%%%%%%
% autoref
%%%%%%%%%%%%%%%%%%%%%%%%%%%%%%%%%%%%%%%%%%%%%%%%%%%%%%%%%%%%

% Labels
\addto\extrasbritish{\def\chapterautorefname~#1\null{Chapter~#1\null}}%
\addto\extrasbritish{\def\sectionautorefname~#1\null{Section~#1\null}}%
\addto\extrasbritish{\def\subsectionautorefname~#1\null{Section~#1\null}}%
\addto\extrasbritish{\def\appendixautorefname~#1\null{Appendix~#1\null}}%
\addto\extrasbritish{\def\figureautorefname~#1\null{Figure~#1\null}}%
\addto\extrasbritish{\def\tableautorefname~#1\null{Table~#1\null}}%
\addto\extrasbritish{\def\equationautorefname~#1\null{Equation~(#1)\null}}%