
%%%%%%%%%%%%%%%%%%%%%%%%%%%%%%%%%%%%%%%%%%%%%%%%%%%%%%%%%%%%
\chapter{Appendix}
\label{chapter:appendix}
%%%%%%%%%%%%%%%%%%%%%%%%%%%%%%%%%%%%%%%%%%%%%%%%%%%%%%%%%%%%



%%%%%%%%%%%%%%%%%%%%%%%%%%%%%%%%%%%%%%%%%%%%%%%%%%%%%%%%%%%%
\section{EFT bases}
\label{chapter:appendix_bases}
%%%%%%%%%%%%%%%%%%%%%%%%%%%%%%%%%%%%%%%%%%%%%%%%%%%%%%%%%%%%

\comment{This entire section still has to be re-written}


%%%%%%%%%%%%%%%%%%%%%%%%%%%%%%%%%%%%%%%%%%%%%%%%%%%%%%%%%%%%
\subsection{SILH basis}
\label{chapter:appendix_bases_SILH}
%%%%%%%%%%%%%%%%%%%%%%%%%%%%%%%%%%%%%%%%%%%%%%%%%%%%%%%%%%%%

The SILH Lagrangian reads
%
\begin{align}
  \label{eq:EFT} \lgr{EFT} =& \lgr{SM} + \frac{\bar{c}_H}{2v^2}
\, \partial^\mu (\phisq) \, \partial_\mu (\phisq) +
\frac{\bar{c}_T}{2v^2} \, (\phi^\dagger \, \overleftrightarrow{D}^\mu
\, \phi) \, (\phi^\dagger \, \overleftrightarrow{D}_\mu \, \phi)
-\frac{\bar{c}_6\lambda}{v^2} (\phisq)^3 \notag \\ &+
\frac{ig\bar{c}_W}{2m^2_W} \, (\phi^\dagger \, \sigma^k
\overleftrightarrow{D}^\mu\phi) \, D^\nu \, {W^k}_{\mu\nu} +
\frac{ig'\bar{c}_B}{2m_W^2} \, (\phi^\dagger \,
\overleftrightarrow{D}^\mu \, \phi) \, \partial^\nu \, B_{\mu\nu}
\notag \\ &+ \frac{ig \, \bar{c}_{HW}}{m_W^2} \, (D^\mu \,
\phi^\dagger) \, \sigma^k \, (D^\nu \, \phi) \, W^k_{\mu\nu} +
\frac{ig'\bar{c}_{HB}}{m_W^2} (D^\mu\phi^\dagger) \, (D^\nu \, \phi)
\, B_{\mu\nu}\notag \\ &+ \frac{g'^2 \bar{c}_\gamma}{m_W^2} \,
(\phisq) \, B_{\mu\nu} \, B^{\mu\nu} + \frac{g_s^2 \bar{c}_g}{m_W^2}
\, (\phisq) \, G^A_{\mu\nu} \, G^{\mu\nu\, A} \notag \\ &- \left[
\frac{\bar{c}_u}{v^2} \, y_u \, (\phisq) (\phi^\dagger\cdot \,
\overline{Q}_L) \, u_R + \frac{\bar{c}_d}{v^2} \, y_d \, (\phisq)
(\phi \, \overline{Q}_L) \, d_R + \frac{\bar{c}_\ell}{v^2} \, y_\ell
\, (\phisq) (\phi \, \overline{L}_L) \, \ell_R + \text{h.c.} \,
\right] \,.
\end{align}
%
Here, $g = e/{s_w}, g' = e/{c_w}$, and $g_s$ stand for the SM gauge
couplings and $\lambda$ denotes the usual Higgs quartic coupling. The
normalization of the dimension-6 Wilson coefficients $\overline{c}_i$
includes conventional prefactors which reflect a bias concerning their
origin. We present further details on the EFT setup, the translation
between the different bases, and the connection to Higgs observables
in Appendix~\ref{sec:validity_ap-eft}.

%%%%%%%%%%%%%%%%%%%%%%%%%%%%%%%%%%%%%%%%%%%%%%%%%%%%%%%%%%%%



As mentioned in Sec.~\ref{sec:theory_eff}, we here adopt the notation
and conventions of Ref.~\cite{Alloul:2013naa}, which is based on the
SILH framework with the decomposition and
normalization of the Wilson coefficients defined in
Ref.~\cite{silh}. For our purposes, it is enough to single
out the subset that encodes all possible new physics contributions to
the Higgs sector compatible with CP conservation and the flavor
structure of the SM. These are given in Tab.~\ref{tab:operators} and
correspond to the Lagrangian in Eq.~\eqref{eq:EFT}.
 
%-----------------------------------------------------------
\begin{table}[b!] 
  \centering
    \renewcommand{\arraystretch}{1.3}
    \begin{tabular}[t]{r @{${}={}$}l} 
      \toprule
      \multicolumn{2}{c}{Higgs fields}  \\
      \midrule
      $\ope{H}$ & $\partial^\mu(\phi^\dagger\,\phi)\,\partial_\mu\,(\phi^\dagger\,\phi)$ \\ 
      $\ope{6}$ & $(\phisq)^3$  \\
      $\ope{T}$ & $(\phi^\dagger\,\overleftrightarrow{D}^\mu\,\phi)\,(\phi^\dagger\,\overleftrightarrow{D}_\mu\,\phi)$ \\
      \bottomrule
      \toprule
      \multicolumn{2}{c}{Higgs and fermion fields} \\
      \midrule
      $\ope{u}$ & $(\phisq)\,(\phi^\dagger\cdot\,\overline{Q}_L)\,u_R$ \\
      $\ope{d}$ & $(\phisq)\,(\phi\, \overline{Q}_L)\,d_R$ \\
      $\ope{\ell}$ & $(\phisq)\,(\phi\, \overline{L}_L)\,l_R$ \\
      \bottomrule 
    \end{tabular}
    \hspace*{1cm}
    \begin{tabular}[t]{r @{${}={}$}l} 
      \toprule
      \multicolumn{2}{c}{Higgs and gauge boson fields} \\
      \midrule
      $\ope{HB}$ & $(D^\mu\phi^\dagger)\,(D^\nu\phi)\,B_{\mu\nu}$ \\
      $\ope{HW}$ & $(D^\mu\phi^\dagger)\,\sigma^k\,(D^\nu\,\phi)\,W^k_{\mu\nu}$ \\ 
      $\ope{g}$ & $(\phisq)\,G^A_{\mu\nu}\,G^{\mu\nu\, A}$ \\
      $\ope{\gamma}$ & $(\phisq)\,B_{\mu\nu}\,B^{\mu\nu}$ \\
      $\ope{B}$ & $(\phi^\dagger\,\overleftrightarrow{D}^\mu\,\phi)\,(\partial^\nu\,B_{\mu\nu})$ \\ 
      $\ope{W}$ & $\left(\phi^\dagger\,\sigma^k\,\overleftrightarrow{D}^\mu\phi\right)\,(D^\nu\,W^k_{\mu\nu})$ \\
      \bottomrule
    \end{tabular}
  \caption{Dimension-6 operators considered in our analysis. These
    correspond to a subset of the most general effective operator basis
    ~\cite{silh} describing new physics effects to the SM
    Higgs sector with CP-invariance and SM-like fermion structures.}
  \label{tab:operators}
\end{table}
%-----------------------------------------------------------

The conventions for how covariant derivatives act on the Higgs,
fermion and gauge vector fields are fixed as follows:
%
\begin{align}
  D_\mu \phi &= \partial_\mu \phi - \dfrac {i g'} 2 B_\mu \phi - i g \dfrac {\sigma^a} 2 \, W^a_\mu \, \phi \,,\notag \\
  D_\mu F_L &= \partial_\mu F_L - i g' \dfrac{Y_{F_L}} 2 B_\mu F_L - i g \dfrac {\sigma^a} 2 \, W^a_\mu \, F_L \,,\notag \\
  D_\mu V_\nu^a &= \partial_\mu V_\nu^a + g \, \varepsilon^{abc} \, W_\mu^b \, V_\nu^c \,,\notag \\
  D_\mu W^a_{\nu \rho} &= \partial_\mu W_{\nu \rho}^a + g \, \varepsilon^{abc} \,W_\mu^b \, W_{\nu\rho}^c \,. 
\label{eq:covariant}
\end{align}
%
While the effective Lagrangian in Eq.~\eqref{eq:EFT} is written in
terms of the fundamental SM gauge fields, the connection to physics
observables is more easily seen in the mass-eigenstate basis, which we
can write as
%
\begin{alignat}{5}
 \lgr{} \supset & -\dfrac{m^2_{H}}{2v}\,g^{(1)}_{HHH}\,HHH  + \dfrac{1}{2}\,g_{HHH}^{(2)}\,H(\partial_\mu H)\,(\partial^\mu H) \notag \\
  & - \dfrac{1}{4}\,g_{ggH}\,G^{\mu\nu\,A}\,G_{\mu\nu}^A\,H - \dfrac{1}{4}\,g_{\gamma\gamma H}\,F^{\mu\nu}\,F_{\mu\nu}\,H \notag \\
  & -\dfrac{1}{4}\,g_{Z}^{(1)}\,Z_{\mu\nu}\,Z^{\mu\nu}\,H -g_{Z}^{(2)}\,Z_{\nu}\,\partial_\mu \,Z^{\mu\nu}\,H + \dfrac{1}{2}\,g^{(3)}_{Z}\,Z_\mu Z^{\mu}\,H \notag 
  \\
  & - \dfrac{1}{2}\,g_{W}^{(1)}\,W^{\mu\nu}\,W^\dagger_{\mu\nu}\,H - 
  \left[g_{W}^{(2)} W^\nu\,\partial^\mu W^\dagger_{\mu\nu} H + \text{h.c.} \right] + g_{W}^{(3)}\,m_W\,W^\dagger_\mu W^\mu\, H \notag \\ 
  & -\left[g_{u}\dfrac{1}{\sqrt{2}}\left(\bar{u}P_R u\right)H + g_{d}\dfrac{1}{\sqrt{2}}\left(\bar{d}P_R d\right)H + g_{\ell}\dfrac{1}{\sqrt{2}}\left(\bar{\ell}P_R \ell\right)H + \text{h.c.}
  \right]
 \label{eq:masslag},
\end{alignat}
%
with the different effective couplings $g_i$ quoted in
Tab.~\ref{tab:coefficients}.  More details on the notation and
conventions can be found in Ref.~\cite{Alloul:2013naa}.

%-----------------------------------------------------------
\begin{table}[t] 
  \renewcommand{\arraystretch}{1.5}
  \centering
  \begin{tabular}{r l l}
    \toprule
    Coupling & Operators & Expression \\
    \midrule
    $g^{(1)}_{Z} $ & $\ope{HB}, \ope{HW}, \ope{H}gam$ & $\frac{2g}{m_Wc^2_w}\left[\bar{c}_{HB}s^2_w-4\bar{c}_{\gamma}s_w^4+c^2_w \,\bar{c}_{HW} \right] $ \\
    $g^{(2)}_{Z} $ & $\ope{HW}, \ope{HB}, \ope{W}, \ope{B}$ & $ \frac{g}{m_Wc^2_w}\left[(\bar{c}_{HW} + \bar{c}_W)c^2_w + (\bar{c}_B + \bar{c}_{HB})s^2_w\right]$\\  
    $g^{(3)}_{Z} $ & $\ope{H}, \ope{T}, \ope{\gamma}$ & $\frac{g m_W}{c_w^2}\left[1-\frac{1}{2}\bar{c}_H - 2\bar{c}_T + 8\bar{c}_\gamma\frac{s_w^4}{c^2_w} \right]$ \\
    \midrule
    $g^{(1)}_{W} 
    $ & $\ope{HW}$ & $\frac{2g}{m_W}\,\bar{c}_{HW} $ \\
    $g^{(2)}_{W}$ & $\ope{HW},\ope{W}$ & $\frac{g}{m_W}\,\left[\bar{c}_W + \bar{c}_{HW}\right]$ \\  
    $g^{(3)}_{W}$ & $\ope{H}$ & $g(1-\frac{1}{2}\bar{c}_H)$\\
    \midrule
    $g_{f}$ & $\ope{H}, \ope{f} \quad (f = u,d,\ell)$ & $\frac{\sqrt{2}m_f}{v}\left[1-\frac{1}{2}\bar{c}_H + \bar{c}_f\right] $ \\
    \midrule
    $g_{g} $ & $\ope{H}, \ope{g}$ & $g_H - \frac{4 \bar{c}_g g_s^2 v}{m_W^2}$ \\
    \midrule
    $g_{\gamma}$ & $\ope{H}, \ope{\gamma}$ & $a_H - \frac{8 g \bar{c}_\gamma s^2_w}{m_W}$\\
    \midrule
    $g^{(1)}_{HHH}$  & $\ope{H}, \ope{6}$ & $1+\frac{5}{2} \bar{c}_6 - \frac{1}{2}\bar{c}_{H}$ \\ 
    $g^{(2)}_{HHH} $  & $\ope{H}$ & $\frac{g}{m_W}\,\bar{c}_H$ \\
    \bottomrule
  \end{tabular} 
  \caption{Subset of the dimension-6 operators which enter the different
    leading-order Higgs couplings which are relevant for LHC
    phenomenology, in the notation and conventions of
    Ref.~\cite{Alloul:2013naa} (see text).  The different superscripts
    denote the various terms in the Lagrangian in Eq.~\eqref{eq:masslag} and
    correspond to either a SM-like interaction with a rescaled coupling
    strength or to genuinely new Lorentz structures. The weak coupling
    constant is written as $g \equiv e/s_w$.  The SM contribution to the
    loop-induced Higgs coupling to the gluons (photons) is denoted by
    $g_H$ ($a_H$).}
  \label{tab:coefficients}
\end{table}
%-----------------------------------------------------------

Note that the Higgs-fermion coupling shift is given by $g_f \propto
y_f (1 - \bar{c}_H/2  +  3 \bar{c}_f/2)$, but $\ope{f}$ also
shifts the fermion masses to $m_f = y_f v (1 + \bar{c}_f/2) /
\sqrt{2}$, yielding the result given above. Similarly, $\ope{H}$ and
$\ope{\gamma}$ generate additional contributions to the Higgs-boson and
gauge-boson kinetic terms, which are restored to their canonical form
by the field re-definitions
%
\begin{align}
H &\to H \, \left(1-\dfrac{1}{2}c_H \right), & 
Z_\mu &\to Z_\mu \, \left( 1+\dfrac{4s_w^4}{c_w^2}c_\gamma \right) \,,\notag \\
&& 
A_\mu &\to A_\mu \, \left( 1+4s_w^2 \, c_\gamma \right) -
  Z_\mu \left( \dfrac{8s_w^3}{c_w} c_\gamma \right).
\end{align}
%
None of the operators considered in this basis affects the relations
between $g$, $m_W$, $v$ and $G_F$, so the SM relations
%
\begin{align}
  m_W &= \dfrac {g\,v} 2 \,, &
  G_F &=\dfrac {\sqrt{2} \, g^2} {8\, m_W^2} = \dfrac 1 {\sqrt{2} v^2} \,,
\end{align}
%
can always be used to translate these coupling shifts from one scheme
of input parameters to another.

Dimension-6 operators result in a modified pattern of Higgs
interactions, leading to coupling shifts $g_{xxH} \equiv
g_{xxH}^\text{SM}(1+\Delta_x)$ and also genuinely novel Lorentz
structures. Interestingly, in general more than one of the
effective operators in Tab.~\ref{tab:operators} contributes to a
given Higgs interaction in the mass basis, implying that it is in
general not possible to establish a one-to-one mapping between Wilson
coefficients and distorted Higgs couplings.

Note that the Wilson coefficients of the operators $\ope{T}$ and $\ope{B}+\ope{W}$
are strongly constrained by electroweak precision data~\cite{silh}. In
this work, we allow ourselves, on occasion, to ignore these bounds to
more distinctly illustrate the effects in the Higgs sector.

Translations between effective operator bases can be performed with
the help of equations of motion, field redefinitions, integration by
parts and Fierz identities. Here we quote a number of such relations
which turn out to be particularly useful for the practitioner. For
example, in addition to the effective operators in the SILH basis,
we often find the operators
%
\begin{align}
  \ope{r} &= \phisq\, (D_\mu \phi)^2\,, &
  \ope{HF}' &= \left( \bar{f}_L \, \gamma^\mu \, \sigma^a \, f_L \right) \left( \phi^\dagger \, \sigma^a \, \overleftrightarrow{D}_\mu \, \phi \right)  \,,
  \notag \\
  \ope{D} &= \left( D^2 \, \phi \right)^2\,, &
  \hat{\ope{}}'_{HH} &= \left( \phi^\dagger \, \sigma^a \, \overleftrightarrow{D}_\mu \, \phi \right)\left( \phi^\dagger \, \sigma^a \, \overleftrightarrow{D}_\mu \, \phi \right)\,.
  \label{eq:EFT_OHf}
\end{align}
%
$\hat{\ope{}}'_{HH}$ can be replaced by using the completeness relation of the Pauli matrices, which for arbitrary SU(2) doublets $\xi,\,\chi,\,\eta,\,\psi$ leads to
\begin{align}
(\xi^\dagger \sigma^a \chi)(\eta^\dagger \sigma^a \psi) &=
\sum_{ijkl} \xi^*_i \sigma^a_{ij} \chi_j \, \eta^*_k \sigma^a_{kl} \psi_l \notag \\
&= \sum_{ijkl} (2\delta_{il}\delta_{jk}-\delta_{ij}\delta_{kl})\xi^*_i \chi_j \, \eta^*_k \psi_l = 2(\xi^\dagger\psi)(\eta^\dagger\chi)-(\xi^\dagger \chi)(\eta^\dagger\psi)\,.
\end{align}
Thus we find
\begin{align}
\hat{\ope{}}'_{HH} &= (\phi^\dagger\,\sigma^a\,D^\mu \phi)^2
+((D^\mu\phi^\dagger)\,\sigma^a\,\phi)^2
-2((D^\mu\phi^\dagger)\,\sigma^a\,\phi)(\phi^\dagger\,\sigma^a\,D_\mu \phi) \notag \\
&= (\phi^\dagger\,D^\mu \phi)^2
+ ((D^\mu\phi^\dagger)\,\phi)^2
-2\left[ 2((D^\mu\phi^\dagger)\,D^\mu \phi)(\phi^\dagger\phi) - 
 ((D^\mu\phi^\dagger)\,\phi)(\phi^\dagger\,D^\mu \phi) \right]
\notag \\
&= \ope{H} - 4 \ope{r} \,.
  \label{eq:EFT_JHJH_replacement}
\end{align}
%
The equation of motion for the $W$ fields,
%
\begin{align}
D^\nu W^a_{\mu \nu} 
= - i g \, \phi^\dagger \dfrac {\sigma^a} {2} \overleftrightarrow{D}_\mu \phi 
  - g \sum_f \bar{f}_L \dfrac {\sigma^a} {2} \gamma_\mu f_L  \,,
 \label{eq:eom-w}
\end{align}
gives rise to the identity
\begin{align}
  \sum_f \ope{HF}' = \dfrac {2} {g} \, \ope{W} - {i} \, \ope{H} + 4 i \, \ope{r}  \,.
  \label{eq:EFT_OHf_replacement}
\end{align}
%
A global redefinition $\phi \to \phi + \alpha \, (\phisq) \phi / v^2$ generates a shift in the Wilson coefficients
\begin{alignat}{5}
 c_H \to c_H + 2\alpha\,, \quad
 c_r \to c_r + 2\alpha\,, \quad c_6 \to c_6 {+} 4\alpha\,, \quad c_f \to c_f {+} \alpha\,,
 \label{eq:wilson-shift}
\end{alignat}
so that
with the choice $\alpha=-c_r/2$ one can eliminate the operator
$\ope{r}$ in favor of other operators:
%
\begin{align}
  \ope{r} \leftrightarrow \biggl\{ - \dfrac 1 2 \, \ope{H} + 2 \, \lambda \, \ope{6} + \sum_f \left[ \dfrac 1 2 \, y_f \, \ope{f} + \text{h.c.}\right] \biggr\} \,.
  \label{eq:EFT_Or_replacement}
\end{align}
%
Finally, $\ope{D}$ can be exchanged for others using the equation of motion for
$\phi$,
%
\begin{align}
  D^2 \phi &= - \mu^2  \, \phi - 2\, \lambda \, \phisq \, \phi - \sum_\text{gen.} \left[y_u \, \bar{Q}^T_L \, u_R + y_d \, \bar{d}_R \, Q_L + y_\ell \, \bar{\ell}_R \, L_L \right] \,. 
\end{align}
%
This leads to
%
\begin{align}
  \ope{D} = \mu^4 \, \phisq 
 + 4 \, \lambda \, \mu^2 \, (\phisq)^2 
 + \mu^2 \, \sum_f y_f \, \bar{f}_L \phi f_R  \
 + 4 \, \lambda^2 \, (\phisq)^3 + 2 \, \lambda \, \sum_f y_f \, \phisq \, \left( \bar{f}_L \phi f_R \right) \,.
\end{align}
%
The first three terms lead to a renormalization of the SM parameters
$\mu$, $\lambda$, $y_f$, without any impact on physical
observables. The last two terms, however, means that $\ope{D}$ is
equivalent to the combination
%
\begin{align}
  \ope{D} \leftrightarrow 4 \, \lambda^2 \, \ope{6} + 2 \, \lambda \, \sum_f \left( y_f \, \ope{f} + \text{h.c.} \right)\,.
  \label{eq:EFT_OD_replacement}
\end{align}





%%%%%%%%%%%%%%%%%%%%%%%%%%%%%%%%%%%%%%%%%%%%%%%%%%%%%%%%%%%%
\subsection{HLM basis}
%%%%%%%%%%%%%%%%%%%%%%%%%%%%%%%%%%%%%%%%%%%%%%%%%%%%%%%%%%%%

Aside from the relatively simple case of the multi-Higgs sector
extensions, we make use of the covariant derivative
expansion~\cite{Gaillard:1985uh,Cheyette:1987qz} to analytically carry out
the matching between the different UV completions to
their corresponding EFT description.  The method has been recently
reappraised in Ref.~\cite{Henning:2014wua} and employed in a number of
studies~\cite{heft_limitations,heft_limitations2,Chiang:2015ura,Huo:2015exa}.
By applying this method, the Wilson coefficients are readily obtained
in a different operator basis (henceforth dubbed HLM),
%
\begin{align}
  \lgr{HLM} = \sum_i \dfrac{k_i} {\Lambda^2} \ope{i}''.
\end{align}
%
The HLM operators involving Higgs fields and their interaction with
gauge bosons are listed in Tab.~\ref{tab:ops2}.  In addition, the HLM
basis contains a subset of operators with no direct correspondence to
the bosonic SILH operators, which must be rewritten with the help of
equations of motion and field redefinitions, as we discuss below.

%-----------------------------------------------------------
\begin{table}[tb] 
  \renewcommand{\arraystretch}{1.3}
  \setlength{\tabcolsep}{1ex}
  \centering
    \begin{tabular}[t]{r @{${}={}$}l} 
    \toprule
    \multicolumn{2}{c}{HLM basis}  \\
    \midrule
    $\ope{H}''$ & $\frac{1}{2} \, \partial^\mu (\phisq) \,\partial_\mu (\phisq)$ \\ 
    $\ope{6}''$ & $(\phisq)^3$  \\
    $\ope{T}''$ & $\frac{1}{2} \left(\phi^\dagger \, \overleftrightarrow{D}^\mu \, \phi \right) \, \left(\phi^\dagger \, \overleftrightarrow{D}_\mu \, \phi \right)$ \\
    $\ope{B}''$ & $\frac{i g'}{2} \left( \phi^\dagger \, \overleftrightarrow{D}^\mu \, \phi \right) \, \partial^\nu \, B_{\mu\nu}$ \\
    $\ope{W}''$ & $\frac{i g}{2} \left(\phi^\dagger\,\sigma^k\,\overleftrightarrow{D}^\mu\phi\right)\,(D^\nu\,W^k_{\mu\nu})$ \\
    $\ope{GG}''$ & $g_s^2 (\phisq)\,G^A_{\mu\nu}\,G^{\mu\nu\, A}$ \\ 
    $\ope{BB}''$ & $g'^2(\phisq)\,B_{\mu\nu}\,B^{\mu\nu}$ \\
    $\ope{WW}''$ & $g^2 (\phisq)\,W^k_{\mu\nu}\,W^{\mu\nu\, k}$ \\
    $\ope{WB}''$ & $g g' \left(\phi^\dagger\sigma^k\phi \right)  \, B_{\mu\nu}\,W^{\mu\nu\, k}$ \\
    \bottomrule 
  \end{tabular}
  \hspace{1cm}
  \begin{tabular}[t]{r @{${}={}$}l} 
    \toprule
    \multicolumn{2}{c}{HISZ basis} \\
    \midrule
    $\ope{\phi1}'$  & $(D_\mu\phi)^\dagger\phi\,\phi^\dagger(D^\mu\phi)$ \\
    $\ope{\phi2}'$  & $\frac{1}{2}\partial^\mu(\phi^\dagger\phi)\,\partial_\mu(\phi^\dagger\phi)$ \\ 
    $\ope{\phi3}'$  &  $\frac{1}{3}(\phisq)^3$ \\
    $\ope{GG}'$  &  $(\phisq)\,G^A_{\mu\nu}\,G^{\mu\nu\, A}$ \\
    $\ope{BB}'$  &  $\phi^\dagger\,\hat{B}_{\mu\nu}\,\hat{B}^{\mu\nu}\,\phi  = -\frac{g'^2}{4}\phisq\,B_{\mu\nu}\,B^{\mu\nu}$ \\
    $\ope{WW}'$  &  $\phi^\dagger\,\hat{W}_{\mu\nu}\,\hat{W}^{\mu\nu}\,\phi  = -\frac{g^2}{4}\phisq\,W^k_{\mu\nu}\,W^{\mu\nu\, k}$ \\
    $\ope{BW}'$  &  $\phi^\dagger\,\hat{B}_{\mu\nu}\,\hat{W}^{\mu\nu}\,\phi  = -\frac{g\,g'}{4}(\phi^\dagger\sigma^k\phi)\,B_{\mu\nu}\,W^{\mu\nu\, k}$ \\
    $\ope{B}'$  &  $(D^\mu\phi)^\dagger \hat{B}_{\mu\nu} (D^\nu\phi)  = i \frac{g}{2}(D^\mu\phi^\dagger)(D^\nu\phi)\,B_{\mu\nu}$ \\
    $\ope{W}'$  &  $(D^\mu\phi)^\dagger \hat{W}_{\mu\nu} (D^\nu\phi)  = i \frac{g}{2}(D^\mu\phi^\dagger)\sigma^k( D^\nu\phi)\,W_{\mu\nu}^k$ \\
    \bottomrule
  \end{tabular}
  \caption{Bosonic CP-conserving Higgs operators in the HLM basis
    (left) and the HISZ basis (right). Here $\hat{B}_{\mu\nu}=i g'/2
    B_{\mu\nu}$ and $\hat{W}_{\mu\nu}=i g \sigma^k/2 W_{\mu\nu}^k$.}
  \label{tab:ops2}
\end{table}
%-----------------------------------------------------------

The operators in Tab.~\ref{tab:ops2} translate to the SILH basis via
\begin{align}
\begin{aligned}
  \ope{H}'' &= \frac 1 2 \ope{H}\,, \quad &
  \ope{6}'' &= \ope{6}\,, \quad &
  \ope{T}'' &= \frac 1 2 \ope{T}\,, \quad &
  \ope{B}'' &= \frac {i g'} 2 \ope{B} \,, \quad &
  \ope{W}'' &= \frac {i g} 2 \ope{W}\,,  \\
  \ope{GG}'' &= g_s^2 \ope{g}\,, &
  \ope{BB}'' &= g'^2 \ope{\gamma}\,, \quad &
  \ope{WB}'' &= 2i g' \ope{B} - 4i g' \ope{HB} - g'^2 \ope{\gamma}\,, \hspace{-10em} \\
  \ope{WW}'' &= -2i g' \ope{B} + 2i g \ope{W} + 4i g' \ope{HB} - 4 i g \ope{HW}  + g'^2 \ope{\gamma}  \, . \hspace{-20em}
\end{aligned}
\end{align}
%
In addition, the HLM basis contains extra operators with no SILH counterpart, 
%
\begin{align}
  \ope{R}'' = \phisq \left(D_\mu \, \phi\right)^\dagger \left(D^\mu \, \phi \right)\,, \qqquad
  \ope{D}'' = \left( D^2 \phi \right)^2 \, ,
  \label{eq:HLM_add_ops}
\end{align}
%
which can be eliminated using Eq.~\eqref{eq:EFT_Or_replacement}
and Eq.~\eqref{eq:EFT_OD_replacement}, respectively.  The Wilson
coefficients $k_i$ of the HLM basis translate to the SILH coefficients
$\bar{c}_i$ as follows:
%
\begin{align}
  \bar{c}_H &= \dfrac {v^2} {\Lambda^2} \, \left( k_H - k_R \right) \,, &
  \bar{c}_B &= \dfrac {v^2} {\Lambda^2} \, \dfrac {g^2} 4 \, \left(k_B + 4 \, k_{WB} - 4 \, k_{WW} \right) \notag \,, \\
  \bar{c}_T &= \dfrac {v^2} {\Lambda^2} \, k_T \,, & 
  \bar{c}_W &= \dfrac {v^2} {\Lambda^2} \, \dfrac {g^2} 4 \, \left(k_W + 4  \, k_{WW} \right) \notag \,, \\
  \bar{c}_6 &= - \dfrac {v^2} {\Lambda^2} \, \left( \dfrac {k_6} {\lambda} + 2\, k_R + 4 \, \lambda \, k_D \right) \,, & 
  \bar{c}_{HB} &= \dfrac {v^2} {\Lambda^2} \, g^2 \, \left(  k_{WW} - k_{WB} \right) \notag \,, \\
  \bar{c}_g &= \dfrac {v^2} {\Lambda^2} \, \dfrac {g^2} 4 \, k_{GG} \,,  &   
  \bar{c}_{HW} &= - \dfrac {v^2} {\Lambda^2} \, g^2 \, k_{WW} \notag \,, \\
  \bar{c}_\gamma &= \dfrac {v^2} {\Lambda^2} \, \dfrac {g^2} 4 \, \left( k_{BB} - k_{WB} + k_{WW} \right) \,, &
  \bar{c}_f &= - \dfrac {v^2} {\Lambda^2} \, \left( \dfrac 1 2 \, k_R + 2 \, \lambda \, k_D \right) \,,
\end{align}
%
where for the sake of completeness we have included the coefficients
of the redundant operators given in Eq.~\eqref{eq:HLM_add_ops}.



%%%%%%%%%%%%%%%%%%%%%%%%%%%%%%%%%%%%%%%%%%%%%%%%%%%%%%%%%%%%
\subsection{SILH to HISZ}
%%%%%%%%%%%%%%%%%%%%%%%%%%%%%%%%%%%%%%%%%%%%%%%%%%%%%%%%%%%%

We also give the conversion to the popular HISZ
basis~\cite{Hagiwara:1993ck} (see also
Refs.~\cite{Corbett:2012ja,sfitter_last} for recent studies in this
framework)
%
\begin{align}
  \lgr{HISZ} = \sum_i \dfrac{f_i} {\Lambda^2} \ope{i}' \, ,
\end{align}
%
with Higgs-gauge operators given in Tab.~\ref{tab:ops2}.  We use the
same conventions for the covariant derivative as above (note that this
is not the case in some of the cited literature). The operators can
then be translated via the relations
%
\begin{align}
\ope{H}  &= 2\ope{\phi2}'\,, \qquad &
\ope{W}  &= \dfrac{2 i}{g}  \left( \ope{WW}' +  \ope{BW}' - 2\ope{W}'   \right)\,,  \quad & 
\ope{HW} &= -\dfrac{2i }{g} \ope{W}'\,,  \notag \\ 
\ope{T}  &= 2\ope{\phi2}'  -  4\ope{\phi1}'\,,  &
\ope{B}  &= \dfrac{2 i}{g'}  \left( \ope{BB}' +  \ope{BW}' - 2\ope{B}' \right)\,, & 
\ope{g}  &= \ope{GG}' \,,  \notag \\
\ope{6} &= 3 \ope{\phi3}' \,, & 
\ope{HB}  &= -\dfrac{2 i }{g'}\ope{B}'\,, & 
\ope{\gamma} &= -\dfrac{4}{g'^2}  \ope{BB}'  \,.
\end{align}
%
The HISZ basis also includes the redundant operator $\ope{\phi4}' =
(D_\mu\phi)^\dagger(D^\mu\phi)\,\phi^\dagger\phi$, which can be
removed using Eq.~\eqref{eq:EFT_Or_replacement}.  For the coefficients,
we find
%
\begin{align}
  \bar{c}_H &= \dfrac{v^2}{\Lambda^2}\,\left(\dfrac{1}{2}f_{\phi1}+f_{\phi2}\right)\,, &
  \bar{c}_W &= -\dfrac{v^2}{\Lambda^2}\,\dfrac{g^2}{4}f_{WW} \notag\,, \\
  \bar{c}_T &= -\dfrac{v^2}{\Lambda^2}\,\dfrac{1}{2}f_{\phi1}\,, & 
  \bar{c}_B &= \dfrac{v^2}{\Lambda^2}\,\dfrac{g^2}{4} (f_{WW}-f_{BW}) \notag\,, \\
  \bar{c}_6 &= -\dfrac{v^2}{\Lambda^2}\,\dfrac{1}{3\lambda}f_{\phi3}\,, &
  \bar{c}_{HW} &= \dfrac{v^2}{\Lambda^2}\,\dfrac{g^2}{8} (f_{W}+2f_{WW}) \notag\,, \\
  \bar{c}_g &= \dfrac{v^2}{\Lambda^2}\,\dfrac{g^2}{4g_s^2}f_{GG}\,, &
  \bar{c}_{HB} &= \dfrac{v^2}{\Lambda^2}\,\dfrac{g^2}{8} (f_{B}+2f_{BW}-2f_{WW}) \notag\,, \\
  \bar{c}_\gamma &= \dfrac{v^2}{\Lambda^2}\,\dfrac{g^2}{16} (f_{BW}-f_{BB}-f_{WW}) \, .
\end{align}




%%%%%%%%%%%%%%%%%%%%%%%%%%%%%%%%%%%%%%%%%%%%%%%%%%%%%%%%%%%%
\section{Model fineprint}
\label{chapter:appendix_models}
%%%%%%%%%%%%%%%%%%%%%%%%%%%%%%%%%%%%%%%%%%%%%%%%%%%%%%%%%%%%


%%%%%%%%%%%%%%%%%%%%%%%%%%%%%%%%%%%%%%%%%%%%%%%%%%%%%%%%%%%%
\subsection{Singlet extension}
\label{sec:appendix_models_singlet}
%%%%%%%%%%%%%%%%%%%%%%%%%%%%%%%%%%%%%%%%%%%%%%%%%%%%%%%%%%%%

The singlet model is defined in
Equations~\eqref{eq:validity_singlet_lagrangian} and
\eqref{eq:validity_singlet_potential} in
Section~\ref{sec:validity_singlet}. Ignoring the Goldstones, the
scalar doublet and singlets fields can be expanded into components as
%
\begin{align}
  \phi &= \frac{1}{\sqrt{2}} \twovec {1} {v + \phi^0} \,, \notag \\
  S &= \frac{1} {\sqrt{2}} (v_s + s^0) \,,
\end{align}
%
where $v \equiv \sqrt{2}\langle \phi \rangle = 246$~GeV and
$v_s \equiv \sqrt{2}\langle S \rangle$ denote their respective
VEVs. The minimisation condition for this potential can be used to
eliminate the parameters $\mu_{1,2}$ in favor of $v$ and
$v_s$. $\phi^0$ and $s^0$ mix to form a light ($h$) and a heavy ($H$)
mass eigenstate,
%
\begin{align}
  h &=\phi^0  \cos\alpha - s^0 \sin\alpha\,, \notag \\
  % 
  H &= \phi^0 \sin\alpha + s^0 \cos\alpha\,,
\end{align}
%
where
%
\begin{equation}
  \tan(2\alpha) = \frac{\lambda_3vv_s}{\lambda_2 v_s^2 - \lambda_1v^2}\,.
\end{equation}
% 
Their masses are
%
\begin{equation}
  m^2_{h,H} =
  \lambda_1\,v^2
  + \lambda_2\,v_s^2
  \mp |\lambda_1\,v^2 - \lambda_2\,v_s^2| \, \sqrt{1+\tan^2(2\alpha)}
\end{equation}
%
with $m_{H}^2 \approx 2 \lambda_2 v_s^2 \gg m_{h}^2$ in the limit
$v^2 \ll v_s^2$.

% To perform the matching to the EFT,
% we identify the UV scale $\Lambda \equiv \sqrt{2 \lambda_2} v_s \approx m_H$ for $v_s \gg v$. From the
% singlet-doublet mixing one then finds a universal coupling shift of
% the SM-like light Higgs to all other SM particles in
% Eq.~\eqref{eq:shift2}, given by
% %
% \begin{alignat}{5}
%  \Delta \approx -\frac{\sin^2 \alpha}{2} \approx
%   -\dfrac{g_\text{eff}^2}{2} \,\biggl(\cfrac{v}{\Lambda} \biggr)^2 \,, \qquad
%   g_\text{eff} = \dfrac{\lambda_3}{\sqrt{2\lambda_2}} \, .
%  \label{eq:singlet-delta4}
% \end{alignat}
% %
% Integrating out the heavy Higgs boson we find
% %
% \begin{align}
%   \lgr{EFT}\supset \dfrac {\sin^2 \alpha} {2v^2} \; 
%    \partial^\mu (\phi^\dagger \phi) \partial_\mu (\phi^\dagger \phi) +
%    \ord{\Lambda^{-4}} \,.
% \label{eq:singlet-matching}
% \end{align}
% %
% We thus see that, up to dimension-6 operators the
% heavy-singlet--induced BSM effects in Higgs production and decay are
% completely captured by the operator $\ope{H}$
% (cf.\ Tab.~\ref{tab:operators}) with coefficient
% %
% \begin{align}
%   \bar{c}_H = \dfrac{\lambda_3^2}{2\lambda_2} \, \left(\dfrac {v} {\Lambda}\right)^2 + \ord {\frac{v^4}{\Lambda^4} } \,.
% \end{align}
% 

% The light Higgs couplings to fermions and gauge bosons in the singlet model 
% are universally suppressed relative to the SM. In the full model and the
% EFT, respectively, they are given by
% %
% \begin{align}
%   1 + \Delta_x =  \cos \alpha \,,
% \qqquad
%   1 + \Delta_x^\text{EFT} = 1-\dfrac{1}{2}\bar{c}_H \,.
% \end{align}

% A more complex pattern emerges for the
% self-interactions involving at least one heavy Higgs field.  We find
% %
% \begin{alignat}{5}
% g_{hhH}
% &= - \cfrac{g_\text{eff} \,(2m^2_{h} + m^2_{H})}{v_s}\,
%       \left[ 1+ g_\text{eff}\,\cfrac{v^2}{v_s^2} + \ord {
%   \frac{v^3}{v_s^3}} \right] 
% \sim \lambda_3 v_s + \ord{v} \,,
% \notag \\ 
% g_{hHH} 
% &=  \cfrac{g_\text{eff} v \,(m^2_{h} + 2 m^2_{H})}{v_s^2}\, 
%    \left[ 1- g_\text{eff} + \ord { \frac{v}{v_s}} \right] 
% \sim 2\lambda_3v \left( 1- \dfrac{\lambda_3}{2\lambda_2} \right) +
% \ord{\dfrac{v^2}{v_s}} \, ,
%  \label{eq:singlet-tripleheavy}
% \end{alignat}
% %
% in which we observe a characteristic non-decoupling behavior which
% manifests itself as a linear growth of $g_{hhH}$ with the heavy
% Higgs mass.  In the EFT, the leading self-interaction contribution enters
% via a dimension-8 operator, which is
% neglected in our dimension-6
% analysis. Therefore, the sole Wilson coefficient $\bar{c}_H = \sin^2 \alpha$ defines
% the singlet model EFT up to dimension 6.

% On the other
% hand, let us emphasize a key structural difference between the
% $\ope{H}$-induced and the UV-complete singlet model contributions to the
% Higgs self-coupling $hhh$. At variance with the latter, the effective
% operators also induces a new momentum structure into the self
% coupling, namely adding derivatives in the Lagrangian or energy
% dependent terms in the Feynman rules
% %
% \begin{align}
%  \lgr{} \supset
% &- \dfrac{m_{h}^2}{2v}\left[
%    \left(1-\dfrac{c_H v^2}{2\Lambda^2} \right) h^3
%    -\dfrac{2c_H v^2}{\Lambda^2 {m_{h}^2}} 
%     h \, \partial_\mu h \, \partial^\mu h \right] \notag \\
% &= - \dfrac{m_{h}^2} {2v} 
%     \left( 1 - \dfrac{1}{2} \bar{c}_H \right) 
%     h^3 
%    + \dfrac {g}{2 m_W} \bar{c}_H \; h \partial_\mu h \partial^\mu h 
% \label{eq:singlet-self},
% \end{align}
% %
% which means that the SM-like $h^3$ term is not only rescaled but
% also endowed with new Lorentz structures involving derivatives.  This
% kind of momentum dependence is encoded in the split into
% $g^{(1)}_{HHH}$ and $g^{(2)}_{HHH}$ in Eq.~\eqref{eq:masslag}.  This
% effect does not correspond to the Higgs singlet mixing, where such a
% momentum dependence can only be generated via loop-induced heavy
% particle exchange with momentum-dependent couplings like a heavy
% fermion triangle.



%%%%%%%%%%%%%%%%%%%%%%%%%%%%%%%%%%%%%%%%%%%%%%%%%%%%%%%%%%%%
\subsection{Two-Higgs-doublet model}
\label{sec:appendix_models_2hdm}
%%%%%%%%%%%%%%%%%%%%%%%%%%%%%%%%%%%%%%%%%%%%%%%%%%%%%%%%%%%%



%%%%%%%%%%%%%%%%%%%%%%%%%%%%%%%%%%%%%%%%%%%%%%%%%%%%%%%%%%%%
\subsubsection{Model setup}
%%%%%%%%%%%%%%%%%%%%%%%%%%%%%%%%%%%%%%%%%%%%%%%%%%%%%%%%%%%%
 
We analyse the most general gauge invariant, $CP$-even theory for two
scalar fields with an additional $\mathbb{Z}_2$ symmetry, as defined
in Equation~\eqref{eq:validity_2hdm_potential} in
Section~\ref{sec:validity_2hdm}.  The Higgs mass-eigenstates follow
from the set of rotations
%
\begin{equation}
  \twovec {H^0} {h^0} = R(\alpha) \, \twovec {h^0_1} {h^0_2} \, ,
  \qqquad
  \twovec {w^0 }{ A^0 } = R(\beta)\,\twovec {a^0_1 }{ a^0_2 } \,,
  \qqquad
  \twovec {w^\pm }{ H^\pm } = R(\beta)\,\twovec {h^\pm_1 }{ h^\pm_2 } \,,
\end{equation}
%
with
%
\begin{equation}
\phi_k = \twovec {h_k^+} {\dfrac{1}{\sqrt{2}} (v_k + h_k^0 + \im a_k) }
\quad \text{and} \quad
R(\theta) = \twomat {\cos\theta} {\sin\theta } {-\sin\theta} {\cos\theta} \,.
\end{equation}

Since the two doublets contribute to giving masses to the weak gauge
bosons, custodial symmetry will impose tight constraints on the viable
mass spectrum of the model~\cite{Veltman:1976rt, Toussaint:1978zm,
  Frere:1982ma, Grimus:2007if, Hollik:1986gg, Beenakker:1988pv,
  Froggatt:1991qw, He:2001tp, Grimus:2008nb}.  Analytic relations
linking the different Higgs masses and mixing angles with the
Lagrangian parameters in Eq.~\eqref{eq:validity_2hdm_potential} can be found
\eg in Appendix A of~\cite{Lopez-Val:2013yba}.  The conventions
%
\begin{equation}
  0 <\beta < \pi/2
  \quad \text{and} \quad
  0 \leq \beta-\alpha < \pi
\end{equation}
%
guarantee that the Higgs coupling to vector bosons has the same sign
in the 2HDM and in the SM.  As we will next show, the decoupling limit
implies that the light Higgs interactions approach the alignment
limit, where $\cos\beta \sim |\sin\alpha|$ and the couplings become
SM-like~\cite{Gunion:2002zf}.



%%%%%%%%%%%%%%%%%%%%%%%%%%%%%%%%%%%%%%%%%%%%%%%%%%%%%%%%%%%%
\subsubsection{Couplings}
%%%%%%%%%%%%%%%%%%%%%%%%%%%%%%%%%%%%%%%%%%%%%%%%%%%%%%%%%%%%

The tree-level coupling shifts of the light Higgs follow from these
rotations and are given in
Eqs.~\eqref{eq:validity_2hdm_higgs_vector_coupling} to
\eqref{eq:validity_2hdm_last_coupling}. The light Higgs
coupling to a charged Higgs pair reads
%
\begin{align}
   \frac {g_{h^0{H^+}{H^{-}}}}{g^\text{SM}_{hhh}}
  = \frac{1}{3 m_{h^0}^2}
  \left[ 
  \sin (\beta - \alpha) \left( 2 m_{H^\pm}^2 - m_{h^0}^2 \right)
  + \frac {\cos (\alpha + \beta)} {\sin (2\beta)}
  \left(2m_{h^0}^2 - \frac {2 m_{12}^2}{\sin \beta \cos \beta} \right)
  \right] \, ,
\end{align}
%
with $g^\text{SM}_{hhh} = -3 m_h^2/v$.

The loop-induced couplings are more involved, giving
%
\begin{align}
  1+\Delta_g
  &=
    \frac{1}{A_{gg}^\text{SM}} 
    \Bigg[\sum_{f = t,b}\,(1+\Delta_f)\,A_f(\tau_f) \Bigg] \,, \\
  %
  1+\Delta_\gamma
  &=
    \frac{1}{A_{\gamma \gamma}^\text{SM}} 
    \Biggl[ \sum_{f = t,b}\, N_C\,Q_f^2\,(1+\Delta_f)\,A_f(\tau_f)
    + Q^2_{\tau}\,(1+\Delta_\tau)\,A_{f}(\tau_\tau)
    + (1+\Delta_W)\, A_v(\tau_W) \notag \\
  %
 &\qqqqquad - {g}_{h^0 {H^+} {H^{-}}} \; \frac {m_W s_w} {e m_{H^\pm}^2} \; A_s(\tau_{H^{\pm}}) \Biggr] \,,
   \label{eq:appendix_models_2hdm_haa}
\end{align}
%
where $A_{xx}^\text{SM}$ are the corresponding contributions in the
SM. The conventional loop form factors read
%
\begin{align}
 A_s(\tau) &= -\frac{\tau}{2}\,\left[1-\tau f(\tau) \right] &&= 1/6 + \mathcal{O}(\tau^{-1}) \,, \notag \\
 A_f(\tau) &= \tau\left[1+(1-\tau)\,f(\tau) \right] &&= 2/3 + \mathcal{O}(\tau^{-1}) \,, \notag \\
 A_V(\tau) &= -\frac{1}{2}\,\left[2+3\tau+3(2\tau-\tau^2)\,f(\tau) \right] &&= -7/2 + \mathcal{O}(\tau^{-1}) \,,
\end{align}
%
where
%
\begin{equation}
 f(\tau) =
\begin{cases}
  - \frac 1 4 \left[ \log \frac{1 + \sqrt{1- \tau}} {1 - \sqrt{1 - \tau}} - i \pi \right]^2 & \quad \text{for } \tau < 1 \\
  \left[ \arcsin \frac 1 {\sqrt{\tau}} \right]^2 & \quad \text{for } \tau \geq 1 \,,
\end{cases}
\end{equation}
%
and $\tau_x = 4 m_x^2 / m_{h^0}^2$.



%%%%%%%%%%%%%%%%%%%%%%%%%%%%%%%%%%%%%%%%%%%%%%%%%%%%%%%%%%%%
\subsubsection{Matching}
%%%%%%%%%%%%%%%%%%%%%%%%%%%%%%%%%%%%%%%%%%%%%%%%%%%%%%%%%%%%

The effect of the second doublet on the phenomenology of the light
Higgs consists purely of shifted couplings $\Delta_x$. This allows us
to match the dimension-six model by setting equal the coupling shifts
from the full model, given by
Eqs.~\eqref{eq:validity_2hdm_higgs_vector_coupling} to
\eqref{eq:validity_2hdm_last_coupling} and
\eqref{eq:appendix_models_2hdm_haa}, to the corresponding couplings in
the dimension-six Lagrangian,
Eq.~\eqref{eq:foundations_higgs_couplings}.

In a second step we then expand in $1/\Lambda$ and keep terms up to
$\ord{1/\Lambda^2}$, where $\Lambda$ is defined in the unbroken phase
for the default matching or as the physical mass $m_{A^0}$ in the
$v$-improved matching, as described in
Section~\ref{sec:validity_2hdm}.

This defines the effective model in a straightforward way. For more
details see Appendix A.2 of Ref.~\cite{Brehmer:2015rna}.
%
\comment{To here}



%%%%%%%%%%%%%%%%%%%%%%%%%%%%%%%%%%%%%%%%%%%%%%%%%%%%%%%%%%%%
\subsection{Scalar top partners}
\label{sec:appendix_models_partners}
%%%%%%%%%%%%%%%%%%%%%%%%%%%%%%%%%%%%%%%%%%%%%%%%%%%%%%%%%%%%

The simplified scalar top-partner generation sector is described by the Lagrangian
%
\begin{alignat}{5}
 \lgr{} &\supset  (D_{\mu}\,\tilde{Q})^\dagger\,(D^\mu\tilde{Q}) + (D_\mu\,\tilde{t}_R)^*\,(D^\mu\,\tilde{t}_R)
 - \underbrace{\tilde{Q}^\dagger\,M^2\,\tilde{Q}\,
 - M^2\,\tilde{t}_R^*\,\tilde{t}_R}_{\lgr{mass}}
 \notag \\
& \qquad\quad  -\underbrace{\kappa_{LL}\,(\phi\cdot\tilde{Q})^\dagger(\phi\cdot\tilde{Q})
 -\kappa_{RR}\,(\tilde{t}_R^*\tilde{t}_R)\,(\phi^\dagger\,\phi) }_{\lgr{Higgs}} 
 -\underbrace{\left[ \kappa_{LR} \, M \, \tilde{t}_R^*\,(\phi \cdot \tilde{Q}) + \text{h.c.} \right]}_{\lgr{mixing}} \,.
\label{eq:lag-partners}
 \end{alignat}
%
We use the customary notation for the $SU(2)_L$
invariant product $\phi^a\cdot \tilde{Q}^b \equiv
\epsilon_{ab}\,\phi^a\,\tilde{Q}^b$, with the help of the antisymmetric
pseudo-tensor $\epsilon^{ab} \equiv (i\sigma^2)^{ab}$, so that
$\epsilon^{12} = -\epsilon^{21} = 1$. 

Notice that the term $\lgr{Higgs}$ gives rise to scalar
partner masses proportional to the Higgs VEV, mirroring the
supersymmetric F-term contribution to the squark masses.  By a
similar token, the explicit mass terms $\lgr{mass}$ are
analogous to the squark soft-SUSY breaking mass terms; while
$\lgr{mixing}$ is responsible for the mixing between the gauge
eigenstates, as a counterpart of the MSSM $A$-terms.
In the absence of an underlying
supersymmetry, the Lagrangian in Eq.~\eqref{eq:lag-partners} features
no equivalent of the D-term contributions.

Collecting all bilinear terms from Eq.~\eqref{eq:lag-partners} we get
%
\begin{alignat}{5}
 \lgr{} &\supset (\tilde{t}_L^*\; \tilde{t}_R^*)
 \,\begin{pmatrix} M_{LL}^2 & M_{LR}^2 \\ M_{RL}^2  & M_{RR}^2 \end{pmatrix}\begin{pmatrix} \tilde{t}_L \\ \tilde{t}_R \end{pmatrix}
  \label{eq:masses1}
\end{alignat}
%
where
%
\begin{alignat}{5}
 M_{LL}^2 &= \kappa_{LL}\cfrac{v^2}{2} + M^2 \,, \qqquad 
 M_{LR}^2=M_{RL}^2 = \kappa_{LR} \, M\cfrac{v}{\sqrt{2}} \,, \qqquad 
 M_{RR}^2 = \kappa_{RR}\,\dfrac{v^2}{2}\, + M^2 \,.
 \label{eq:masses2}
\end{alignat}
%
Assuming all parameters in Eq.~\eqref{eq:lag-partners} to be real,
the above mass matrix can be diagonalized through the usual
orthogonal transformation $R(\theta_{\tilde{t}})$ which rotates the gauge
eigenstates $(\tilde{t}_L, \tilde{t}_R)$ onto the mass basis $(\tilde{t}_1,\tilde{t}_2)$,
% 
\begin{alignat}{5}
R(\theta_{\tilde{t}})\,\mathcal{M}_{\tilde{t}}^2\,R^\dagger(\theta_{\tilde{t}}) = \text{diag}(m^2_{\tilde{t}_1}, m^2_{\tilde{t}_2})\,,
\qquad \begin{pmatrix} \tilde{t}_1 \\ \tilde{t}_2 \end{pmatrix} = R(\theta_{\tilde{t}}) \begin{pmatrix} \tilde{t}_L \\ \tilde{t}_R \end{pmatrix} =
\begin{pmatrix} \cos\theta_{\tilde{t}} & \sin\theta_{\tilde{t}} \\ -\sin\theta_{\tilde{t}}  & \cos\theta_{\tilde{t}}\end{pmatrix}
 \begin{pmatrix} \tilde{t}_L \\ \tilde{t}_R \end{pmatrix}.
\label{eq:rotation}
\end{alignat}
% 
The physical scalar partner masses and the mixing angle are then given by
% 
\begin{align}
m_{\tilde{t}_1}^2 &= M_{LL}^2\cos^2\theta_{\tilde{t}} + M_{RR}^2\,\sin^2\theta_{\tilde{t}} + 2M_{LR}^2\,\sin\theta_{\tilde{t}}\cos\theta_{\tilde{t}} \,,\notag \\
%
m_{\tilde{t}_2}^2 &= M_{LL}^2\,\sin^2\theta_{\tilde{t}} + M_{RR}^2\,\cos^2\theta_{\tilde{t}} - 2M_{LR}^2\,\sin\theta_{\tilde{t}}\,\cos\theta_{\tilde{t}} \, ,
\label{eq:masses3} \\
\tan(2\theta_{\tilde{t}}) &= \frac{2M_{LR}^2}{M_{LL}^2-M_{RR}^2} \, .
\label{eq:mixing}
\end{align}
%
As we assume the right-handed partner $\tilde{b}_R$ to be heavy and thus decoupled, the
sbottom-like scalar eigenstate $\tilde{b}_L$ undergoes no mixing and can be
readily identified with the physical eigenstate.

To derive the effective theory, we compute the effective action at one
loop with the help of the covariant derivative
expansion~\cite{Gaillard:1985uh,Cheyette:1987qz,Henning:2014wua},
which is fully consistent with our mass degeneracy setup.  Notice
that, since the Lagrangian Eq.~\eqref{eq:lag-partners} lacks any linear
terms in the heavy scalar fields $\Psi \equiv (\tilde{Q}, \tilde{t}_R^*)$, the
tree-level exchange of such heavy partners cannot generate any
effective interaction at dimension 6. 

Following our default matching prescription, we set the matching scale as
$\Lambda = M$. The relevant Wilson coefficients in the SILH basis
then read:
% 
\begin{alignat}{5}
& \bar{c}_{g} =  
 \cfrac{m_W^2}{24\,(4\pi)^2\,M^2}\,\left[(\kappa_{LL} + \kappa_{RR}) - {\kappa_{LR}^2}\right] \notag \\
& \bar{c}_{\gamma} =  
 \cfrac{m_W^2}{9\,(4\pi)^2\,M^2}\,\left[(\kappa_{LL} + \kappa_{RR}) - {\kappa_{LR}^2}\right] \notag \\
& \bar{c}_{B} =  
 -\cfrac{5m_W^2}{12\,(4\pi)^2\,M^2}\,\left[\kappa_{LL} - \cfrac{31}{50}{\kappa_{LR}^2}\right] \notag \\
& \bar{c}_{W} =  
 \cfrac{m_W^2}{4\,(4\pi)^2\,M^2}\,\left[\kappa_{LL} - \cfrac{3}{10}{\kappa_{LR}^2}\right] \notag \\
& \bar{c}_{HB} =  
 \cfrac{5m_W^2}{12\,(4\pi)^2\,M^2}\,\left[\kappa_{LL} - \cfrac{14}{25}{\kappa_{LR}^2}\right] \notag \\
& \bar{c}_{HW} =  -
 \cfrac{m_W^2}{4\,(4\pi)^2\,M^2}\,\left[\kappa_{LL} - \cfrac{2}{5}{\kappa_{LR}^2}\right] \notag \\
 &  \bar{c}_{{H}} = 
 \cfrac{v^2}{4(4\pi)^2\,M^2}\,\left[(2\kappa_{RR}^2-\kappa_{LL}^2) - 
 {\Bigl( \kappa_{RR} - \frac{1}{2}\kappa_{LL} \Bigr) \kappa_{LR}^2 + \cfrac{\kappa_{LR}^4}{10}}\right]
 \notag \\
%
 & \bar{c}_{{T}}  = \cfrac{v^2}{4(4\pi)^2\,M^2}\,\left[\kappa_{LL}^2 - \cfrac{\kappa_{LL}\,\kappa_{LR}^2}{{2}} + \cfrac{\kappa_{LR}^4}{{10}}\right].
\label{eq:triplet_coefficients_ap}
\end{alignat}

We also consider a $v$-improved matching. The only difference
to the default matching is the choice of the matching scale $\Lambda = m_{\tilde{t}_1}$,
which manifests itself as a rescaling of the Wilson coefficients in~Eq.\,\eqref{eq:triplet_coefficients_ap}
by a factor of $M^2 / m_{\tilde{t}_1}^2$.


The scalar partner couplings to the Higgs boson can be written as
%
\begin{alignat}{5}
 g_{h\tilde{t}_1\tilde{t}_1}/v &= \kappa_{LL}\,\cos^2\theta_{\tilde{t}} + \kappa_{RR}\,\sin^2\theta_{\tilde{t}} + \sin(2\theta_{\tilde{t}})\,\kappa_{LR} \,, \notag \\
 g_{h\tilde{t}_2\tilde{t}_2}/v &= \kappa_{LL}\,\sin^2\theta_{\tilde{t}} + \kappa_{RR}\,\cos^2\theta_{\tilde{t}} - \sin(2\theta_{\tilde{t}})\,\kappa_{LR} \,, \notag \\
 g_{h\tilde{b}_L\tilde{b}_L}/v & = \kappa_{LL} \,.
 \label{eq:higgscouplings}
\end{alignat}



%%%%%%%%%%%%%%%%%%%%%%%%%%%%%%%%%%%%%%%%%%%%%%%%%%%%%%%%%%%%
\subsection{Vector triplet}
\label{sec:appendix_models_triplet}
%%%%%%%%%%%%%%%%%%%%%%%%%%%%%%%%%%%%%%%%%%%%%%%%%%%%%%%%%%%%

We consider a real vector triplet field $V_\mu^{a=1,2,3}$ transforming
under the SM gauge group as $(r_c,r_L,r_Y) =
(\textbf{1},\textbf{3},0)$.  Its dynamics can be effectively described
by means of the Lagrangian~\cite{Pappadopulo:2014qza}
%
\begin{alignat}{5}
 \lgr{} \supset& -\cfrac{1}{4}\,V_{\mu\nu}^a\,V^{\mu\nu\, a}\, + \cfrac{M_V^2}{2}\,V_{\mu}^a\,V^{\mu\,a}
 + i\,g_V\,c_H\,V_{\mu}^a\,\left[\phi^\dagger\tau^a\,\overleftrightarrow{D}^\mu\,\phi\,\right]
  +\cfrac{g_W^2}{g_V}\,V_{\mu}^a\,c_F\sum_F\, \overline{F}_L\,\gamma^\mu\,\tau^a\,F_L
 \notag \\
 %
 &+
 \cfrac{g_V}{2}\,c_{VVV}\,\epsilon_{abc}\,V_{\mu}^a\,V_{\nu}^b\,D^{[\mu}V^{\nu]c}\, + g_V^2\,c_{VVHH}\,V_{\mu}^a\,V^{\mu a}\,\phisq\,
 - \cfrac{g_W}{2}\,c_{VVW}\,\epsilon_{abc}\,W^{\mu\nu}\,V_\mu^b\,V_\nu^c \,,
 \label{eq:lag-vectortriplet-app}
\end{alignat}
%
where the vector triplet field-strength tensor is $V_{\mu\nu}^a \equiv
D_{\mu}V_{\nu}^a - D_{\nu}\,V_{\mu}^a$ and $\tau^a\equiv \sigma^a/2$ are the
$SU(2)_L$ generators in the fundamental representation.  The covariant
derivative acts on the vector triplet field as $D_\mu\,V_\nu^a =
\partial_\mu\,V_\nu^a+g\epsilon^{abc}\,V^b_{\mu}V_{\nu}^c$.

The coupling constant $g_V$ stands for the characteristic strength of
the heavy vector-mediated interactions, while $g_W$ denotes the
$SU(2)_L$ weak gauge coupling (which differs from the coupling
strength $g$ of the observable $W$ boson due to $W$-$V$ mixing, see
below).  The different
dimensionless coefficients $c_i$ quantify the relative strengths of
the individual couplings.  This parametrization weights the extra $V$
and $\phi$ field insertions by one factor of $g_V$ each, while gauge
boson insertions are weighted by one power of the weak coupling.  An
exception is made for the couplings to fermions, where an extra
weighting factor $g_W^2/g_V^2$ is introduced for a convenient power
counting in certain UV embeddings~\cite{Pappadopulo:2014qza}.  For
simplicity, it is assumed that the fermion current in
Eq.~\eqref{eq:lag-vectortriplet-app} is universal.

Equation~\eqref{eq:lag-vectortriplet-app} is the most general Lagrangian
compatible with the SM gauge group and CP invariance, provided that
$V_\mu^a$ transforms as $V_\mu^a(\vec{x},t)
\sin\theta_{\tilde{t}}ackrel{\text{CP}}{\longrightarrow}
-(-1)^{\delta_{a2}}\,V_\mu^a(-\vec{x},t)$ as the SM vectors. 
Moreover, the Lagrangian obeys a
global $SO(4) = SU(2)_L \times SU(2)_R$ symmetry, which is typical of
strongly interacting dynamics.

Since $V_\mu^a$ is not manifestly gauged, this simplified vector
triplet model in itself is not renormalizable.  However, it can be
easily linked to a gauge-invariant theory \eg via the Higgs or the
St\"uckelberg mechanisms~\cite{Pappadopulo:2014qza}.

An alternative model setup, which is particularly useful to construct
the effective theory,
introduces an explicit kinetic $V$-$W$ mixing via the Lagrangian
% 
\begin{alignat}{5}
 \lgr{} \supset &-\cfrac{1}{4}\,V_{\mu\nu}^a\,V^{\mu\nu\, a}\, + \cfrac{\tilde{M}_V^2}{2}\,V_{\mu}^a\,V^{\mu\,a}
 + \,g_V\,\tilde{c}_H\,V_{\mu}^a\,J_H^{\mu,a}
  +\cfrac{g_W^2}{2g_V}\,V_{\mu}^a\,\tilde{c}_F\sum_F\,J_F^{\mu,a} + \tilde{c}_{WV}\,\cfrac{g_W}{2g_V}\,D_{[\mu}\,V_{\nu]}^a\,W^{\mu\nu\, a}
 \notag \\
 &+
 \cfrac{g_V}{2}\,\tilde{c}_{VVV}\,\epsilon_{abc}\,V_{\mu}^a\,V_{\nu}^b\,D^{[\mu}V^{\nu]c}\, 
 + g_V^2\,\tilde{c}_{VVHH}\,V_{\mu}^a\,V^{\mu a}\,\phisq\,
 - \cfrac{g_W}{2}\,\tilde{c}_{VVW}\,\epsilon_{abc}\,W^{\mu\nu}\,V_\mu^b\,V_\nu^c \,
 \label{eq:lag-vectortriplet-tilded},
\end{alignat}
%
where for convenience we have introduced the Higgs, fermion and vector
current bilinears
%
\begin{align}
J_\mu^{H, a} &= \cfrac{i}{2}\,\left[\phi^\dagger\,\sigma^a\,\overleftrightarrow{D}_\mu
\,\phi \right],
 & J_\mu^{F, a} &= \overline{F}_L\gamma_\mu\,\sigma^a\,F_L\,,
 & J_\mu^{W,a} &= D^\nu\,W_{\mu\nu}^a\, \label{eq:currents}. 
\end{align}
%
An appropriate field redefinition absorbs the kinetic mixing term
$V^{\mu a} \left(D^\nu W_{\mu \nu} \right)^a$ ~\cite{delAguila:2010mx}
and connects the parameters in the \emph{tilded} basis of
Eq.~\eqref{eq:lag-vectortriplet-tilded} and \emph{untilded} basis of
Eq.~\eqref{eq:lag-vectortriplet-app} through the relations
%
\begin{align}
  M_V^2 &= \dfrac {g_V^2} {g_V^2 - \tilde{c}_{WV}^2 g_W^2} \tilde{M}_V^2 \,,\notag \\
  c_H &= \dfrac {g_V} {\sqrt{g_V^2 - \tilde{c}_{WV}^2 g_W^2}} \left[ \tilde{c}_H + \dfrac {g_W^2}{g_V^2} \tilde{c}_{WV} \right] ,\notag \\
  c_F &= \dfrac{g_V} {\sqrt{g_V^2 - \tilde{c}_{WV}^2 g_W^2}} \left[ \tilde{c}_F + \tilde{c}_{WV} \right] ,\notag \\
  c_{VVHH} &= \dfrac {g_V^2} {g_V^2 - \tilde{c}_{WV}^2 g_W^2} \left[ \tilde{c}_{VVHH} + \dfrac {g_W^2} {2 g_V^2} \tilde{c}_{WV} \tilde{c}_H + \dfrac {g_W^4} {4g_V^4} \tilde{c}_{WV}^2 \right] ,\notag \\
  c_{VVW} &= \dfrac {g_V^2} {g_V^2 - \tilde{c}_{WV}^2 g_W^2} \left[ \tilde{c}_{VVW} - \dfrac{g_W^2}{g_V^2} \tilde{c}_{WV}^2 \right] ,\notag \\
  c_{VVV} &= \dfrac {g_V^2} {\left(g_V^2 - \tilde{c}_{WV}^2 g_W^2\right)^{3/2}} \left[
    \tilde{c}_{VVV} - \dfrac{g_W^2}{g_V^2} \tilde{c}_{WV} (\tilde{c}_{VVW} + 2) + 2 \dfrac {g_W^2} {g_V^4} \tilde{c}_{WV}^3 \right]  \,.
  \label{eq:VectorTriplet_tilded_to_untilded}
\end{align}



%%%%%%%%%%%%%%%%%%%%%%%%%%%%%%%%%%%%%%%%%%%%%%%%%%%%%%%%%%%%
\subsubsection*{Spectrum}
%%%%%%%%%%%%%%%%%%%%%%%%%%%%%%%%%%%%%%%%%%%%%%%%%%%%%%%%%%%%

The heavy vector sector in the gauge basis contains one neutral state
$V_\mu^{0}\equiv V_\mu^3$ and two charged states $V_\mu^{\pm} \equiv
(V_\mu^1\mp V_\mu^2)/\sqrt{2}$.  Upon EWSB only one vector state
remains massless, which we readily identify with the standard photon
field $A_\mu = c_w\,B_\mu + s_w\,W_\mu^3$.  Here, the Weinberg angle
is linked as usual to the electroweak gauge couplings $e = g_W\,s_w =
g'\,c_w$, although at this stage we cannot yet relate it to
electroweak observables before the mixing with the heavy vectors is
included.  The latter involves, for the neutral fields, the heavy
vector component $V^0$ and the linear combination of $B,W^3$
orthogonal to the photon field. A similar mixing pattern appears in
the charged sector, involving the field components $V^{1,2}_\mu,
W^{1,2}_\mu$.  The physical mass eigenstates can be written as
%
\begin{align}
  Z_\mu &= \cos \theta_N \left(-s_w B_\mu + c_w W_\mu^3 \right) + \sin \theta_N \, V^3_\mu \,,\notag \\
  \xi^0_\mu &= - \sin \theta_N \left(-s_w B_\mu + c_w W_\mu^3 \right) + \cos \theta_N  \, V^3_\mu \,,\notag \\
  W^\pm_\mu &= \cos \theta_C \, \dfrac{W^1_\mu \mp W^2_\mu} {\sqrt{2}} + \sin \theta_C \, \dfrac{V^1_\mu \mp V^2_\mu} {\sqrt{2}} \,,\notag \\
  \xi^\pm_\mu &= - \sin \theta_C \, \dfrac{W^1_\mu \mp W^2_\mu} {\sqrt{2}} + \cos \theta_C \, \dfrac{V^1_\mu \mp V^2_\mu} {\sqrt{2}} \,.
\end{align}  
%
The mass eigenvalues are given by
%
\begin{align}
  m_{Z/\xi^0}^2 &= \dfrac 1 2 \left[ \hat{m}_V^2 + \hat{m}_Z^2 \mp \sqrt{ \left( \hat{m}_Z^2 - \hat{m}_V^2 \right)^2 + c_H^2 \, g_V^2 \, \hat{m}_Z^2 \, \hat{v}^2} \, \right] \notag \\
    &=
  \begin{cases}
    \hat{m}_Z^2 \left(1 - \dfrac {c_H^2 g_V^2} {4} \, \dfrac {\hat{v}^2} {\hat{m}_V^2} + \ord{\hat{v}^4/ \hat{m}_V^4}  \right)  \\
    \hat{m}_V^2 \left(1 + \dfrac {c_H^2 g_V^2} {4} \, \dfrac {\hat{v}^2} {\hat{m}_V^2} + \ord{\hat{v}^4/ \hat{m}_V^4}  \right) \,,
  \end{cases}
  \label{eq:VectorTriplet_mZxi}
\end{align}
\begin{align}
  m_{W^\pm/\xi^\pm}^2 &= \dfrac 1 2 \left[ \hat{m}_V^2 + \hat{m}_W^2 \mp \sqrt{ \left( \hat{m}_W^2 - \hat{m}_V^2 \right)^2 + c_H^2 \, g_V^2 \, \hat{m}_W^2 \, \hat{v}^2} \, \right] \notag \\
    &=
  \begin{cases}
    \hat{m}_W^2 \left(1 - \dfrac {c_H^2 g_V^2} {4} \, \dfrac {\hat{v}^2} {\hat{m}_V^2} + \ord{\hat{v}^4/ \hat{m}_V^4}  \right) \\
    \hat{m}_V^2 \left(1 + \dfrac {c_H^2 g_V^2} {4} \, \dfrac {\hat{v}^2} {\hat{m}_V^2} + \ord{\hat{v}^4/ \hat{m}_V^4}  \right) \,.
  \end{cases} 
  \label{eq:VectorTriplet_mWxi}
\end{align}
%
For the mixing angles, we find
%
\begin{align}
  \tan (2 \theta_N) &= \dfrac {c_H \, g_V \, \hat{v} \, \hat{m}_Z } {\hat{m}_V^2 - \hat{m}_Z^2}  
  = \dfrac{c_H \, g \, g_V} {2 \, c_w} \, \dfrac {\hat{v}^2} {\hat{m}_V^2}  +  \ord {\hat{v}^4/ \hat{m}_V^4} \,,\notag \\
  \tan (2 \theta_C) &= \dfrac {c_H \, g_V \, \hat{v} \, \hat{m}_W } {\hat{m}_V^2 - \hat{m}_W^2} 
  = \dfrac{c_H \, g \, g_V} 2 \, \dfrac {\hat{v}^2} {\hat{m}_V^2}  + \ord {\hat{v}^4/ \hat{m}_V^4} \,,
  \label{eq:VectorTriplet_mixingangles}
\end{align}
%
or
%
\begin{align}
  \sin \theta_C = \dfrac {c_H \, g \, g_V} {4} \, \dfrac {v^2} {M_V^2}  + \ord{\hat{v}^4/ \hat{m}_V^4}\,.
\end{align}
%
Here we define
%
\begin{align}
  \hat{m}_Z  = \dfrac{g_W \, \hat{v} } {2 \, c_w} \qquad
  \hat{m}_W  = \dfrac{g_W \, \hat{v} } {2} \qquad 
  \hat{m}_V^2  = M_V^2 + g_V^2 \, c_{VVHH} \, \hat{v}^2
  \label{eq:VectorTriplet_mVhat}
\end{align}
%
where $\hat{v}$ is the actual vev of $\phi$, which does not necessarily have the SM value of $v = 2m_W/g \approx 246 \ \gev$. 

Notice that the $V$-$W$ mixing also affects the weak current interactions, which are no longer governed by $g_W$. Instead, the physical $Wff'$ coupling reads
%
\begin{align}
  g  = \cos \theta_C \, g_W - \sin \theta_C \, c_F \, \dfrac {g_W^2} {g_V} 
     = g_W \, \left(1 - \dfrac{c_F \, c_H \, g_W^2} 4  \, \dfrac {v^2}{M_V^2} \right)+ \ord{v^4 / M_V^4 } \,.
  \label{eq:VectorTriplet_gratio}
\end{align}
%
The relation between $\hat{v}$ and $v$ can be read off from Eq.~\eqref{eq:VectorTriplet_mWxi}, giving approximately
%
\begin{align}
  \dfrac{\hat{v}} v = 1 + \dfrac {c_H^2 \, g_V^2} 8 \, \dfrac {v^2} {M_V^2}  -
  \dfrac {c_F \, c_H \, g_W^2} 4 \, \dfrac {v^2} {M_V^2}  + \ord{v^4 / M_V^4 } \,.
  \label{eq:VectorTriplet_vevratio}
\end{align}

The global $SU(2)_V$ custodial symmetry connects the charged and
neutral current strengths through $m_W^2\,m^2_{\xi^{\pm}} =
c^2_w\,m_Z^2\,m^2_{\xi^0}$, which generalizes the SM relation $m_W^2 =
c^2_w\,m_Z^2$.  Compatibility with EWPO enforces nearly mass-degenerate
states $m_{\xi^0} \simeq m_{\xi^{\pm}}$ for phenomenologically viable
scenarios.  In practice, we set up our model in the $m_W$-$g$ scheme,
\ie taking as input parameters $g$, $m_W$, $\alpha$, $m_{h^0}$,
$\alpha_s$; the model-specific parameters $c_i$; as well as the
physical masses $m_{\xi^\pm}$.  The mass spectrum and mixing angles we
obtain by solving Eq.~\eqref{eq:VectorTriplet_mZxi} and
Eq.~\eqref{eq:VectorTriplet_mWxi} iteratively.



%%%%%%%%%%%%%%%%%%%%%%%%%%%%%%%%%%%%%%%%%%%%%%%%%%%%%%%%%%%%
\subsubsection*{Effective theory}
%%%%%%%%%%%%%%%%%%%%%%%%%%%%%%%%%%%%%%%%%%%%%%%%%%%%%%%%%%%%

To construct the vector triplet EFT following the default matching, we identify 
the new physics scale $\Lambda = M_V$.  Starting from the heavy
triplet Lagrangian defined by Eq.~\eqref{eq:lag-vectortriplet-tilded},
we first integrate by parts the kinetic mixing term,
%
\begin{alignat}{5}
\tilde{c}_{WV}{\dfrac{g_W}{2g_V}}\,D_{[\mu}\,V_{\nu]}^a\,W^{\mu\nu\, a} =
\tilde{c}_{WV}\,\cfrac{g_W}{g_V}\,V^{\mu, a}\,(D^\nu\,W^a_{\mu\nu}) = 
 \tilde{c}_{WV}\,\cfrac{g_W}{g_V}\,V^{\mu, a}\,J_\mu^{W\,a} \,  
\label{eq:kinmix}, 
\end{alignat}
%
such that we can rewrite it in terms of the gauge current from
Eq.~\eqref{eq:currents}.  Integrating out the heavy vector field
$V^a_\mu$ one obtains the effective Lagrangian
%
\begin{alignat}{5}
 \lgr{EFT}&\supset \cfrac{\tilde{M}_V^2}{2}V^{\mu,a}\,V_\mu^a + V_{\mu}^a\left[g_V\,\tilde{c}_H\,J_H^{\mu,a}
  +\cfrac{g_W^2}{2g_V}\,\tilde{c}_F\,\sum_F\,J_F^{\mu,a} +
  \tilde{c}_{WV}\,\cfrac{g_W}{g_V}\,J_\mu^{W\,a} \right] +  \ord{V^3}\,,
 \label{eq:lageff1}
\end{alignat}
%
where we neglect those contributions involving higher powers in the
heavy field, as they play no role in our analysis.

The Euler-Lagrange equation for $V_{\mu}^a$,
%
\begin{alignat}{5}
& [\partial^\mu\partial^\nu - g^{\mu\nu}\,\partial^2 - \tilde{M}_V^2] \,V^{a}_{\nu}
= g_V\,\tilde{c}_H\,J_H^{\mu,a}+\cfrac{g_W^2}{2g_V}\,\tilde{c}_F\sum_FJ_F^{\mu,a}\,+\tilde{c}_{WV}\,\cfrac{g_W}{g_V}\,J_W^{\mu,a} + 
 \,\text{h.o.\ terms in}\; V_{\mu}^a\,, \notag
\end{alignat}
%
leads to
%
\begin{align}
V^{\mu,a} = 
-\cfrac{1}{\tilde{M}_V^2}\,\left[\tilde{c}_{WV}\,\cfrac{g_W}{g_V}\,J_W^{\mu,a} +g_V\,\tilde{c}_H\,J_H^{\mu,a}+\cfrac{g_W^2}{2g_V}\,\tilde{c}_F\,\sum_F\,J_F^{\mu,a}\,\right] {+ {\cal O}(p_V^2/\tilde{M}_V^4) + {\cal O}(V^2)}  
 \label{eq:eom-v}.
\end{align}

Plugging Eq.~\eqref{eq:eom-v} into Eq.~\eqref{eq:lageff1}, $\lgr{EFT}$ can
be expressed in terms of current products as
%
\begin{alignat}{5}
 \lgr{EFT}&\supset {-}\cfrac{g_W^{{4}}\,\tilde{c}_F^2}{8g_V^2\,\tilde{M}_V^2}\,J_F^{\mu,a}\,J_\mu^{F\,a} - 
 \cfrac{g_V^2\,\tilde{c}_H^2}{2\,\tilde{M}_V^2}\,J_H^{\mu,a}\,J_\mu^{H\,a}
 - \cfrac{g_W^2\,\tilde{c}_F\,\tilde{c}_H}{2\,\tilde{M}_V^2}\,J_H^{\mu,a}\,J_\mu^{F\,a} {-}  \cfrac{g_W\,\tilde{c}_H\,\tilde{c}_{WV}}{\tilde{M}_V^2}\,J_H^{\mu,a}\,J_\mu^{W\,a}
 \notag \\
& \qquad - \cfrac{g_W^2\,\tilde{c}_{WV}^2}{{2}\,g_V^2\,\tilde{M}_V^2}\,J_W^{\mu,a}\,J_\mu^{W\,a} {-} \cfrac{g_W^3\,\tilde{c}_F\,\tilde{c}_{WV}}{{ 2}\,g_V^2\,\tilde{M}_V^2}\,J_W^{\mu,a}\,J_\mu^{F\,a}
 \label{eq:lageff2}. 
\end{alignat}
%
In the following, we disregard 4-fermion operators since they are
irrelevant for our analysis. The remaining five current products in
Eq.~\eqref{eq:lageff2} can be expressed in terms of two independent
ones by using Eq.~\eqref{eq:eom-w} (with the replacement $g \to g_W$),
which corresponds to $J_W^{\mu,a} = g_WJ_H^{\mu,a}+ g_WJ_F^{\mu,a}/2$:
%
\begin{alignat}{5}
 \lgr{EFT}&\supset - 
 \cfrac{(g_V^2\tilde{c}_H+g_W^2\tilde{c}_{WV})^2}{2\,g_V^2\,\tilde{M}_V^2}\,J_H^{\mu,a}\,J_\mu^{H\,a}
 - \cfrac{g_W^2\,(\tilde{c}_F+\tilde{c}_{WV})\,(g_V^2\tilde{c}_H+g_W^2\tilde{c}_{WV})}{2\,g_V^2\,\tilde{M}_V^2}\,J_H^{\mu,a}\,J_\mu^{F\,a} + \text{4-fermion}
 \label{eq:lageffeom}.
\end{alignat}
%
Using Eq.(A.4) in~\cite{Pappadopulo:2014qza}, it can be checked that
this equation is invariant when changing between the tilded and
untilded bases.  With the help of Eqs.~\eqref{eq:EFT_JHJH_replacement},
\eqref{eq:EFT_OHf_replacement}, and \eqref{eq:EFT_Or_replacement}
(and again relabeling $g \to g_W$ in these relations) the two
independent current products can be expressed in terms of dimension-6
operators as follows:
%
\begin{alignat}{5}
J_H^{\mu,a}\,J_\mu^{H\,a} &= -\cfrac{1}{4}\,(\ope{H} - 4\,\ope{r}) = 
-\cfrac{1}{4}\,\left[ 3\ope{H} -8\lambda\ope{6} - 2\sum_f\left[y_f\ope{f} + \text{h.c.} \right] \right]
\notag \\
  J_F^{\mu,a}\,J_\mu^{H\,a} &= 
  \cfrac{i}{2}\,\ope{HF}' 
  = \cfrac{i\ope{W}}{g_W} + \cfrac{1}{2}\,\left[ 3\ope{H} - 8\lambda\ope{6} - 2\sum_f\left[y_f\ope{f} + \text{h.c.}\right]\right]
\label{eq:current-summary1} 
\end{alignat}
%
where $y_f$ denotes the bare Yukawa coupling $y_f \equiv
\sqrt{2}m_f/v$.  Plugging the above into Eq.~\eqref{eq:lageffeom}, one
can easily read off the relevant Wilson coefficients of the EFT:
%
\begin{alignat}{5}
 \bar{c}_H &= \cfrac{3\,g_W^2\,v^2}{4\,\tilde{M}_V^2}\,\left[\tilde{c}_H^2\cfrac{g_V^2}{g_W^2} -2 \tilde{c}_F\,\tilde{c}_{WV}\,\cfrac{g^2}{g_V^2} - 2\,\tilde{c}_F\,\tilde{c}_H { -\tilde{c}_{WV}^2\,\cfrac{g^2}{g_V^2}}\right] ,\notag \\
 \bar{c}_6 &= \cfrac{g_W^2\,v^2}{\tilde{M}_V^2}\,\left[\tilde{c}_H^2\cfrac{g_V^2}{g_W^2} -2 \tilde{c}_F\,\tilde{c}_{WV}\,\cfrac{g^2}{g_V^2} - 2\,\tilde{c}_F\,\tilde{c}_H {-\tilde{c}_{WV}^2\,\cfrac{g^2}{g_V^2}}\right] ,\notag \\
 \bar{c}_f &= \cfrac{g_W^2\,v^2}{4\,\tilde{M}_V^2}\,\left[\tilde{c}_H^2\cfrac{g_V^2}{g_W^2} -2 \tilde{c}_F\,\tilde{c}_{WV}\,\cfrac{g^2}{g_V^2} - 2\,\tilde{c}_F\,\tilde{c}_H { -\tilde{c}_{WV}^2\,\cfrac{g^2}{g_V^2}}\right] ,\notag \\
 \bar{c}_W &= \cfrac{m_W^2}{\tilde{M}_V^2}\,\left[-\tilde{c}_F\tilde{c}_H -\tilde{c}_H\,\tilde{c}_{WV} - \tilde{c}_F\,\tilde{c}_{WV}\,\cfrac{g_W^2}{g_V^2}  -\tilde{c}_{WV}^2\,\cfrac{g^2}{g_V^2}\right] \, .
 \label{eq:wilsonmatched}
\end{alignat}
%
In the untilded basis, these correspond to
\begin{align}
 \bar{c}_{H} &= \dfrac{3\,g_W^2\,v^2}{4\,M_V^2}\,\left[c_H^2\dfrac{g_V^2}{g_W^2}  - 2\,c_F\,c_H\right] , \notag \\
 \bar{c}_{6} &= \dfrac{g_W^2\,v^2}{M_V^2}\,\left[c_H^2\dfrac{g_V^2}{g_W^2} - 2\,c_F\,c_H \right] , \notag \\
 \bar{c}_{f} &= \dfrac{g_W^2\,v^2}{4\,M_V^2}\,\left[c_H^2\dfrac{g_V^2}{g_W^2}  - 2\,c_F\,c_H \right] , \notag \\
 \bar{c}_{W} &= - \dfrac{m_W^2}{M_V^2}\, c_Fc_H \,.
  \label{eq:VectorTriplet_EFT2}
\end{align}
with $f = u, d, \ell$. Other than that, only four-fermion interactions
are generated at tree level and at $\ord{v^2/M_V^2}$; these are not
relevant for our analysis and are not considered here.

As in the 2HDM and scalar partner models, we define an additional
$v$-improved EFT by $\Lambda = m_{\xi^0}$, leading the same Wilson
coefficients as above except that $M_V$ is replaced by $m_{\xi^0}$.


%%%%%%%%%%%%%%%%%%%%%%%%%%%%%%%%%%%%%%%%%%%%%%%%%%%%%%%%%%%%
\subsubsection*{Higgs couplings}
%%%%%%%%%%%%%%%%%%%%%%%%%%%%%%%%%%%%%%%%%%%%%%%%%%%%%%%%%%%%

On the EFT side, it is illustrative to discuss the origin of the Higgs
coupling shifts within two different approaches. First we consider the
EFT that keeps the fermionic operator $\ope{HF}'$ (\ie instead of
using the conventional replacement in
Eq.~\eqref{eq:EFT_OHf_replacement} that maximizes the use of bosonic
operators). In this case, similar to
Eq.~\eqref{eq:VectorTriplet_gratio}, a renormalization effect of the
weak coupling occurs from $V$-$W$ mixing,
%
\begin{alignat}{5}
g &= g_W(1-i\bar{c}'_{HF})
\end{alignat}
%
where $g$ is the observable coupling between the $W$ boson and SM
fermions. In this EFT and using the untilded basis, the relevant
Wilson coefficients are
%
\begin{alignat}{5}
\bar{c}_H &= c_H^2 \dfrac{3g_V^2v^2}{4M_V^2}, \qquad
\bar{c}_f &= \dfrac{1}{3}\bar{c}_{H}, \qquad
\bar{c}'_{HF}= -ic_Fc_H\dfrac{g_W^2v^2}{4M_V^2}\, .
\end{alignat}
%
Instead, if we now consider the EFT with the bosonic operator
$\ope{W}$, \ie after applying the replacement in
Eq.~\eqref{eq:EFT_OHf_replacement}, there is no additional
renormalization of the weak coupling, so that $g=g_W$.  The relevant
Wilson coefficient are given in Eq.~\eqref{eq:VectorTriplet_EFT2}.


Now we are in a position to determine the Higgs coupling shifts in
the three models. For the Yukawa couplings we find
%
\begin{align}
&\text{Full model:} & \Delta_f^\text{full} &= \frac{g_W}{g} \,\frac{v}{\hat{v}} -1 =
                           \frac{1}{c_{\theta_C}^{}-c_F\frac{g_W}{g_V}s_{\theta_C}^{}}\, \frac{v}{\hat{v}} -1 \notag \\
&&&=  c_H^2\frac{g_V^2 v^2}{8M_V^2} + c_Fc_H\frac{g^2 v^2}{4M_V^2} + {\cal O}(M_V^{-4}) \notag \\[2ex]
&\text{EFT with $\ope{HF}'$:} &
\Delta_f^{\ope{HF}'} &= \frac{\bar{c}_{H}}{2}+ \bar{c}_{f} =  \frac{\bar{c}_{H}}{2}+ \bar{c}_{f} + i\bar{c}'_{HF}  \notag \\
&&&= c_H^2\frac{g_V^2 v^2}{8M_V^2} + c_Fc_H\frac{g^2 v^2}{4M_V^2} \notag \\[2ex]
&\text{EFT with $\ope{W}$:} &
\Delta_f^{\ope{W}} &= \frac{\bar{c}_{H}}{2}+ \bar{c}_{f}  \notag \\
  &&&=  c_H^2\frac{g_V^2 v^2}{8M_V^2} + c_Fc_H\frac{g^2 v^2}{4M_V^2}
\end{align}

Similarly for the Higgs coupling to on-shell $W$ bosons we get
%
\begin{align}
&\text{Full model:} &
\Delta_W^\text{full} &= \frac 1 {g m_W} \left( \frac{c_{\theta_C}^2 g^2 \hat{v}}{2(c_{\theta_C}^{}-c_F\frac{g_W}{g_V}s_{\theta_C}^{})^2}
-c_H\frac{s_{\theta_C}^{}c_{\theta_C}^{} g g_V \hat{v}}{c_{\theta_C}^{}-c_F\frac{g_W}{g_V}s_{\theta_C}^{}}+
2c_{ VVHH} s_{\theta_C}^2 g_V^2 \hat{v} \right) - 1 \notag \\
&&&= c_H^2\frac{3g_V^2 v^2}{8M_V^2} + c_Fc_H\frac{g^2 v^2}{4M_V^2} + {\cal O}(M_V^{-4}) \notag \,, \\[2ex]
&\text{EFT with $\ope{HF}'$:} &
\Delta_W^{\ope{HF}'} &= \frac {g_W} {g}  \left( 1-\frac{\bar{c}_{H}}{2} \right) - 1 = \frac{\bar{c}_{H}}{2}+ i\bar{c}'_{HF}  \notag \\
&&&= c_H^2\frac{3g_V^2 v^2}{8M_V^2} + c_Fc_H\frac{g^2 v^2}{4M_V^2} \,, \notag \\[2ex]
&\text{EFT with $\ope{W}$:} &
\Delta_W^{\ope{W}} &=  \frac{\bar{c}_{H}}{2}+ 2\bar{c}_{W} \notag \\
  &&&= c_H^2\frac{3g_V^2 v^2}{8M_V^2} + c_Fc_H\frac{g^2 v^2}{4M_V^2}  \,.
\end{align}




%%%%%%%%%%%%%%%%%%%%%%%%%%%%%%%%%%%%%%%%%%%%%%%%%%%%%%%%%%%%
\section{Fisher information derivations}
\label{chapter:appendix_information}
%%%%%%%%%%%%%%%%%%%%%%%%%%%%%%%%%%%%%%%%%%%%%%%%%%%%%%%%%%%%