
% \setchapterpreamble[ur]{%
% \dictum[W.~Allen~\cite{allen1971getting}]{%
% Can we actually know the universe? My God, it’s hard enough finding your way around in Chinatown.}%
% \vspace*{2cm}}
 
%%%%%%%%%%%%%%%%%%%%%%%%%%%%%%%%%%%%%%%%%%%%%%%%%%%%%%%%%%%%
\chapter{Conclusions}
\label{chapter:conclusions}
%%%%%%%%%%%%%%%%%%%%%%%%%%%%%%%%%%%%%%%%%%%%%%%%%%%%%%%%%%%%

\firstword{M}{easuring the properties} of the Higgs boson is one of
the most important missions for Run~2 of the LHC and may help us
understand some of the open questions of fundamental physics. The
efficient combination of different experimental channels and their
interpretation in the plethora of models of physics beyond the
Standard Model benefit from a universal parametrisation of these
properties. The dimension-six operators of linear Higgs effective
field theory provide such a framework that is theoretically
well-motivated, mostly independent of model assumptions, and
phenomenologically powerful. In this thesis we discussed two aspects
of this approach, both of immediate practical relevance: we analysed
whether these effective operators accurately capture the relevant
signatures of specific scenarios of new physics, and we developed
statistical tools that can help to design efficient measurements of
the Higgs properties.

\newparagraph
%
The starting point of the first part was the observation that the
validity of the dimension-six model for Higgs physics at the LHC is
not at all obvious. The effective field theory approach is based on
the assumption that the typical mass scale of new physics is
significantly larger than the experimentally probed energy scale. But
the limited precision of Higgs measurements at the LHC cannot
guarantee such a scale separation: signatures of weakly coupled new
physics can only have a relevant size if the new physics scale is
close to the electroweak scale, and the effective theory may not be
valid in general.

We studied whether the dimension-six model is nevertheless useful for
Higgs signatures
%as a parametrisation of new physics signatures
by comparing the phenomenology of different scenarios of physics
beyond the Standard Model to the corresponding effective theories. For
extended Higgs sectors with an additional scalar singlet or doublet,
scalar top partners, and heavy vector bosons, we analysed total rates
and kinematic distributions in the most important Higgs production and
decay channels.

Our results show that the agreement between the full models and the
dimension-six model crucially depends on the matching procedure, \ie
the construction of the effective theory from a full model. Standard
procedures for the matching, defined in the unbroken phase of the
electroweak symmetry, lead to large errors of the effective field
theory description already for total rates, which only deteriorate
in kinematic distributions.

This does not mean, though, that the dimension-six model is not useful for
Higgs physics. We introduced $v$-improved matching, a procedure that
improves the performance of the dimension-six model by resumming
certain terms that arise during electroweak symmetry breaking. While
formally of subleading order in the EFT expansion, these effects can
be large under LHC conditions. With such a matching, the effective
model provides a good description even in many scenarios where the EFT
validity is not obvious, and typically only breaks down at new
resonances or in the far high-energy tails of certain distributions.

In addition, we discussed the role of squared amplitudes from dimension-six
operators. Even though they appear at the same order in the EFT
expansion as the leading effects from the neglected dimension-eight
operators, we argued that it is often preferable to include them
in calculations. In a detailed study of Higgs production in weak boson
fusion we finally showed that the transverse momenta of the tagging
jets provide the best measure of the unobservable momentum transfer.

\newparagraph
%
Having established that Higgs EFT works well as a largely
model-independent language for Higgs physics at the LHC, we turned to
the question how its parameters can be measured optimally. The
high-dimensional theory space defined by the Higgs properties and the
intricate kinematics of some Higgs channels present challenges for
traditional analysis methods based on cuts and kinematic
distributions, while modern multivariate methods can be non-transparent.

Information geometry allows us to understand the information contained
in LHC signatures and can help optimise measurement strategies. It is
based on the Fisher information matrix, which according to the
Cram\'er-Rao bound quantifies the maximal precision with which any
continuous model parameters can be measured in an experiment. It can be
interpreted geometrically, and it natively supports high-dimensional
parameter space such as Higgs effective field theory.

We developed an algorithm to calculate the Fisher information in
arbitrary high-energy processes. In addition to the total Fisher
information, we also calculated the distribution of the differential
information over phase space, and the information in individual
kinematic distributions. These tools define the important phase-space
regions and observables for an analysis and let us compare the
discrimination power in traditional histogram-based analyses to that
in modern multivariate ones.

Applying these techniques to Higgs production in weak boson fusion
with decays into taus or four leptons, we showed how the kinematics of
the tagging jets is significantly more sensitive to new physics in the
form of dimension-six operators than observables that characterise the
Higgs decay. The most powerful kinematic observables are the
transverse momenta of these jets and their angular correlations. Still, a
multivariate analysis with matrix-element-based methods or
machine-learning tools can potentially extract significantly more
information. Finally, we analysed the structure of Higgs production
with a single top. Higgs decay products and angular distributions
between the Higgs and top daughters are sensitive to different
operators, but overall the sensitivity of this process to
dimension-six physics is limited.

This is the first application of information geometry to high-energy
physics. While there is no shortage of statistical tools in the field,
these new methods can help to plan and optimise measurement strategies
for high-dimensional, continuous parameter spaces in an intuitive
way. We demonstrated this approach in different Higgs channels for
dimension-six operators, but our tools can easily be translated to
other processes and models.

\newparagraph
%
To conclude, effective dimension-six operators provide a powerful
framework to measure and understand the properties of the Higgs
boson. We analysed the validity of this approach at the LHC and showed
how it can be improved with a suitable matching procedure. In a next
step, we developed statistical tools based on information geometry to
guide the design of efficient measurement strategies. These new ideas
can contribute to a better understanding of the nature of the Higgs
boson, which may ultimately point us to what lies beyond the Standard
Model.
