
% \setchapterpreamble[ur]{%
% \dictum[W.~Allen~\cite{allen1971getting}]{%
% Can we actually know the universe? My God, it’s hard enough finding your way around in Chinatown.}%
% \vspace*{2cm}}
 
%%%%%%%%%%%%%%%%%%%%%%%%%%%%%%%%%%%%%%%%%%%%%%%%%%%%%%%%%%%%
\chapter{Conclusions}
\label{chapter:conclusions}
%%%%%%%%%%%%%%%%%%%%%%%%%%%%%%%%%%%%%%%%%%%%%%%%%%%%%%%%%%%%

\firstword{M}{easuring the properties} of the Higgs boson is one of
the most important missions for Run~II of the LHC and might help us
understand some of the open questions of fundamental physics. The
efficient combination of different experimental channels and their
interpretation in the plethora of models of physics beyond the
Standard Model benefit from a universal parametrisation of these
properties. The dimension-six operators of linear Higgs effective
field theory provide such a framework that is theoretically
well-motivated, mostly independent of model assumptions, and
phenomenologically powerful. This thesis discussed two aspects of this
approach, both of immediate practical relevance: we analysed whether
these effective operators accurately capture the relevant signatures
of specific scenarios of new physics, and developed statistical tools
that can help to design efficient measurements of the Higgs
properties.

\newparagraph
%
The starting point of the first part was the observation that the
validity of the dimension-six model for Higgs physics at the LHC is
not at all obvious. The effective field theory is based on the
assumption that the typical mass scale of new physics is significantly
larger than the energy scale of the experiments described. But the
limited precision of Higgs measurements at the LHC cannot guarantee
such a scale separation: signatures of weakly coupled new physics can
only have a relevant size if the new physics scale is close to the
electroweak scale, and the effective theory might not be valid in
general.

We analysed whether the dimension-six model is nevertheless useful as
a universal description of Higgs properties by comparing the Higgs
phenomenology of different scenarios of new physics to their
corresponding approximations in the effective theory. We analysed
extended Higgs sectors with an additional scalar singlet or doublet,
scalar top partners, and heavy vector bosons for total rates and
kinematic distributions in the most important Higgs production and
decay channels.

Our results show that the agreement between the full models and the
dimension-six model crucially depends on the matching procedure that
links the coefficients of the effective theory to the full
models. Standard procedures for the matching, defined in the unbroken
phase of the electroweak symmetry, lead to large errors of the
effective field theory description already for total rates, which only
become worse when analysing kinematic distributions.

But this does not mean that the dimension-six model is not useful for
Higgs physics. We introduced $v$-improved matching, a procedure that
improves the performance of the dimension-six model by resumming
certain terms that arise during electroweak symmetry breaking. While
formally of subleading order in the EFT expansion, these effects can
be large under LHC conditions. With such a matching, the effective
model provides a good description even in many scenarios where the EFT
validity is not obvious, and in many scenarios only breaks down at new
resonances or in the far high-energy tails of certain distributions.

\newparagraph
%
Having established that Higgs EFT works well as a largely
model-independent language for Higgs physics at the LHC, we turned to
the question how its parameters can be measured optimally. The
high-dimensional theory space defined by the Higgs properties and the
intricate kinematics of some Higgs channels present challenges for
traditional analysis methods based on cuts and kinematic
distributions, while modern multivariate methods can be nontransparent.

Information geometry allows us to understand the information contained
in LHC signatures and can help optimise measurement strategies. It is
based on the Fisher information matrix, which according to the
Cram\'er-Rao bound quantifies the maximal precision with which any
continuous model parameters can be measured in an experiment. It can be
interpreted geometrically and natively supports to high-dimensional
parameter space such as Higgs effective field theory.

We developed an algorithm to calculate the Fisher information in
arbitrary high-energy processes. In addition to the total Fisher
information which quantifies the maximal knowledge that can be derived
from a measurement, we also calculated the the distribution of the
differential information over phase space, and the information in
individual kinematic distributions. These tools define the important
phase-space regions and observables for an analysis and let us combine
the discrimination power in traditional histogram-based analyses to
that in modern multivariate ones.

Applying these techniques to Higgs production in weak boson fusion
with decays into taus or four leptons, we showed how the kinematics of
the tagging jets is significantly more sensitive to new physics in the
form of dimension-six operators than the decay process. The most
powerful observables are the transverse momenta of these jets and
their angular correlations, but an analysis with matrix-element-based
methods or machine-learning tools can extract significantly more
information than that encoded in these distributions. Finally, we
analysed the structure of Higgs production with a single top. Higgs
decay products and angular distributions between the Higgs and top
daughters are sensitive to different operators, but overall the
sensitivity of this process to dimension-six physics is limited.

This is the first application of information geometry to high-energy
physics. While there is no shortage of statistical tools in the field,
these new methods can help to plan and optimise measurement strategies
for high-dimensional continuous models in an intuitive but powerful
way. We demonstrated this approach in different Higgs channels for
dimension-six operators, but our tools can easily be translated to
other processes and models.

\newparagraph
%
All in all, effective dimension-six operators provide a powerful
framework for measurements of the Higgs properties. We analysed the
validity of this approach at the LHC and showed how it can be improved
with a suitable matching procedure, and developed tools based on
information geometry to guide the design of measurement
strategies. Hopefully, these new ideas can contribute to a better
understanding of the nature of the Higgs boson, which might ultimately
point us to what lies beyond the Standard Model.
