
\setchapterpreamble[ur]{%
\dictum[T.~Plehn~\cite{plehn_defense}]{%
Did you know that in my PhD defence, I didn't answer a single question correctly?}%
\vspace*{2cm}}

%%%%%%%%%%%%%%%%%%%%%%%%%%%%%%%%%%%%%%%%%%%%%%%%%%%%%%%%%%%%
\chapter*{Acknowledgements}
\addcontentsline{toc}{chapter}{Acknowledgements}
%%%%%%%%%%%%%%%%%%%%%%%%%%%%%%%%%%%%%%%%%%%%%%%%%%%%%%%%%%%%

Without doubt, the three years of my PhD were an amazing time.

For that, I first and foremost want to thank my advisor Tilman Plehn:
for taking me on as a student, for working with me on interesting
questions, and for disagreeing with me on many points over coffee. In
particular I want to thank him for always having my back, including
coming into the office on a Sunday morning to get me a backup laptop
when mine broke down days before a deadline. And last but certainly
not least, I am very grateful to him for clearing the way to an
academic career for me, even though I finally decided against this
path.

I would like to express my gratitude to Bj\"orn Malte Sch\"afer for
reading and refereeing this thesis, and to Monica Dunford and Susanne
Westhoff for completing my examination committee. Further praise is
reserved until after the defence.

The Research Training Group ``Particle physics at the LHC'' (DFG GRK
1940) paid my salary and, together with the Heidelberg Graduate School
of Fundamental Physics, my travels. I am grateful for their support.

During the different PhD projects, I had the great pleasure to
collaborate with a number of amazing scientists. Together with Tilman
Plehn, Ayres Freitas and David L\'opez-Val saw me off to a great
start. Anke Biek\"otter joined us a little later. Chasing the diboson
ambulance with JoAnne Hewett, Joachim Kopp, Tom Rizzo, and Jamie
Tattersall was a wild and fun ride. One of the best ideas during my
PhD was to invite Kyle Cranmer to a beer in Caf\'e Botanik. Working
with him was truly inspiring, and was even more fun when Felix Kling
joined our team.  Heartfelt thanks to all of my collaborators.

Much of the work presented in this thesis would not have been possible
without the help from others. In particular I would like to express my
gratitude to Juan Gonzalez-Fraile for many discussions about effective
field theories; Peter Schichtel for helping me set up and use
\toolfont{MadMax}; and Torben Schell, who was an unwavering source of
physics knowledge as well as a saviour in the hour of incomprehensible
linker errors.

From Marstall lunches to ``class trips'' to Pheno, with nut cakes, PhD
hats and the eternal coffee list, the Heidelberg pheno group was akin
to a second home in the last three years. I would like to thank Martin
Bauer, Anke Biek\"otter, Katja B\"ohnke, Anja Butter, Nishita Desai,
Karin Firnkes, Josua G\"ocking, Juan Gonzalez-Fraile, Jamil Hetzel,
Sebastian Hoof, Jan Horak, Thomas Hugle, J\"org J\"ackel, Florian
Jetter, Martin Klassen, Lara Kuhn, Rabea Link, Viraf Mehta, Luminita
Mihaila, Rhea Moutafis, Tilman Plehn, Peter Reimitz, Michael Russel,
Torben Schell, Sebastian Schenk, Peter Schichtel, Linda Shen, Beatriz
Tapia Oregui, Jamie Tattersall, Valentin Tenorth, Jennifer Thompson,
Susanne Westhoff, and Nikolai Zerf. I am grateful to Anke, Patrick,
and Sebastian for the awesome wine\,\&\,cheese seminar. But most of
all, I would like to thank Anja and Torben, who shared nearly the
entire PhD journey with me. Without the two, these three years would
not have been half as much fun.

\dots 

% Further away from the Philosophenweg, I want to thank the friends
% whose company I enjoyed during the last years. I cannot list
% everyone here, but there is enough space for a shout out to the
% Dossenheim gang\,---\,Marcel Gutsche, Clemens Hassel, and Julia
% Velte\,---\,and to Christina Eilers. I want to especially thank
% Astrid Hiller Blin for teaching me one thing or two, for loving the
% academic life, and for jump-starting my days over coffee and
% arXiv. A big thank you to my parents, who were always there when I
% needed them, and to my awesome sisters. And then there is Merle, who
% I cannot thank enough, who in just about any circumstances makes
% life better.

% If this thesis is legible, it is only because of the trustworthy
% proof readers. I am indebted to Astrid, and hopefully many more\dots

% To all these people, I am lucky to know you, to work with you, and
% to enjoy your company. Thank you.
