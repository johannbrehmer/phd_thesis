
% \setchapterpreamble[ur]{%
% \dictum[I.~Berlin~\cite{berlin_blueskies}]{%
% Blue skies smilin' at me\\
% Nothin' but blue skies do I see}%
% \vspace*{2cm}}

% \setchapterpreamble[ur]{%
% \dictum[G.~R.~R.~Martin~\cite{martin1997game}]{%
% ``It is known,'' Irri agreed.}%
% \vspace*{2cm}}



%%%%%%%%%%%%%%%%%%%%%%%%%%%%%%%%%%%%%%%%%%%%%%%%%%%%%%%%%%%%
\chapter{Foundations}
\label{chapter:foundations}
%%%%%%%%%%%%%%%%%%%%%%%%%%%%%%%%%%%%%%%%%%%%%%%%%%%%%%%%%%%%

\firstword{I}{n this chapter} we review some of the essential concepts
that underlie the research presented in the rest of this
thesis. First, we briefly summarise the role of the Higgs boson in the
Standard Model (SM) and its phenomenology at the LHC .
\autoref{sec:foundations_eft} then presents a pedagogical introduction
to the effective field theory (EFT) concept. In
\autoref{sec:foundations_higgs_eft} we combine these ideas and
construct an effective field theory for the Higgs sector.

Our introduction to Higgs physics is superficial, and the EFT part
eschews mathematical rigour in favour of a broad picture of the
central ideas. For a more thorough introduction to Higgs physics, see
for instance Reference~\cite{Plehn:2009nd}. For an extensive
introduction to EFTs, see References~\cite{Georgi:1994qn,
  Kaplan:2005es}.

The EFT section is based on a lecture given by the author and largely
identical to the lecture notes~\cite{Brehmer:EFTlecture}. Some of the
examples are taken from References~\cite{Georgi:1994qn,
  Kaplan:2005es}.



%%%%%%%%%%%%%%%%%%%%%%%%%%%%%%%%%%%%%%%%%%%%%%%%%%%%%%%%%%%%
\section{The Higgs boson at the LHC}
\label{sec:foundations_Higgs}
%%%%%%%%%%%%%%%%%%%%%%%%%%%%%%%%%%%%%%%%%%%%%%%%%%%%%%%%%%%%


%%%%%%%%%%%%%%%%%%%%%%%%%%%%%%%%%%%%%%%%%%%%%%%%%%%%%%%%%%%%
\subsection{The Standard Model Higgs sector}
%%%%%%%%%%%%%%%%%%%%%%%%%%%%%%%%%%%%%%%%%%%%%%%%%%%%%%%%%%%%

In the Standard Model, the Higgs boson is part of a scalar $SU(2)_L$
doublet $\phi$. The relevant terms in the Lagrangian are
%
\begin{multline}
  \lgr{SM} \supset (D^\mu \phi)^\dagger (D_\mu \phi) - \mu^2 \phisq - \lambda (\phisq)^2 \\
            - \sum_{\text{generations}} \left(    y_u {\twovec {\overbar u} {\overbar d}}_L \tilde \phi \, u_{R} 
                                                           + y_d {\twovec {\overbar u} {\overbar d}}_L \phi \, d_{R}
                                                           + y_\ell {\twovec {\overbar \nu} {\overbar \ell^-}}_L \phi \, \ell_{R}  + \hc \right ) \,.
  \label{eq:foundations_sm_lagrangian}
\end{multline}
%
Here $u$, $d$, and $\ell$ are the up-type quarks, down-type quarks,
and leptons of the three generations, all appearing in a left-handed
and right-handed version marked by the subscripts $L$ and $R$. $\mu^2$
and $\lambda$ are real parameters, and the Yukawa couplings $y_i$ are
complex-valued matrices in flavour space. The covariant derivatives
are defined as
%
\begin{equation}
  D_\mu \phi = \left( \partial_\mu - \im g \frac {\sigma^a} 2 W^a_\mu
    - \im \frac {g'} 2 B_\mu \right) \phi
  \label{eq:foundations_covariant_derivative}
\end{equation}
%
and 
%
\begin{equation}
  \tilde \phi \equiv \im \sigma_2 \phi^* \,,
\end{equation}
%
with $SU(2)_L$ gauge bosons $W_\mu^a$ and $U(1)_Y$ gauge boson
$B_\mu$. $g$ and $g'$ are the two corresponding coupling constants,
and $\sigma_i$ the Pauli matrices. We give the full Lagrangian of the
Standard Model including all definitions and conventions in
Appendix~\ref{sec:appendix_bases_sm}.

For $\mu^2 < 0$, the Higgs doublet develops a non-zero vacuum
expectation value (VEV)
%
\begin{equation}
  v^2 \equiv 2 \left| \langle {\phi} \rangle \right|^2  = - \frac {\mu^2} \lambda \,.
\end{equation}
%
Using some of the gauge freedom, we can rotate the scalar doublet such that
%
\begin{equation}
  \phi = \frac 1 {\sqrt{2}} \twovec  {-w_2 - \im w_1} {v + h + \im w_3} \,.
  \label{eq:foundations_sm_phi}
\end{equation} 
%
Here $w_i$ are the would-be Goldstone bosons resulting from the
spontaneous breaking of the electroweak symmetry. They combine with
the gauge bosons $W^a$ and $B$ to the mass eigenstates $\gamma$,
$W^\pm$, and $Z$. The remaining degree of freedom, the scalar field
$h$, is the Higgs boson.

Plugging \autoref{eq:foundations_sm_phi} into
\autoref{eq:foundations_sm_lagrangian}, we find its mass
%
\begin{equation}
  m_h^2 = {-2\mu^2} = {2\lambda} v^2 \,.
  \label{eq:foundations_higgs_mass_sm}
\end{equation}
%
In addition, the fermions and the vector bosons $W^\pm$ and $Z$ get
mass terms proportional to $v$, as well as couplings to the Higgs
boson $h$. Since both terms stem from the same coupling to
$\phi \sim v + h$, the Higgs couplings to other particles $x$ are
always proportional to $g_{hxx} \sim m_x / v$. Finally, there are
$h^3$ and $h^4$ self-couplings. The SM Higgs sector is very
predictive: with the measurement of the Higgs mass
$m_h = 125~\gev$~\cite{Aad:2012tfa, Chatrchyan:2012xdj,
  Khachatryan:2016vau}, there are no more free parameters in the SM
and all couplings are fixed.



%%%%%%%%%%%%%%%%%%%%%%%%%%%%%%%%%%%%%%%%%%%%%%%%%%%%%%%%%%%%
\subsection{Production and decay}
\label{sec:foundations_channels}
%%%%%%%%%%%%%%%%%%%%%%%%%%%%%%%%%%%%%%%%%%%%%%%%%%%%%%%%%%%%

\begin{figure}
  \centering
  \fmfframe(0,15)(15,15){ %(L,T) (R,B)
    \begin{fmfgraph*}(80,60) 
      \feynmansetup
      \fmfleft{i2,i1}
      \fmfright{o1}
      \fmflabel{\small $g$}{i1}
      \fmflabel{\small $g$}{i2}
      \fmflabel{\small $h$}{o1}
      \fmf{gluon}{i1,v1}
      \fmf{gluon}{i2,v2}
      \fmf{fermion,tension=0.1,label=\small $t$,label.side=left}{v1,v2}
      \fmf{fermion,tension=1}{v2,v3,v1}
      \fmf{dashes,tension=3}{v3,o1}
    \end{fmfgraph*}
  }
  \hspace{1cm}
  \fmfframe(0,15)(15,15){ %(L,T) (R,B)
    \begin{fmfgraph*}(100,60)
      \feynmansetup
      \fmfleft{i2,i1}
      \fmfright{o4,o2,o1}
      \fmflabel{\small $q$}{i1}
      \fmflabel{\small $q$}{i2}
      \fmflabel{\small $q$}{o1}
      \fmflabel{\small $h$}{o2}
      \fmflabel{\small $q$}{o4}
      \fmf{fermion,tension=4}{i1,v3}
      \fmf{fermion,tension=4}{i2,v4}
      \fmf{fermion,tension=2.5}{v3,o1}
      \fmf{fermion,tension=2.5}{v4,o4}
      \fmf{wiggly,label=\small $W$,, $Z$,label.side=right}{v3,v5}
      \fmf{wiggly,label=\small $W$,, $Z$,label.side=left}{v4,v5}
      \fmf{dashes,tension=0.5}{v5,o2}
      %\fmfv{decoration.shape=circle,foreground=(0.776,, 0.094,, 0.149),decoration.size=5}{v5}
    \end{fmfgraph*}
  }
  \hspace{1cm}
  \fmfframe(0,15)(15,15){ %(L,T) (R,B)
    \begin{fmfgraph*}(80,60)
      \feynmansetup
      \fmfleft{i2,i1}
      \fmfright{o2,o1}
      \fmflabel{\small $q$}{i1}
      \fmflabel{\small $q$}{i2}
      \fmflabel{\small $h$}{o1}
      \fmflabel{\small $W$, $Z$}{o2}
      \fmf{fermion}{i1,v1,i2}
      \fmf{photon,label=\small $W$,, $Z$}{v1,v2}
      \fmf{dashes}{v2,o1}
      \fmf{photon}{v2,o2}
    \end{fmfgraph*}
  }
  \caption[Feynman diagrams for main Higgs production modes]{Feynman diagrams for
    the most important Higgs production modes considered in this
    thesis. Left: gluon fusion. Middle: weak boson fusion. Right:
    Higgs-strahlung.}
  \label{fig:foundations_production_diag}
\end{figure}

At the LHC, most Higgs bosons are produced in \emph{gluon-gluon
  fusion} (ggF) as shown in the left panel of
\autoref{fig:foundations_production_diag}. Due to its large Yukawa
coupling, the top plays the dominant role in the loop, with small
contributions from the bottom. The total cross section at
$\sqrt{s} = 13~\tev$ is approximately
$49~\pb$~\cite{deFlorian:2016spz}, a large part of which comes from
NLO and NNLO corrections. This sizeable rate comes at the price of a
lack of discerning kinematic features that could help to separate the
Higgs signal from QCD backgrounds.

This is certainly different for Higgs production in \emph{weak boson
  fusion} (WBF)\footnote{This channel is also known as Vector Boson
  Fusion or VBF. But this is slightly misleading since the gluon also
  has spin 1.}, as shown in the middle panel of
\autoref{fig:foundations_production_diag}. The production rate for
this quark-initiated process is only
$3.8~\pb$~\cite{deFlorian:2016spz}, but the Higgs is accompanied by
two highly energetic jets that point nearly back-to-back into the two
forward regions of the detector. This translates to a large invariant
mass $m_{jj}$ between them as well as a large separation in
(pseudo-)rapidity $\Delta \eta_{jj}$. A second important property is
provided by the colour structure of the process: at leading order,
there is no colour exchange between the two quark lines, which means
there is very little QCD radiation in this process. Both of these
features set the WBF process apart from QCD backgrounds, which
typically have many central jets. Such backgrounds can therefore be
reduced significantly by requiring two so-called ``tagging jets'' with
large $\Delta \eta_{jj}$ and large $m_{jj}$, and vetoing any
additional central jets~\cite{Rainwater:1998kj}.

But the tagging jets are not only useful to discriminate Higgs
production from non-Higgs backgrounds. Since they recoil against the
intermediate vector bosons that couple to the Higgs, they provide
access to the momentum flow through the Higgs production vertex. Their
properties, in particular their transverse momenta and the angular
correlations between them, thus provide probes of the Higgs-gauge
coupling~\cite{Eboli:2000ze, Plehn:2001nj, Hankele:2006ma,
  Hagiwara:2009wt, Englert:2012xt, Buckley:2014fqa,
  Brehmer:2014pka}. We revisit this important feature from different
perspectives in this thesis.

The right panel of \autoref{fig:foundations_production_diag} shows
Higgs production in association with a vector boson, or
\emph{Higgs-strahlung}. The rate is $1.4~\pb$ for a $Wh$ final state
plus $0.9~\pb$ for $Zh$. Similarly to the tagging jets in WBF, the
final-state gauge boson both helps to discriminate the Higgs from
backgrounds and provides a handle on the momentum flow through the
virtual intermediate vector boson.

\begin{figure}
  \centering
  \fmfframe(0,15)(15,15){ %(L,T) (R,B)
    \begin{fmfgraph*}(120,60)
      \feynmansetup
      %
      \fmfleft{i2,i1}
      \fmfright{o3,o2,o1}
      \fmflabel{\small $q$}{i1}
      \fmflabel{\small $b$}{i2}
      \fmflabel{\small $q$}{o1}
      \fmflabel{\small $h$}{o2}
      \fmflabel{\small $t$}{o3}
      %
      % Upper quark line
      \fmf{fermion,tension=4}{i1,v1}
      \fmf{fermion,tension=2.5}{v1,o1}
      %
      % Lower quark line
      \fmf{fermion,tension=4}{i2,v3}
      \fmf{fermion,tension=2.5}{v3,o3}
      %
      % W exchange 
      \fmf{wiggly,label=\small $W$,label.side=right}{v1,v2}
      \fmf{wiggly,label=\small $W$,label.side=right}{v2,v3}
      %
      % Higgs 
      \fmf{dashes,tension=0.5}{v2,o2}
    \end{fmfgraph*}
  }
  \hspace{1cm}
  \fmfframe(0,15)(15,15){ %(L,T) (R,B)
    \begin{fmfgraph*}(120,60)
      \feynmansetup
      %
      \fmfleft{i2,i1}
      \fmfright{o3,o2,o1}
      \fmflabel{\small $q$}{i1}
      \fmflabel{\small $b$}{i2}
      \fmflabel{\small $q$}{o1}
      \fmflabel{\small $h$}{o2}
      \fmflabel{\small $t$}{o3}
      %
      % Upper quark line
      \fmf{fermion,tension=4}{i1,v1}
      \fmf{fermion,tension=2.5}{v1,o1}
      %
      % Lower quark line
      \fmf{fermion,tension=4}{i2,v2}
      \fmf{fermion,label=\small $t$,label.side=right,tension=5}{v2,v3}
      \fmf{fermion,tension=5}{v3,o3}
      %
      % W exchange 
      \fmf{wiggly,label=\small $W$,label.side=right}{v1,v2}
      %
      % Higgs 
      \fmf{dashes,tension=0.5}{v3,o2}
    \end{fmfgraph*}
  }
  \caption[Feynman diagrams for Higgs plus single top production]{Feynman diagrams for Higgs production with a single top quark.}
  \label{fig:foundations_th_diag}
\end{figure}

Higgs production in association with a $t \bar t$ pair is another
production mode relevant for the LHC, but do not discuss it in
this thesis. Instead, we briefly analyse \emph{Higgs production
  with a single top quark}. This process exists as an $s$-channel and
a $t$-channel version with very different kinematic features, and can
be calculated either in the four-flavour scheme (with a gluon in the
initial state) or in the five-flavour scheme (with a $b$ quark in the
initial state). We focus on the dominant $t$-channel process and
calculate it in the five-flavour scheme, as shown in
\autoref{fig:foundations_th_diag}. Diagrams where the Higgs is
radiated off a top quark interfere destructively with amplitudes in
which the Higgs couples to a $W$. The SM rate is small at
$74~\fb$~\cite{deFlorian:2016spz}, but this interference pattern makes
it very sensitive to modified Higgs couplings. In fact, this process
is the only direct probe of the sign or complex phase of the top
Yukawa coupling ($t \bar{t} h$ production is only sensitive to the
absolute value of the top Yukawa, while the total rate in gluon fusion
can be influenced by many effects such as new particles in the loop).

\begin{figure}[b]
  \centering
  \fmfframe(0,15)(15,15){ %(L,T) (R,B)
    \begin{fmfgraph*}(120,60) 
      \feynmansetup
      \fmfleft{i2,i1}
      \fmfright{o2,o1}
      \fmflabel{\small $g$}{i1}
      \fmflabel{\small $g$}{i2}
      \fmflabel{\small $h$}{o1}
      \fmflabel{\small $h$}{o2}
      \fmf{gluon}{i1,v1}
      \fmf{gluon}{i2,v2}
      \fmf{fermion,tension=0.5,label=\small $t$,label.side=left}{v1,v2}
      \fmf{fermion}{v2,v3}
      \fmf{fermion,tension=0.5}{v3,v4}
      \fmf{fermion}{v4,v1}
      \fmf{dashes}{v3,o2}
      \fmf{dashes}{v4,o1}
    \end{fmfgraph*}
  }
  \hspace{1cm}
  \fmfframe(0,15)(15,15){ %(L,T) (R,B)
    \begin{fmfgraph*}(120,60) 
      \feynmansetup
      \fmfleft{i2,i1}
      \fmfright{o2,o1}
      \fmflabel{\small $g$}{i1}
      \fmflabel{\small $g$}{i2}
      \fmflabel{\small $h$}{o1}
      \fmflabel{\small $h$}{o2}
      \fmf{gluon}{i1,v1}
      \fmf{gluon}{i2,v2}
      \fmf{fermion,tension=0.1,label=\small $t$,label.side=left}{v1,v2}
      \fmf{fermion,tension=1}{v2,v3,v1}
      \fmf{dashes,tension=1.5,label=\small $h$}{v3,v4}
      \fmf{dashes,tension=2}{o2,v4,o1}
    \end{fmfgraph*}
  }
  \caption[Feynman diagrams for Higgs pair production]{Feynman
    diagrams for Higgs pair production.}
  \label{fig:foundations_hh_diag}
\end{figure}

Our final channel is \emph{Higgs pair production}, showing in
\autoref{fig:foundations_hh_diag}, which allows to measure the Higgs
self-coupling~\cite{Plehn:1996wb, Baur:2002rb}. It is another example
of destructive interference between different amplitudes: diagrams in
which the two Higgses couple to a top box loop interfere with those in
which a single Higgs is produced in gluon fusion and then splits into
two Higgses through the self-coupling. Close to threshold, these two
contributions approximately cancel in the SM~\cite{Plehn:1996wb,
  Li:2013rra}, and the total rate is very small at $33~\fb$. But
modified Higgs sectors can spoil this cancellation and increase the
rate drastically, as we demonstrate in the next chapter.

\newparagraph
%
The Higgs decay patterns are rather simple. Since it couples to all
particles proportional to their mass, it prefers to decay into the
heaviest particles allowed by phase space. The dominant decay mode
with a branching ratio of $58 \%$~\cite{deFlorian:2016spz} is
therefore $h \to b\bar{b}$. This signature is clearly useless for
Higgs bosons produced in gluon fusion because of the overwhelming QCD
$gg \to b \bar{b}$ background. WBF and $Vh$ production provide handles
to tame these backgrounds, but the channel is still difficult. Easier
to detect are $\tau^+ \tau^-$ pairs with a branching ratio of
$6.3 \%$. Their semi-leptonic and purely leptonic decays involve
neutrinos. But if the taus are boosted enough and not exactly
back-to-back, the neutrino momentum can be reconstructed for instance
using a collinear approximation~\cite{Plehn:2009nd}.

The decays through $W^+W^-$ or $ZZ$ pairs into four-lepton final
states are particularly important due to their clean signatures and
because they provide access to Higgs-gauge couplings. Since the Higgs
mass is below the $W^+W^-$ and $ZZ$ thresholds, one of the vectors has
to be off-shell.\footnote{This also means that the branching ratios
  for $h\to ZZ$ and $h \to WW$ are not really well-defined. What is
  often quoted is in fact a term like
  $\br(h \to 4 \ell) / (\br(Z \to \ell^+ \ell^-))^2$.}  The channel
$h \to W^+W^- \to (\ell^+ \nu) (\ell^- \overbar{\nu})$ with
$\ell = e, \mu$ has a respectable branching fraction of
$1.1 \%$~\cite{deFlorian:2016spz}, but comes with two neutrinos in the
final state. Still, it is one of the most important channels to
measure WBF Higgs production. The decay $h \to ZZ \to 4 \ell$ with
$\ell = e, \mu$ provides an extremely clean signal. Despite its small
branching ratio of $1.3 \cdot 10^{-4}$, it was one of the most
important channels for the discovery of the Higgs
boson~\cite{Aad:2012tfa,Khachatryan:2016vau}. From a post-discovery
perspective, its four leptons provide a rich spectrum of angular
correlations and other observables that allow us to measure the Higgs
behaviour in detail. We discuss this feature in more depth in
\autoref{chapter:information}.

Finally, the small tree-level couplings of the Higgs to light
particles mean that the loop-induced decay into photon pairs can
compete with them. The dominant contribution comes from a $W$ loop,
which interferes destructively with the top loop, resulting in a
branching ratio of $0.23 \%$~\cite{deFlorian:2016spz}. The ATLAS and
CMS detectors are designed to reconstruct photons well, in fact with
exactly this Higgs decay channel in mind. Together with $h \to 4 \ell$
it constitutes the most important channel for the discovery.



%%%%%%%%%%%%%%%%%%%%%%%%%%%%%%%%%%%%%%%%%%%%%%%%%%%%%%%%%%%%
\subsection{How I Learned to Stop Worrying and Love the Higgs}
\label{sec:foundations_relevance}
%%%%%%%%%%%%%%%%%%%%%%%%%%%%%%%%%%%%%%%%%%%%%%%%%%%%%%%%%%%%

There are several facets of the Higgs boson that make it special. From
an experimental point of view, the properties of this shiny new thing
in particle physics are still relatively unknown. Its couplings to
vector bosons and heavy fermions are constrained at the $\ord{10\%}$
level, while for the couplings to light fermions, invisible decays,
and the total decay width of the Higgs there are only weak upper
bounds~\cite{Khachatryan:2016vau, Corbett:2015ksa}. Many of these
limits also rely on specific model assumptions. The top Yukawa
coupling, for instance, is most strongly constrained from the total
Higgs production rate, but only under the assumption that there no new
physics plays a role in the gluon-fusion loop. The total Higgs width
can be constrained indirectly from the contribution of
$g g \to h \to ZZ \to 4 \ell$ in the off-shell Higgs region, but again
relying on strong model assumptions. All in all, the Higgs is still
the least well measured elementary particle (in some sense with the
exception of neutrinos), leaving plenty of room for physics beyond the
Standard Model.

\newparagraph
%
From a theory perspective, there are several reasons to suspect
manifestations of new physics in the Higgs sector. Starting with a
rather general argument, the Higgs doublet is the key component of
electroweak symmetry breaking (EWSB), which is often seen as the very
core of the SM construction. A test of the Higgs properties therefore
provides a test of the fundamental structure of Nature.

The Higgs boson is the only fundamental scalar discovered so far. This
is interesting in its own right, but also leads to the famous
electroweak \emph{hierarchy problem}: in the absence of any protective
symmetry, the mass of a scalar field should receive quantum
corrections of the order of the largest scale in the theory. If the SM
is valid all the way to the Planck scale, severe fine-tuning between
the bare parameter and these quantum corrections is necessary to keep
the electroweak mass scale at the observed value. Note that this
argument interchangeably applies to the mass parameter of the Higgs
doublet $\mu^2$, the physical Higgs mass $m_h$, or the electroweak VEV
$v$. Since the strength of the weak force is suppressed by powers of
$m_W \sim v$, and the gravitational force by the Planck scale, the
hierarchy problem is often phrased in terms of the surprising weakness
of gravity compared to the weak force. This naturalness problem is of
a purely aesthetic nature, but similar aesthetic problems have in the
past led to new insights. Many models have been proposed to solve the
hierarchy by introducing a new symmetry that protects the Higgs mass
against quantum corrections.\footnote{An entirely different and
  somewhat metaphysical argument is based on the (weak) anthropic
  principle that observations of the universe are conditional upon its
  laws of physics allowing conscious life~\cite{Weinberg:1987dv,
    Barrow:1988yia}. First, this explanation requires some mechanism
  that generates many different vacua with different values of the
  physics parameters, including the Higgs mass. Most of these vacua
  have ``natural'' parameters in which the weak and gravitational
  scales are comparable. String theory is hypothesised to provide such
  a sampling mechanism (the ``multiverse''). Second, there has to be a
  reason why larger (and thus more abundant) values of the weak scale
  would not allow any type of intelligent life to form and make
  observations. This question is difficult to answer, and the jury is
  still out~\cite{Agrawal:1997gf, Harnik:2006vj, Clavelli:2006di,
    Giudice:2008bi, Donoghue:2009me, Gedalia:2010iy,
    Adams:2015hvd}. Given the speculative nature of the two questions,
  anthropic reasoning is being criticised as unverifiable or as based
  on arguments from lack of imagination.}\footnote{Finally, other
  models such as the relaxion~\cite{Graham:2015cka} or
  Nnaturalness~\cite{Arkani-Hamed:2016rle} modify the cosmological
  evolution such that a small Higgs mass is generated dynamically
  during inflation or reheating.}\comment{Citation needed for
  criticism of anthropic principle.} Famous examples are
supersymmetry, composite Higgs models in which the Higgs is the
pseudo-Goldstone boson of some broken symmetry, conformal
symmetries~\cite{Bardeen:1995kv}, or extra dimensions. To reduce
tuning to an acceptable level, this new physics should reside at
energy scales not too far from the electroweak scale. These models
usually modify the Higgs sector in a way that translates into Higgs
couplings different from their SM values.

Another hierarchy unexplained in the SM is the large difference
between the \emph{fermion masses}. There are more than five orders of
magnitude between the top and the electron mass, and neutrinos are
even lighter. Since the fermion masses are generated by the Yukawa
couplings of the Higgs doublet, models that explain the fermion masses
usually also shift the Higgs-fermion coupling patterns.

The question of \emph{vacuum stability} is still being discussed. The
renormalisation group (RG) allows us to link this question
to the running the quartic coupling $\lambda$ to higher
energies. Current results~\cite{Degrassi:2012ry} indicate that indeed
the quartic coupling becomes negative at large energies, leading to a
second vacuum with lower energy at much larger values of
$\phi$. Fortunately for us, the tunnelling probability is very small,
and in the SM ``our'' vacuum with $v \approx 246~\gev$ seems to be
metastable with a lifetime longer than the age of the universe. While
this indicates there is no pressing need for physics below the Planck
scale to save the electroweak vacuum from a horrible fate, this result
crucially depends on the measured top and Higgs masses, higher-order
corrections to the beta functions, and higher-dimensional operators
stemming from ultra-violet (UV) physics~\cite{Eichhorn:2015kea}.

In addition to these theoretical and to some degree aesthetic
arguments, there is solid experimental evidence for physics beyond the
SM that might be linked to the Higgs sector. First, the nature of
\emph{dark matter} (DM)~\cite{Plehn:DM} is still unclear. It is experimentally
established that this form of matter is electrically neutral, stable
over cosmological timescales, clumps (i.\,e.\ is now
non-relativistic), and makes up roughly a fourth of the energy density
of the universe. In many models DM is in thermal equilibrium with
ordinary matter in the early universe. Interestingly, the observed
dark matter density is in good agreement with electroweak-scale masses
and weak couplings. This ``WIMP miracle'' is one main reason behind
the popularity of weakly interacting massive particles (WIMPs) as DM
candidates. In this scenario, good candidates for the mediator between
dark matter and the SM are the Higgs boson or other scalars in an
extended Higgs sector. Such ``Higgs portal'' scenarios often predict
signatures in Higgs physics such as modified couplings or invisible
Higgs decays.

Another mystery is the \emph{matter-antimatter asymmetry} of the
universe. Assuming that the cosmos was initially perfectly symmetric,
the observed excess of matter can be generated dynamically if the
three Sakharov conditions are satisfied: there have to be processes
with baryon-number violation as well as $C$ and $CP$ violation, which
take place out of thermal equilibrium. In the SM, these effects are
too small to account for the observed asymmetry. Models that
accommodate larger effects often affect the Higgs sector. In
particular, extended Higgs sectors allow for electroweak symmetry
breaking to be a strong first-order phase transition, providing the
required out-of-equilibrium dynamics. Again, such scenarios predict
signatures in Higgs measurements.

Finally, the Higgs could play another role in the cosmological
evolution of the universe. The origin of the large-scale structure of
the cosmos, the surprising isotropy of the cosmic microwave background
(CMB), and the flatness of the Universe are all explained by an epoch
of exponential expansion of space in the early universe called
\emph{inflation}. This process is often thought to be caused by a
scalar field, the inflaton, slowly rolling down a potential of a
certain shape. In principle the Higgs can be the inflaton, though this
scenario of Higgs inflation requires unnaturally large couplings
between the Higgs and the Ricci scalar, and requires a UV completion.

\newparagraph
%
The null results of the LHC searches for new particles have led to
some disappointment among particle physicists. But with the discovery
of the Higgs boson, the LHC might not only have completed the SM
particle zoo, but rather opened the door to the unknown.  The Higgs
boson is not just another SM particle.  Some of the big open questions
of fundamental physics are deeply rooted in the Higgs sector, and many
other ideas can at least be linked to the Higgs sector under some
assumptions. On the other hand, the current experimental precision
leaves quite some room for signatures of new physics in Higgs
observables. A precise determination of the Higgs properties might be
one of the most exciting measurements at the LHC, and hopefully
improve our understanding of Nature significantly. Hopefully, the
Higgs boson is not just the last puzzle piece of the Standard Model,
but the first sign of what lies beyond.



%%%%%%%%%%%%%%%%%%%%%%%%%%%%%%%%%%%%%%%%%%%%%%%%%%%%%%%%%%%%
\section{The effective field theory idea}
\label{sec:foundations_eft}
%%%%%%%%%%%%%%%%%%%%%%%%%%%%%%%%%%%%%%%%%%%%%%%%%%%%%%%%%%%%

\comment{Check that all arguments of
  \url{http://indico.cern.ch/event/477407/contributions/2200060/attachments/1369793/2076900/AM_HiggsCouplings2016.pdf}
  are in this section. In particular slide 5.}

This plethora of possible BSM models means that a model-independent
universal theory framework is invaluable for TeV signatures of new
physics. We consider such a model based on the effective field
theory (EFT) paradigm. Before discussing the specific realisation for
Higgs physics in the next section, here we introduce the general EFT idea.

Effective field theories are a powerful tool that play a role in many,
if not all, areas of physics. Whenever phenomena are spread out over
different energy or length scales, an effective description can be
valuable, either to simplify calculations, or to actually allow
model-independent statements that would be impossible without such a
framework.



%%%%%%%%%%%%%%%%%%%%%%%%%%%%%%%%%%%%%%%%%%%%%%%%%%%%%%%%%%%%
\subsection{Different physics at different scales}
\label{sec:foundations_scales}
%%%%%%%%%%%%%%%%%%%%%%%%%%%%%%%%%%%%%%%%%%%%%%%%%%%%%%%%%%%%

Our world behaves very differently depending on which energy and
length scales we look at. At extremely high energies (or short
distances), Nature might be described by a quantum theory of
gravity. At energies of a few hundred GeV, the Standard Model is
(maybe disappointingly) in agreement with all measurements. Going to lower
energies (or larger distances), we do not have to worry about the Higgs or
$W$ bosons anymore: electromagnetic interactions are described by QED,
weak interactions by Fermi theory, strong physics by QCD. Below a GeV,
quarks and gluons are replaced by pions and nucleons as the relevant
degrees of freedom. Then by nuclei, atoms, molecules. At this point
most physicists give up and let chemists (and ultimately biologists
and sociologists) analyse the systems.

The important point here is that the observables at one scale are not
directly sensitive to the physics at significantly different
scales. This is nothing new: for molecules to stick together, the
details of the Higgs sector are not relevant, just as we can calculate how
an apple falls from a tree without knowing about quantum gravity. To
do physics at one scale, we do not have to (and often cannot) take
into account the physics from all other scales. Instead, we isolate
only those features that play a role at the scale of interest.

An effective field theory is a physics model that includes all effects
relevant at a given scale, but not those that only play a role at
significantly different scales. In particular, EFTs ignore spatial
substructures much smaller than the lengths of interest, or effects at
much higher energies than the energy scale considered.
% In other words, EFTs provide an organized way of approximating
% phenomena that\,---\,at a scale of interest\,---\,are very small as
% zero, and quantities that are very large as infinite.

We often use examples with one full or underlying theory and
one effective theory. For simplicity, we pretend that the full theory
describes physics correctly at all scales. The EFT is a simpler model
than the full theory and neglects some phenomena (such as heavy
particles) at an energy scale $\Lambda$. However, it correctly describes
the physics as long as the observables probe energy scales
%
\begin{equation}
  E \ll \Lambda \,,
  \label{eq:foundations_scale_separation}
\end{equation}
%
within some finite precision. This \emph{scale hierarchy} between the
energy of interest and the scale of high-energy physics not included
in the EFT is the basic requirement for the EFT idea. A validity
range~\eqref{eq:foundations_scale_separation} is a fundamental property of each
EFT.





%%%%%%%%%%%%%%%%%%%%%%%%%%%%%%%%%%%%%%%%%%%%%%%%%%%%%%%%%%%%
\subsubsection{Fermi theory}
%%%%%%%%%%%%%%%%%%%%%%%%%%%%%%%%%%%%%%%%%%%%%%%%%%%%%%%%%%%%

The textbook example for an EFT is Fermi theory, which
describes the charged current interactions between quarks (or
hadrons), leptons and neutrinos at low energies. The underlying model
here is the SM, in which this weak interaction is mediated by the
exchange of virtual $W$ bosons with mass $m_W$ and coupling constant
$g$:
%
\begin{equation}
  \mathcal{M}_\text{full} \sim {}
  \raisebox{-27pt}{
    %\fbox{
      \fmfframe(0,10)(0,10){ %(L,T) (R,B)
        \begin{fmfgraph*}(70,40) 
          \feynmansetup
          \fmfleft{i2,i1}
          \fmfright{o2,o1}
          \fmflabel{\small $f_1$}{i1}
          \fmflabel{\small $f_2$}{i2}
          \fmflabel{\small $f_3$}{o1}
          \fmflabel{\small $f_4$}{o2}
          \fmf{fermion}{i1,v1,i2}
          \fmf{fermion}{o1,v2,o2}
          \fmf{boson,tension=1,label=\small $W$}{v1,v2}
          \marrow{m}{up}{top}{$p$}{v1,v2}
          \fmflabel{\small $g$}{v1}
          \fmflabel{\small $g$}{v2}
        \end{fmfgraph*}
      }
    %}
  }
  {} \sim - \frac {g^2} {p^2 - m_W^2} \,.
  \label{eq:foundations_4fermion_full}
\end{equation}
%
In Fermi theory, there are no $W$ bosons, just a direct
interaction between four fermions with coupling constant
$G_F \propto g^2 / m_W^2$:
%
\begin{equation}
  \mathcal{M}_\text{EFT} \sim {}
  \raisebox{-27pt}{
    %\fbox{
      \fmfframe(0,10)(0,10){ %(L,T) (R,B)
        \begin{fmfgraph*}(50,40) 
          \feynmansetup
          \fmfleft{i2,i1}
          \fmfright{o2,o1}
          \fmflabel{\small $f_1$}{i1}
          \fmflabel{\small $f_2$}{i2}
          \fmflabel{\small $f_3$}{o1}
          \fmflabel{\small $f_4$}{o2}
          \fmf{fermion}{i1,v,i2}
          \fmf{fermion}{o1,v,o2}
          \fmflabel{\small $G_F$}{v}
          \fmfv{decor.shape=circle,decor.filled=full, decor.size=5}{v}
        \end{fmfgraph*}
      }
    %}
  }
  {} \sim G_F \propto \frac {g^2} {m_W^2} \,.
  \label{eq:foundations_4fermion_EFT}
\end{equation}
%
So the EFT turns the $W$ propagator into a contact interaction between
the fermions, shrinking the distance bridged by the virtual $W$ to
zero. Clearly, the two amplitudes agree as long as the momentum
transfer through the vertex is small,
$E^2 = p^2 \ll \Lambda^2 = m_W^2$.
% %
% \begin{equation}
%   - \frac {g^2} {p^2 - m_W^2}
%   = \frac {g^2} {m_W^2} \left(1 +  \frac {p^2} {m_W^2} + \ord{p^4/m_W^4} \right)
%   \approx  \frac {g^2} {m_W^2} \,.
% \end{equation}

One process described by this interaction is muon decay. Its typical
energy scale $E \approx m_\mu$ is well separated from $\Lambda = m_W$,
and Fermi theory describes the process quite accurately. The relative
\emph{EFT error}, i.\,e.\ the mistake we make when calculating an
observable with the EFT rather than with the full model, should be of
order
$\Delta_\text{EFT} = \Gamma_\text{EFT} / \Gamma_\text{full} \sim E^2 /
\Lambda^2 \sim m_\mu^2 / m_W^2 \approx 10^{-6}$.

In proton collisions at the LHC the same interaction takes place, but
at potentially much larger momentum transfer $E \lesssim 13$~TeV. The
EFT error increases with $E$. For $E \gtrsim m_W$, the full model
allows on-shell $W$ production, a feature entirely missing in the
EFT. Here the two descriptions obviously diverge and Fermi theory is
no longer a valid approximation of the weak interaction.



%%%%%%%%%%%%%%%%%%%%%%%%%%%%%%%%%%%%%%%%%%%%%%%%%%%%%%%%%%%%
\subsubsection{Down and up the theory ladder}
%%%%%%%%%%%%%%%%%%%%%%%%%%%%%%%%%%%%%%%%%%%%%%%%%%%%%%%%%%%%

In reality there are of course more than two theories, and the notion
of underlying and effective model becomes relative. The SM itself is
not valid up to arbitrary large energies: it does not explain dark
matter, the matter-antimatter asymmetry, or gravity. It is probably
also internally inconsistent since at some very large energy the
quartic coupling $\lambda$ and the coupling constant $g'$ hit Landau
poles. So the SM is an effective theory with validity range
$E \ll \Lambda \le M_{Pl}$ and has to be replaced by some other
description at larger energies. On the other hand, going to energies
lower than a few GeV, the relevant physics changes again and we should
switch to a new effective theory. In this way, all theories can be
thought of as a series of EFTs, where the model valid at one scale is
the underlying model for the effective theory at the next lower scale.

If you think you know a theory that describes our world at
sufficiently large energies, then in principle there is no need to use
effective theories: you can calculate every single observable in your
full model (at least if the full model is perturbative at these
energies or other approximations such as lattice calculations are
available). This however makes hard calculations necessary even
for the simplest low-energy processes. One can save a lot of
computational effort and focus on the relevant physics by dividing the
phase space into regions with different appropriate effective
descriptions.

Starting from a high energy scale where the parameters of the
fundamental theory are defined, these parameters are run to lower
energies until the physics changes substantially. At this
\emph{matching scale} an effective theory is constructed from the full
model, and its coefficients are determined from, or matched to, the
underlying model. Then the coefficients of this EFT are run down to
the next matching scale, where a new EFT is defined and its parameters
are calculated, and so on. This is the \emph{top-down} view of
EFTs. For instance, we can start from the SM and construct Fermi
theory as a simpler model valid at low energies. While we can
certainly use the SM to calculate the muon lifetime, it is not
necessary, and a calculation in Fermi theory is quite accurate and
simpler.

But often we do not know the underlying theory. As mentioned above,
there has to be physics beyond the SM, and there is still hope it will
appear around a few TeV. If we want to parametrise the effects of such
new physics on electroweak-scale observables, we do not know how the
full model looks like. But even without knowing the underlying model,
we can still construct an effective field theory based on a few very
general assumptions. We go through these ingredients in the next
section. For this \emph{bottom-up} approach, an effective theory is
not only useful, but actually the only way we can discuss new physics
without choosing a particular model of BSM physics.

High-energy physics can be seen as the field of working ourselves up a
chain of EFTs to ever higher energies. But how does this chain end?
Does it end at all? Even if we one day find a consistent theory that
can explain all observations to date, how would we check if it indeed
describes Nature up to arbitrarily high energies? Understanding all
theories as effective, these questions do not matter. The EFT
framework provides us with the tools to do physics without having to
worry about the far ultraviolet.

% To summarize this introduction, the basic EFT idea is to take into
% account only the phenomena relevant at the scale of interest. With
% this broad definition one can even say that effective theories are
% pretty much the only way we can do physics at all. The question is
% not \emph{if} one should use an EFT for a given process (after all,
% what would be the alternative?), but \emph{which} effective theory
% is the best one for a given process, especially in light of the
% validity range of all effective theories.

% There is nothing strange or complicated about effective field
% theories. They simply provide an organized way of doing what we always
% do in physics: neglecting effects that do not matter for a given
% question. EFTs in the form of quantum field theories consist of a set
% of (typically non-renormalizable) operators. You have seen how this
% framework allows us to start with a full theory and constructive an
% effective approximation from the top down, and how it even allows us
% to construct an approximate description of physics even if we do not
% know the underlying theory.



%%%%%%%%%%%%%%%%%%%%%%%%%%%%%%%%%%%%%%%%%%%%%%%%%%%%%%%%%%%%
\subsection{EFT construction and the bottom-up approach}
\label{sec:foundations_eft_bottom_up}
%%%%%%%%%%%%%%%%%%%%%%%%%%%%%%%%%%%%%%%%%%%%%%%%%%%%%%%%%%%%

EFTs are especially useful in the framework of quantum field theory
(QFT). Before showing how to construct the effective operators of such
a theory in a bottom-up approach, let us recapitulate how QFTs are
organised.



%%%%%%%%%%%%%%%%%%%%%%%%%%%%%%%%%%%%%%%%%%%%%%%%%%%%%%%%%%%%
\subsubsection{Reminder: operators and power counting}
%%%%%%%%%%%%%%%%%%%%%%%%%%%%%%%%%%%%%%%%%%%%%%%%%%%%%%%%%%%%

The basic object describing perturbative QFTs in $d=4$ flat space-time
dimensions is the action
%
\begin{equation}
  S = \intfourx \lgr{}(x) \,.
  \label{eq:foundations_action}
\end{equation}
%
The Lagrangian $\lgr{}(x)$ is a sum of couplings times operators,
where the operators are combinations of fields and derivatives
evaluated at one point $x$. These are either kinematic terms, mass
terms or represent interactions between three or more fields. For
instance, the Lagrangian
%
\begin{equation}
  \lgr{} =  \im  \bar{\psi}_i \gamma^\mu \partial_\mu \psi_i - \frac 1 4 V_{\mu \nu} V^{\mu \nu} 
  - m_i \bar{\psi}_i \psi_i + m_V^2 V_\mu V^\mu
  - g \bar{\psi}_i \gamma_\mu \psi_i V^\mu
  \label{eq:foundations_action_V_fermions}
\end{equation}
%
with implicit sum over $i$ describes fermions $\psi_i$, a vector boson
$V_\mu$ of mass $m_V$, and an interaction between them governed by a
coupling $g$.\footnote{Massive vector bosons have issues with
  renormalisability and unitarity, which can be solved by generating
  the mass in a Higgs mechanism at a higher scale, but this is irrelevant
  for the discussion here.}

A key property of each coupling or operator is its \emph{canonical
  dimension} or mass dimension. In simple terms this can be formulated
as the following question: if you assign a value to a quantity, which
power of a mass unit such as GeV would this value carry? Since we work
in units with $\hbar = c = 1$, length and distance dimensions are just
the inverse of mass dimensions. We denote the mass dimension of
any object with squared brackets, where $[\ope{}] = D$ means that
$\ope{}$ is of dimension mass$^D$, or mass dimension $D$.

In QFT, the action can appear in exponentials such as $e^{\im S}$, so
it must be dimensionless: $[S] = 0$. The space-time integral in
\autoref{eq:foundations_action} then implies $[\lgr{}] = d = 4$, so every term
in the Lagrangian has to be of mass dimension 4. Applying this to the
kinetic terms, we can calculate the mass dimension of all fields. This
then allows us to calculate the mass dimension or canonical dimension of operators and
couplings in the theory.

In the example in \autoref{eq:foundations_action_V_fermions}, the kinetic term
for the fermions contains one space-time derivative, $[\partial] =
1$.
To get $[\bar{\psi} \partial \psi] = 4$, the fermion fields must have
dimension $[\psi_i] = 3/2$. Similarly, the field strength $V_{\mu\nu}$
contains a derivative, so we end up with $[V_\mu] = 1$. With these
numbers we can check the other operators. In addition to the expected
$[m] = [m_V] = 1$, we find $[\bar{\psi} \psi V^\mu] = 4$ or $[g] = 0$.

The canonical dimension of an operator has two important
consequences. First, the renormalisation group flow of a theory,
i.\,e.\ the running of the couplings between different energy scales,
largely depends on the mass dimensions of the operators. Operators
with mass dimension $D < d$ (``relevant'' operators) receive
substantial quantum corrections when going from high to low energies.
This is a key argument for many fine-tuning problems such as the
hierarchy problem or the cosmological constant problem. On the other
hand, operators with $D > d$ (``irrelevant'' ones) are typically
suppressed when going to lower energies. Operators with $D = d$ are
called ``marginal''.

The second consequence of the mass dimension affects the
renormalisability of a theory. Theories with operators with $D > d$
are \emph{non-renormalisable}:\footnote{The opposite is not true: some
  theories contain only operators with $D \le d$, but are still not
  renormalisable.} particles in loops with energies $E \to \infty$
lead to infinities in observables, and they are too many to be
hidden in a renormalisation of the parameters.



%%%%%%%%%%%%%%%%%%%%%%%%%%%%%%%%%%%%%%%%%%%%%%%%%%%%%%%%%%%%
\subsubsection{Effective operators}
%%%%%%%%%%%%%%%%%%%%%%%%%%%%%%%%%%%%%%%%%%%%%%%%%%%%%%%%%%%%

From now on we only consider EFTs realised as a local QFT in 4
space-time dimensions, an approach that has proven very successful in
high-energy physics so far. EFTs are then defined as a sum of
operators $\ope{i}$, each with a specific canonical dimension $D_i$. We can
split the coupling in front of each operator into a dimensionless
constant, the \emph{Wilson coefficient} $f_i$, and some powers of a
mass scale, for which we use the scale of heavy physics $\Lambda$:
%
\begin{equation}
  \lgr{EFT} = (\text{kinetic and mass terms}) + \sum_i \frac {f_i} {\Lambda^{D_i - d}} \, \ope{i} \,.
  \label{eq:foundations_EFT_lagrangian}
\end{equation}
%
Why do we force $\Lambda$ to appear in front of the operators like
this? If we do not know anything about the underlying model at scale
$\Lambda$, our best guess (which can be motivated with arguments based
on the renormalisation group flow) is that it consists of
dimensionless couplings $g \sim \ord{1}$ and mass scales
$M \sim \ord{\Lambda}$. Indirect effects mediated by this high-energy
physics should therefore be proportional to a combination of these
factors, as given in \autoref{eq:foundations_EFT_lagrangian} with
couplings $f_i \sim \ord{1}$. This is certainly true in Fermi theory,
where the effective coupling $G_F$ is suppressed by
$\Lambda^2 = m_W^2$.


  
% %%%%%%%%%%%%%%%%%%%%%%%%%%%%%%%%%%%%%%%%%%%%%%%%%%%%%%%%%%%%
% \subsubsection{Full and effective descriptions of physics}
% %%%%%%%%%%%%%%%%%%%%%%%%%%%%%%%%%%%%%%%%%%%%%%%%%%%%%%%%%%%%

% Let us go back to the simple picture of one full and one effective
% theory and summarise the typical differences between the two setups.
% %
% \begin{itemize}
% \item The full model contains high-energy physics, for instance heavy
%   particles with mass $\gtrsim \Lambda$ that are not dynamical degrees
%   of freedom in the EFT. In the effective model their effects are
%   mapped onto additional higher-dimensional operators involving only
%   the light fields.
% %
% \item At least in our simple picture we assume that the full model is
%   valid at all energies. The EFT, in any case, is only valid at
%   $E \ll \Lambda$. Only in this low-energy region the two descriptions
%   agree, at $E \gtrsim \Lambda$ the EFT predictions will not reproduce
%   the full model.
% %
% \item The full model is fully renormalisable, while the EFT is
%   typically only renormalisable order by order.
% %
% \item An interaction mediated by heavy fields in the full model is
%   described by the higher-dimensional operators in the EFT, see for
%   instance Esq.~\eqref{eq:foundations_4fermion_full} and
%   \eqref{eq:foundations_4fermion_EFT}. This means that the non-local
%   interaction in the full model is approximated as a local contact
%   interaction in the EFT.
% \end{itemize}



%%%%%%%%%%%%%%%%%%%%%%%%%%%%%%%%%%%%%%%%%%%%%%%%%%%%%%%%%%%%
\subsubsection{Ingredients}
%%%%%%%%%%%%%%%%%%%%%%%%%%%%%%%%%%%%%%%%%%%%%%%%%%%%%%%%%%%%

How the operators $\ope{i}$ look like might be clear in a top-down
situation where we know the underlying theory. In a bottom-up
approach, however, we need a recipe to construct a list of operators
in a model-independent way. It turns out that this is surprisingly
straightforward, and the list of operators we need to include in the
EFT is defined by three ingredients: the particle content, the
symmetries, and a counting scheme that decides which operators are
relevant at the scale of interest. We go through them one by one.

\begin{enumerate}
\item \emph{Particle content}: one has to define the fields that are
  the dynamical degrees of freedoms in the EFT, i.\,e.\ that can form
  either external legs or internal propagators in Feynman diagrams. At
  least all particles with masses $m \ll \Lambda$ should be
  included. The operators are then combinations of these fields and
  derivatives.
%
\item \emph{Symmetries}: some symmetry properties of the world have
  been measured with high precision, and we can expect that a
  violation of these symmetries has to be extremely small or happens
  at very high energies. These can be gauge symmetries (such as the
  $SU(3) \times SU(2) \times U(1)$ of the SM), space-time symmetries
  (such as Lorentz symmetry), or other global symmetries (such as
  flavour symmetries). Requiring that the effective operators do not
  violate these symmetries is well motivated and can reduce the
  complexity of the theory significantly.
%
\item \emph{Counting scheme}: with a set of particles and some
  symmetry requirements we can construct an infinite tower of
  different operators. We therefore need some rule to decide which of
  the operators we can neglect. Here the dimensionality of the
  operators becomes important. As argued above, we expect an operator
  with canonical dimension $D > d$ to be suppressed by a factor of roughly
  $1 / \Lambda^{D-d}$. Operators of higher mass dimension are
  therefore more strongly suppressed. Setting a maximal operator
  dimension is thus a way of limiting the EFT to a finite number of
  operators that should include the leading effects at energies
  $E \ll \Lambda$.
\end{enumerate}

One property that is often required of theories is missing in this
list: an EFT (with its intrinsic UV cutoff $\Lambda$) does not have to
be renormalisable in the traditional sense. In fact, most EFTs include
operators with mass dimension $D > d$ and are thus non-renormalisable.
However, EFTs are still renormalisable order by order in the counting
scheme, and loop effects can be calculated without any fundamental
issues. 



%%%%%%%%%%%%%%%%%%%%%%%%%%%%%%%%%%%%%%%%%%%%%%%%%%%%%%%%%%%%
\subsubsection{Basis choices}
%%%%%%%%%%%%%%%%%%%%%%%%%%%%%%%%%%%%%%%%%%%%%%%%%%%%%%%%%%%%

Usually not all operators that can be constructed in this way are
independent. This can be seen from a field redefinition of the form
%
\begin{equation}
  \phi(x) \to \phi'(x) = \phi(x) + \varepsilon f(x)
\end{equation}
%
where $\varepsilon$ is some small parameter and $f(x)$ can contain any
combination of fields evaluated at $x$. The action in terms of the new
field is then (after integration by parts)
%
\begin{equation}
  \intfourx \lgr{}[\phi] \to \intfourx \lgr{}[\phi']
  % &= \intfourx \left( \lgr{}[\phi] + \varepsilon \fder {\lgr{}} {\phi} f + \varepsilon \fder {\lgr{}} {\partial_\mu \phi} \partial_\mu f + \ord{\varepsilon^2} \right) \notag \\
  = \intfourx \left( \lgr{}[\phi] + \varepsilon \left[ \fder {\lgr{}} {\phi} - \partial_\mu \fder {\lgr{}} {\partial_\mu \phi} \right] f + \ord{\varepsilon^2} \right) \,.
  \label{eq:foundations_field_redefinitions}
\end{equation}
%
Such a transformation does not change the physics, i.\,e.\ the
$S$-matrix elements~\cite{Politzer:1980me, Georgi:1991ch, Arzt:1993gz,
  Simma:1993ky}, so we can equivalently use the new action instead of
the original one. In this way, each equation of motion provides us
with a degree of freedom to swap operators for a combination of other
operators. Similarly, Fierz identities and integration by parts can be
used to manipulate the form of operators. Together these tools reduce
the number of operators and coefficients necessary in an EFT basis,
and lead to some freedom to choose which operators to work with.



% %%%%%%%%%%%%%%%%%%%%%%%%%%%%%%%%%%%%%%%%%%%%%%%%%%%%%%%%%%%%
% \subsubsection{Fermi theory again}
% %%%%%%%%%%%%%%%%%%%%%%%%%%%%%%%%%%%%%%%%%%%%%%%%%%%%%%%%%%%%
% %
% As an example, let us pretend to not know anything about the SM, and
% construct an EFT of the weak interaction around or below a few GeV.
% %
% \begin{enumerate}
% \item Above $\Lambda_{QCD}$, the particle content is given by the
%   leptons and quarks, excluding the top. For a general EFT at these
%   energy scales we would have to include photons and gluons as well,
%   but for simplicity we leave them out here. For energies below
%   $\Lambda_{QCD}$ we should in principle write down a different EFT
%   based on baryons and mesons, but this does not really change the
%   result.
% \item The low-energy symmetries observed at these energies are Lorentz
%   invariance as well as the conservation of electromagnetic charge,
%   lepton number, and baryon number. Since we already leave out the
%   gluons, we pretend colour charges do not exist.
%   \footnote{Based on the experience with electromagnetism, and without
%     taking into account the measurement of $P$ and $C$ violation, one
%     might be tempted to also prescribe $P$ and $C$ invariance, which
%     would lead to the wrong EFT.}
%   % Interestingly, the weak interaction does not respect parity and
%   % charge conjugation invariance, an EFT based on these symmetries
%   % would fail.
% \item Finally, let us only keep the operators with the lowest mass
%   dimension (not counting kinetic and mass terms).
% \end{enumerate}

% The kinetic and mass terms for the fermions read
% %
% \begin{equation}
%   \lgr{} \supset \im  \bar{\psi}_i \gamma^\mu \partial_\mu \psi_i - m_i \bar{\psi}_i \psi_i \,.
%   \label{eq:foundations_Fermi_theory_kin}
% \end{equation}

% As before, we can calculate the mass dimension of all objects and find
% %
% \begin{equation}
%   [\psi] = \frac 3 2 \quad \text{and} \quad [\partial] = 1 \,.
% \end{equation}
% %
% Adding operators composed of two fermion fields only give us more
% kinetic and mass terms and not change anything. Operators with three
% fermion fields violate both fermion number conservation and Lorentz
% invariance. So the lowest-dimensional operators that we can write down
% include four fermion fields and no derivatives:
% %
% \begin{align}
%   \lgr{} &\supset \frac {f_{1 \, ijkl} } {\Lambda^2}  \left( \bar{\psi}_i \psi_j \right) \left( \bar{\psi}_k \psi_l \right)
%                      + \frac {f_{2 \, ijkl} } {\Lambda^2}  \left( \bar{\psi}_i  \gamma_5 \psi_j \right) \left( \bar{\psi}_k \psi_l \right) \notag \\
%            &\quad + \frac {f_{3 \, ijkl} } {\Lambda^2}  \left( \bar{\psi}_i  \gamma_5 \psi_j \right) \left( \bar{\psi}_k \gamma_5 \psi_l \right)
%                      + \frac {f_{4 \, ijkl} } {\Lambda^2}  \left( \bar{\psi}_i  \gamma_\mu \psi_j \right) \left( \bar{\psi}_k \gamma^\mu \psi_l \right) \notag \\
%           &\quad  + \frac {f_{5 \, ijkl} } {\Lambda^2}  \left( \bar{\psi}_i  \gamma_5 \gamma_\mu \psi_j \right) \left( \bar{\psi}_k \gamma^\mu \psi_l \right)
%                      + \frac {f_{6 \, ijkl} } {\Lambda^2}  \left( \bar{\psi}_i  \gamma_5 \gamma_\mu \psi_j \right) \left( \bar{\psi}_k \gamma_5 \gamma^\mu \psi_l \right) \notag \\
%          &\quad  + \frac {f_{7 \, ijkl} } {\Lambda^2}  \left( \bar{\psi}_i  \gamma_\mu \gamma_\nu \psi_j \right) \left( \bar{\psi}_k \gamma^\mu \gamma^\nu \psi_l \right) \,,
% \end{align}
% %
% Here some entries of the Wilson coefficient matrices $f_1$ to $f_7$
% have to be zero to conserve lepton and baryon number. We will drop
% these flavour indices from now on.

% In this bottom-up approach, all remaining coefficients are free
% parameters and have to be determined by experiment.
% % A priori we also do not know the value of $\Lambda$, only that it
% % should be significantly separated from the light quark and lepton
% % masses for our EFT to make sense.
% With the measurement of the muon lifetime, beta decay, and parity
% violation it turns out that the $f_5$ and $f_6$ coefficients are equal
% and of opposite sign, while the others are zero (ignoring $Z$ and $H$
% interactions):
% %
% \begin{align}
%   \lgr{} &=  \im  \bar{\psi}_i \gamma^\mu \partial_\mu \psi_i - m_i \bar{\psi}_i \psi_i 
%            + \frac {c } {\Lambda^2}  \left( \bar{\psi}_i  (1 - \gamma_5) \gamma_\mu \psi_j \right) \left( \bar{\psi}_k (1 - \gamma_5) \gamma^\mu \psi_l \right) \,.
% \end{align}
% %
% This is exactly Fermi theory, with $G_F = \sqrt{2} c / \Lambda^2 = 1.16 \cdot 10^{-5}$~GeV$^{-2}$.

% The dimension-six operators in this theory are not renormalizable, so
% Fermi theory cannot be valid at arbitrary large energies. But even
% knowing the Wilson coefficients, one cannot determine the scale
% $\Lambda$ where the EFT breaks down. By postulating that the
% underlying theory is perturbative, $c \lesssim 4 \pi$, one can set an
% upper limit $\Lambda \lesssim \sqrt{4 \pi / G_F} \approx 1040$~GeV,
% much larger than the observed $\Lambda = m_W = 80$~GeV.


% \newparagraph
% %
% Finally, here is one last simple example for how the EFT framework
% lets us estimate physics effects even when we do not know the full
% theory. It is taken from Reference~\cite{Kaplan:2005es}.



%%%%%%%%%%%%%%%%%%%%%%%%%%%%%%%%%%%%%%%%%%%%%%%%%%%%%%%%%%%%
\subsubsection{Why is the sky blue?}
%%%%%%%%%%%%%%%%%%%%%%%%%%%%%%%%%%%%%%%%%%%%%%%%%%%%%%%%%%%%

Following Reference~\cite{Kaplan:2005es}, we demonstrate this
bottom-up approach with a simple question: why is the sky blue?
In other words, why is blue light coming from the sun scattered more
strongly by particles in the atmosphere than red light?  A full
derivation of this takes some time and requires knowledge of the
underlying electrodynamic interactions. Instead, we write down an
effective field theory for this process of Rayleigh scattering. The
only thing we have to know are the basic scales of the process:
photons with energy $E_\gamma$ scatter off basically static nuclei
characterised by an excitation energy $\Delta E$, mass $M$ and radius
$a_0$. Looking at these numbers, we see that these scales are clearly
separated:
%
\begin{equation}
  E_\gamma \ll \Delta E,  a_0^{-1} \ll M \,.
\end{equation}
%
This is good news, since such a scale hierarchy is the basic
requirement for an EFT. We are interested in elastic scattering, so we
set the cutoff of the EFT as\footnote{In reality there are two orders
  of magnitude between $\Delta E$ and $a_0^{-1}$, but this does not
  affect the line of argument at all and we choose to ignore this
  fact.}
\begin{equation}
  \Lambda \sim \Delta E, a_0^{-1} \,.
\end{equation}

With this we can put together the building blocks for our EFT as
discussed above:
%
\begin{enumerate}
\item As fields we need photons and atoms, where we can
  approximate the latter as infinitely heavy.
  %
\item The relevant symmetries are the $U(1)_{\text{em}}$ and Lorentz
  invariance. At these energies atoms can also not be created or
  destroyed, which one can see as another symmetry requirement on the
  effective Lagrangian.
  %
\item We include the lowest-dimensional operators that describe
  photon-atom scattering.
\end{enumerate}

The kinetic part of such an EFT reads
%
\begin{equation}
  \lgr{kin} = \phi_v^\dagger \im v^\alpha  \partial_\alpha \phi_v - \frac 1 4 F_{\mu \nu} F^{\mu \nu} \,,
\end{equation}
%
where $\phi_v$ is the field operator representing an infinitely heavy
atom at constant velocity $v$, and $F_{\mu \nu}$ is the photon field
strength tensor. Boosting into the atom's rest frame, $v = (1,0,0,0)$
and the first term becomes the Lagrangian of the Schr\"odinger
equation.

The usual power counting based on $[\lgr{}]= 4$ gives the mass
dimensions
%
\begin{equation}
  [\partial] = 1 \,, \quad [v] = 0 \,, \quad [\phi] = \frac 3 2 \quad \text{and} \quad [F_{\mu \nu} ] = 2 \,.
\end{equation} 

The interaction operators must be Lorentz-invariant combinations of
$\phi^\dagger \phi$, $F_{\mu\nu}$, $v_\mu$, and $\partial_\mu$. Note
that operators directly involving $A_\mu$ instead of $F_{\mu \nu}$ are
forbidden by gauge invariance, and single instances of $\phi$
correspond to the creation or annihilation of atoms, which is not
possible at these energies. The first such operators appear at mass
dimension 7:
%
\begin{equation}
  \lgr{int} = \frac {f_1} {\Lambda^3} \phi_v^\dagger \phi_v F_{\mu \nu} F^{\mu \nu} 
  + \frac {f_2} {\Lambda^3} \phi_v^\dagger \phi_v v^\alpha F_{\alpha \mu} v_{\beta} F^{\beta \mu} 
  % + \frac {f_3} {\Lambda^3} \phi_v^\dagger \phi_v v^\alpha \partial_\alpha (F_{\mu \nu} F^{\mu \nu})
  + \ord{1/\Lambda^4} \,,
\end{equation}
%
with Wilson coefficients $f_1$ and $f_2$. These two operators should
capture the dominant effects of Rayleigh scattering at energies
$E_\gamma \ll \Lambda$.

The scattering amplitude of light off the atmospheric atoms should
therefore scale as $\mathcal{M} \sim 1 / \Lambda^3$, which means that
the cross section scales with $\sigma \sim 1 / \Lambda^6$. Since the
cross section has the dimension of an area, $[\sigma] = -2$, and the
only other mass scale in this low-energy process is the photon energy
$E_\gamma$, we know that the effective cross section must be
proportional to
%
\begin{equation}
  \sigma \propto \frac {E_\gamma^4} {\Lambda^6} \left( 1 + \ord{E_\gamma / \Lambda} \right) \,.
\end{equation}
%
In other words, blue light is much more strongly scattered than red
light. Our effective theory, built just from a few simple assumptions,
explains the colour of the sky!

Finally, we should check the validity range of our EFT. We expect it
to work as long as
%
\begin{equation}
  E_\gamma \ll \Lambda \sim \Delta E \sim \ord{\ev} \,,
\end{equation}
%
equivalent to wavelengths above $\ord{ 100 \ \text{nm}}$. Our
approximation is probably safe for visible light. In the near
ultraviolet we expect deviations from the $E_\gamma^4$ proportionality
and the EFT to lose its validity.





%%%%%%%%%%%%%%%%%%%%%%%%%%%%%%%%%%%%%%%%%%%%%%%%%%%%%%%%%%%%
\subsection{Top-down approach and matching}
\label{sec:foundations_matching}
%%%%%%%%%%%%%%%%%%%%%%%%%%%%%%%%%%%%%%%%%%%%%%%%%%%%%%%%%%%%

In the top-down approach to effective field theories, we start from a
known model of UV physics and calculate the corresponding effective
operators and Wilson coefficients in the EFT. The defining criterion
of this \emph{matching procedure} is that at low energies the
effective and underlying descriptions agree, at least up to a given
order in the loop expansion (e.\,g.\ in $\alpha_s$) and up to a given
order in the EFT expansion in $1/\Lambda$.

This can be achieved either by functional methods or with Feynman
diagrams. Here we sketch the conceptual foundation involving
functional methods, before arriving at a simple diagrammatic
method. Note that the matching cannot be reversed: one cannot uniquely
reconstruct a full theory only based on the EFT. Details of the
matching procedure play a crucial role in
\autoref{chapter:validity}.



%%%%%%%%%%%%%%%%%%%%%%%%%%%%%%%%%%%%%%%%%%%%%%%%%%%%%%%%%%%%
\subsubsection{The effective action}
%%%%%%%%%%%%%%%%%%%%%%%%%%%%%%%%%%%%%%%%%%%%%%%%%%%%%%%%%%%%

The central object that allows us to systematically analyse the
low-energy effects of heavy physics is the effective action
$\seff$. Following Reference~\cite{Gaillard:1986dz, Henning:2014wua}, we
now outline its calculation at the one-loop level. Note that this is
just a conceptual sketch and not mathematically rigorous, and that we
omit higher-order terms irrelevant for this thesis as well as certain
cases of mixed loops with light and heavy
particles~\cite{Henning:2016lyp}. For a more thorough derivation see
the quantum field theory textbook of your choice.

For simplicity, let us assume that our theory $S[\phi, \Phi]$ consists
of light particles $\phi$ and a heavy scalar $\Phi$ that should not be
part of the effective theory as dynamical degree of freedom. The
effective action is calculated by \emph{integrating out} the heavy
particles from the partition function,
%
\begin{equation}
  e^{\im \seff [\phi]} = \int \! \mathcal{D} \Phi \,  e^{\im S[\phi, \Phi]} \,.
  \label{eq:effective_action_definition}
\end{equation}
%
While the path integral over the heavy fields is computed, the light
fields are kept fixed as ``background fields''.

The effective action can be calculated with a saddle-point
approximation. For this we expand $\Phi$ around its classical value
$\Phi_c$:
%
\begin{equation}
  \Phi(x) = \Phi_c(x) + \eta(x) \,.
\end{equation}
%
$\Phi_c$ is defined by the classical equation of motion
%
\begin{equation}
  \left. \frac {\delta S[\phi,\Phi]} {\delta \Phi} \right|_{\Phi = \Phi_c} = 0 \,,
  \label{eq:classical_eom}
\end{equation}
%
so expanding the action around this extremum leads to
%
\begin{equation}
  S[\phi, \Phi_c + \eta] = S[\phi,\Phi_c] + \frac 1 2 \left. \frac {\delta^2 S[\phi,\Phi]} {\delta \Phi^2} \right|_{\Phi = \Phi_c} \eta^2 + \ord{\eta^3} \,.
\end{equation}
%
Plugging this into \autoref{eq:effective_action_definition}, we find
%
\begin{equation}
  e^{\im \seff [\phi]} \approx e^{\im S[\phi,\Phi_c]} \, \int \! \mathcal{D} \eta \,  \exp \left( \frac 1 2 \left. \frac {\delta^2 S[\phi,\Phi]} {\delta \Phi^2} \right|_{\Phi = \Phi_c} \eta^2 \right) \,.
\end{equation}
%
The last term is a Gaussian integral with a known solution,
%
\begin{equation}
  e^{\im \seff [\phi]} \approx e^{\im S[\phi,\Phi_c]} \, \left[ \det \left( - \left. \frac {\delta^2 S} {\delta \Phi^2} \right|_{\Phi = \Phi_c} \right) \right]^{-1/2} 
\end{equation}
%
and finally
%
\begin{equation}
  \seff [\phi] \approx S[\phi,\Phi_c] + \frac \im 2 \tr \log \left( - \left. \frac {\delta^2 S} {\delta \Phi^2} \right|_{\Phi = \Phi_c} \right) \,,
  \label{eq:effective_action_result}
\end{equation}
%
where the functional trace is defined as an integral over momentum
space $k$ together with a sum over internal states $i$ such as spin or
flavour,
%
\begin{equation}
  \tr x \equiv \sum_{i} \intfourk \braket {k, i | x | k, i} \,.
  \label{eq:functional_trace}
\end{equation}
%
% \footnote{If we would have considered fermions instead, there would
% be a minus sign in front of the second term.}

This result can be directly evaluated with functional methods. The
first term in \autoref{eq:effective_action_result} can be easily
calculated by solving the classical equations of motion in
\autoref{eq:classical_eom}. Computing the functional trace is more
involved, but can be simplified with a procedure called
\emph{covariant derivative expansion}~\cite{Gaillard:1985uh,
  Gaillard:1986dz, Cheyette:1987qz}. Universal results that can be
adapted to many scenarios are available in the
literature~\cite{Henning:2014wua, Drozd:2015rsp, Henning:2016lyp}.

The effective action is in general non-local, visible as (covariant)
derivatives $D$ appearing in the denominator (formally defined as
Green's functions). We expand these terms schematically as
%
\begin{equation}
    \phi^\dagger \frac 1 {D^2 - M^2} \phi = - \phi^\dagger \frac 1 {M^2}  \left[ 1 + \frac {D^2} {M^2} \right] \phi + \ord{1 / M^6}\,,
\end{equation}
%
so that only a rest term of higher order in $1/\Lambda = 1/M$ remains
non-local~\cite{Henning:2016lyp}. In a last step, we truncate the
resulting tower of operators at some order in our counting scheme, in
our case in the expansion in $1/\Lambda$. The resulting effective
theory consists of a finite set of local operators up to some order in
a counting scheme, compatible with our definition of effective
theories in the previous section. Unlike in the bottom-up approach,
not all operators have to appear, and we can calculate the Wilson
coefficients based on the underlying theory. 


  
%%%%%%%%%%%%%%%%%%%%%%%%%%%%%%%%%%%%%%%%%%%%%%%%%%%%%%%%%%%%
\subsubsection{Scalar example}
%%%%%%%%%%%%%%%%%%%%%%%%%%%%%%%%%%%%%%%%%%%%%%%%%%%%%%%%%%%%

As a simple example consider a theory of two real scalar fields. The
light field $\phi$ has mass $m$, the heavy field $\Phi$ with mass $M$
is being integrated out. The underlying theory is given by
%
\begin{multline}
  S[\phi,\Phi] = \intfourx \Biggl[
    \frac 1 2 \partial_\mu \phi \partial^\mu \phi
    - \frac {m^2} 2 \phi^2
    + \frac 1 2 \partial_\mu \Phi \partial^\mu \Phi
    - \frac {M^2} 2 \Phi^2 \\
    - \frac {\lambda_0} {4!} \phi^4
    - \frac {\lambda_2} {4} \phi^2 \Phi^2
    - \frac {\lambda_4} {4!} \Phi^4
    \Biggr] \,.
\end{multline}
%
Odd interactions and a mixing term $\phi\Phi$  are forbidden with suitable $\mathbb{Z}_2$
symmetries.

The classical equation of motion for $\Phi$ is
%
\begin{equation}
  \left( \partial^2 + M^2 + \frac {\lambda_2} 2 \phi^2 + \frac {\lambda_4} {3!} \Phi_c^2 \right) \Phi_c = 0
\end{equation}
%
with the trivial solution $\Phi_c = 0$.

The first term in the effective action then just gives back $\phi^4$
theory for the light field, without any new effective interactions:
%
\begin{equation}
  S[\phi,\Phi_c] = \intfourx \left[\frac 1 2 \partial_\mu \phi \partial^\mu \phi - \frac {m^2} 2 \phi^2 - \frac {\lambda_0} {4!} \phi^4 \right] \,.
  \label{eq:foundations_scalar_example_effective_action_tree_part}
\end{equation}
%
The second term is
%
\begin{align}
  \frac \im 2 \tr \log \left( - \left. \frac {\delta^2 S} {\delta \Phi^2} \right|_{\Phi = \Phi_c} \right) 
  %
  &=  \frac \im 2  \tr \log \left( \partial^2 + M^2 + \frac {\lambda_2} 2 \phi^2 \right) \notag \\
  %
  &=  \frac \im 2  \tr \log \left( \partial^2 + M^2 \right) + \frac \im 2  \tr \log \left( 1 + \frac {\lambda_2} 2 \frac {1} {\partial^2 + M^2 + \im \varepsilon} \phi^2 \right) \,,
  %
  % &=  \frac \im 2  \intfourk \braket {k| \log \left( \partial^2 + M^2 \right) | k} \notag \\
  % &\phantom{=} \quad + \frac \im 2  \intfourk \braket {k| \log \left( 1 + \frac {\lambda_2} 2 \frac {1} {\partial^2 + M^2 + \im \varepsilon} \phi^2 \right) | k} \,.
\end{align}
%
where derivatives in the denominator are defined as Green's
functions. Since $\tr \log \left( \partial^2 + M^2 \right) $ is just a
constant that can be calculated for instance in dimensional
regularisation, the first part does not give us any higher-dimensional
operators of the light fields $\phi$. Expanding the logarithm in the
second term, we find
%
\begin{equation}
  \seff \supset \frac {\im \lambda_2} 4  \tr \frac {1} {\partial^2 + M^2 - \im \varepsilon} \phi^2
  - \frac {\im \lambda_2^2} 8  \tr \left( \frac {1} {\partial^2 + M^2 - \im \varepsilon} \phi^2 \right)^2
  + \frac {\im \lambda_2^3} {12}  \tr \left( \frac {1} {\partial^2 + M^2 - \im \varepsilon} \phi^2\right)^3 
  + \ord{\lambda_2^4} \,.
  \label{eq:foundations_scalar_example_effective_action_powers}
\end{equation}
%
The first of these terms renormalises the $\phi$ mass term, and
the second contributes to the $\phi^4$ interaction. This is
important for RG running, but does not create the kind of new
effective interactions we are interested in here. We instead focus on
the last term and evaluate the functional trace:
%
\begin{align}
  \seff &\supset - \frac {\im \lambda_2^3} {12} \intfourk \braket {k| \left( \frac {1} {\partial^2 + M^2 - \im \varepsilon} \phi^2 \right)^3 |k} \notag \\
  %
  &\supset - \frac {\im \lambda_2^3} {12} \;
    \intfourx \!\! \intfoury \!\! \intfourz \!\!
    \intfourk \!\! \intfourp \!\! \intfourq 
    \braket {k| \frac {1} {\partial^2 + M^2 - \im \varepsilon} |x} \braket{x| \phi^2 |p}\notag \\
  &\phantom{\supset} \quad \quad
    \times \braket {p| \frac {1} {\partial^2 + M^2 - \im \varepsilon} |y} \braket{y| \phi^2 |q} 
    \braket {q| \frac {1} {\partial^2 + M^2 - \im \varepsilon} |z} \braket{z| \phi^2 |k} \,.
\end{align}
%
Here we have used the definition of the functional trace in
\autoref{eq:functional_trace} and inserted unity,
$1 = \intfourx \ket{x} \bra{x} = \intfourp \ket{p} \bra{p}$.
$\ket{k}$, $\ket{p}$, and $\ket{q}$ are eigenstates of the derivative
operator $\partial$,
i.\,e.~$\bra{k} \im \partial_\mu = \bra{k} k_\mu$, while $\ket{x}$,
$\ket{y}$, and $\ket{z}$ denote the eigenstates of local operators,
$\bra{x} \phi^2= \bra{x} \phi^2(x)$. Their inner product is
$\braket {x|k} = e^{-\im k x}$. Using these properties and shifting
the integration variables, we get
%
\begin{align}
  \seff &\supset - \frac {\im \lambda_2^3} {12} \;
          \intfourx \!\! \intfoury \!\! \intfourz \!\!
          \intfourk \!\! \intfourp \!\! \intfourq
          \frac {1} {-k^2 + M^2 - \im \varepsilon} e^{\im k x}  \phi(x)^2 e^{-\im p x} \notag \\
  &\phantom{\supset} \qquad
    \times \frac {1} {-p^2 + M^2 - \im \varepsilon} e^{\im p y}  \phi(y)^2 e^{-\im q y}  \;
    \frac {1} {-q^2 + M^2 - \im \varepsilon} e^{\im q z}  \phi(z)^2 e^{-\im k z} \notag \\
  %
  &\supset \frac {\im \lambda_2^3} {12} \;
    \intfourx \!\! \intfoury \!\! \intfourz \!\!
    \intfourk \!\! \intfourp \!\! \intfourq
    \phi(x)^2 \phi(y)^2 \phi(z)^2 \notag \\
  &\phantom{\supset} \qqquad 
    \times \frac {e^{\im p (z-x)} \, e^{\im q (z-y)}}
    { (k^2 - M^2 + \im \varepsilon) \, ((k+p)^2 - M^2 + \im \varepsilon) \,  ((k+p+q)^2 - M^2 + \im \varepsilon)} \,.
\end{align}

We can now perform the integral over the loop momentum $k$ with
Feynman parameters:
%
\begin{align}
  &\intfourk \frac {1}
              { (k^2 - M^2 + \im \varepsilon) \,
              ((k+p)^2 - M^2 + \im \varepsilon) \,
              ((k+p+q)^2 - M^2 + \im \varepsilon)} \notag \\
  %
   &\quad = 2 \int_0^1 \!\! \diff x_1 \int_0^{1-x_1} \!\! \diff x_2 \intfourk
     \Bigl[  x_1  (k^2 - M^2 + \im \varepsilon)
     + x_2 ((k+p)^2 - M^2 + \im \varepsilon) \notag \\
  &\quad \phantom{=} \qqqquad \qqquad
    + (1-x_1 -x_2)  ((k+p+q)^2 - M^2 + \im \varepsilon)  \Bigr]^{-3} \notag \\
  %
   &\quad = 2 \int_0^1 \!\! \diff x_1 \int_0^{1-x_1} \!\! \diff x_2 \intfourk
     \frac 1 { \left[ (k + a)^2 - B  + \im \varepsilon \right]^3 }
\end{align}
%
with $a = (1 - x_1) p  + (1 - x_1 - x_2) q $ and
$B = M^2 - (1 - x_1 - x_2) (p+q)^2 - x_2 p^2 + a^2$.
Shifting the loop momentum as $k \to k + a$, we finally arrive at
%
\begin{equation}
  T_3(p,q)  
   = 2 \int_0^1 \!\! \diff x_1 \int_0^{1-x_1} \!\! \diff x_2 \intfourk
     \frac 1 { \left[ k^2 - B + \im \varepsilon \right]^3 } \,.
  \label{eq:foundations_scalar_example_loop_function}
\end{equation}

To evaluate this, we first Wick-rotate $k^0 = \im k_E^0$. Formally,
this means shifting the integration path in the complex plane of $k^0$
from along the real axis to along the imaginary axis. The Cauchy
theorem assures that this does not change the value of the integral as
long as we chose the contour such that the poles are not caught
between the two contours. Defining
$k_E^2 = (k^0_E)^2 + \boldsymbol{k}^2 = - k^2$, we find
%
%\footnote{This
%  step requires $B>0$. In our case, $p$ and $q$ correspond to momenta
%  of the light fields, and in the validity regime of the EFT we should
%  always have $M^2 \gg p^2, q^2$ and therefore $B > 0$.}
%
\begin{equation}
  I_{0,3} 
   \equiv \intfourk \frac 1 { \left[ k^2 - B + \im \varepsilon \right]^3 } 
   = \im \intfourke \frac 1 { \left[ - k_E^2 - B \right]^3 } \,,
\end{equation}
%
where the $+ \im \varepsilon$ is no longer necessary. With
$\overbar{k} = |k_E |$ we can finally calculate the integral:
%
\begin{align}
  I_{0,3} = \frac {2 \pi^2} {(2 \pi)^4} \,
            \int \!\! \diff \overbar{k} {\overbar{k}}^3 \;
    \frac 1 { \left[ {\overbar{k}}^2 + B \right]^3 }
            = \frac {- \im } {32 \pi^2 B} \,.
\end{align}

Collecting all the pieces, we have
%
\begin{multline}
  \seff \supset \frac {\lambda_2^3} {192 \pi^2} \,
    \intfourx \!\! \intfoury \!\! \intfourz \phi(x)^2 \phi(y)^2 \phi(z)^2 
    \intfourp \!\! \intfourq e^{\im p (z-x)} \, e^{\im q (z-y)} \\
    \times \int_0^1 \!\! \diff x_1 \int_0^{1-x_1} \!\! \diff x_2 \;
    \left[ M^2 - (1 - x_1 - x_2) (p+q)^2 - x_2 p^2 + ((1-x_1) p  + (1 - x_1 - x_2) q )^2 \right]^{-1} \,.
\end{multline}
%
At first glance, this is disappointing: this effective action looks
non-local and involves highly non-trivial integrals. It turns out that
these can in fact be calculated and give a finite
result~\cite{tHooft:1978jhc, Denner:1991kt}. The full expression is
quite ugly, but we fortunately do not need it. Instead, we expand the
integrand in powers of $1/M^2$. We only calculate the leading term at
$\ord{1/M^2}$. Since it produces a finite result as well, the rest
term at $\ord{1/M^4}$ also has to be finite. Even more, we can argue
that the rest term at $\ord{1/M^4}$ has to vanish: the coefficient at
a given order $1/M^k$ in this expansion is an integral without any
mass scales, and has to lead to a result of mass dimension $k-2$. Only
$k = 2$ can give a non-zero and finite result, all higher orders
therefore have to vanish.\comment{Something is fishy. Check this
  argument, and check discrepancy with Denner's habil!}
% Physically, this corresponds to approximating the loop momenta
% as lighter then the heavy mass scale, which makes sense in the
% validity region of the EFT.
In this way, we find the much simpler result
%
\begin{align}
  \seff &\supset \frac {\lambda_2^3} {192 \pi^2 \,M^2 } 
    \intfourx \!\! \intfoury \!\! \intfourz 
    \phi(x)^2 \phi(y)^2 \phi(z)^2
    \intfourp \!\! \intfourq  e^{\im p (z-x)} e^{\im q (z-y)}  \notag\\
        &\phantom{\supset} \qqqquad \qqqquad \qqqquad + \ord{1/M^4} \notag \\
  %
  &\supset \frac {\lambda_2^3} {192 \pi^2 M^2 } 
    \intfourx \!\! \intfoury \!\! \intfourz
    \phi(x)^2 \phi(y)^2 \phi(z)^2 \,
    \delta (z-x) 
    \delta (z-y) \notag \\
  %
  &\supset \frac {\lambda_2^3} {192 \pi^2 M^2 } 
    \intfourx
    \phi(x)^6 \,.
    \label{eq:foundations_scalar_example_effective_action_loop_part}
\end{align}
%
After the expansion in $1/M$, we have finally arrived at a local
theory!

What about the higher terns in
\autoref{eq:foundations_scalar_example_effective_action_powers}?
Their calculation is analogous to the one presented here and leads
to operators like $\phi^8$ and higher. They are suppressed at
least with $ 1 / M^4$ and are thus irrelevant for our dimension-six
effective theory.

Collecting the pieces in
\autoref{eq:foundations_scalar_example_effective_action_tree_part}
and
\autoref{eq:foundations_scalar_example_effective_action_loop_part},
up to one loop and $\ord{1/M^2}$ the full effective action is given by
%
\begin{equation}
  \seff[\phi] = \intfourx \left[\frac 1 2 \partial_\mu \phi \partial^\mu \phi
    - \frac {m^2} 2 \phi^2 - \frac {\lambda_0} {4!} \phi^4
    + \frac {\lambda_2^3} {12 (4 \pi)^2 M^2 } \, \phi^6\right] \,.
\end{equation}
%
As expected, the dimension-six operator is suppressed by two powers of
the heavy scale $\Lambda \equiv M$, and the Wilson coefficient
consists of the couplings $\lambda_2^3$ times a loop factor.

  

%%%%%%%%%%%%%%%%%%%%%%%%%%%%%%%%%%%%%%%%%%%%%%%%%%%%%%%%%%%%
\subsubsection{Diagrammatic matching}
%%%%%%%%%%%%%%%%%%%%%%%%%%%%%%%%%%%%%%%%%%%%%%%%%%%%%%%%%%%%

As an alternative to this functional approach, the effective action in
\autoref{eq:effective_action_result} can be calculated in an
intuitive diagrammatic way.  Since the light fields are kept fixed in
\autoref{eq:effective_action_definition}, the effective action is
given by all connected Feynman diagrams with only $\phi$ as external
legs and only $\Phi$ fields as internal propagators. A more rigorous
derivation than the one in the precious section in fact reveals that
also certain connected loop diagrams with only $\phi$ as external legs
and both $\Phi$ and $\phi$ fields as internal propagators contribute
if they cannot be disconnected by cutting a single internal $\phi$
line~\cite{Henning:2016lyp}. The first term in
\autoref{eq:effective_action_result} corresponds to all such
tree-level diagrams, the second term describes one-loop
pieces. Higher-loop corrections play no role in this thesis and
were left out.

In practice, the effective operators and their Wilson coefficients can
be calculated without the need for any functional methods as follows:
%
\begin{enumerate}
\item Start with the particle content of the full model. Choose
  $\Lambda$ and divide the particles of the full model into light and
  heavy fields. Light fields, which should include at least those with
  masses below $\Lambda$, make up the particle content of the
  effective theory. Heavy fields are integrated out, \ie removed as
  dynamical degrees of freedom in the EFT.
%
\item Based on the particles and interactions of the full model, draw
  all connected Feynman diagrams that satisfy two conditions:
%
  \begin{itemize}
    \item all external legs are light fields; and
    \item the diagram cannot be disconnected by cutting a single
      internal light-field line. For tree-level diagrams this is
      equivalent to requiring that only heavy fields appear as
      internal lines.
  \end{itemize}
%
  Using the Feynman rules of the full model, calculate the expressions
  for these diagrams. Do not treat the external legs as incoming or
  outgoing particles, but keep the field operator expressions.
%
\item Express quantities of the full model in terms of
  $\Lambda$. Truncate this infinite series of diagrams at some order
  in $1/\Lambda$, depending on the dimension of the operators that you
  want to keep. Together with kinetic and mass terms for the light
  fields, these form the Lagrangian of the EFT.
\end{enumerate}



  
%%%%%%%%%%%%%%%%%%%%%%%%%%%%%%%%%%%%%%%%%%%%%%%%%%%%%%%%%%%%
\subsubsection{Fermi theory again}
%%%%%%%%%%%%%%%%%%%%%%%%%%%%%%%%%%%%%%%%%%%%%%%%%%%%%%%%%%%%

Let us apply this top-down procedure to our standard example of Fermi
theory. For simplicity, we do not take the full SM, but just the
interactions between massive $W$ bosons and fermions as the underlying
theory. The Lagrangian of these interactions is similar to that given
in \autoref{eq:foundations_action_V_fermions}.
%
\begin{enumerate}
\item Our full model consists of the quarks and leptons and the $W$
  boson. We want to analyse weak interactions below the $W$ mass, so
  we set $\Lambda = m_W$. The light particles of the EFT thus consist
  of the quarks and leptons except for the top, while the $W$ boson
  and the top quark are heavy and are integrated out.
    %
\item The only diagram with the requested features that has only one
  heavy propagator has the form
  %
    \begin{equation}
      \raisebox{-0.5\height}{
       %\fbox{
        \fmfframe(0,10)(0,10){ %(L,T) (R,B)
          \begin{fmfgraph*}(70,50)
            \fmfleft{i2,i1}
            \fmfright{o2,o1}
            \fmflabel{\small $\psi_j$}{i1}
            \fmflabel{\small $\bar{\psi}_i$}{i2}
            \fmflabel{\small $\psi_l$}{o1}
            \fmflabel{\small $\bar{\psi}_k$}{o2}
            \fmf{fermion}{i1,v1,i2}
            \fmf{fermion}{o1,v2,o2}
            \fmf{dbl_wiggly,tension=1,label=$W$}{v1,v2}
            \marrow{m}{up}{top}{$p$}{v1,v2}
          \end{fmfgraph*}
        }
       %}
      }
    \end{equation}
    % 
    Double lines denote a heavy field. There are additional diagrams
    with $W$ self-interactions or $W$ loops, but they involve at least
    two $W$ propagators, which means that all contributions from them
    are of order $\ord{1/\Lambda^4}$, which we neglect.
  
    Applying the SM Feynman rules, this diagram evaluates to
    % 
    \begin{align}
       &\quad  \left( \bar{\psi}_i \frac{\im g} {\sqrt{2}}  \frac {1 - \gamma_5} 2 \gamma^\mu \psi_j \right)  \frac {- g_{\mu \nu}} {p^2 - m_W^2}  \left( \bar{\psi}_k \frac{\im g} {\sqrt{2}} \frac {1 - \gamma_5} 2  \gamma^\nu \psi_l \right) \notag \\
      {} &= \frac { g^2 \left( \bar{\psi}_i (1 - \gamma_5) \gamma^\mu \psi_j \right)  \left( \bar{\psi}_k (1 - \gamma_5)  \gamma_\mu \psi_l \right) }  {8 (p^2 - m_W^2)} \,.
    \end{align}
    %
  \item The only dimensionful parameter is $m_W = \Lambda$, and for
    the EFT to be valid we assume $p^2 \ll \Lambda^2$. We can then
    expand this expression as
    %
    \begin{equation}
       \frac { g^2 } {8 m_W^2}  \left( \bar{\psi}_i (1 - \gamma_5) \gamma^\mu \psi_j \right)  \left( \bar{\psi}_k (1 - \gamma_5)  \gamma_\mu \psi_l \right) + \ord{1 / \Lambda^4} \,.
    \end{equation}
    %
    With this, we again rediscover the dimension-six EFT matched to the
    weak interactions of the SM:
    %
   \begin{equation}
     \lgr{} =  \im  \bar{\psi}_i \gamma^\mu \partial_\mu \psi_i - m_i \bar{\psi}_i \psi_i 
     + \frac {c } {\Lambda^2}  \left( \bar{\psi}_i  (1 - \gamma_5) \gamma_\mu \psi_j \right) \left( \bar{\psi}_k (1 - \gamma_5) \gamma^\mu \psi_l \right) \,,
     \label{eq:foundations_fermi_theory}
   \end{equation}
   %
   with heavy scale $\Lambda = m_W$ and Wilson coefficient
   $c = g^2 / 8$. Replacing $c / \Lambda^2$ by
   $G_F / \sqrt{2} = g^2 / (8 m_W^2)$ restores the historic form of
   Fermi theory.
\end{enumerate}





%%%%%%%%%%%%%%%%%%%%%%%%%%%%%%%%%%%%%%%%%%%%%%%%%%%%%%%%%%%%
\subsubsection{Operator mixing}
%%%%%%%%%%%%%%%%%%%%%%%%%%%%%%%%%%%%%%%%%%%%%%%%%%%%%%%%%%%%

So far we have neglected that like all parameters in a QFT, the Wilson
coefficients of an EFT depend on the energy scale. Running the model
from one energy to a different one leads to \emph{operator mixing}:
loop effects from one operator affect the coefficients of other
operators. If the Wilson coefficients are given at the matching scale
$\Lambda$ (we use this symbol since the matching scale is usually
chosen only slightly below the EFT cutoff), at the scale of interest
$E$ they take on values of the form
%
\begin{equation}
  f_i (E) \sim f_i(\Lambda) \pm \sum_j \frac {g^2} {16 \pi^2} \, \log \frac {\Lambda^2} {E^2} \,f_j(\Lambda) \,,
  \label{eq:foundations_EFT_running}
\end{equation}
%
where $g$ are the typical couplings in the loops.

If the matching scale is not too far away from the energy scale of
interest and if all Wilson coefficients are already sizeable at the
matching scale, this is often negligible. There is an important
consequence, though: even if an operator is zero at the matching
scale, operator mixing will give it a small but non-zero value at
lower energies. So regardless of what the underlying model is, it can
be expected that eventually all effective operators allowed by the
symmetries receive contributions from it. 



%%%%%%%%%%%%%%%%%%%%%%%%%%%%%%%%%%%%%%%%%%%%%%%%%%%%%%%%%%%%
\section{Dimension-six Higgs physics}
\label{sec:foundations_higgs_eft}
%%%%%%%%%%%%%%%%%%%%%%%%%%%%%%%%%%%%%%%%%%%%%%%%%%%%%%%%%%%%

We now apply these general ideas to electroweak and in particular
Higgs physics at the TeV scale and construct the Standard Model
effective field theory (interchangeably called linear Higgs effective
field theory) up to dimension six~\cite{Burges:1983zg, Leung:1984ni,
  Buchmuller:1985jz, Arzt:1994gp}. This is the framework we use
throughout this thesis. We first argue why such an effective
theory is very useful, and then construct its effective operators
following the recipe laid out in
\autoref{sec:foundations_eft_bottom_up}. \autoref{sec:foundations_heft_pheno}
takes a closer look at the phenomenology of these
operators. Finally, in \autoref{sec:foundations_heft_alternatives}
we briefly discuss a few alternative frameworks.


  

%%%%%%%%%%%%%%%%%%%%%%%%%%%%%%%%%%%%%%%%%%%%%%%%%%%%%%%%%%%%
\subsection{Motivation}
\label{sec:foundations_heft_motivation}
%%%%%%%%%%%%%%%%%%%%%%%%%%%%%%%%%%%%%%%%%%%%%%%%%%%%%%%%%%%%


% The Higgs boson~\cite{Higgs:1964ia, Higgs:1964pj, Englert:1964et}
% discovery announced on July 4th 2012~\cite{Aad:2012tfa,
% Chatrchyan:2012xdj} is a historical milestone in the physics of the
% 21st century.  The thorough scrutiny of the LHC Run I data has so
% far confirmed that the narrow resonance observed at a mass around
% 125~GeV is compatible with the minimal Standard Model (SM) agent of
% electroweak symmetry breaking~\cite{Plehn:2009nd}. To date, this
% agreement is limited to around $20\%$ precision in the Higgs
% couplings~\cite{Lafaye:2009vr, Corbett:2012ja, Klute:2012pu,
% Dobrescu:2012td, Cheung:2013kla, Giardino:2013bma, Chang:2013cia,
% Djouadi:2013qya, Corbett:2015ksa}, which is not sensitive to the
% deviations that one would expect from typical perturbatively
% extended Higgs sectors.  This accuracy, based on a large set of
% on-shell and most recently off-shell Higgs
% measurements~\cite{Corbett:2015ksa}, will soon improve with data
% from Run~II.  Odds are high that the upcoming runs will shed light
% on a possible UV completion of the Standard
% Model~\cite{Englert:2014uua, Morrissey:2009tf}.

% Based on everything we know, such an underlying theory should be
% described by a gauge field theory. While the measurement of Higgs
% couplings from inclusive rates has been extremely successful at
% Run~I, it needs to be extended, for example to include kinematic
% distributions. For this purpose, Higgs effective field theories
% (EFT)~\cite{Weinberg:1980wa, Coleman:1969sm, Callan:1969sn,
% Burges:1983zg, Leung:1984ni, Buchmuller:1985jz} have become the
% \emph{koin\'e} for discussing the phenomenology of extended Higgs
% sectors.  In the effective field theory language, beyond the
% Standard Model (BSM) effects are described in terms of a Lagrangian
% with local operators of increasing mass dimension $d > 4$. Each of
% them includes a suppression by inverse powers of a new physics
% scale, which should be well separated from the experimentally
% accessible scale, in our case the electroweak scale,
% $\Lambda \gg v$.

% After the discovery of a light Higgs boson~\cite{higgs,discovery}, one
% of the key tasks of the LHC is to test if the observed particle indeed
% corresponds to the minimalist setup of the Higgs sector in the
% Standard Model. Because of the many intricacies of the electroweak
% sector of the Standard Model, it is not straightforward to define a
% theoretical framework which describes possible deviations in the Higgs
% sector. If we want to remain more general than testing specific
% models~\cite{bsmreview}, we can use an effective field theory ansatz.
% Here the Lagrangian is organized by the field or particle content, the
% symmetry structure, and the mass
% dimension~\cite{eftfoundations,eftorig,higgsreview}.

% Extensions of the SM Higgs sector involve new degrees of freedom with
% electroweak charges and\,/\,or color charges, coupled to or mixing
% with the SM-like Higgs boson. Hidden sectors coupled to the Higgs
% potential without any SM charge lead to non-standard Higgs
% decays. Since the Higgs potential is closely linked to the electroweak
% sector, any model that affects the SM gauge bosons will also affect
% Higgs physics. This way, a wide range of new physics models can be
% probed in Higgs signatures at the LHC, both in total rates and
% kinematic distributions. 

% At TeV energies find ourselves in the latter bottom-up
% situation. There has to be physics beyond the standard model, and it
% better be accessible by the LHC experiments, but we do not know what
% it is.  Higgs effective field theory is designed as a
% model-independent language that captures the effects of such new
% physics on electroweak-scale observables. Its minimal version consists
% of 59 dimension-six operators, some of which parametrize changes in
% kinematic structures in the interactions of Higgs and gauge bosons. A
% global fit to these operators works fine, especially if distributions
% are included. Concerns about the validity of this effective theory
% have to be taken seriously, but so far it seems that this language is
% the way to go as long as no new light particles are discovered.

As argued in \autoref{sec:foundations_relevance}, there are many
reasons to suspect new physics in the Higgs sector. Some of these
arguments, such as the hierarchy problem or the WIMP miracle of dark
matter, point towards BSM physics close to the electroweak scale or,
depending on the level of acceptable fine-tuning, up to a few
TeV. Unfortunately these (purely aesthetic) arguments do not tell us
how exactly such physics should look like.

This leaves us with a question highly relevant for upcoming ATLAS and
CMS analyses: what is the best language to discuss indirect signs of
new physics at the electroweak scale, in particular in the Higgs
sector? Which parametrisation of Higgs properties provides a good
interface between different experiments, and between experiment and
theory?

Directly interpreting measurements in complete models of new physics
is impractical: for $n_y \gg 1$ signatures and $m_a \gg 1$ models this
requires $m_a n_y$ limits to be derived.\footnote{The weird notation
  is necessary because the author cannot resist a stupid pun.} Also,
the parameter space of such models (think of the relatively simple
MSSM) can be huge, and many of their features do not matter at the
electroweak scale at all. It makes more sense to define an
intermediate framework that can be linked both to measurements and to
full theories, so only $n_y$ sets of limits plus $m_a$ translation
rules from complete theories to the intermediate language have to be
calculated. Such a framework should include all necessary physics, but
no phenomena irrelevant at this scale. This is exactly the defining
feature of an effective field theory.




%%%%%%%%%%%%%%%%%%%%%%%%%%%%%%%%%%%%%%%%%%%%%%%%%%%%%%%%%%%%
\subsection{Operators}
\label{sec:foundations_heft_operators}
%%%%%%%%%%%%%%%%%%%%%%%%%%%%%%%%%%%%%%%%%%%%%%%%%%%%%%%%%%%%

%%%%%%%%%%%%%%%%%%%%%%%%%%%%%%%%%%%%%%%%%%%%%%%%%%%%%%%%%%%%
\subsubsection{Building blocks}
%%%%%%%%%%%%%%%%%%%%%%%%%%%%%%%%%%%%%%%%%%%%%%%%%%%%%%%%%%%%

Since we do not know what physics lays beyond the SM, we have to
construct our EFT from a bottom-up perspective. As discussed above,
this means we have to write down all operators based on a set of
particles that are compatible with certain symmetries and are
important according to some counting scheme. Let us go through these
one by one:
%
\begin{enumerate}
\item As degrees of freedoms we use the SM fields. In particular, we
  assume the Higgs boson $h$ and the Goldstone bosons $w_i$ are
  combined in an $SU(2)_L$ doublet $\phi$ as in the SM, see
  \autoref{eq:foundations_sm_phi}. This is consistent with
  data and presents the correct choice if new physics decouples, that is, if in
  the limit $\Lambda \to \infty$ the SM is
  recovered~\cite{Krause:2016uhw}. We discuss an alternative
  construction based on the physical scalar $h$ instead of $\phi$ as
  the fundamental building block in
  \autoref{sec:foundations_heft_alternatives}.
  %
\item All operators have to be invariant under Lorentz transformations
  and under the SM gauge group $SU(3)_C \times SU(2)_L \times U(1)_Y$,
  and conserve lepton and baryon number.
  %
\item We order the operators by their mass dimension and thus their
  suppression in powers of $1/\Lambda$. We keep those up to mass
  dimension 6, i.\,e.~$\ord{1/\Lambda^2}$.
\end{enumerate}

Simple dimensional analysis of the kinetic terms of the SM fields
tells us the mass dimensions of all building blocks:
%
\begin{equation}
  [f] = \frac 3 2\,, \quad [V_\mu] = 1 \,, \quad [V_{\mu \nu}] = 2 \,, \quad
  [\phi] = 1 \,, \quad [\partial_\mu] = 1 \quad \text{and} \quad [D_\mu] = 1 \,.
\end{equation}

The only dimension-five operator that can be built from the SM fields
is the ``Weinberg operator''
$(\overbar{L}_L \tilde{\phi}^* )(\tilde{\phi}^\dagger L_L)$. It
generates a Majorana mass term for the neutrinos, violates lepton
number, and is entirely irrelevant for Higgs physics. The leading
effects in our EFT are expected to come from dimension-six operators:
%
\begin{equation}
  \lgr{EFT} = \lgr{SM} + \sum_i \frac {f_i} {\Lambda^2} \ope{i} + \ord{1/\Lambda^4}
  \label{eq:sm_eft}
\end{equation}
%
with the unknown cutoff scale $\Lambda$ and Wilson coefficients
$f_i$. For convenience, from now on we drop the higher-order terms.

As discussed in \autoref{sec:foundations_eft_bottom_up}, field
redefinitions (or, relatedly, equations of motion), Fierz identities
and integration by parts provide equivalence relations between certain
operators and give us some freedom to define a basis of
operators. Taking these into account, there are 59 independent types
of dimension-six operators, not counting flavour structures and
Hermitian conjugation~\cite{Grzadkowski:2010es}. Counting all possible
flavour structures, there are 2499 distinct operators. Fortunately, in
practice only a small subset of these are relevant: first, the strong
constraints on flavour-changing neutral currents motivate the
assumption of flavour-diagonal or even flavour-universal Wilson
coefficients. Second, only a small number of these operators directly
affects Higgs physics. At higher orders in the EFT expansion, the
number of operators increases rapidly, explaining why we stick to the
leading effects at dimension 6: not counting flavour structures, there
are $\ord{10^3}$ operators at dimension eight and $\ord{10^4}$
dimension-ten operators~\cite{Henning:2015alf}.

Three different conventions have become popular: the complete
``Warsaw'' basis~\cite{Grzadkowski:2010es}, the ``Strongly Interacting
Light Higgs'' convention (SILH)~\cite{Giudice:2007fh, Contino:2013kra}
and the Hagiwara-Ishihara-Szalapski-Zeppenfeld
basis (HISZ)~\cite{Hagiwara:1993ck}. All three maximise the use of bosonic
operators to describe Higgs and electroweak observables. For a
comparison of and conversion between these bases see
Appendix~\ref{sec:appendix_bases} and
References~\cite{Falkowski:2015wza, Brehmer:2015rna}\comment{More
  references?}. Throughout this thesis we use the basis developed
in References~\cite{Corbett:2012ja, Juan_thesis, Tyler_thesis}, which is strongly
based on the HISZ basis and now widely used in global
fits~\cite{Corbett:2015ksa, Butter:2016cvz}.

We classify the operators based on their field content and on their
behaviour under $CP$ transformations. This combined charge conjugation
and parity inversion is an approximate symmetry of the SM that is only
violated by the complex phase of the CKM matrix. In addition, there
are rather tight bounds on $CP$ violation in many processes. This
motivates many analyses to restrict their set of operators to the
$CP$-conserving ones. On the other hand, new sources of $CP$ violation
are needed to explain the matter-antimatter asymmetry in the universe,
and their effects at low energies could be visible as $CP$-violating
effective operators. Both types of operators are analysed in this
thesis.



%%%%%%%%%%%%%%%%%%%%%%%%%%%%%%%%%%%%%%%%%%%%%%%%%%%%%%%%%%%%
\subsubsection{Operator basis}
%%%%%%%%%%%%%%%%%%%%%%%%%%%%%%%%%%%%%%%%%%%%%%%%%%%%%%%%%%%%

\begin{table}
  \renewcommand{\arraystretch}{1.8}
  \begin{tabular}{r @{${} = {}$} l @{\hspace*{0.8cm}} r @{${} = {}$} l } 
    \toprule 
    %
    $\ope{\phi1}$ & $(D_\mu\phi)^\dagger \, (\phi\,\phi^\dagger) \, (D^\mu\phi)$  &
    $\ope{GG}$ & $(\phisq)\,G^a_{\mu\nu}\,G^{\mu\nu\, a}$ \\
    %
    $\ope{\phi2}$ & $\frac{1}{2}\partial^\mu(\phisq)\,\partial_\mu(\phisq)$ &
    $\ope{BB}$ & $-\frac{g'^2}{4}(\phisq)\,B_{\mu\nu}\,B^{\mu\nu}$ \\
    %
    $\ope{\phi3}$ & $\frac{1}{3}(\phisq)^3$ &
    $\ope{WW}$ & $-\frac{g^2}{4}(\phisq)\,W^k_{\mu\nu}\,W^{\mu\nu\, k}$ \\
    %
    $\ope{\phi4}$  & $(\phisq) \, (D_\mu \phi)^\dagger (D^\mu \phi)$ &
    $\ope{BW}$ & $-\frac{g\,g'}{4}(\phi^\dagger\sigma^k\phi)\,B_{\mu\nu}\,W^{\mu\nu\, k}$ \\
    %
    \multicolumn{2}{c}{\quad} &
    $\ope{B} $ & $\frac{\im g}{2}(D^\mu\phi^\dagger)(D^\nu\phi)\,B_{\mu\nu}$ \\
    %
    \multicolumn{2}{c}{\quad} &
    $\ope{W}$ & $\frac{\im g}{2}(D^\mu\phi^\dagger)\sigma^k( D^\nu\phi)\,W_{\mu\nu}^k$ \\
    %
    \bottomrule
  \end{tabular}
  \caption[$CP$-even Higgs and Higgs-gauge operators]{Bosonic $CP$-conserving
    dimension-six operators relevant for Higgs physics.}
  \label{tbl:foundations_operators_bosonic_even}
\end{table}

\begin{table}
  \renewcommand{\arraystretch}{1.8}
  \begin{tabular*}{\textwidth}{r @{${} = {}$} l @{\hspace{0.8cm}} r @{${} = {}$} l @{\hspace{0.8cm}} r @{${} = {}$} l} 
    \toprule 
    %
    $\ope{\ell}$ & $(\phisq) \, \overbar{L}_L \phi \, \ell_{R} $ &
    $\ope{\phi L}^{(1)}$ & $\im (\phi^\dagger \overleftrightarrow{D}_\mu \phi) (\overbar{L}_L \gamma^\mu L_L)$ &
    $\ope{\phi L}^{(3)}$ & $\im (\phi^\dagger \overleftrightarrow{D}_\mu^a \phi) (\overbar{L}_L \gamma^\mu \sigma_a L_L)$ \\
    %
    $\ope{u}$ & $(\phisq) \, \overbar{Q}_L \tilde \phi \, u_{R} $ &
    $\ope{\phi Q}^{(1)}$ & $\im (\phi^\dagger \overleftrightarrow{D}_\mu \phi) (\overbar{Q}_L \gamma^\mu Q_L)$ &
    $\ope{\phi Q}^{(3)}$ & $\im (\phi^\dagger \overleftrightarrow{D}_\mu^a \phi) (\overbar{Q}_L \gamma^\mu \sigma_a Q_L)$ \\
    %
    $\ope{d}$ & $(\phisq) \,  \overbar{Q}_L \phi \, d_{R} $ &
    $\ope{\phi \ell}^{(1)}$ & $\im (\phi^\dagger \overleftrightarrow{D}_\mu \phi) (\overbar{\ell}_R \gamma^\mu \ell_R)$ \\
    %
    \multicolumn{2}{c}{\quad} &
    $\ope{\phi u}^{(1)}$ & $\im (\phi^\dagger \overleftrightarrow{D}_\mu \phi) (\overbar{u}_R \gamma^\mu u_R)$ \\
    %
    \multicolumn{2}{c}{\quad} &
    $\ope{\phi d}^{(1)}$ & $\im (\phi^\dagger \overleftrightarrow{D}_\mu \phi) (\overbar{d}_R \gamma^\mu d_R)$ \\
    %
    \multicolumn{2}{c}{\quad} &
    $\ope{\phi ud}^{(1)}$ & $\im (\phi^\dagger \overleftrightarrow{D}_\mu \phi) (\overbar{u}_R \gamma^\mu d_R)$ \\
    %
    \bottomrule
  \end{tabular*}
  \caption[$CP$-even Higgs-fermion operators]
  {$CP$-conserving dimension-six operators relevant for the Higgs-fermion
    couplings. All operators contain an implicit Hermitian conjugation. For
    readability, flavour indices are omitted.}
  \label{tbl:foundations_operators_fermionic_even}
\end{table}

\begin{table}
  \renewcommand{\arraystretch}{1.8}
  \begin{tabular*}{\textwidth}{r @{${} = {}$} l @{\hspace{0.8cm}} r @{${} = {}$} l @{\hspace{0.8cm}} r @{${} = {}$} l} 
    \toprule 
    %
    $\ope{u W}$ & $(\overbar{Q}_L \sigma^{\mu \nu} u_R ) \sigma^a \tilde{\phi} W^a_{\mu\nu}$ &
    $\ope{u B}$ & $(\overbar{Q}_L \sigma^{\mu \nu} u_R ) \tilde{\phi} B_{\mu\nu}$ &
    $\ope{u G}$ & $(\overbar{Q}_L \sigma^{\mu \nu} T^a u_R ) \tilde{\phi} G^a_{\mu\nu}$ \\
    % 
    $\ope{d W}$ & $(\overbar{Q}_L \sigma^{\mu \nu} d_R ) \sigma^a \phi W^a_{\mu\nu}$ &
    $\ope{d B}$ & $(\overbar{Q}_L \sigma^{\mu \nu} d_R ) \phi B_{\mu\nu}$ &
    $\ope{d G}$ & $(\overbar{Q}_L \sigma^{\mu \nu} T^a d_R ) \phi G^a_{\mu\nu}$ \\
    % 
    $\ope{\ell W}$ & $(\overbar{L}_L \sigma^{\mu \nu} \ell_R ) \sigma^a \phi W^a_{\mu\nu}$ &
    $\ope{\ell B}$ & $(\overbar{L}_L \sigma^{\mu \nu} \ell_R ) \phi B_{\mu\nu}$ \\
    %
    \bottomrule
  \end{tabular*}
  \caption[Dipole operators]
  {Dipole operators affecting the Higgs-gauge-fermion
    couplings. All operators contain an implicit Hermitian conjugation. For
    readability, flavour indices are omitted.}
  \label{tbl:foundations_operators_dipole}
\end{table}

We begin with the $CP$-conserving dimension-six operators relevant for
Higgs physics, following References~\cite{Corbett:2012ja,
  Juan_thesis, Tyler_thesis}.  In
\autoref{tbl:foundations_operators_bosonic_even} we list the bosonic
ones, \autoref{tbl:foundations_operators_fermionic_even} gives the
Higgs-fermion operators, and the ``dipole operators'' made of Higgs
fields, gauge bosons, and fermions are listed in
\autoref{tbl:foundations_operators_dipole}. We use the convention for
the sign in the covariant derivative given in
\autoref{eq:foundations_covariant_derivative}. $T^a$ are the $SU(3)$
generators, and
%
\begin{equation}
  \phi^\dagger \overleftrightarrow{D}_\mu \phi \equiv \phi^\dagger D_\mu \phi - (D_\mu \phi)^\dagger \phi
 \quad \text{and} \quad 
  \phi^\dagger \overleftrightarrow{D}_\mu^a \phi \equiv \phi^\dagger \sigma^a D_\mu \phi - (D_\mu \phi)^\dagger \sigma^a \phi \,.
\end{equation}
%
All other symbols appearing in these operators are defined in
Appendix~\ref{sec:appendix_bases_sm}.

% These operators can be separated into two groups: first, since
% $\phi^\dagger \phi$ is a scalar gauge singlet, it can be attached it
% to any dimension-4 operator of the SM to form a dimension-six
% operator. Second, additional (covariant) derivatives allow for new
% structures very different from the SM vertices.

\begin{table}
  \renewcommand{\arraystretch}{1.8}
  \begin{tabular}{r @{${} = {}$} l @{\hspace*{0.8cm}} r @{${} = {}$} l} 
    \toprule 
    %
    $\ope{G\widetilde{G}}$ & $(\phisq)\,G^a_{\mu\nu}\,\widetilde{G}^{\mu\nu\, a}$ &
    $\ope{\widetilde{B}} $ & $\frac{\im g}{2}(D^\mu\phi^\dagger)(D^\nu\phi)\,\widetilde{B}_{\mu\nu}$ \\
    %
    $\ope{B\widetilde{B}}$ & $-\frac{g'^2}{4}(\phisq)\,B_{\mu\nu}\,\widetilde{B}^{\mu\nu}$ &
    $\ope{B\widetilde{W}}$ & $-\frac{g\,g'}{4}(\phi^\dagger\sigma^k\phi)\,B_{\mu\nu}\,\widetilde{W}^{\mu\nu\, k}$ \\
    %
    $\ope{W\widetilde{W}}$ & $-\frac{g^2}{4}(\phisq)\,W^k_{\mu\nu}\,\widetilde{W}^{\mu\nu\, k}$ \\
    %
    % $\ope{\text{need a name}} $ & $(\phi^\dagger \overleftrightarrow{D^2} \phi ) (\phisq)$ \\
    % $\ope{\text{another name}} $ & $g' \widetilde{B}_{\mu\nu} \left[ (\phi^\dagger \overleftrightarrow{D^2} \phi ) + 2 (D_\rho \phi )^\dagger D_\sigma \phi \right] $ \\
    % $\ope{\text{and one more}} $ & $g \widetilde{W}_{\mu\nu} \left[ (\phi^\dagger \sigma_a \overleftrightarrow{D^2} \phi ) + 2 (D_\rho \phi )^\dagger \sigma_a D_\sigma \phi \right] $ \\
    %
    \bottomrule
  \end{tabular}
  \caption[$CP$-odd Higgs and Higgs-gauge operators]{Bosonic $CP$-violating
    dimension-six operators relevant for Higgs physics. The dual field strengths
    $\widetilde{V}_{\mu\nu}$ with $V = G, B, W$  are defined in
    \autoref{eq:foundations_dual_field_strengths}.}
  \label{tbl:foundations_operators_bosonic_odd}
\end{table}

In addition to these $CP$-conserving structures, there are a number of
$CP$-violating operators. We only list the bosonic ones relevant for
Higgs physics~\cite{Gavela:2014vra, Hankele:2006ma} in
\autoref{tbl:foundations_operators_bosonic_odd}. They involve the
dual field strength tensors
%
\begin{equation}
  \widetilde{V}_{\mu \nu} = \frac 1 2 \varepsilon_{\mu \nu \rho \sigma} V^{\mu\nu} \,, \quad
  V = B,W,G \,.
  \label{eq:foundations_dual_field_strengths}
\end{equation}

Finally, there are a few $CP$-even and $CP$-odd pure gauge operators
made from field strength tensors and (covariant) derivatives, and a
large number of four-fermion operators similar to the one in
\autoref{eq:foundations_fermi_theory}, which are not important in this
thesis.

% \begin{table}
%   \renewcommand{\arraystretch}{1.8}
%   \begin{tabular}{r @{${} = {}$} l @{\hspace*{0.8cm}} r @{${} = {}$} l} 
%     \toprule 
%     %
%     $\ope{WWW}$ & $\im \varepsilon^{abc} W_{\,\mu}^{a\, \nu} W_{\,\nu}^{b\, \rho} W_{\,\rho}^{c\, \mu}$ &
%     $\ope{WW\widetilde{W}}$ & $\im \varepsilon^{abc} W_{\,\mu}^{a\, \nu} W_{\,\nu}^{b\, \rho} \widetilde{W}_{\,\rho}^{c\, \mu}$ \\
%     %
%     $\ope{GGG}$ & $\im f^{abc} G_{\,\mu}^{a\, \nu} G_{\,\nu}^{b\, \rho} G_{\,\rho}^{c\, \mu}$ &
%     $\ope{GG\widetilde{G}}$ & $\im f^{abc} G_{\,\mu}^{a\, \nu} G_{\,\nu}^{b\, \rho} \widetilde{G}_{\,\rho}^{c\, \mu}$ \\
%     % 
%     %$\ope{DB}$ & $(\partial^\mu B_{\mu\nu}) (\partial_\rho B^{\rho \nu})$\\
%     %$\ope{DW}$ & $(D^\mu W_{\mu\nu})^i (D_\rho W^{\rho \nu})^i$\\
%     %$\ope{DG}$ & $(D^\mu G_{\mu\nu})^i (D_\rho G^{\rho \nu})^i$\\
%     % 
%     \bottomrule
%   \end{tabular}
%   \caption[Pure gauge operators]{Dimension-six operators made only of gauge fields.}
%   \label{tbl:foundations_operators_pure_gauge}
% \end{table}

% \begin{table}
%   \renewcommand{\arraystretch}{1.8}
%   \begin{tabular}{r @{${} = {}$} l @{\hspace*{0.8cm}} r @{${} = {}$} l} 
%     \toprule 
%     %
%     $\ope{}$ & $\dots$ \\
%     %
%     \bottomrule
%   \end{tabular}
%   \caption[Four-fermion operators]{Dimension-six four-fermion operators, omitting flavour indices.}
%   \label{tbl:foundations_operators_four_fermion}
% \end{table}

% For the sake of completeness, we finally list the operators not
% containing Higgs fields as well, following References~\ref{Juan_thesis, Tyler_thesis,
%   Grzadkowski:2010es}. \autoref{tbl:foundations_operators_pure_gauge}
% shows the operators made from gauge fields and covariant derivatives, where
% %
% \begin{equation}
%   (D^\mu W_{\mu\nu})^a \equiv \partial^\mu W^a_{\mu\nu} - g \varepsilon^{abc} W^b_\mu W^c_\nu \,, \quad
%   (D^\mu G_{\mu\nu})^a \equiv \partial^\mu G^a_{\mu\nu} - g_s f^{abc} G^b_\mu G^c_\nu \,.
% \end{equation}
% %
% Finally, \autoref{tbl:foundations_operators_four_fermion} offers a
% glimpse into the vast world of four-fermion operators. 

As argued above, not all of these operators are independent.  The
equations of motion for the Higgs field and the electroweak gauge
bosons read~\cite{Grzadkowski:2010es}
%
\begingroup%
\allowdisplaybreaks%
\begin{align}
  %
  D^2 \phi
  &=
    - \mu^2 \phi
    - 2 \lambda (\phisq) \phi
    - \sum_f y_f \overbar{f}_R f_L
    + \ord{1/\Lambda^2} \,, 
  \label{eq:foundations_sm_eom1} \\
  %
  \partial^\rho B_{\rho \mu}
  &=
    - \frac {\im g'} 2  \phi^\dagger \overleftrightarrow{D}_\mu \phi
    - \sum_f g' Y_f \overbar{f} \gamma_\mu f + \ord{1/\Lambda^2} \,, \\
  %
  (D^\rho W_{\rho \mu})^a
  &=
    - \frac {\im g} 2 \, \phi^\dagger \overleftrightarrow{D}_\mu^a \phi
    - \frac g 2  \, \sum_f \overbar{f}_L \gamma_\mu \sigma^a f_L
    + \ord{1/\Lambda^2} \,,
    %
  \label{eq:foundations_sm_eom2}
\end{align}%
\endgroup
%
where $Y_f$ are the weak hypercharges of the fermions.  Following
\autoref{eq:foundations_field_redefinitions}, this provides us with
three equivalence relations between dimension-six
operators~\cite{Corbett:2012ja, Juan_thesis, Tyler_thesis}:
%
\begingroup%
\allowdisplaybreaks%
\begin{align}
  % \ope{\phi,2} + \ope{\phi,4} - \mu^2 (\phisq)^2 - 6 \lambda \ope{6} &= \sum_f y_f \ope{f} + \ord{1/\Lambda^4}\\ % My version
  %
  2 \ope{\phi,2} + 2 \ope{\phi,4}  + 2 \mu^2 (\phisq)^2 + 12 \lambda \ope{6} &= \sum_f y_f \ope{f} + \ord{1/\Lambda^4}\\ % Tyler's version, and Juan believes it's true
  %
  \ope{BB}  + \ope{BW} - 2 \ope{B} + g'^2 \left( \ope{\phi,1} - \frac 1 2 \ope{\phi,2} \right) &= - \frac {g'^2} 2 \, \sum_f Y_f \ope{\phi f}^{(1)} + \ord{1/\Lambda^4}\\
  %
  \ope{WW}  + \ope{BW} - 2 \ope{W} + g^2 \left( \ope{\phi,4} - \frac 1 2 \ope{\phi,2} \right) &= - \frac {g^2} 4 \, \sum_{F=L,Q} \ope{\phi F}^{(3)} + \ord{1/\Lambda^4}\,.
\end{align}%
\endgroup
%
This lets us eliminate three of the operators listed in
Tables~\ref{tbl:foundations_operators_bosonic_even} to
\ref{tbl:foundations_operators_bosonic_odd}.

There are different strategies for picking the operators to keep. In a
top-down approach, one could choose operators based on the underlying
physics. In a bottom-up approach, calculations can be simplified if
the operators are chosen based on their contributions to physical
observables, for instance to avoid non-trivial blind
directions. Following Reference~\cite{Corbett:2012ja}, we choose
to discard $\smash{\ope{\phi L}^{(1)}}$, $\smash{\ope{\phi L}^{(3)}}$, and $\ope{\phi,4}$.



%%%%%%%%%%%%%%%%%%%%%%%%%%%%%%%%%%%%%%%%%%%%%%%%%%%%%%%%%%%%
\subsubsection{Constraints}
%%%%%%%%%%%%%%%%%%%%%%%%%%%%%%%%%%%%%%%%%%%%%%%%%%%%%%%%%%%%

Some of the remaining operators are tightly constrained from
experimental data. Electroweak precision measurements limit the Wilson
coefficients of $\ope{\phi,1}$, $\ope{BW}$, $\ope{B\widetilde{W}}$,
$\ope{\widetilde{B}}$, $\smash{\ope{\phi L}^{(3)}}$, and the remaining
$\smash{\ope{\phi f}^{(1)}}$ to a level where their effects in Higgs
physics are small. Measurements of electric dipole moments put tight
constrains on the dipole operators. Therefore we mostly ignore these
operators in this thesis.\footnote{This simple argument is suitable
  for our rather conceptual work. In a thorough global fit, however,
  it should be checked carefully whether these constraints are
  actually strong enough to make all these operator irrelevant for
  Higgs physics in all cases. Such a check should include RGE effects
  when comparing constraints from different scales. The increasing
  precision in Higgs observables means that many of these operators
  will become relevant again in the future.}

Limits on flavour-changing neutral currents constrain off-diagonal
fermion-Higgs couplings. Also, flavour-diagonal $\ope{f}$ involving
fermions of the first and second generation are irrelevant for many
signatures considered in this thesis. We therefore only keep the
Higgs-fermion operators $\ope{f}$ of the third generation.

This leaves us with a list of thirteen operators relevant for LHC
Higgs physics: ten $CP$-even operators,
%
\begin{equation}
  \ope{\phi,2} \,, \quad 
  \ope{\phi,3} \,, \quad 
  \ope{GG} \,, \quad 
  \ope{BB} \,, \quad 
  \ope{WW} \,, \quad 
  \ope{B} \,, \quad 
  \ope{W} \,, \quad 
  \ope{\tau} \,, \quad
  \ope{t} \,, \quad  \text{and} \quad
  \ope{b} \,;
  \label{eq:foundations_operators_even}
\end{equation}
%
and three $CP$-odd ones,
%
\begin{equation}
  \ope{G\widetilde{G}} \,, \quad 
  \ope{B\widetilde{B}} \,, \quad  \text{and} \quad
  \ope{W\widetilde{W}} \,. 
  \label{eq:foundations_operators_odd}
\end{equation}




%%%%%%%%%%%%%%%%%%%%%%%%%%%%%%%%%%%%%%%%%%%%%%%%%%%%%%%%%%%%
\subsubsection{Renormalisation group evolution}
%%%%%%%%%%%%%%%%%%%%%%%%%%%%%%%%%%%%%%%%%%%%%%%%%%%%%%%%%%%%

The Wilson coefficients of these operators depend on the energy scale.
In the last years, the contributions of all dimension-six operators on
the running of the SM parameters, as well as the whole $59\times 59$
anomalous dimension matrix of dimension-six operators, have been
calculated at one-loop level~\cite{Jenkins:2013zja, Jenkins:2013wua,
  Alonso:2013hga}. This provides all necessary tools to run the EFT
parameters from the matching scale $\Lambda$ to the experimental scale
$E$. Following \autoref{eq:foundations_EFT_running}, this shifts the
Wilson coefficients by a term proportional to a loop factor and
$\log \Lambda^2 / E^2$.

As we discuss at some length in \autoref{chapter:validity},
the LHC Higgs measurements are only sensitive to new physics scales
between the electroweak scale and the TeV scale. The corresponding
logarithm typically cannot compensate for the loop factor, and the RGE
effects on Wilson coefficients that are already non-zero at the
matching scale are small. We therefore neglect operator running
for our analyses.



%%%%%%%%%%%%%%%%%%%%%%%%%%%%%%%%%%%%%%%%%%%%%%%%%%%%%%%%%%%%
\subsection{Phenomenology}
\label{sec:foundations_heft_pheno}
%%%%%%%%%%%%%%%%%%%%%%%%%%%%%%%%%%%%%%%%%%%%%%%%%%%%%%%%%%%%

After picking a set of operators, the next question is how they affect
Higgs observables. We first discuss two examples, $\ope{\phi,2}$
and $\ope{W}$, in detail, before listing the effects of all operators
in Equations~\eqref{eq:foundations_operators_even} and
\eqref{eq:foundations_operators_odd}.



%%%%%%%%%%%%%%%%%%%%%%%%%%%%%%%%%%%%%%%%%%%%%%%%%%%%%%%%%%%%
\subsubsection{$\ope{\phi2}$: rescaled Higgs couplings}
%%%%%%%%%%%%%%%%%%%%%%%%%%%%%%%%%%%%%%%%%%%%%%%%%%%%%%%%%%%%

Our first example is the operator $\ope{\phi,2}$. Ignoring the
Goldstones, it consists only of derivatives and Higgs fields
$\phisq = (v^2 + 2 v \tilde{h} + \tilde{h}^2) / 2$, where we use a
tilde on $h$ because this field is not yet a mass eigenstate. Its
contribution to the Lagrangian reads
%
\begin{align}
  \lgr{EFT} &\supset \frac {f_{\phi, 2}} {2\Lambda^2} \, \partial^\mu(\phi^\dagger\phi) \, \partial_\mu(\phi^\dagger\phi) \notag \\
  %{} &= \frac {f_{\phi, 2}} {2\Lambda^2} \, \frac {( 2 v \partial_\mu \tilde{h} + 2 \tilde{h} \partial_\mu \tilde{h} )^2} 4  \notag \\
  {} &= \frac {f_{\phi, 2} v^2} {2\Lambda^2} \, \partial_\mu \tilde{h} \partial^\mu \tilde{h} + \frac {f_{\phi, 2} \, v} {\Lambda^2} \, \tilde{h} \partial_\mu \tilde{h} \partial^\mu \tilde{h} + \frac {f_{\phi, 2} \, v} {2 \Lambda^2} \, \tilde{h}^2 \partial_\mu \tilde{h} \partial^\mu \tilde{h} \,.
  \label{eq:foundations_ophi2_terms}
\end{align}
%
The first term rescales the kinetic term of the Higgs boson:
%
\begin{equation}
  \lgr{EFT} \supset \left( 1 + \frac {f_{\phi 2} v^2} {\Lambda^2} \right) \, \frac 1 2 \, \partial_\mu \tilde{h} \partial^\mu \tilde{h} \,.
\end{equation}
%
To restore the canonical form of the kinetic term, we have to rescale
the Higgs boson $\tilde{h}$ to
%
\begin{equation}
  h = \sqrt{1 + \frac {f_{\phi 2} v^2} {\Lambda^2} } \, \tilde{h} \,.
  \label{eq:foundations_ophi2_rescaling}
\end{equation}

This universally shifts all Higgs couplings to other particles as
%
\begin{equation}
  g_{hxx} = \frac 1 {\sqrt{1 + \frac {f_{\phi 2} v^2} {\Lambda^2} } }  \, g_{hxx}^{\text{SM}}\,. 
\end{equation}
%
There is a second, more involved effect on the Higgs
self-coupling. First, the rescaling in
\autoref{eq:foundations_ophi2_rescaling} also affects the Higgs mass
term given in \autoref{eq:foundations_higgs_mass_sm}. For fixed $v$
and $m_h$, this amounts to shifting the Higgs self-coupling $\lambda$
to
%
\begin{equation}
  \lambda = \frac {m_h^2} {2 v^2} \, \left( 1 + \frac {f_{\phi 2} v^2} {\Lambda^2} \right) \,.
\end{equation}
%
In addition, the second term in \autoref{eq:foundations_ophi2_terms}
introduces a new Lorentz structure into the Higgs self-interaction
that depends on the Higgs momenta. A non-zero Wilson coefficient
$f_{\phi 2}$ therefore has a strong impact on Higgs pair production,
changing not only the total rate, but also kinematic shapes.



%%%%%%%%%%%%%%%%%%%%%%%%%%%%%%%%%%%%%%%%%%%%%%%%%%%%%%%%%%%%
\subsubsection{$\ope{W}$: new Higgs-gauge structures}
%%%%%%%%%%%%%%%%%%%%%%%%%%%%%%%%%%%%%%%%%%%%%%%%%%%%%%%%%%%%

Our second example is $\ope{W}$, $\upsilon v \nu {\mathrm{v}}$ which contracts covariant derivatives
acting on Higgs doublets, defined in
\autoref{eq:foundations_covariant_derivative}, with a field strength
tensor
$W_{\mu\nu}^k = \partial_\mu W^k_\nu - \partial_\nu W^k_\mu + g
\varepsilon^{kmn} W^m_\mu W^n_\nu$.
Expanding $\ope{W}$ and only keeping the pieces that affect the $hWW$
coupling, we find
%
\begin{align}
  \lgr{EFT} &\supset \frac {f_{W}} {\Lambda^2} \, \frac{\im g}{2} \, (D^\mu\phi)^\dagger \sigma^k ( D^\nu\phi) \, W_{\mu\nu}^k \notag \\
  {} &= \frac {\im g f_{W}} {2 \Lambda^2} 
       \left(\partial^\mu \phi^\dagger + \frac {\im  g} 2 W^{m\,\mu} \phi^\dagger \sigma^m + \frac {\im  g'} 2 B^\mu \phi^\dagger \right) \sigma^k \left(\partial^\nu \phi - \frac {\im g} 2 \sigma^n  W^{n\,\nu} \phi - \frac {\im  g'} 2 B^\nu \phi\right) W_{\mu\nu}^k \notag \\
  {} &\supset \frac {\im g f_{W}} {2 \Lambda^2}  \, \Biggl\{
       \frac {\partial^\mu h} {\sqrt 2} \, \left[ \sigma^k  \sigma^n \right]_{22} \, \frac {-\im g} 2 \, W^{n \,\nu} \, \frac v {\sqrt{2}} %\notag \\
   % &\phantom{=} \quad  \quad \quad \quad 
+ \frac {\im g} 2 \, W^{m \,\nu} \, \frac v {\sqrt{2}}  \, \left[ \sigma^m  \sigma^k \right]_{22} \, \frac {\partial^\mu h} {\sqrt 2}
       \Biggr\} \, W_{\mu\nu}^k \notag \\
  {} &= \frac {f_{W}} {\Lambda^2} \, \frac{g^2 v}{8} \,  \left[ \sigma^k,  \sigma^n \right]_{22} \, (\partial^\mu h)  \, W^{n \,\nu}  \, W_{\mu\nu}^k \notag \\
  %{} &= \frac {f_{W}} {\Lambda^2} \, \frac{g^2 v}{8} \,  2 \im \varepsilon^{knm} \, \sigma^m_{22} \, (\partial^\mu h)  \, W^{n \,\nu}  \, W_{\mu\nu}^k\notag \\
  {} &= \frac {f_{W}} {\Lambda^2} \, \frac{\im g^2 v}{4} \, \varepsilon^{nk3} \, (\partial^\mu h)  \,  W^{n \,\nu}  \, W_{\mu\nu}^k  \,.
\end{align}
%
With $m_W = gv / 2$ and
$W^\pm_\mu = (W^1_\mu \mp \im W^2_\mu ) / \sqrt{2}$ this finally
yields
%
\begin{equation}
  \lgr{EFT} \supset \frac {f_{W}} {\Lambda^2} \, \frac{\im gm_W}{2} \, (\partial^\mu h) \left( W^{+ \,\nu} \, W^-_{\mu\nu} + W^{- \,\nu} \, W^+_{\mu\nu} \right) \,.
\end{equation}

This is another contribution to the $hWW$ vertex. But unlike the SM-like coupling
%
\begin{equation}
  \lgr{SM} \supset g m_W \, h W^{+ \,\mu} \, W^-_{\mu} \,,
\end{equation}
%
the $\ope{W}$ term includes derivatives. This means that the
interaction gains a momentum dependence:
%
\begin{equation}
  \raisebox{-0.475\height}{
      \fmfframe(10,15)(10,15){ %(L,T) (R,B)
        \begin{fmfgraph*}(60,60)
          \feynmansetup
          \fmfleft{i}
          \fmfright{o2,o1}
          \fmflabel{$H$}{i}
          \fmflabel{$W^+_\mu$}{o1}
          \fmflabel{$W^-_\nu$}{o2}
          \fmf{dashes}{i,v}
          \fmf{boson}{v,o1}
          \fmf{boson}{v,o2}
        \end{fmfgraph*}
      }
  }
  =  \im g m_W  
  \left[ g_{\mu \nu} +  \frac{f_W}{2 \Lambda^2} \, p_H^2 \, g_{\mu \nu} + \frac{f_W}{2 \Lambda^2} \left( p^H_\mu p^+_\nu + p^-_\mu p^H_\nu \right) \right] \, ,
  \label{eq:foundations_OW_HWW_Feynman_rule}
\end{equation}
%
where $p^\pm_\mu$ and $p^H_\mu$ are the incoming momenta of the
$W^\pm$ and the $H$, respectively. 

%-----------------------------------------------------------
\begin{figure}
  \centering
  \includegraphicsdummy[width=0.49\textwidth]{fig/general/dim6demo1.pdf}
  \includegraphicsdummy[width=0.49\textwidth]{fig/general/dim6demo2.pdf}
  \caption[Momentum dependence from $\ope{W}$ in $Zh$
  production]{Distribution of the $Zh$ invariant mass in the
    Higgs-strahlung process $pp \to Zh$ at LHC conditions. We compare
    the contributions from the SM, the operator $\ope{W}$ squared, and
    their interference, which can be constructive (left) or
    destructive (right).}
    % Note that this plot is based on a different
    % operator basis, which is why the operator $\ope{W}$ used in this
    % plot is closely related, but not identical to that defined in
    % \autoref{tbl:foundations_operators_bosonic_even}.
  \label{fig:foundations_OW_Zh_demo}
\end{figure}
%-----------------------------------------------------------

This operator illustrates two key features of the EFT approach. First,
$\ope{W}$ does not only affect the $hWW$ vertex, but also $hZZ$
interactions and triple-gauge couplings such as $WWZ$. This means that
the dimension-six operator language allows us to combine different
measurements in a global fit.

Second, $\ope{W}$ changes the shape of distributions, for instance in
Higgs-strahlung at the LHC,
%
\begin{equation}
  p p \to Z h \,.
\end{equation}
%
In this process, the intermediate $Z$ can carry arbitrary large energy
and momentum, which we can measure for instance as the invariant mass
of the final $Zh$ system. From
\autoref{eq:foundations_OW_HWW_Feynman_rule} we expect that the
effects from $\ope{W}$ grow with $m_{Zh}$. In
\autoref{fig:foundations_OW_Zh_demo} we demonstrate this by comparing
the distribution of $m_{Zh}$ based on the SM diagrams alone, on the
contributions from only the dimension-six operator $\ope{W}$ only, and
on the interference between the two components. Indeed we see that
$\ope{W}$ contributes mostly in the high-energy tail of the
distribution.




%%%%%%%%%%%%%%%%%%%%%%%%%%%%%%%%%%%%%%%%%%%%%%%%%%%%%%%%%%%%
\subsubsection{All those couplings}
%%%%%%%%%%%%%%%%%%%%%%%%%%%%%%%%%%%%%%%%%%%%%%%%%%%%%%%%%%%%

After these two worked-out examples, we now give the complete list of
single-Higgs couplings induced by the dimension-six operators of
Equations~\eqref{eq:foundations_operators_even} and
\eqref{eq:foundations_operators_odd}~\cite{Corbett:2012ja,
  Juan_thesis, Tyler_thesis}.\footnote{When comparing with
  References~\cite{Corbett:2012ja, Juan_thesis, Tyler_thesis}, note
  the different sign conventions in the covariant derivative.}

These interactions read
%
\begin{align}
  \lgr{EFT} &\supset
              g^{(1)}_{hgg} \; h G^a_{\mu\nu} G^{a \, \mu\nu}
              + g^{(2)}_{hgg} \; \varepsilon_{\mu \nu \rho \sigma} h G^{a\, \mu\nu} G^{a \, \rho\sigma}
              + g_{h\gamma\gamma} \; h A_{\mu \nu} A^{\mu \nu} \notag \\
            &\phantom{=} \quad
              + g^{(1)}_{hZ\gamma} \; A_{\mu \nu} Z^{\mu} \partial^{\nu} h
              + g^{(2)}_{hZ\gamma} \; h A_{\mu \nu} Z^{\mu \nu} \notag \\
            &\phantom{=} \quad
              + g^{(1)}_{hZZ}  \; Z_{\mu \nu} Z^{\mu} \partial^{\nu} h
              + g^{(2)}_{hZZ}  \; h Z_{\mu \nu} Z^{\mu \nu}
              + g^{(3)}_{hZZ}  \; h Z_\mu Z^\mu 
              + g^{(4)}_{hZZ}  \; h \varepsilon_{\mu \nu \rho \sigma}  Z^{\mu \nu} Z^{\rho \sigma} \notag \\
            &\phantom{=} \quad
              + g^{(1)}_{hWW}  \; \left (W^+_{\mu \nu} W^{- \, \mu} \partial^{\nu} h + \hc \right)
              + g^{(2)}_{hWW}  \; h W^+_{\mu \nu} W^{- \, \mu \nu}
              + g^{(3)}_{hWW}  \; h W^+_{\mu} W^{- \, \mu} \notag \\
            &\phantom{=} \qqqqquad
              + g^{(4)}_{hWW}  \; h \varepsilon_{\mu \nu \rho \sigma} W^{+ \, \mu \nu} W^{- \, \rho \sigma} \notag \\
            &\phantom{=} \quad
              + \sum_{f=\tau,t,b} \left( g_{hff} h \overbar{f}_L f_R + \hc \right) 
  \label{eq:foundations_higgs_interactions}
\end{align}
%
with couplings
%
\begingroup%
\allowdisplaybreaks%
\begin{align}
  g^{(1)}_{hgg} &= \frac {f_{GG} v} {\Lambda^2} \,, \quad & \quad 
  g^{(1)}_{hZ\gamma}  &= - \frac {g^2 v s_W (f_{W} - f_{B}) } {4 c_W \Lambda^2} \,, \notag \\
  %
  g^{(2)}_{hgg} &= \frac {f_{G\widetilde{G}} v} {2 \Lambda^2} \,, \quad & \quad 
  g^{(2)}_{hZ\gamma} &= \frac {g^2 v s_W (2 s_W^2 f_{BB} - 2 c_W^2 f_{WW}) } {4 c_W \Lambda^2} \,, \notag \\
  %
  g_{h\gamma\gamma} &= - \frac {g^2 v s_W^2 (f_{WW} + f_{BB}) } {4 \Lambda^2} \,, \quad & \quad 
  g_{hff} &= - \frac {m_f} {v} \left(1 +  \frac{v^2  f_{\phi,2} } {\Lambda^2} \right)^{-1/2}
  + \frac {v^2 f_f} {\sqrt{2} \Lambda^2} \,, \notag \\
  %
  g^{(1)}_{hZZ} &= - \frac {g^2 v (c_W^2 f_{W} + s_W^2 f_{B}) } {4 c_W^2 \Lambda^2} \,, \quad & \quad 
  g^{(1)}_{hWW} &= - \frac {g^2 v f_W } {4 \Lambda^2} \,, \notag \\
  %
  g^{(2)}_{hZZ} &= - \frac {g^2 v (s_W^4 f_{BB} + c_W^4 f_{WW}) } {4 c_W^2 \Lambda^2} \,, \quad & \quad 
  g^{(2)}_{hWW} &= - \frac {g^2 v f_{WW}} {2 \Lambda^2} \,, \notag \\
  %
  g^{(3)}_{hZZ} &= \frac {g^2 v} {4 c_W^2} \left(1 +  \frac{v^2  f_{\phi,2} } {\Lambda^2} \right)^{-1/2} \,, \quad & \quad 
  g^{(3)}_{hWW} &= \frac {g^2 v} {2} \left(1 +  \frac{v^2  f_{\phi,2} } {\Lambda^2} \right)^{-1/2}  \,, \notag \\
  %
  g^{(4)}_{hZZ} &= - \frac {g^2 v (s_W^4 f_{B\widetilde{B}} + c_W^4 f_{W\widetilde{W}}) } {8 c_W^2 \Lambda^2} \,, \quad & \quad 
  g^{(4)}_{hWW} &= - \frac {g^2 v f_{W\widetilde{W}}} {4 \Lambda^2}  \,.
  \label{eq:foundations_higgs_couplings}
\end{align}%
\endgroup
%
Here $s_W = g' / \sqrt{g^2 + g'^2}$ and $c_W = g / \sqrt{g^2 + g'^2}$
are the sine and cosine of the weak mixing angle, and
$V_{\mu\nu} = \partial_\mu V_\nu - \partial_\nu V_\mu$ for
$V = A, W^\pm, Z$.

Note that the clear majority of the couplings in
\autoref{eq:foundations_higgs_interactions} does not exist in the SM
and contains derivatives. Dimension-six operators predict a variety of
novel kinematic features in Higgs interactions, making their
measurement at the LHC both exciting and challenging.



%%%%%%%%%%%%%%%%%%%%%%%%%%%%%%%%%%%%%%%%%%%%%%%%%%%%%%%%%%%%
\subsection{Alternative frameworks}
\label{sec:foundations_heft_alternatives}
%%%%%%%%%%%%%%%%%%%%%%%%%%%%%%%%%%%%%%%%%%%%%%%%%%%%%%%%%%%%

The linear Higgs EFT discussed above is not the only useful
parametrisation of Higgs properties. We briefly go through some
of the alternative frameworks and explain their main properties,
before finishing this chapter with a comparison between the different
approaches.



%%%%%%%%%%%%%%%%%%%%%%%%%%%%%%%%%%%%%%%%%%%%%%%%%%%%%%%%%%%%
\subsubsection{Non-linear Higgs effective field theory}
%%%%%%%%%%%%%%%%%%%%%%%%%%%%%%%%%%%%%%%%%%%%%%%%%%%%%%%%%%%%

In the SM EFT (or linear Higgs EFT), constructed in
\autoref{sec:foundations_heft_operators}, the Higgs boson $h$ and
the Goldstone bosons $w_a$ form an $SU(2)$ doublet $\phi$ as given in
\autoref{eq:foundations_sm_phi}. But in some models of new physics
the Higgs is not part of an elementary $SU(2)$ doublet. Typical
examples are composite Higgs models in which the Higgs boson is a
pseudo-Goldstone from some strongly interacting
dynamics~\cite{Kaplan:1983fs, Kaplan:1983sm, Banks:1984gj,
  Agashe:2004rs, Gripaios:2009pe}. Non-linear Higgs EFT, sometimes
simply called ``Higgs EFT'', is an effective theory designed for these
scenarios~\cite{Appelquist:1980vg, Longhitano:1980iz,
  Appelquist:1984rr, Grinstein:2007iv, Alonso:2012px,
  Buchalla:2013rka, Buchalla:2013eza, Brivio:2013pma, Gavela:2014vra,
  Buchalla:2015wfa, Brivio:2016fzo}. This model is also often referred
to as ``chiral Lagrangian'', and indeed its structure is similar to
that of chiral perturbation theory, for an introduction see for
instance References~\cite{Scherer:2002tk, HillerBlin:2016jpb}.  Again, we
begin by going through the ingredients to the effective theory before
constructing the Lagrangian.

The \emph{particle content} of the non-linear Higgs EFT constitutes
the main difference to the linear Lagrangian: the physical scalar $h$
is separated from the Goldstones $w_a$, both are included as
independent degrees of freedom rather than as part of the doublet
$\phi$. The Higgs boson $h$ is now a singlet under the SM gauge
symmetry.  The Goldstone bosons $w_a$, no longer a part of a Higgs
doublet, are organised in the exponential form
% 
\begin{equation}
  U = e^{\im \sigma_a w_a} \,,
\end{equation}
% 
where $\sigma_a$ are the Pauli matrices.
The Goldstones transform non-linearly under the (approximate) global
custodial symmetry $SU(2)_L \times SU(2)_R$, giving the EFT its name.

The \emph{symmetries} are the same as in the linear case. We require
invariance under Lorentz transformations as well as under the SM gauge
group $SU(3)_C \times SU(2)_L \times U(1)_Y$ and baryon and lepton
number conservation. For simplicity, we also restrict our brief
discussion to $CP$-even operators that conserve lepton flavour.

Choosing the \emph{counting scheme} is a little more involved. To
account for strongly interacting scenarios, we now have to distinguish
three different scales~\cite{Buchalla:2013eza}:
% 
\begin{itemize}
\item The electroweak scale $v = 246~\gev$, which defines the $W$ and
  $Z$ mass, but is not necessarily the Higgs VEV.
  % 
\item The scale $f$ associated to the Goldstone bosons $w_a$ and the
  Higgs boson $h$ due to some breaking of the underlying dynamics, in
  analogy to the pion decay constant $f_\pi$.\footnote{In general, the
    scales associated with $w_a$ and $h$, $f_w$ and $f_h$, can be
    different, making the power-counting even more complicated.}
  % 
\item The cut-off $\Lambda$ of the theory. For weakly coupled physics
  this can be arbitrary. But one can calculate that the low-energy
  effective theories from spontaneously broken strongly coupled
  dynamics break down around
  $\Lambda \approx 4 \pi f$~\cite{Scherer:2002tk}. A cut-off of this
  size guarantees that the EFT is renormalisable order by order.
\end{itemize}

The existence of three scales means there are two dimensionless
parameters, so in general the EFT terms are organised in a double
expansion~\cite{Buchalla:2013eza}. The first is
% 
\begin{equation}
  \xi \equiv \frac {v^2} {f^2} \,.
\end{equation}
% 
The value of $\xi$ defines the non-linearity of the model: the limit
$\xi \to 0$ restores the linear Lagrangian. An expansion in $\xi$
exactly corresponds to the power-counting scheme of the linear EFT,
i.\,e.\ it orders operators by their canonical dimension.

The second dimensionless parameter is
% 
\begin{equation}
  \frac {f^2} {\Lambda^2} \approx \frac 1 {16 \pi^2}
\end{equation}
% 
for strongly coupled scenarios. Expanding in $f^2/\Lambda^2$
corresponds to a loop expansion similar to that in chiral perturbation
theory. Equivalently, one can define a \emph{chiral dimension}
$\chi = [\ope{}]_c$ for each operator $\ope{}$ with the
assignments~\cite{Buchalla:2013eza}
% 
\begin{align}
  [f]_c &= 1 \,, \quad &
  [V_{\mu}]_c &= 0 \,, \quad & 
  [V_{\mu\nu}]_c &= 1 \,, \quad &
  [U]_c &= 0 \,, \quad &
  [h]_c &= 0 \,, \quad \notag \\
  [\partial_\mu]_c &= 1 \,, \quad  &
  [D_\mu]_c &= 1 \,, \quad &
  [g]_c &= 1 \,, \quad &
  [y_f]_c &= 1 \,.
\end{align}
%
The loop order $L$ of an operator is equivalent to the chiral dimension
$\chi = 2L + 2$. This chiral counting can also be linked to an
expansion in $\hbar$~\cite{Gavela:2016bzc}.

The correct expansion scheme depends on the value of $\xi$. For
$\xi \gg 1 / 16 \pi^2$ or $f \ll 3~\tev$, the chiral expansion is more
appropriate. For $\xi \ll 1 / 16 \pi^2$ or $f \gg 3~\tev$, the
canonical expansion is correct. In the intermediate region, a combined
expansion gives the best results. Since LHC Higgs physics is mostly
sensitive to new physics scenarios with $f \ll 3~\tev$, the chiral expansion
can be considered phenomenologically more relevant. For a more
thorough discussion of power counting in this framework, see
Reference~\cite{Krause:2016uhw}.

\newparagraph
%
At the leading chiral order $\chi = 2$ or $L = 0$, the Higgs sector of
the Lagrangian is given by~\cite{Alonso:2012px}\footnote{Two comments
  on this Lagrangian are in order. First, in principle there could be
  further functions of $h$ coupling to the kinetic terms of the gauge
  bosons. Such interactions arise in typical strongly coupled theories
  at one-loop level with a coefficient $\sim 1 / (16 \pi^2)$ and are
  therefore usually classified as NLO operators~\cite{Alonso:2012px,
    Buchalla:2013rka}. Second, the function $\mathcal{F}_C(h)$ (and a
  corresponding one for the fermion kinetic terms) can be removed with
  field redefinitions, shifting its physics into the other
  couplings~\cite{Buchalla:2013rka, Brivio:2016fzo}.}
%
\begin{align}
  \lgr{non-linear EFT}
  &\supset \frac 1 2 \partial_\mu h \partial^\mu h \left( 1 + c_H \xi \mathcal{F}_H(h) \right)
    - V(h) \notag \\
  &\phantom{=} \qquad - \frac {v^2} 4 \tr [ V_\mu V^\mu]  \mathcal{F}_C(h) 
    + c_T \xi \frac {v^2} 4 \tr [T V^\mu] \tr [T V_\mu] \mathcal{F}_T(h) \notag \\
  &\phantom{=} \qquad - \frac v {\sqrt{2}} \left[ \sum_f \bar{f}_L U y_f \mathcal{F}_Y^{f}(h) P_f f_R + \hc \right]
  \label{eq:foundations_nonlinear_EFT_LO}
\end{align}
%
with $V_\mu \equiv (D_\mu U) U^\dagger$,
$T \equiv U \sigma_3 U^\dagger$ and projectors
$P_u = (\mathds{1} + \sigma_3) / 2$,
$P_d = P_\ell = (\mathds{1} - \sigma_3)/2$.  The functions
$\mathcal{F}_C(h)$, $V(h)$, $\mathcal{F}_T(h)$, $\mathcal{F}_T(h)$,
$\mathcal{F}_Y^{u}(h)$, $\mathcal{F}_Y^{d}(h)$, and
$\mathcal{F}_Y^{\ell}(h)$ encode the coupling of the Higgs $h$ and are
arbitrary functions. They can be expanded as a power series in $h/f$,
or to simplify the expressions in $h/v$, for instance
%
\begin{equation}
  \mathcal{F}_C(h) = 1 + 2a_C \frac h v + b_C \left(\frac h v \right)^2 + \dots \,.
\end{equation}

At next-to-leading order in the chiral expansion many more terms
relevant for Higgs physics appear. We do not list them here and refer
the interested reader for instance to Reference~\cite{Brivio:2013pma}.

\newparagraph
%
Finally, the relationship between the linear and non-linear effective
theories deserves some discussion. The two approaches in principle
provide different parametrisations of the same physics, as can be
seen by expanding the non-linear Lagrangian in $\xi$ rather then
$\chi$. The difference is the ordering of the operators in the EFT
expansion and, equivalently, the expected size of different
effects. Operators that appear at one order in the $1/\Lambda$
expansion of the linear EFT may appear at a very different order in
the chiral expansion of the non-linear EFT.

Since the symmetry requirements on the non-linear setup are smaller,
we expect it to be more general than the linear Lagrangian at a
comparable order in the expansions. This is exactly what is found when
comparing the linear dimension-six operators to the NLO chiral
Lagrangian: the dimension-six operators predict certain correlations,
while the non-linear description has more operators that can break
these correlations~\cite{Brivio:2013pma}. A straightforward example is
the relation between $hxx$ and $hhxx$ couplings. For dimension-six
operators, it is fixed due to the appearance of
$\phisq \sim (v^2 + 2vh + h^2) / 2$, while these couplings are always
independent in the chiral approach, as can be seen in
\autoref{eq:foundations_nonlinear_EFT_LO}.

The current experimental limits leave room for both strongly or weakly
coupled new physics, for $\xi$ smaller or larger than
$1 / (16 \pi^2)$, for scenarios in which the linear or non-linear
effective theories work better. Only a precise measurement of the
Higgs properties and a global analysis of correlations will tell us
which approach is correct. As a general rule, more SM-like results
favour the linear approach that we follow throughout this
thesis~\cite{Krause:2016uhw}. On the other hand, certain deviations
that do not follow the correlations predicted by dimension-six
operators point towards non-linear physics~\cite{Brivio:2013pma}.



%%%%%%%%%%%%%%%%%%%%%%%%%%%%%%%%%%%%%%%%%%%%%%%%%%%%%%%%%%%%
\subsubsection{$\kappa$ framework}
%%%%%%%%%%%%%%%%%%%%%%%%%%%%%%%%%%%%%%%%%%%%%%%%%%%%%%%%%%%%

Effective field theories are of course not the only way to describe
the Higgs sector. During run~1 of the LHC, the most widely used
parametrisation was the $\kappa$
framework~\cite{LHCHiggsCrossSectionWorkingGroup:2012nn} or the
closely related $\Delta$ framework~\cite{Lafaye:2009vr}. Its
construction is remarkably simple: starting from the SM Higgs sector,
all Higgs couplings are dressed with form factors,
%
\begin{equation}
  g_{hxx} = \kappa_x g_{hxx}^{\text{SM}} = (1 + \Delta_x) g_{hxx}^{\text{SM}} \,.
  \label{eq:foundations_kappa_delta}
\end{equation}
%
Some care has to be taken to treat the Higgs-gluon and Higgs-photon
couplings consistently, where indirect effects of shifted Higgs-top or
Higgs-$W$ couplings compete with direct effects from new
physics~\cite{Lafaye:2009vr}.

From a theoretical point of view, the $\kappa$ framework is not
gauge-invariant and does not present a consistent quantum field
theory. In practice, this means that electroweak loop effects may
introduce divergences that cannot be renormalised.~\comment{Citation?}
This problem can be solved by embedding the $\kappa$ framework in a UV
completion~\cite{Lopez-Val:2013yba}.

From a more phenomenological point of view, this approach is well
suited to parametrise measurements of total rates. Simple shifts of
SM-like Higgs coupling structures are expected in some scenarios of
new physics, for instance in many scalar extensions of the Higgs
sector~\cite{Lopez-Val:2013yba}. But many other models predict new
kinematic features, visible as changed kinematic shapes, and the
$\kappa$ framework is unable to describe these. For better or worse,
it is also agnostic about correlations between different Higgs
couplings, and about correlations between Higgs observables and triple
gauge vertices or electroweak precision measurements.

The strength of the $\kappa$ framework is clearly not its theoretical
foundation. Its allure comes from its simplicity and the fact that it
is designed around the simple question of measuring the couplings of
the (SM-like) Higgs boson. This parametrisation made sense as a
common denominator for the first Higgs measurements with limited
statistics of run~1 of the LHC. But the increased amount of data and
crucial kinematic information collected during run~2 require a
different, more sophisticated language.



%%%%%%%%%%%%%%%%%%%%%%%%%%%%%%%%%%%%%%%%%%%%%%%%%%%%%%%%%%%%
\subsubsection{Pseudo-Observables}
%%%%%%%%%%%%%%%%%%%%%%%%%%%%%%%%%%%%%%%%%%%%%%%%%%%%%%%%%%%%

Higgs pseudo-observables (POs)~\cite{Isidori:2013cga, Bordone:2015nqa,
  Greljo:2015sla} are designed as a generalisation of the $\kappa$
framework to include BSM kinematic features. In a very broad sense,
this term encompasses any number that is field theoretically defined
and can be experimentally accessed~\cite{Krause:2016uhw}. Signal
strengths, cross sections, partial widths, total widths, and
individual form factors or couplings all fall under this umbrella
term. Here we follow the more narrow definition of effective-coupling
PO~\cite{deFlorian:2016spz}. Pseudo-observables are then constructed
process by process by writing down all contributing amplitudes under
some broad assumptions on new physics. These expressions are then
decomposed in a pole expansion, and the resulting residues are
identified as pseudo-observables. This procedure also requires an
expansion in the inverse of the new physics scale $\Lambda$. This
means that just as the EFT approach, pseudo-observables rely on new
physics being heavy, $E \ll \Lambda$. So far, this framework has been
developed for Higgs production in WBF or Higgs-strahlung, as well as
for all phenomenologically relevant Higgs decays.

Phenomenologically, pseudo-observables can describe shifts in SM
couplings as well as kinematic shapes. Like the EFT approach, their
construction requires certain minimal assumptions on the symmetries of
new physics as well as an expansion in $1/\Lambda$. In fact, at tree
level pseudo-observables can be mapped linearly to an EFT constructed
with the same ingredients (which is the non-linear Higgs EFT discussed
below).

The main difference between pseudo-observables and the EFT approach is
a conceptual one. Pseudo-observables are designed from the perspective
of a given process: they describe the coefficients of the different
contributing amplitudes. They are not parameters of a Lagrangian and
do not define a consistent quantum field theory. In particular, the
values of POs measured in one process have no meaning for other
processes. Effective operators on the other hand are a proper,
gauge-invariant quantum field theory that universally describes any
physics below the cutoff scale, and the same Wilson coefficients
predict the behaviour of very different processes.

Proponents of the PO approach favour a multi-layer interpretation of
LHC data, where the data is first presented in terms of
pseudo-observables, which can then be interpreted within EFTs or
specific models of UV physics. They argue that this approach provides
a clear separation between measurement and
interpretation~\cite{deFlorian:2016spz}. On the other hand, proponents
of the ``direct EFT approach'' argue that there is no need for such an
intermediate layer, and suggest to directly fit effective
operators. The debate about which approach is better is still
ongoing~\cite{deFlorian:2016spz}. Ultimately, both effective operators
and POs are well-defined frameworks that can describe all relevant
kinematic effects, and thus present a suitable interface between
experiment and theory.



%%%%%%%%%%%%%%%%%%%%%%%%%%%%%%%%%%%%%%%%%%%%%%%%%%%%%%%%%%%%
\subsubsection{Simplified models}
%%%%%%%%%%%%%%%%%%%%%%%%%%%%%%%%%%%%%%%%%%%%%%%%%%%%%%%%%%%%

All these approach have in common that they assume the absence of new
light particles. Simplified models are designed to close this gap and
to describe kinematic effects from new light resonances. In addition
to resonance peaks these include threshold effects in loops and Higgs
decays into (invisible) new light degrees of freedom. Simplified
models also allow to combine information from direct searches with
indirect measurements. Except for the key element of adding new light
propagating degrees of freedom to the SM, the term is not particularly
well-defined, and there is a lot of freedom to construct such models.

The simplest version of a simplified model consists of the SM
supplemented with another particle, with ad-hoc coupling structures
based on phenomenological requirements~\cite{Biekotter:2016ecg}. Such
a setup might even be not gauge-invariant and thus inconsistent beyond
tree level. At the other end of the spectrum, simplified models can be
consistently defined quantum field theories, potentially involving
higher-dimensional operators. The only difference to the linear and
non-linear EFT approaches discussed above is the extended particle
content. The additional flexibility of course comes at the price of an
increased number of parameters.

Examples of such models for Higgs physics include an extended Higgs
sector with an additional singlet and a doublet, which offers great
flexibility to tune the Higgs couplings~\cite{Lopez-Val:2013yba}. The
authors of Reference~\cite{Dolan:2016eki} develop a model with an
additional singlet and vector-like quarks. Finally,
References~\cite{Gripaios:2016xuo, Bauer:2016hcu} discuss additional
scalar singlets supplemented with higher-dimensional operators.



%%%%%%%%%%%%%%%%%%%%%%%%%%%%%%%%%%%%%%%%%%%%%%%%%%%%%%%%%%%%
\subsubsection{Comparison}
%%%%%%%%%%%%%%%%%%%%%%%%%%%%%%%%%%%%%%%%%%%%%%%%%%%%%%%%%%%%

\begin{table}
  \renewcommand{\arraystretch}{1.8}
  \footnotesize
  \begin{tabularx}{\textwidth}{LLLLLL} 
    \toprule 
    %
    & $\kappa$ framework & POs & Non-linear EFT & Linear EFT & Simplified models \\
    \midrule
    %
    Motivation & 
    experiment: \newline simplest Higgs parametrisation &
    experiment: \newline amplitude decomposition for a given process &
    theory: \newline complete low-energy effects of NP with singlet $h$& 
    theory: \newline complete low-energy effects of NP with doublet $\phi$ &
    exp.\,/\,theory:\newline new light particles \\
    %
    Input & 
    SM Higgs couplings &
    process amplitudes, \newline pole expansion,  \newline NP expansion ($1/\Lambda $)&
    SM particles ($h$), \newline symmetries, \newline counting scheme (loops) & 
    SM particles ($\phi$), \newline symmetries, \newline counting scheme ($1/\Lambda $) & 
    new particles (masses, charges, interactions) \\
    %
    Parameters & 
    coefficients of SM amplitude &
    coefficients of amplitudes &
    Lagrangian parameters of consistent QFT & 
    Lagrangian parameters of consistent QFT &
    depends \\
    %
    Validity conditions & 
    SM-like &
    NP heavy, symmetries &
    NP heavy, symmetries & 
    NP heavy, symmetries &
    single light new particles, other NP decouples \\
    \midrule 
    %
    Shifted SM couplings & 
    yes &
    yes &
    yes & 
    yes &
    depends \\
    %
    Kinematic effects & 
    no &
    yes &
    yes & 
    yes &
    depends \\
    %
    New resonances, \newline loop thresholds, \newline invisible decays & 
    no &
    no &
    no & 
    no &
    yes \\
    %
    Correlations & 
    no &
    no &
    some & 
    many &
    depends \\
    % \midrule
    % %
    % WBF parameters &
    % $\kappa_W$
    % &
    % $\kappa_{WW}$, $\varepsilon_{WW}$, $\varepsilon_{WW}^{CP}$, $\varepsilon_{Wu_L}$ &
    % &
    % $\frac {f_{\phi,2}} {\Lambda^2}$, $\frac {f_{W}} {\Lambda^2}$, $\frac {f_{WW}} {\Lambda^2}$, $\frac {f_{W\widetilde{W}}} {\Lambda^2}$ & 
    % depends \\
    %                                                                                                
    \bottomrule
  \end{tabularx}
  \caption[Comparison between different parametrisations of Higgs properties]{Comparison
    between different parametrisations of Higgs properties. The upper part of the table
    focuses on the theoretical foundation, the lower on the phenomenology.
% Finally, we list some parameters relevant for WBF Higgs
% production in the different approaches, restricting this example to
% the dominant $W$-mediated amplitude.
    Since ``simplified models'' describe a rather general idea, many
    details depend on the specific realisation.}
  \label{tbl:foundations_framework_comparison}
\end{table}

All these approaches define parametrisations of the Higgs properties
that can be used as interfaces between different measurements and
between experiment and theory. In
\autoref{tbl:foundations_framework_comparison} we summarise and
compare their different properties.

The different frameworks can be classified into consistent quantum
field theories, which include the EFTs, and process-based
parametrisations of amplitudes through form factors. The $\kappa$
framework and pseudo-observables belong to the latter
category. Simplified models can fall into either category. The QFT
formalism allows to link different processes and to incorporate any
loop effects.

More important for practical purposes is the range of phenomena that
can be described. The $\kappa$ framework is limited to rescalings of
the SM Higgs couplings and is not able to incorporate kinematic
information. Pseudo-observables and the two EFT approaches are much
more flexible and can describe a large number of kinematic
features. However, they rely on new physics being substantially
heavier than the experimentally probed energies around the weak
scale. Features from light new particles, for instance resonances or
loop thresholds, are only covered by appropriate simplified
models. The dimension-six operators of linear Higgs EFT and to a
lesser extent the leading operators of the chiral EFT also predict
certain correlations between different couplings and measurements,
while by definition the pseudo-observables are only valid for a given
process.

To summarise, in the absence of new light particles, Higgs properties
can be adequately parametrised by pseudo-observables, non-linear
Higgs EFT, and linear Higgs EFT. The linear EFT approach is
theoretically well-motivated and phenomenologically powerful, and we
focus on this framework during this thesis.



% Maybe the simplest model of new physics is the extension of the SM by one real scalar singlet, also known as a ``Higgs portal''. This new field $s$ only couples to the Higgs doublet $\phi$ proportional to its mass $m_s$ times a coupling $\lambda_s$. Identifying $\Lambda = m_s$ and integrating out the singlet gives rise to the diagram
% \begin{align}
%   \raisebox{-0.5\height}{
%     %\fbox{
%       \fmfframe(0,5)(-1,5){ %(L,T) (R,B)
%         \begin{fmfgraph*}(35,20)
%           \fmfleft{i2,i1}
%           \fmfright{o2,o1}
%           \fmflabel{$\phi^\dagger$}{i1}
%           \fmflabel{$\phi$}{i2}
%           \fmflabel{$\phi^\dagger$}{o1}
%           \fmflabel{$\phi$}{o2}
%           \fmf{dashes}{i1,v1,i2}
%           \fmf{dashes}{o1,v2,o2}
%           \fmf{dbl_dashes,tension=1,label=$s$}{v1,v2}
%           \marrow{m}{up}{top}{$p$}{v1,v2}
%         \end{fmfgraph*}
%       }
%     %}
%   }
%   &\sim (\phisq) \frac {m_s^2 \lambda_s^2}{p^2 - m_s^2} (\phisq) \notag \\
%   {} &\sim  \lambda_s^2  (\phisq) \left(1 + \frac {p^2} {\Lambda^2} + \ord{1/\Lambda^4}\right) (\phisq) \notag \\
%   {} &\sim  \lambda_s^2  (\phisq) \left(1 - \frac {\partial^2} {\Lambda^2} + \ord{1/m_s^4}\right) (\phisq) \\
%   {} &\sim  \lambda_s^2  (\phisq)^2 + \frac {\lambda_s^2} {\Lambda^2} \, \partial_\mu (\phisq) \partial^\mu (\phisq) \,.
% \end{align}

% The first term is an unobservable renormalisation of an SM operator, but the second one is just $\ope{\phi2}$ with Wilson coefficient $f_{\phi 2} = 2\lambda_s^2$. At tree level, this is the only operator generated by this model.