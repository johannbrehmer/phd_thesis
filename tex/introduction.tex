
\setchapterpreamble[ur]{%
\dictum[T.~Plehn~\cite{plehn_defense}]{%
Did you know that in my PhD defence, I didn't answer a single question correctly?}%
\vspace*{2cm}}
 
%%%%%%%%%%%%%%%%%%%%%%%%%%%%%%%%%%%%%%%%%%%%%%%%%%%%%%%%%%%%%%%%%%%%%%%%%%%%%%%%
\chapter{Introduction}
\label{chapter:Introduction}
%%%%%%%%%%%%%%%%%%%%%%%%%%%%%%%%%%%%%%%%%%%%%%%%%%%%%%%%%%%%%%%%%%%%%%%%%%%%%%%%

The Higgs boson~\cite{Higgs:1964ia, Higgs:1964pj, Englert:1964et} is a
key element of the Standard Model of particle physics (SM). Its
discovery in 2012~\cite{Aad:2012tfa, Chatrchyan:2012xdj} completed the
particle zoo of the SM. It is a triumph of a decade-old model, but it
also offers us a way forward: the Higgs provides us with an
unprecedented chance to understand some of the biggest unsolved
mysteries of physics.

As the only known fundamental scalar, it suffers from the famous
electroweak hierarchy problem: why is its mass scale (and therefore
the electroweak scale) so much smaller than the Planck scale, while
there is no sign of a symmetry protecting it against quantum
corrections? Is the electroweak vacuum, defined by the Higgs
potential, stable?  Why are the Yukawa couplings, and consequently the
fermion masses, spread out over so many orders of magnitude?  Is the
Higgs related to Dark Matter, for instance as mediator to a dark
sector? Or might it even be the source of inflation, explaining the
surprising level of isotropy in the Cosmic Microwave Background?

Many models of physics beyond the Standard Model have been proposed to
answer at least some of these questions. Very often they predict Higgs
coupling patterns different from the SM. A precise measurement of the
Higgs properties thus provides a crucial probe of such models, and
might be one of the most important missions for present and future
runs of the Large Hadron Collider (LHC).

This poses two immediate questions:
%
\begin{enumerate}
\item Which framework should be used to parametrise the Higgs
  properties?
\item How can these parameters be measured efficiently at the LHC
  experiments?
\end{enumerate}
%
These two questions drive the research presented in this thesis, and
we will tackle them one by one.

\newparagraph
%
Ideally, all Higgs measurements should use the same universal language
to parametrise their results, allowing for an efficient comparison
and combination. Such a framework should be general enough to describe
the effects of any interesting new physics scenario without strong model
assumptions. On the other hand, too large a number of parameters makes
combinations of different experiments and global fits impractical.

A simple example for such a universal parametrisation is the $\kappa$
framework, which was widely used during run~1 of the LHC. It is based
on the SM Lagrangian, but promotes all Higgs couplings to free
parameters. There are several issues with this approach: it is not
gauge-invariant, and it can only describe structures that are already
present in the SM. So while a measurement based on the $\kappa$
framework can be useful for total rates, it will not be able to
utilise information in kinematic distributions.

Instead, we work in an approach based on effective field theory
(EFT)~\cite{Coleman:1969sm, Callan:1969sn, Weinberg:1980wa}. Based
only on the assumption that new physics has a typical energy scale
significantly larger than the experimental energies, all new physics
effects are captured by a tower of higher-dimensional operators. The
leading effects for Higgs physics should come from only a handful of
operators with mass dimension 6~\cite{Burges:1983zg, Leung:1984ni,
  Buchmuller:1985jz}. These operators are manifestly gauge-invariant
and can be used beyond tree level. They describe both coupling
rescalings as well as novel kinematic structures not present in the
SM, allowing us to access information in distributions in addition to
total rates~\cite{Corbett:2012ja, Corbett:2015ksa}. Effective
operators also let us combine Higgs data with results from other
experiments, including electroweak precision data or gauge boson
production at the LHC~\cite{Butter:2016cvz}. However, the limited
precision of the LHC Higgs measurements means that only models that
are either strongly coupled or relatively light can be probed. In the
latter case, the characteristic energy scale of new physics is not
sufficiently separated from the momentum transfers in the experiments,
casting doubt on the validity of the EFT approach.

We analyse the usefulness of higher-dimensional operators at the LHC
by comparing the predictions of UV-complete scenarios of new physics
to their dimension-6 approximations~\cite{Brehmer:2015rna}. Our
analysis covers additional scalar singlets, two-Higgs-doublet models,
scalar top partners, and heavy vector bosons, focusing on parameter
ranges that the LHC will be sensitive to. We take into account rates
and distributions in the most important Higgs production modes and
various representative decay channels as well as Higgs pair
production. For this array of models, benchmark points, and
observables, we ask if and where the effective description of new
physics breaks down, and how it can be improved.

As it turns out, the agreement between the approaches crucially
depends on the matching procedure that links the coefficients of the
dimension-6 model to the full theory. We introduce $v$-improved
matching, a procedure that improves the performance of the dimension-6
model by resumming certain terms that arise during electroweak
symmetry breaking. While formally of higher order in the EFT
expansion, these effects can be large under LHC conditions. With such
a matching, the effective model provides a good description even in
many scenarios where the EFT validity is not obvious. We then discuss
a number of practical questions on the role of squared dimension-6
terms in the differential cross sections, effects on fits, and the
correlation between different observables and the momentum
transfer~\cite{Biekotter:2016ecg}.

\newparagraph
%
Having established that Higgs EFT works well as a largely
model-independent language for Higgs physics at the LHC, the
next question is how its parameters can be measured optimally. Higgs
measurements are affected by many different operators, and each of
them affects different couplings, often introducing non-trivial
kinematic structures. This leads to a complicated relation between the
high-dimensional model parameter space and often also high-dimensional
phase spaces.

Traditional analyses based on selection cuts and histograms of
kinematic observables are often not sensitive to such subtle signatures.
At the other end of the spectrum, experiments resort more and more to
high-level statistical tools, including machine learning techniques or
the matrix element method~\cite{Kondo:1988yd, Atwood:1991ka,
  Gao:2010qx, Bolognesi:2012mm, Cranmer:2015nia}. Many of these tools
are designed for the comparison between two discrete hypotheses, and
applying them to high-dimensional parameter spaces such as Higgs EFT
is computationally expensive. Extending machine learning techniques to
such high-dimensional spaces is a current area of
research~\cite{Cranmer:2015bka}. These multivariate techniques are
powerful, but can be non-transparent. It is therefore increasingly
important to be able to characterise the information contained in LHC
signatures.

We use information geometry to understand and optimise Higgs
measurements~\cite{Brehmer:2016nyr}.  The central building block is
the Fisher information, which according to the Cram\'er-Rao bound
encodes the maximal knowledge on theory parameters we can derive from
an experiment. Unlike many other statistical tools, the Fisher
information is intrinsically designed for continuous, high-dimensional
parameter spaces, and this approach does not require any
discretisation of the theory space and leads to results that do not
depend on arbitrary parameter or basis choices. In addition, the
Fisher information defines a metric on the model parameter space. This
not only provides an intuitive geometric picture of the sensitivity of
measurements, but also allows us to track the impact of higher
orders in the EFT expansion.

We calculate the Fisher information for Higgs production in weak boson
fusion with decays into tau pairs and four leptons, and for Higgs
production in association with a single top quark. To this end, we
develop an algorithm to calculate the Fisher information in
particle-physics observables based on Monte-Carlo methods. Our results
give the maximum precision with which dimension-6 operators can be
measured in these processes, or the maximal new physics reach of these
signatures. We then analyse how the differential information is
distributed over phase space, which defines optimal event
selections. In a next step, we calculate the information in individual
kinematic distributions, and compare it to the maximal information in
the full event kinematics. This provides a ranking of the most
powerful production and decay observables. It allows us to compare how
much we can learn from a simple fit to histograms compared to fully
multivariate methods.

This is the first application of information geometry to high-energy
physics. While there is no shortage of statistical tools in the field,
these new methods can help to plan and optimise measurement strategies
for high-dimensional continuous models in an intuitive but powerful
way. While we demonstrate this approach in different Higgs channels
for dimension-6 operators, our tools can easily be translated to other
processes and models.

\newparagraph
%
This thesis begins by recapitulating some basic ideas of Higgs physics
and effective field theory in Chapter~\ref{chapter:foundations}. In
Chapter~\ref{chapter:validity}, we discuss the validity of effective
field theory for LHC measurements and the matching between full models
and effective operators. Chapter~\ref{chapter:information} presents
our work on information geometry and optimal Higgs measurements. Both
of these chapters will contain separate and more detailed introductions
and conclusions. We summarise the results in
Chapter~\ref{chapter:conclusions}.
