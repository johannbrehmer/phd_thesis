
% \setchapterpreamble[ur]{%
%   \dictum[L.~Carroll~\cite{carroll1898alice}]{%
%     `Begin at the beginning,' the King said, very gravely, `and go
%     on till you come to the end: then stop.' }%
%   \vspace*{2cm}}

%%%%%%%%%%%%%%%%%%%%%%%%%%%%%%%%%%%%%%%%%%%%%%%%%%%%%%%%%%%%
\chapter{Introduction}
\label{chapter:Introduction}
%%%%%%%%%%%%%%%%%%%%%%%%%%%%%%%%%%%%%%%%%%%%%%%%%%%%%%%%%%%%

\firstword{T}{he Higgs boson}~\cite{Higgs:1964ia, Higgs:1964pj,
  Englert:1964et, Guralnik:1964eu, Higgs:1966ev} is the cornerstone
% centrepiece
of the Standard Model of particle physics (SM)~\cite{Glashow:1961tr,
  Weinberg:1967tq, Salam:1968rm}. Its experimental discovery in
2012~\cite{Aad:2012tfa, Chatrchyan:2012xdj}
%
%completed the particle zoo
%of the SM. It
%
% This feat of experimental particle physics
%
is a triumph for a decades-old model, but it also defines a way
forward: the Higgs provides us with an unprecedented chance to
understand some of the biggest unsolved mysteries of physics.

As the only known fundamental scalar, it suffers from the famous
electroweak hierarchy problem: why is its mass scale (and therefore
the electroweak scale) so much smaller than the Planck scale, while
there is no sign of a symmetry protecting it against quantum
corrections? In addition, the Higgs sector is intimately tied to the
stability of the electroweak vacuum and to the unexplained large
hierarchy between the fermion masses. It might also be related to
the open questions of the baryon asymmetry, of the nature of dark
matter, and of inflation.

Many models of physics beyond the Standard Model have been proposed to
explain at least some of these aspects. Often they predict Higgs
coupling patterns different from the SM. A precise measurement of
the Higgs properties thus provides a crucial probe of such models, and
might be one of the most important goals for present and future
runs of the Large Hadron Collider (LHC).

This prospect poses two immediate phenomenological questions:
%
\begin{enumerate}
\item Which framework should be used to parametrise the Higgs
  properties?
\item How can these parameters be measured efficiently at the LHC?
\end{enumerate}
%
% These two topics drive the research presented in this thesis, and we
% will address them one by one.
%
The research presented in this thesis consists of two major parts,
each driven by one of these questions.
%
% These two topics drive the research presented in this thesis, and
% each is addressed in a dedicated chapter.

\newparagraph
%
Regarding the first point, all Higgs measurements should ideally use
the same universal language to parametrise their results, allowing for
an efficient comparison, combination, and interpretation. Such a
framework should be general enough to describe the effects of many
interesting new physics (NP) scenarios without strong model
assumptions.  On the other hand, practical considerations such as
computational resources limit the number of parameters.
%
% On the other hand, too large a number of parameters makes
% statistical analyses and the combinations of different experiments in
% global fits impractical.

A simple example for such a universal parametrisation that was widely
used during Run~1 of the LHC is the $\kappa$ framework. It is based on
the SM Lagrangian, but promotes all Higgs couplings to free
parameters. The main limitation of this approach is that it can only
describe structures that are already present in the SM. While a
measurement based on the $\kappa$ framework can be useful for total
rates, it is not able to utilise information in kinematic
distributions.

Instead, we parametrise new physics signatures with an effective field
theory (EFT)~\cite{Coleman:1969sm, Callan:1969sn,
  Weinberg:1980wa}. Based only on the assumption that new physics has
a typical energy scale significantly larger than the experimental
energies, all new physics effects are captured by a tower of
higher-dimensional operators. The leading effects for Higgs physics
should come from a handful of operators with mass dimension
six~\cite{Burges:1983zg, Leung:1984ni, Buchmuller:1985jz}. These
operators
%
% are manifestly gauge-invariant and can be used beyond tree
% level. They
%
describe both coupling rescalings as well as novel kinematic
structures not present in the SM, allowing us to access information in
distributions in addition to total rates~\cite{Corbett:2012ja,
  Corbett:2015ksa}. Effective operators also let us combine Higgs
measurements with constraints from other processes, including
electroweak precision data or gauge boson production at the
LHC~\cite{Butter:2016cvz}.

However, the limited precision of LHC Higgs measurements is only
sensitive to signatures from models that are either strongly coupled
or relatively light. In the latter case, the characteristic energy
scale of new physics is not clearly separated from the momentum
transfer in the experiments, casting doubt on the validity of the EFT
approach.

We analyse the usefulness of higher-dimensional operators at the LHC
by comparing the predictions of specific scenarios of new physics to
their dimension-six approximations~\cite{Brehmer:2015rna}. Our
analysis covers additional scalar singlets, a two-Higgs-doublet model,
scalar top partners, and heavy vector bosons, focusing on parameter
ranges that the LHC will be sensitive to. We take into account rates
and distributions in the most important Higgs production modes and
representative decay channels as well as in Higgs pair production. For
this array of models, benchmark points, and observables, we ask if and
where the effective description of new physics breaks down, and how it
can be improved.

As it turns out, the performance of the effective model strongly
depends on the matching procedure that links the coefficients of the
dimension-six operators to the full theory. We analyse how electroweak
symmetry breaking affects the validity of the effective theory, and
discuss how the standard matching procedure can be adapted to
situations where these effects are large. In addition, we discuss
whether squared amplitudes from dimension-six operators should be
included in calculations, and which observables provide the best
probes of the momentum transfer in Higgs production in weak boson
fusion~\cite{Biekotter:2016ecg}.

\newparagraph
%
Having chosen a parametrisation of the Higgs properties, the second part
of this thesis focuses on the question of how its parameters can be
measured optimally. Higgs processes are sensitive to many effective
operators. Each of them affects different couplings, typically
introducing non-trivial kinematic structures. This leads to a
complicated relation between the high-dimensional model parameter
space and often also high-dimensional phase spaces.

Conventional analyses based on selection cuts and histograms of
kinematic observables are in many cases not sensitive to such subtle
signatures.  At the other end of the spectrum, experiments resort more
and more to high-level statistical tools, including machine learning
techniques~\cite{Cranmer:2015bka, Louppe:2016ylz, Louppe:2016aov,
  Cranmer:2016lzt, Baldi:2016fzo, Brehmer:ghost_probability,
  Cogan:2014oua, Baldi:2014pta, deOliveira:2015xxd, Almeida:2015jua,
  Baldi:2016fql, Guest:2016iqz, Komiske:2016rsd, Kasieczka:2017nvn,
  Louppe:2017ipp, Baldi:2014kfa, Searcy:2015apa, Santos:2016kno,
  Alves:2016htj, Buckley:2011kc, Bornhauser:2013aya, Bechtle:2017vyu}
or matrix-element-based methods~\cite{Kondo:1988yd, Abazov:2004cs,
  Gao:2010qx, Alwall:2010cq, Avery:2012um, Andersen:2012kn,
  Campbell:2013hz, Artoisenet:2013vfa, Martini:2015fsa,
  Gritsan:2016hjl, Soper:2011cr, Soper:2012pb, Soper:2014rya,
  Atwood:1991ka, Davier:1992nw, Diehl:1993br}. While these
multivariate techniques are powerful, it is often not transparent
which physical properties they probe. It is therefore increasingly
important to be able to characterise the information contained in LHC
signatures.

%We introduce new statistical tools based on information geometry that
%can guide the measurement of continuous theory parameters such as Higgs
%properties~\cite{Brehmer:2016nyr}.
%
We use information geometry~\cite{efron1975, amari1982, amari2000joho}
to understand and optimise the measurement of Higgs properties at the
LHC~\cite{Brehmer:2016nyr}.
%
The central building block of our approach is the Fisher information
matrix, which according to the Cram\'er-Rao bound~\cite{Rao:1945,
  Cramer:1946} encodes the maximal knowledge on theory parameters we
can derive from an experiment. We show that the properties of the
Fisher information make it particularly well-suited to continuous,
high-dimensional parameter spaces, and in particular to effective
field theories.

We develop an algorithm to calculate the Fisher information in
particle physics processes based on Monte-Carlo methods. It allows us
to calculate the maximum precision with which continuous parameters
can be measured in a process. We also analyse how the differential
information is distributed over phase space and how much information
is carried by individual kinematic distributions. This defines the
most powerful phase-space regions and observables for an analysis. It
also allows us to compare how much we can learn from a simple fit to
histograms compared to fully multivariate methods.

These new instruments are then applied to Higgs measurements in three
different channels. We calculate the information on dimension-six
operators in Higgs production in weak boson fusion with decays into
tau pairs and four leptons, and in Higgs production in association
with a single top quark. Finally, we show how our approach can be
extended to include systematic and theory uncertainties, and compare
it to the likelihood ratio.

\newparagraph
%
This thesis begins with a synopsis of essential aspects of Higgs
physics and effective field theories in
\autoref{chapter:foundations}. In \autoref{chapter:validity}, we
discuss the validity of effective field theory for LHC Higgs
measurements and the matching between full models and effective
operators. \autoref{chapter:information} presents our work on
information geometry and efficient measurements of Higgs
properties. Both these chapters contain separate, detailed
introductions and conclusions. We summarise our results in
\autoref{chapter:conclusions}. In a set of appendices we list our
conventions, explain technical details, and provide additional
examples.
