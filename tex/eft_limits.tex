
\setchapterpreamble[ur]{%
\dictum[G.~Box~\cite{box1987empirical}]{%
All models are wrong, but some are useful.}%
\vspace*{2cm}}



%%%%%%%%%%%%%%%%%%%%%%%%%%%%%%%%%%%%%%%%%%%%%%%%%%%%%%%%%%%%
\chapter{Higgs effective theory at its limits}
\label{chapter:validity}
%%%%%%%%%%%%%%%%%%%%%%%%%%%%%%%%%%%%%%%%%%%%%%%%%%%%%%%%%%%%

\firstword{H}{iggs effective theories} can universally describe all
signatures expected from new physics as long as its mass scale
$\Lambda$ is sufficiently separated from the experimentally accessible
scale. However, the limited precision of the LHC means that Higgs
measurements are often only sensitive to models with scales not much
higher than the electroweak scale. In this chapter we discuss if and
when the EFT approach is nevertheless useful, and how its validity can
be improved.

In \autoref{sec:validity_introduction} we discuss the LHC
sensitivity and its implications for effective theories, and formulate
our strategy to test the dimension-six approach. In
\autoref{sec:validity_matching} we discuss some crucial details of
the matching between full models and EFTs, and introduce the novel
``$v$-improved'' matching
procedure. \autoref{sec:validity_full_vs_effective} contains the
bulk of our results, the comparison of full models to their EFT
approximations for a variety of models and observables. We go into
more detail and analyse some practical questions in a specific example
case in \autoref{sec:validity_practical_questions}, and finally
present our conclusions in \autoref{sec:validity_conclusions}.

Most of the work presented in this chapter was previously published in
Reference~\cite{Brehmer:2015rna}, while the content of
\autoref{sec:validity_practical_questions} was published in
Reference~\cite{Biekotter:2016ecg}. A part of the content of this
chapter was also included in
Reference~\cite{deFlorian:2016spz}. Nearly all of the results and most
of their presentation in this chapter\,---\,including all plots, most
of the tables, and a significant part of the text\,---\,are identical
to that in these three publications.



%%%%%%%%%%%%%%%%%%%%%%%%%%%%%%%%%%%%%%%%%%%%%%%%%%%%%%%%%%%%
\section{Introduction}
\label{sec:validity_introduction}
%%%%%%%%%%%%%%%%%%%%%%%%%%%%%%%%%%%%%%%%%%%%%%%%%%%%%%%%%%%%

%%%%%%%%%%%%%%%%%%%%%%%%%%%%%%%%%%%%%%%%%%%%%%%%%%%%%%%%%%%%
\subsection{Energy scales in Higgs measurements}
%%%%%%%%%%%%%%%%%%%%%%%%%%%%%%%%%%%%%%%%%%%%%%%%%%%%%%%%%%%%

There is no doubt that effective field theories work extraordinarily
well as long as there is a scale hierarchy between the experimentally
probed energy and the new physics scale, $E \ll \Lambda$. On the other
hand, it is clear that the EFT expansion breaks down at
$E \geq \Lambda$, where an infinite number of operators contributes at
the same size and the model is neither predictive nor
renormalisable. The validity of the effective theory in the
intermediate region,
%
\begin{equation}
  E \lesssim \Lambda \,,
  \label{eq:validity_muddy_waters}
\end{equation}
%
is less obvious and will depend on the specific underlying model as
well as on the observable studied. So in which of these categories do
LHC Higgs measurements belong?

Higgs production at hadron colliders does not probe a single
experimental energy scale over the full relevant phase space. The
momentum transfer is bounded from below by the Higgs mass. Selection
criteria necessary to separate a signal from the QCD backgrounds often
require a higher momentum transfer $E > m_h$. More importantly, most
of the information on operators with derivatives comes from
high-energy tails, as demonstrated in
\autoref{sec:foundations_heft_pheno}. During run~1, significant
event numbers are recorded within the range
%
\begin{equation}
  m_h \leq E \lesssim \ord{400~\gev} \,,
  \label{eq:validity_experimental_energy_scale}
\end{equation}
%
depending on the process, observable, collected amount of data, and
analysis methods.

On the other hand, we can roughly estimate the new physics scale
$\Lambda$ that LHC Higgs measurements are able to probe. Assuming a
$10\%$ precision on total Higgs rate measurements and no loop
suppression of new physics effects, such a signature lies within the
experimental reach of the LHC if
%
\begin{equation}
  \left| \frac{\sigma \times \text{BR}}{\left( \sigma \times \text{BR} \right)_\text{SM}} - 1 \right|
  = \frac{g^2 m_h^2}{\Lambda^2}
  \gtrsim 10\%
  \qquad \Leftrightarrow \qquad 
  \Lambda < \frac{g \, m_h}{\sqrt{10\%} } \, \approx g \, 400~\gev \,.
  \label{eq:validity_probed_scale_estimation}
\end{equation}
%
where $g$ are the typical couplings of the underlying theory.
%
% For loop-induced new physics effects, the corresponding loop
% suppression factor pulls $\Lambda$ to even lower values,
% %
% \begin{equation}
%   \left| \frac{\sigma \times \text{BR}}{\left( \sigma \times \text{BR} \right)_\text{SM}} - 1 \right|
%   = \frac{ g^2 m_h^2}{16 \pi^2 \Lambda^2}
%   \gtrsim 10\%
%   \qquad \Leftrightarrow \qquad 
%   \Lambda < \frac{g \, m_h}{4 \pi \sqrt{10\%} } \, \approx g \,30~\gev \,.
%   \label{eq:validity_probed_scale_estimation_loops}
% \end{equation}

This simple estimation shows that the scale separation $E/\Lambda$ is
limited by the experimental precision and crucially depends on the
size of the couplings of underlying physics. For very weakly coupled
theories, $g^2 < 1/2$, only new physics models with new particles at
or below the electroweak scale can leave measurable signatures in
Higgs observables, and the EFT approach clearly does not make
sense. For truly strongly coupled theories, $1 < g \lesssim 4 \pi$,
new physics scenarios up to $\Lambda \lesssim 5~\tev$ are relevant,
and the EFT expansion converges flawlessly. In fact, the EFT approach
to Higgs observables has largely been motivated by the desire to
describe models with strongly interacting electroweak symmetry
breaking~\cite{Giudice:2007fh}. For moderately weakly to moderately
strongly coupled theories, $1/2 \lesssim g^2 \lesssim 2$, the LHC
Higgs programme is sensitive to scales
%
\begin{equation}
  280~\gev \lesssim \Lambda \lesssim 560~\gev
  \label{eq:validity_probed_scale_weakly_interaction}
\end{equation}
%
This corresponds exactly to the intermediate region defined in
\autoref{eq:validity_muddy_waters}.

In this simple argument we ignored that new physics might also change
distributions and especially affect the high-energy tails or off-shell
regions.
%
% In particular the increased statistics and Higgs production cross
% sections at Run~II will enable us to add a wide range of
% distributions and off-shell processes to the Higgs observables.  A
% well-known example is weak boson fusion, where the details of the
% ultraviolet completion can have a huge effect for example on the
% transverse momenta of the tagging jets~\cite{bad_one, spins,
% phi_jj,higgs_pole}.
%
A thorough global fit of Higgs results including kinematic information
confirms the rough estimate given in
\autoref{eq:validity_probed_scale_weakly_interaction}~\cite{Corbett:2015ksa}.

We conclude that for moderately weakly coupled scenarios of new
physics, the limited precision of LHC Higgs measurements cannot
guarantee a clear scale hierarchy, and the effective field theory
approach cannot be trusted blindly. But this does not mean that an
analysis of LHC data in terms of a truncated dimension-six Lagrangian
cannot be useful. Instead, the applicability of the dimension-six
model now depends on the nature of the underlying physics as well as
on the process and observable, and has to be carefully checked for
each situation.

From a practical perspective, in Reference~\cite{Corbett:2015ksa} is
has been shown that a fit of dimension-six operators to the Higgs data
at Run~I is a sensible and practicable extension of the usual Higgs
couplings fit.  As discussed at length in the previous chapter,
dimension-six operators including derivatives complement the Higgs
coupling modifications and allow us to extract information from
kinematic distributions. Even if the LHC constraints do not induce a
hierarchy of scales, and the EFT approach is not formally well
defined, there appears to be no problem in using the truncated
dimension-six Lagrangian as a phenomenological model to describe the
LHC Higgs data. This description induces theory uncertainties if we
want to interpret the LHC results in terms of an effective field
theory~\cite{Berthier:2015gja}.



%%%%%%%%%%%%%%%%%%%%%%%%%%%%%%%%%%%%%%%%%%%%%%%%%%%%%%%%%%%%
\subsection{Questions for the EFT approach}
%%%%%%%%%%%%%%%%%%%%%%%%%%%%%%%%%%%%%%%%%%%%%%%%%%%%%%%%%%%%

In this chapter we analyse the usefulness of the dimension-six
description of the Higgs sector with a comprehensive comparison of
full models and their dimension-six description. As argued above, the
EFT validity has to be tested on a process-to-process as well as
model-to-model basis.  We therefore select four represantative models
of new physics, map them onto dimension-six operators, and compare the
predictions of the full and the effective model in various Higgs
observables.

We select moderately weakly interacting extensions of the Higgs sector
of the Standard Model by
%
\begin{enumerate}
\item a scalar singlet,
\item a scalar doublet,
\item a coloured top-partner scalar, and
\item a massive vector triplet.
\end{enumerate}
%
For each of these models we pick a number of parameter benchmark
points, designed to highlight phenomenogical features of the model,
and to be within the experimental reach of the LHC Higgs programme.

The corresponding EFT descriptions are constructed by integrating out
the heavy fields and expanding the effective action to
$\ord{1/\Lambda^2}$. In other words, we match the theories to the
dimension-six operators of the linear Higgs EFT introduced in
\autoref{sec:foundations_heft_operators}. The details of this
matching procedure are a key element of our analysis and will be
discussed in detail in the next section.

For all these scenarios, we calculate the Higgs couplings, and in a
next step rates and distributions for selected Higgs production modes
and decay channels. The key questions we aim to address are:
%
\begin{itemize}
\item Which masses and coupling predict Higgs signatures relevant for
  the LHC? Is the corresponding new physics scale sufficiently
  separated from $v$?
%
\item Which observables are correctly described by dimension-six
  model?  Where does the EFT description break down, and why?
%
\item Does this breakdown pose a problem for LHC analyses?
\end{itemize}
%
In this way, we analyse what problems the lack of a clear hierarchy of
scales leads to in practice, and discuss how these might affect global
LHC-Higgs fits including kinematic distributions. Turning the argument
around, we ask whether and when the analysis of a UV-complete model
offers an advantage compared to the effective theory.

% It will turn out that two limitations of the EFT description will
% guide us through the different models. First, we need to ensure that
% the new physics scale and with it all new particles are properly
% decoupled, in particular when we go beyond total cross
% sections. Second, when we define our effective field theory in terms
% of a Higgs-Goldstone doublet, it is crucial that the electroweak
% vacuum expectation value (VEV) does not have a destabilising effect on
% the hierarchy of scales.

\newparagraph
%
After this broad survey of the applicability of dimension-six
operators, we focus on the vector triplet scenario and WBF Higgs
production and briefly discuss some practical aspects. The first
question is whether it is justified or preferable to include the
square of dimension-six operators in calculations while neglecting
dimension-eight operators interfering with the SM. Both contribute at
the same order $\ord{1/\Lambda^4}$ to the squared amplitude.

We also discuss is whether a naive simplified model with a new light
scalar can improve the description where the EFT breaks down, analyse
the correlation of different kinematic observables with the
unobservable momentum transfer in WBF Higgs production, and check
whether the differences between full and effective descriptions
survive a realistic parton shower and jet reconstruction.

\newparagraph
%
In addition and mostly simultanously to our work, published in
References~\cite{Brehmer:2015rna, Biekotter:2016ecg}, the
applicability of EFTs to Higgs physics at the LHC was studied in a
range of different situations~\cite{Biekoetter:2014jwa,
  Arnesen:2008fb, Englert:2014cva, deVries:2014apa, Craig:2014una,
  Dawson:2015gka, Edezhath:2015lga, Gorbahn:2015gxa,
  Edelhaeuser:2015zra, Drozd:2015kva, Englert:2015hrx,
  Contino:2016jqw, Freitas:2016iwx, deFlorian:2016spz}. The
differences to our work lies in the considered new physics scenarios
and observables. A lot of attention focussed on Higgs production in
weak boson fusion and its sensitivity to UV
physics~\cite{Alwall:2007ed, Hagiwara:2009wt, Englert:2012xt,
  Brehmer:2014pka}. Similar points were discussed for
Higgs-strahlung~\cite{Biekoetter:2014jwa}, in the production of
(potentially off-shell) Higgs bosons in gluon
fusion~\cite{Azatov:2014jga, Buschmann:2014sia, Dawson:2015gka,
  Drozd:2015kva, Azatov:2016xik}, and in electroweak precision
observables as well as Higgs decays to
photons~\cite{Freitas:2016iwx}. With the notable exception of
Reference~\cite{Freitas:2016iwx}, these other studies generally do not
discuss ambiguities in the matching procedure, a central aspect of the
research presented here.

The problem of a lack of scale hierarchy is not unique to the EFT
approach. As discussed at the end of
\autoref{sec:foundations_heft_alternatives}, pseudo-observables rely
on the same expansion in $1/\Lambda$, and the breakdown of this
expansion has been studied in Reference~\cite{Greljo:2015sla}.

As an aside, similar concerns have fuelled an intense investigation in
the context of dark matter searches~\cite{Shoemaker:2011vi,
  Busoni:2013lha, Buchmueller:2013dya, Busoni:2014sya, Racco:2015dxa,
  Bauer:2016pug}.  While in that field EFT-based predictions are
usually robust for early-universe and late-time annihilation rates as
well as for dark matter-nucleon scattering, the required hierarchy of
scales often breaks down for dark matter signals at colliders.



%%%%%%%%%%%%%%%%%%%%%%%%%%%%%%%%%%%%%%%%%%%%%%%%%%%%%%%%%%%%
\section{Matching in the Time of LHC Run~2}
\label{sec:validity_matching}
%%%%%%%%%%%%%%%%%%%%%%%%%%%%%%%%%%%%%%%%%%%%%%%%%%%%%%%%%%%%

%%%%%%%%%%%%%%%%%%%%%%%%%%%%%%%%%%%%%%%%%%%%%%%%%%%%%%%%%%%%
\subsection{Ambiguities}
%%%%%%%%%%%%%%%%%%%%%%%%%%%%%%%%%%%%%%%%%%%%%%%%%%%%%%%%%%%%

Before discussing the individual models and presenting our results, we
have to define how we construct the effective models.  Matching the
dimension-six Lagrangian to a full model is a three-step procedure.
Its starting point is the definition of a heavy mass scale $\Lambda$.
Second, we integrate out the degrees of freedom above $\Lambda$ as
described in \autoref{sec:foundations_matching}, which leads to an
infinite tower of higher-dimensional operators.  Finally, this
effective action is truncated so that only the dimension-six terms,
suppressed by $1 / \Lambda^2$, remain.

The matching is not unambiguous: on the one hand, $\Lambda$ is usually
not uniquely defined. A typical case is a new physics scenario with
only one heavy mass scale $M$ in the Lagrangian, but also some mixing
terms of the new fields with the SM Higgs doublet. In the unbroken
electroweak phase the only new physics scale is then
%
\begin{equation}
  \Lambda_1 = M \,.
\end{equation}
%
But after electroweak symmetry breaking, the electroweak VEV
contributes through the mixing term to the actual physical masses $m$
of the new particles. Even if there is only one dimensionful scale of
new physics in the Lagrangian, this defines additional scales of the
form
%
\begin{equation}
  \Lambda_2^2 = m^2 = M^2 \pm g^2 v^2 \,,
\end{equation}
%
where $g$ is a combination of couplings or mixing angles. Of course
there can be many such scales.

Further ambiguities arise in the third step since we can choose which
parameters to keep constant while expanding in $1/\Lambda$. For
instance we can often choose to define Wilson coefficients in terms of
Lagrangian couplings or in terms of mixing angles. Again, the first
choice corresponds to the natural choice in the unbroken phase of the
electroweak symmetry, while the latter is often only defined in the
broken phase.

In both scenarios, switching from one choice to another corresponds to
including additional contributions suppressed by $v^2/\Lambda^2$ to
the Wilson coefficients of the dimension-six operators. In the first
example,
%
\begin{align}
  \frac {f_x} {\Lambda_2^2} \, \ope{x}
  %
  &= \frac  {f_x}  {M^2 \pm g^2 v^2}\, \ope{x}
    %
  = \frac  {f_x}  {M^2} \left (1 \mp \frac {g^2 v^2} {M^2} + \ord{1/M^4} \right) \ope{x} \notag \\
  %
  &= \left( \frac  {f_x}  {\Lambda_1^2} \mp \frac { f_x g^2 v^2}{\Lambda_1^4} \right) \ope{x}
    + \ord{1/\Lambda_1^6} \,.
\end{align}

It should be stressed that these effects always contribute to
observables at $\ord{1/\Lambda^4}$, the same order in the EFT
expansion as the leading effects from dimension-eight operators, which
we always neglect. From a purely theoretical point of view these terms
are subleading, and indeed in the obvious validity regime of the EFT
they are irrelevant. But this is not the situation we find at the
LHC. In practice, these formally suppressed terms may be important
even if the dimension-eight terms are small.

These ambiguities in the matching procedure raise the question if we
can improve the agreement between full model and dimension-six
Lagrangian by incorporating effects of the non-zero electroweak VEV in
the matching. To answer this question we now define two different
matching prescriptions.



%%%%%%%%%%%%%%%%%%%%%%%%%%%%%%%%%%%%%%%%%%%%%%%%%%%%%%%%%%%%
\subsection{Default vs.\ $v$-improved matching}
%%%%%%%%%%%%%%%%%%%%%%%%%%%%%%%%%%%%%%%%%%%%%%%%%%%%%%%%%%%%

Our \emph{default matching} follows a purely theoretical motivation
and represents the conventional approach to matching. The linear Higgs
EFT is formulated in terms of the doublet $\phi$ and based on the
assumption $\Lambda \gg v$. It should therefore be matched to the full
theory in the unbroken phase of the electroweak symmetry. An obvious
choice for the matching scale is then the mass scale of new particles
in the limit of $v \to 0$, which as in our simple example above we
denote
%
\begin{equation}
  \Lambda_{\text{default}} = M \,.
\end{equation}
%
For simplicity we assume there is only one such scale, \ie that all
new particles are mass-degenerate in the unbroken electroweak
phase. Otherwise the new particles would have to be integrated out
consecutively at different scales $M_i$. We then expand the effective
action, expressed in parameters of the Lagrangian, and drop all terms
of $\ord{\Lambda^{-4}}$.

Alternatively, we define a \emph{$v$-improved matching} procedure that
accounts for additional terms suppressed by $v^2 / \Lambda^2$ in the
Wilson coefficients of the dimension-six Lagrangian. This corresponds
to matching the linear EFT in the broken electroweak phase. In the
first matching step, we define $\Lambda$ as the physical mass $m$ of
the new particles in the broken phase including contributions from
$v$,
%
\begin{equation}
  \Lambda_{\text{$v$-improved}} = m \,,
\end{equation}
%
rather than the mass scale in the unbroken phase. Again, multiple
particles with substantial mass splittings will require a multi-step
matching procedure, but there is no fundamental problem to describe
them. The Wilson coefficients are expressed in terms of
phenomenologically relevant quantities such as mixing angles and
physical masses, again defined in the broken phase. This is a somewhat
subjective criterion that depends on model and process: the
$v$-improved matching procedure is not a unique definition, but rather
a general guideline. We will demonstrate an example for a natural
choice in the singlet scenario.



%%%%%%%%%%%%%%%%%%%%%%%%%%%%%%%%%%%%%%%%%%%%%%%%%%%%%%%%%%%%
\subsection{Making sense of $v$-improvement}
%%%%%%%%%%%%%%%%%%%%%%%%%%%%%%%%%%%%%%%%%%%%%%%%%%%%%%%%%%%%

We can interpret the two matching schemes from two different
perspectives. First we analyse it from the unbroken electroweak
phase. Some of the operators with dimension 8 or higher are of the
form
%
\begin{equation}
  \ope{i}^{(d=6 + 2n)} = ( \phisq)^n \ope{i}^{(6)} \,,
  \label{eq:validity_dim8_phisq_dim6}
\end{equation}
%
where $\ope{i}^{(6)}$ is a dimension-six operator. The untruncated,
infinite tower of higher-dimensional operators generated from the full
theory can be re-organised as
%
\begin{align}
  \lgr{EFT}
  %
  &\equiv \lgr{SM}
    + \sum_{d=6}^\infty \sum_i \frac {f_i^{(d)}} {\Lambda^{d-4}} \ope{i}^{(d)} \notag \\
  %
  &= \lgr{SM}
  + \sum_i \sum_{n=0}^\infty \frac {f_i^{(6+2n)}} {\Lambda^{2+2n}} \left( \phisq \right)^n \ope{i}^{(6)}
  + \sum_{d=8}^\infty \sum_k \frac {f_k^{(d)}}  {\Lambda^{d-4}} \ope{k}^{(d)} \,,
\end{align}
%
where $\smash{\ope{k}^{(d)}}$ are the dimension-eight and higher
operators of a different form than
\autoref{eq:validity_dim8_phisq_dim6}.  A $v$-improved matching
corresponds to replacing $\phisq \to v^2 / 2$ in (part of) the first
sum:\footnote{The argument is slightly more complicated if additional
  powers of $\phi$ appear in $\smash{\ope{i}^{(6)}}$. One can then
  also take into account terms where instances of $\phi$ in
  $\smash{\ope{i}^{(6)}}$ are replaced by the VEV and fields in the
  prefactor are left alone. This effectively adds a combinatorical
  factor to \autoref{eq:validity_v-improvement_resummation}.}
%
\begin{equation}
  \lgr{$v$-improved dim-6} = \lgr{SM}
  + \sum_i
  \frac { \overbrace{\sum_{n=0}^\infty f_i^{(6+2n)}  \left( \frac {v^2} {2\Lambda^{2}} \right)^n}^{f_i^{(6), \text{$v$-improved}}} }
  {\Lambda^2}
  \ope{i}^{(6)} \,.
  \label{eq:validity_v-improvement_resummation}
\end{equation}
%
So from the perspective of the unbroken phase of the electroweak
symmetry, $v$-improvement corresponds to a \emph{partial resummation
  of dimension-eight and higher operators} into the Wilson
coefficients of the dimension-six operators.
 
From an experimental point of view, or in the broken electroweak
phase, physical masses and mixing angles are simply the natural
choices to describe a model. We then expect that the $v$-improved
matching procedure can improve the validity of the dimension-six
model. For a more precise statement we have to distinguish between an
expansion in $v/\Lambda$ and $\mathbf{p}/\Lambda$. We expect the
$v$-improved matching prescription can lead to a better agreement with
the full models in situations where the expansion in $v/\Lambda$ is
relevant, while it cannot help with the expansion in
$\mathbf{p}/\Lambda$, corresponding to genuine new dimension-eight
operators not of the form in
\autoref{eq:validity_dim8_phisq_dim6}. We will come back to this
difference later and demonstrate it in the vector triplet scenario.

Again, the truncation of the EFT Lagrangian is formally justified as
long as $v \ll \Lambda$ and we only probe energies $E \ll \Lambda$.
In this limit the dimension-eight operators as well as the
$\Lambda$-suppressed terms in the Wilson coefficients are negligible;
our two matching procedures then give identical results. In the
absence of a large enough scale separation, our bottom-up approach
allows us to treat them independently. This way we can use the
$v$-improved matching to enhance the validity of the dimension-six
Lagrangian.

% The external energy scale depends on the specific process and
% observable, \eg $E_{\text{phys}} \sim m_h$ for on-shell Higgs coupling
% measurements, $E_{\text{phys}} \sim m_{4 \ell}$ for off-shell Higgs
% coupling measurements, $E_{\text{phys}} \sim m_{hh}$ for Higgs pair
% production at threshold, or $E_{\text{phys}} \sim p_{T,h}$ for boosted
% single or double Higgs production. In kinematic distributions the
% high-energy tails can probe significantly larger energy scales.  This
% implies that the energy range where the EFT description is applicable
% is model-dependent and observable-dependent. Successively adding
% higher-dimensional operators should improve the situation, as long as
% the key scales $E_{\text{phys}}, \Lambda$ are sufficiently
% separated. Of course, the EFT description fails spectacularly in the
% presence of new resonances in the relevant energy range, and we have
% to adjust the field content of the effective Lagrangian.



%%%%%%%%%%%%%%%%%%%%%%%%%%%%%%%%%%%%%%%%%%%%%%%%%%%%%%%%%%%%
\section{Full models vs.\ effective theory}
\label{sec:validity_full_vs_effective}
%%%%%%%%%%%%%%%%%%%%%%%%%%%%%%%%%%%%%%%%%%%%%%%%%%%%%%%%%%%%

The main aim of this chapter is to compare a comprehensive set of LHC
predictions from specific new physics models to their corresponding
effective field theory predictions. In this way we test the
applicability of the dimension-six model for four different, more or
less UV-complete, scenarios of underlying physics:
%
\begin{enumerate}
\item a scalar singlet extension with mixing effects and a second
  scalar resonance;
\item a two-Higgs doublet model, adding a variable Yukawa structure, a
  CP-odd, and a charged Higgs;
\item scalar top partners, contributing to Higgs couplings at one
  loop; and
\item a vector triplet with gauge boson mixing.
\end{enumerate}
%
This ensemble of models covers a wide range of $CP$-even new physics
signatures in the Higgs sector.

After describing our technical setup, we analyse these four scenarios
one by one. For each model we first define the theory and introduce
the main phenomenological features at the LHC. We discuss the
decoupling in the Higgs sector, and derive the dimension-six
setup. Finally, we define a number of benchmark points and give a
detailed account of the full and dimension-six phenomenology at the
LHC.

Effects in the SM-like Higgs couplings will be parametrised with the
relative shifts from the SM values
%
\begin{equation}
  \Delta_x \equiv \frac {g_{hxx}} {g_{hxx}^{\text{SM}}} - 1\,,
\end{equation}
%
as defined in \autoref{eq:foundations_kappa_delta}. Unlike in the
published version~\cite{Brehmer:2015rna}, we express the effective
Lagrangian in the HISZ basis with the ten dimension-six operators of
\autoref{eq:foundations_operators_even}.



%%%%%%%%%%%%%%%%%%%%%%%%%%%%%%%%%%%%%%%%%%%%%%%%%%%%%%%%%%%%
\subsection{Setup}
%%%%%%%%%%%%%%%%%%%%%%%%%%%%%%%%%%%%%%%%%%%%%%%%%%%%%%%%%%%%

Our comparison covers the most relevant observables for LHC Higgs
physics. Acceptance and background rejections cuts are kept to a
minimum to be able to test the effective field theory approach over as
much of the phase space as possible.

In the case of Higgs production through gluon fusion, we analyse the
production process with a Higgs decay to four leptons or to photons,
%
\begin{align}
  g g &\to h \to 4 \ell \,,\notag \\
  g g &\to h \to \gamma \gamma \,.
  \label{eq:validity_gg_4l_process}
\end{align}
%
For the photons we do not apply any cuts, while for $\ell = e, \mu$ we
require
%
\begin{equation}
  m_{4 \ell} > 100 \ \gev \quad \text{and} \quad
  m_{\ell^+\ell^-}^\text{same flavour} > 10 \ \gev \,
\end{equation}
%
to avoid too large contributions from the $Z$ peak and
bremsstrahlung.

For Higgs production in weak boson fusion (WBF), we evaluate the
production process
%
\begin{equation}
  u d \to h \, u d
  \to W^+ W^- \, ud
  \to (\ell^+ \nu) \, (\ell^- \bar{\nu}) \, ud \,,
\label{eq:validity_wbf_proc}
\end{equation}
%
which is the dominant partonic contribution at the LHC. We require the standard
WBF cuts
%
\begin{align}
  p_{T,j} &> 20 \ \gev \,, &
 \Delta \eta_{jj} &> 3.6 \,, &
  m_{jj} &> 500 \ \gev \,, \notag \\
  p_{T,\ell} &> 10 \ \gev  \,, &
  \met &> 10 \ \gev \,.
  \label{eq:validity_wbf_cuts}
\end{align}
%
Unlike for gluon fusion, the kinematics of the final state can now
introduce new scales and a dependence on the UV structure of the
model. The process is particularly interesting in the context of
perturbative unitarity~\cite{Cornwall:1974km, Cornwall:1973tb,
  LlewellynSmith:1973yud, Weldon:1984th, Weldon:1984wt, Gunion:1990kf,
  Corbett:2015lfa}. While the latter is satisfied in a UV-complete
model by construction, deviations from the SM Higgs-gauge couplings in
the EFT may lead to an increasing rate at very large
energies~\cite{Han:2009em, Brehmer:2014pka}, well outside the EFT
validity range $E / \Lambda \ll 1$.  To look for such signatures, we
focus on the high-energy tail of the transverse mass distribution,
%
\begin{equation}
  m_T^2 = \left( E_{T,\ell \ell } + E_{T,\nu \nu }
  \right)^2 - \left( \mathbf{p}_{T,\ell \ell } +
  \mathbf{p}_T^{\text{miss}} \right)^2 
  \label{eq:validity_mT}
\end{equation}
%
with
%
\begin{equation}
  E_{T,\ell \ell } = \sqrt{\mathbf{p}_{T,\ell \ell }^2 + m_{\ell \ell}^2} \,, \quad
  E_{T,\nu \nu } = \sqrt{\met + m_{\ell \ell }^2} \,.
\end{equation} 

As the last single Higgs production process we evaluate
Higgs-strahlung
%
\begin{equation}
  q q \to V h
\end{equation}
%
with $V = W^\pm, Z$. We do not simulate the Higgs and gauge boson
decays, assuming that we can always reconstruct for example the full
$Zh \to \ell^+ \ell^- \, b \bar{b}$ final state. No cuts are applied.

Finally, Higgs pair production,
%
\begin{equation}
  g g \to h h \,,
\end{equation}
%
is well known to be problematic when it comes to the effective theory
description~\cite{Baur:2002rb, Gillioz:2012se, Dawson:2012mk}. Again,
neither Higgs decays nor kinematic cuts are expected to affect our
analysis, so we leave them out.

\newparagraph
%
We test all these channels for the singlet and doublet Higgs sector
extensions. For the top partner and vector triplet models we focus on
the WBF and Higgs-strahlung modes.

In the dimension-six simulations we always include the square of the
dimension-six operator contributions. While these terms are
technically of the same mass dimension as dimension-eight operators,
which we neglect, we keep them to avoid negative values of the squared
matrix element in extreme phase-space regions. Notice that these
situations do not necessarily imply a breakdown of the EFT
expansion. On the contrary, they may appear in scenarios where new
physics contributions dominate over the SM part, while the EFT
expansion is fully valid (with $E/\Lambda \ll 1$). In such cases, the
bulk effects stem from the squared dimension-six terms instead of the
interference with the SM, while the effects from dimension-eight
operators are smaller and can be safely neglected. We will discuss
this aspect in more detail in the next section.

Most amplitudes are calculated at leading order in $\alpha_s$
and $\alpha_{ew}$, which is sufficient given the size of new physics
effects that the LHC is sensitive to. We always take into account
interference terms between Higgs and gauge amplitudes. 

For tree-level processes we generate event samples with
\toolfont{MadGraph~5}~\cite{Alwall:2014hca}, using publicly available
our own model files implemented in
\toolfont{FeynRules}~\cite{Alloul:2013bka}, which provides the
corresponding UFO files~\cite{Degrande:2011ua}.  For the dimension-six
predictions we resort to a modified version of the \toolfont{HEL}
model file~\cite{Alloul:2013naa}.

The Higgs-gluon and Higgs-photon couplings are evaluated with the full
one-loop form fac\-tors~\cite{Djouadi:2005gi}, including top, bottom and
$W$ loops as well as new particles present in the respective
models. For Higgs pair production, we use a modified version of
Reference~\cite{higgspaircode}, see also Reference~\cite{Hespel:2014sla}.

Other loop effects are analysed using reweighting: we generate event
samples using appropriate general couplings. Next, we compute the
one-loop matrix element for each phase space point and reweight the
events with the ratio of the renormalised one-loop matrix element
squared to the tree-level model. For the one-loop matrix elements we
utilise \toolfont{FeynArts} and \toolfont{FormCalc}~\cite{Hahn:2000kx}
with our own model files that include the necessary counterterms. The
loop form factors are handled with dimensional regularisation in the
't~Hooft-Veltman scheme, and written in terms of standard loop
integrals. These are further reduced via Passarino-Veltman
decomposition and evaluated with the help of
\toolfont{LoopTools}~\cite{Hahn:1998yk}.

Generally we create event samples of at least 100\,000 events per
benchmark point and process for $pp$ collisions at
$\sqrt{s} = 13$~\tev. We use the \toolfont{CTEQ6L}
pdf~\cite{Pumplin:2002vw} and the default dynamical choices of the
factorisation and renormalisation scale implemented in
\toolfont{MadGraph}. For this broad survey of EFT validity we limit
ourselves to parton level and do not apply a detector simulation, a
more realistic simulation will be discussed in the next section. The
mass of the SM-like Higgs is fixed to
$m_h = 125$~\gev~\cite{Aad:2015zhl}. For the top mass we take
$m_t = 173.2$~\gev~\cite{Tevatron:2014cka, ATLAS:2014wva}. The Higgs
width in each model is based on calculations with
\toolfont{Hdecay}~\cite{Djouadi:1997yw}, which we rescale with the
appropriate coupling modifiers and complement with additional decay
channels where applicable.



%%%%%%%%%%%%%%%%%%%%%%%%%%%%%%%%%%%%%%%%%%%%%%%%%%%%%%%%%%%%
\subsection{Singlet extension}
\label{sec:validity_singlet}
%%%%%%%%%%%%%%%%%%%%%%%%%%%%%%%%%%%%%%%%%%%%%%%%%%%%%%%%%%%%

%%%%%%%%%%%%%%%%%%%%%%%%%%%%%%%%%%%%%%%%%%%%%%%%%%%%%%%%%%%%
\subsubsection{Model setup}
%%%%%%%%%%%%%%%%%%%%%%%%%%%%%%%%%%%%%%%%%%%%%%%%%%%%%%%%%%%%

The simplest extension of the minimal Higgs sector of the Standard
Model is by a real scalar singlet $S$~\cite{Silveira:1985rk,
  Schabinger:2005ei, Patt:2006fw, Pruna:2013bma, Lopez-Val:2014jva,
  Robens:2015gla, Robens:2016xkb}. For the sake of simplicity we
consider a minimal version of the singlet model, in which a discrete
$\mathbb{Z}_2$ parity forbids additional (\eg cubic) terms in the
potential. The theory is then given by
%
\begin{equation}
  \lgr{singlet}
  \supset (D_\mu\phi)^\dagger\, (D^\mu\phi)
  + \frac 1 2 \, \partial_\mu S \, \partial^\mu S
  - V(\phi,S) \,,
  \label{eq:validity_singlet_lagrangian}
\end{equation}
%
where the scalar potential has the form
%
\begin{align}
  V(\phi,S) =
  \mu^2_1\, \phisq
  + \lambda_1 \left( \phisq \right)^2
  + \mu^2_2\,S^2 + \lambda_2\,S^4
  + \lambda_3 \left( \phisq \right) S^2 \,.
  \label{eq:validity_singlet_potential}
\end{align}
%
The new scalar $S$ can mix with the SM doublet $\phi$ provided
the singlet develops a VEV,
%
\begin{equation}
  \langle S\rangle = \frac {v_s} {\sqrt{2}} \,.
\end{equation}
%
The mixing angle is given by
%
\begin{equation}
  \tan(2\alpha) = \frac{\lambda_3vv_s}{\lambda_2 v_s^2 - \lambda_1v^2}\,.
  \label{eq:validity_singlet_mixing_angle}
\end{equation}
%
Details on the parametrisation and Higgs mass spectrum are given in
Appendix~\ref{sec:appendix_models_singlet}.



%%%%%%%%%%%%%%%%%%%%%%%%%%%%%%%%%%%%%%%%%%%%%%%%%%%%%%%%%%%%
\subsubsection{Signatures and decoupling patterns}
%%%%%%%%%%%%%%%%%%%%%%%%%%%%%%%%%%%%%%%%%%%%%%%%%%%%%%%%%%%%

The additional scalar singlet affects Higgs physics in three
ways. First, it mixes with the Higgs via the mixing angle $\alpha$,
which leads to a universal rescaling of all Higgs couplings to
fermions and vectors. Second, it modifies the Higgs
self-coupling. Finally, it introduces a new, heavy resonance $H$
coupled to the Standard Model through mixing.

The key parameter is the portal interaction between the doublet and
the singlet fields $\lambda_3(\phisq)\,S^2$, which is responsible for
the mixed mass eigenstates. The mixing reduces the coupling of the
SM-like Higgs $h$ to all Standard Model particles universally,
%
\begin{align}
  \Delta_x = \cos \alpha - 1 \quad 
  \label{eq:validity_singlet_shift}
\end{align}
% 
for $x=W,Z,t,b,\tau,g,\gamma,\dots$.  It also affects the
self-coupling of the light Higgs, which takes on the form
%
\begin{equation}
  g_{hhh} =
  6 \cos^3 \alpha\, \lambda_1 v
  - 3 \cos^2 \alpha \sin \alpha\, \lambda_3 v_s
  + 3 \cos \alpha \sin^2 \alpha\, \lambda_3 v
  - 6 \sin^3 \alpha\, \lambda_2 v_s \,.
\end{equation}

The parameter $\sin\alpha$ quantifies the departure from
the SM limit $\alpha \to 0$.  This limit can be attained in two ways:
first, a small mixing angle can be caused by a weak portal
interaction,
%
\begin{equation}
  \left| \tan(2\alpha) \right|
  = \left| \frac{\lambda_3\,v\,v_s}{\lambda_2 v_s^2 - \lambda_1 v^2} \right|
  \ll 1 \quad \text{if} \quad \lambda_3 \ll 1 \,.
  \label{eq:validity_singlet_limit1}
\end{equation}
%
The Higgs couplings to SM particles approach their SM values, but
there is no large mass scale associated with this limit. In the
extreme case of $\lambda_2,\lambda_3 \ll \lambda_1$ we find small
$\alpha \approx - \lambda_3/\lambda_1 \times v_s/(2v)$ even for
$v_s \lesssim v$.  This situation is to some extent the singlet model
counterpart of the ``alignment without decoupling'' scenario in the
Two-Higgs-doublet model (2HDM)~\cite{Gunion:2002zf, Craig:2013hca} or
the MSSM~\cite{Carena:2013ooa, Delgado:2013zfa}. It relies on a weak
portal coupling and a small scale separation, which cannot be properly
described by an effective field theory.

As a second possibility, the additional singlet can introduce a large
mass scale $v_s \gg v$, giving us
% 
\begin{equation}
  \tan \alpha \approx \frac{\lambda_3}{2\lambda_2}\,\frac{v}{v_s}
  \ll 1 \quad \text{if} \quad v \ll v_s \,,
  \label{eq:validity_singlet_limit2}
\end{equation}
% 
where $\lambda_3/(2\lambda_2)$ is an effective coupling of up to order
one. In this limit the heavy Higgs mass, which we identify as the
heavy mass scale, is given by
%
\begin{equation}
  m_H \approx \sqrt{2\lambda_2} \, v_s \,.
\end{equation}

In terms of $m_H$, the Higgs couplings scale like
%
\begin{equation}
  \Delta_x = - \frac{\alpha^2}{2} + \ord{\alpha^3}
  \approx - \frac{\lambda_3^2}{4 \lambda_2}\, \left( \frac{v}{m_H} \right)^2 \,.
  \label{eq:validity_singlet_decoup}
\end{equation}
%
This is a dimension-six effect. If we require $|\Delta_x| \gtrsim 10\%$
to keep our discussion relevant for the LHC, this implies
%
\begin{equation}
  m_H \approx \sqrt{2\lambda_2} \, v_s
  < \frac{\sqrt{5} \lambda_3}{\sqrt{2 \lambda_2}} \, v
  = 390~\text{GeV} \times \frac{\lambda_3}{\sqrt{\lambda_2}} \,.
 \label{eq:validity_singlet_delta3}
\end{equation}
%
If we also assume that the ratio of quartic couplings is of the order
of a perturbative coupling, $\lambda_3/\sqrt{\lambda_2} \lesssim 0.5$,
the LHC reach in the Higgs coupling analysis translates into heavy
Higgs masses below 200~GeV. For strongly coupled scenarios,
$\lambda_3/\sqrt{\lambda_2} \lesssim 1 \dots \sqrt{4\pi}$, the heavy
mass reach increases to $m_H \lesssim 0.4 \dots 1.5$ TeV.  This
suggests that a weakly coupled Higgs portal will fail to produce a
sizeable separation of scales when looking at realistic Higgs coupling
analyses. The question becomes if and where this lack of scale
separation hampers our LHC analyses.



%%%%%%%%%%%%%%%%%%%%%%%%%%%%%%%%%%%%%%%%%%%%%%%%%%%%%%%%%%%%
\subsubsection{Dimension-six description}
%%%%%%%%%%%%%%%%%%%%%%%%%%%%%%%%%%%%%%%%%%%%%%%%%%%%%%%%%%%%

In the EFT approach the singlet model only generates $\ope{\phi,2}$ at
dimension six~\cite{Gorbahn:2015gxa}. Before electroweak symmetry
breaking, the only mass scale in the Lagrangian that describes the new
physics is $\mu_2^2 < 0$. Defining the Wilson coefficients suppressed
by this new physics scale gives clearly wrong results, as we will
discuss in the analogous case for the two-Higgs-doublet model in the
next section. Instead identify the leading contribution to the heavy
Higgs mass as the new physics scale in our default matching, in
agreement with the logic
\autoref{sec:validity_matching}. Following the discussion of
decoupling patterns above this means
%
\begin{equation}
  \Lambda_{\text{default}} = \sqrt{2\lambda_2} \, v_s \approx m_H \,.
\end{equation}
%
The corresponding Wilson coefficient, expressed in terms of Lagrangian
parameters, is
%
\begin{equation}
  \bar{f}_{\phi,2}^{\text{default}} = \frac{\lambda_3^2}{2\lambda_2} \,.
\end{equation}

For the $v$-improved matching, we instead use the actual physical mass
%
\begin{equation}
    \Lambda_{\text{$v$-improved}} = m_H \,.
\end{equation}
%
In the broken phase the Higgs couplings are fully expressed through
the mixing angle $\alpha$ as given in
\autoref{eq:validity_singlet_shift}. We define
%
\begin{equation}
  f_{\phi,2}^{\text{$v$-improved}} = 2 ( 1 - \cos \alpha) \frac {m_H^2} {v^2} \,.
\end{equation} 
%
which ensures that the Higgs couplings exactly agree between the full
model and the $v$-improved dimension-six description.



%%%%%%%%%%%%%%%%%%%%%%%%%%%%%%%%%%%%%%%%%%%%%%%%%%%%%%%%%%%%
\subsubsection{Benchmark points}
%%%%%%%%%%%%%%%%%%%%%%%%%%%%%%%%%%%%%%%%%%%%%%%%%%%%%%%%%%%%

\begin{table}
  \begin{tabular}{c c rrrr c rrr c rrr}
    \toprule
    %
    \multirow{2}{*}{} && \multicolumn{4}{c}{Singlet} &&
    \multicolumn{3}{c}{Default EFT} && \multicolumn{3}{c}{$v$-improved EFT} \\
    %
    \cmidrule{3-6} \cmidrule{8-10} \cmidrule{12-14}
    %
    && $m_H$ & $\sin\alpha$ & $v_s/v$ & $\Delta_x$ &&
    $\Lambda$ & ${f}_{\phi,2}$ & $\Delta_x$ &&
    $\Lambda$ & ${f}_{\phi,2}$ & $\Delta_x$ \\
    %
    \midrule
    %
    S1 && $500$ & $0.2$ & $10$ & $-0.020$ && $491$ & $0.14$ & $-0.018$ && $500$ & $0.15$ & $-0.020$ \\
    S2 && $350$ & $0.3$ & $10$ & $-0.046$ && $336$ & $0.14$ &  $-0.037$ && $350$ & $0.16$ & $-0.046$ \\
    S3 && $200$ & $0.4$ & $10$ & $-0.083$ && $190$ & $0.04$ & $-0.031$ && $200$ & $0.06$ & $-0.083$ \\
    S4 && $1000$ & $0.4$ & $10$ & $-0.083$ && $918$ & $2.60$ & $-0.092$ && $1000$ & $3.13$ & $-0.083$ \\
    S5 && $500$ &  $0.6$ & $10$ & $-0.200$ && $407$ &$1.26$ & $-0.231$ && $500$ &  $1.24$ & $-0.200$ \\
    %
    \bottomrule
    \end{tabular}
    \caption[Benchmarks for the singlet extension]{Benchmarks for the singlet extension.
      We show the model parameters and the universal coupling modification for the complete
      model, as well as the cutoff scales $\Lambda$, Wilson coefficients $\bar{f}_{\phi,2}$, and the
      universal coupling modification in the EFT approach for the default and $v$-improved
      matching schemes. All mass scales are given in \gev.}
  \label{tbl:validity_singlet_benchmarks}
\end{table}

We start our numerical analysis by defining five singlet benchmark
points in \autoref{tbl:validity_singlet_benchmarks}, with heavy Higgses
ranging from $200$ to $1000$~\gev. The first three scenarios are in
agreement with current experimental and theoretical constraints.  This
includes direct mass bounds from heavy Higgs searches at colliders,
Higgs coupling measurements, electroweak precision observables,
perturbative unitarity and vacuum stability~\cite{Pruna:2013bma,
  Lopez-Val:2014jva, Robens:2015gla}. Note that for S4 and S5 the
combination of large heavy Higgs masses together with large mixing
angles is incompatible with perturbative unitarity and electroweak
precision constraints.  We nevertheless keep such benchmarks for
illustration purposes.



%%%%%%%%%%%%%%%%%%%%%%%%%%%%%%%%%%%%%%%%%%%%%%%%%%%%%%%%%%%%
\subsubsection{Higgs couplings and total production rates}
%%%%%%%%%%%%%%%%%%%%%%%%%%%%%%%%%%%%%%%%%%%%%%%%%%%%%%%%%%%%

\autoref{tbl:validity_singlet_benchmarks} also shows the universal
shift $\Delta_x$ of the light Higgs couplings, both for the full
singlet model and its dimension-six EFT descriptions. For the default
matching, we find reasonable agreement with the full model for the
scenarios with a heavy additional Higgs, while large discrepancies
appear when the new physics is lighter. In particular, note the
difference between S3 and S4. Both describe the same coupling shift
$\Delta_x = - 0.08$. But while S3 realises this with a weakly coupled
light Higgs, which the default EFT cannot describe, S4 has a heavier,
more strongly coupled Higgs, and the EFT description works.  In all
cases, the $v$-improved EFT by construction predicts the Higgs
couplings correctly.

\begin{table}
  \begin{tabular}{c c rrr c rrr}
    \toprule
    %
    \multirow{2}{*}{}
    && \multicolumn{3}{c}{$\sigma_\text{default EFT} / \sigma_\text{singlet}$}
    && \multicolumn{3}{c}{$\sigma_\text{$v$-improved EFT} / \sigma_\text{singlet}$} \\
    %
    \cmidrule{3-5} \cmidrule{7-9}
    %
    && ggF & WBF & $Vh$ && ggF & WBF & $Vh$ \\
    %
    \midrule
    %
    S1 && $1.006$ & $1.006$ & $1.004$ && $1.001$ & $1.001$ & $1.000$ \\
    S2 && $1.019$ & $1.021$ & $1.019$ && $1.000$ & $1.001$ & $1.000$ \\
    S3 && $1.119$ & $1.118$ & $1.118$ && $1.000$ & $0.999$ & $1.000$ \\
    S4 && $0.982$ & $0.982$ & $0.982$ && $0.999$ & $0.999$ & $1.000$ \\
    S5 && $0.925$ & $0.925$ & $0.925$ && $0.999$ & $0.999$ & $1.000$ \\
    %
    \bottomrule
  \end{tabular}
  \caption[Total Higgs production rates in the singlet extension]{Cross
    section ratios of the matched dimension-six EFT
    approximation to the full singlet model at the LHC. We show the
    leading Higgs production channels for all singlet benchmark
    points. The statistical uncertainties on these ratios are below
    0.4\%.}
  \label{tbl:validity_singlet_rates}
\end{table}

In \autoref{tbl:validity_singlet_rates} we show how well the
effective models describe the total Higgs production cross sections in
gluon fusion, WBF and Higgs-strahlung. These numbers confirm what we
expect from the coupling modifications: while the default
dimension-six model predict qualitatively similar shifts in the total
rates, there are rate deviations of up to $10 \%$. In the $v$-improved
EFT we find that the Higgs couplings and total rates agree exactly
with the full model predictions. The dimension-six operators are
entirely sufficient to capture the coupling shifts, but a significant
part of their coefficients are formally of $\ord{v^4/\Lambda^4}$.



%%%%%%%%%%%%%%%%%%%%%%%%%%%%%%%%%%%%%%%%%%%%%%%%%%%%%%%%%%%%
\subsubsection{Distributions}
%%%%%%%%%%%%%%%%%%%%%%%%%%%%%%%%%%%%%%%%%%%%%%%%%%%%%%%%%%%%

\begin{figure}
  \includegraphicsdummy[width=0.49\textwidth]{fig/validity/Singlet_4l.pdf}%
  \includegraphicsdummy[width=0.49\textwidth]{fig/validity/Singlet_VH.pdf}\\%
  \includegraphicsdummy[width=0.49\textwidth]{fig/validity/Singlet_WBF.pdf}%
  \includegraphicsdummy[width=0.49\textwidth]{fig/validity/Singlet_HH.pdf}%
  \caption[Kinematic distributions in the singlet extension]{Selected kinematic
    distributions in the singlet model.  The
    different curves show the SM, the full singlet model, and the
    dimension-six model. Top left: $m_{4\ell}$ distribution in
    the $gg \to h \to 4 \ell$ channel for S2. Top right: $m_{Vh}$
    distribution in $Vh$ production for S1.  Bottom left: $m_T$
    distribution in the WBF $h \to \ell^+ \ell^- \; \met$ channel for
    S5. Bottom right: $m_{hh}$ distribution in Higgs pair production
    for S4. For $m_{hh}$ we show several contributions in the full
    theory and the dimension-six approach. In all plots, the error
    bars give the statistical uncertainties.}
  \label{fig:validity_singlet_distributions}
\end{figure}


The most obvious source of discrepancy between the full model and the
EFT is the heavy resonance $H$. It can for example be produced in
gluon fusion and then observed as a peak in the $m_{4\ell}$
distribution. By construction, it will not be captured by the
dimension-six model. We illustrate this in the upper left panel of
\autoref{fig:validity_singlet_distributions}. For Higgs-strahlung
production (\autoref{fig:validity_singlet_distributions}, upper
right panel), where the novel $H$ resonance does not appear in an
intermediate Born-level propagator and hence has no impact, we find
instead excellent agreement between both descriptions over the entire
phase space.

The second Higgs has a second, more subtle effect.  In the full model,
both Higgs exchange diagrams are needed to unitarise $WW$
scattering. Correspondingly, the EFT description breaks perturbative
unitarity roughly at the scale~\cite{Han:2009em}
%
\begin{equation}
  m_{WW}^2
  %
  \sim \frac{16 \pi } {\frac {f_{\phi,2}}{\Lambda^2} \left( 1 - \frac{f_{\phi,2} v^2 /\Lambda^2}{4 (1+f_{\phi,2} v^2 /\Lambda^2)} \right)}
  %
  \approx \left( \frac {1.7\ \tev} {\sin \alpha} \right)^2 \,.
  \label{eq:validity_singlet_unitarity_violation}
\end{equation}
%
In our benchmark point S5, this is around 2.8~TeV. The incomplete
cancellations between Higgs and gauge amplitudes means that the
dimension-six model tends to have a larger rate at energies already
below this scale. For this specific benchmark choice, this can be seen
in the lower left panel of
\autoref{fig:validity_singlet_distributions}, where we show the
distribution of the transverse mass defined in
Equation\,\eqref{eq:validity_mT} in the process
$ u d \to W^+ W^- \, ud \to (\ell^+ \nu) \, (\ell^- \bar{\nu}) \, ud$,
to which WBF production of both $h$ and $H$ contributes.  We observe
that the dimension-six model predicts a slightly higher rate at large
$m_T$ than both the full singlet model and the SM. Given the very mild
signal, which results from the fast decrease in the parton densities
and the small mixing angle for realistic scenarios, such effect is
likely of no relevance for LHC physics. 

A more interesting channel to study in the singlet model is Higgs pair
production. The Higgs self-coupling is the only Higgs coupling which
gains a momentum dependence in the matched EFT. In addition, there
exists an approximate cancellation between the two leading amplitudes
in the SM at threshold~\cite{Plehn:1996wb, Li:2013rra}. This induces a
second relevant scale and with it a sensitivity to small deviations in
the Higgs couplings.

In the bottom right panel of
\autoref{fig:validity_singlet_distributions} we give the $m_{hh}$
distribution in the full and dimension-six models.  In addition, we
show how the distributions would look in the full model without a $H$
state, and in the EFT without the momentum-dependent (derivative)
terms given in Equation\,\eqref{eq:validity_singlet-self}.  Already at
threshold and far away from the $H$ resonance, the interference of the
SM-like terms with the $H$ diagrams makes up a significant part of the
amplitude.  In the EFT, the derivative terms are similarly relevant
already at low energies. Close to threshold, the ($v$-improved)
dimension-six model approximates the full theory well. But this
agreement becomes worse already at moderately larger energies, and
clearly breaks down towards the $H$ pole.



%%%%%%%%%%%%%%%%%%%%%%%%%%%%%%%%%%%%%%%%%%%%%%%%%%%%%%%%%%%%
\subsubsection{Summary}
%%%%%%%%%%%%%%%%%%%%%%%%%%%%%%%%%%%%%%%%%%%%%%%%%%%%%%%%%%%%

If we limit ourselves to Higgs properties relevant for single Higgs
production at the LHC, the modifications from a singlet extension are
very simple: all Standard Model couplings acquire a common scaling
factor, and no relevant new Lorentz structures appear at tree-level.
The dimension-six setup reproduces this effect correctly: the reduced
couplings to all SM fields alone do not require a large hierarchy of
scales. A standard matching procedure that expands the coefficients to
leading order in $v^2/\Lambda^2$ may lead to sizeable deviations from
the full model. However, a $v$-improved EFT construction that takes
into account higher orders in $v^2/\Lambda^2$ gives perfect agreement
with the full theory. In other words, many of the dimension-eight and
higher operators generated from the singlet model are of the form
$( \phisq)^n \ope{i}^{(6)}$. After EWSB they can be resummed into the
Wilson coefficients of the dimension-six operators as given by
Eq.~\eqref{eq:validity_v-improvement_resummation}.

Higgs pair production is different. There is a large contribution from
off-shell $H$, while in the EFT the $h$ self-coupling involves a
derivative. These different structures lead to discrepancies between
full and effective description that increase with momentum
transfer. Of course, the effective theory by definition does not
include the second resonance, so it fails whenever a heavy Higgs
appears on-shell in the full theory.



%%%%%%%%%%%%%%%%%%%%%%%%%%%%%%%%%%%%%%%%%%%%%%%%%%%%%%%%%%%%
\subsection{Two-Higgs-doublet model}
\label{sec:validity_2hdm}
%%%%%%%%%%%%%%%%%%%%%%%%%%%%%%%%%%%%%%%%%%%%%%%%%%%%%%%%%%%%

%%%%%%%%%%%%%%%%%%%%%%%%%%%%%%%%%%%%%%%%%%%%%%%%%%%%%%%%%%%%
\subsubsection{Model setup}
%%%%%%%%%%%%%%%%%%%%%%%%%%%%%%%%%%%%%%%%%%%%%%%%%%%%%%%%%%%%

The two-Higgs-doublet model (2HDM)~\cite{Gunion:1989we, Branco:2011iw}
adds a second weak doublet with weak hypercharge $Y = +1$ to the SM
Higgs sector. The combined potential reads
%
\begin{multline}
  V(\phi_1,\phi_2) =
  m^2_{11}\,\phi_1^\dagger\phi_1
  + m^2_{22}\,\phi_2^\dagger\phi_2
  + \frac{\lambda_1}{2} \, (\phi_1^\dagger\phi_1)^2
  + \frac{\lambda_2}{2} \, (\phi_2^\dagger\phi_2)^2
  + \lambda_3 \, (\phi_1^\dagger\phi_1)\,(\phi_2^\dagger\phi_2)  \\
  + \lambda_4 \, |\phi_1^\dagger\,\phi_2|^2
  + \left[ - m^2_{12}\,\phi_1^\dagger\phi_2
    + \frac{\lambda_5}{2} \, (\phi_1^\dagger\phi_2)^2 + \hc \right] \,.
  \label{eq:validity_2hdm_potential}
\end{multline}
%
The mass terms $m^2_{ij}$ and the dimensionless self-couplings
$\lambda_i$ are real parameters. The doublet VEVs
$\langle \phi_j^0 \rangle = v_j / \sqrt{2} $ are parametrised by their
ratio $\tan \beta = v_2/v_1$.  For the Yukawa couplings, there are
four possible scenarios that satisfy the SM flavour symmetry and
preclude tree-level flavour-changing neutral
currents (FCNCs)~\cite{Glashow:1976nt}:
%
\begin{itemize}
\item type I, where all fermions couple to just one Higgs doublet
  $\phi_2$;
\item type II, where up-type (down-type) fermions couple exclusively
  to $\phi_2$ ($\phi_1$);
\item lepton-specific, with a type-I quark sector and a type-II lepton
  sector; and
\item flipped, with a type-II quark sector and a type-I lepton sector.
\end{itemize}
%
For simplicity, we restrict our discussion to type I and type II. In
all four cases, the absence of tree-level FCNCs is protected by a
global $\mathbb{Z}_2$ discrete symmetry $\phi_i \to (-1)^{i}\,\phi_i$
(for $i=1,2$). This symmetry is softly broken by the mixed mass term
$m_{12}$.

The physical degrees of freedoms are two neutral $CP$-even scalars
$h^0$, $H^0$, one neutral $CP$-odd scalar $A^0$, and a set of charged
scalars $H^\pm$, parametrised by the mixing angle between the
$CP$-even scalars $\alpha$. For a detailed account of the model setup,
see Appendix~\ref{sec:appendix_models_2hdm}.



%%%%%%%%%%%%%%%%%%%%%%%%%%%%%%%%%%%%%%%%%%%%%%%%%%%%%%%%%%%%
\subsubsection{Signatures and decoupling patterns}
%%%%%%%%%%%%%%%%%%%%%%%%%%%%%%%%%%%%%%%%%%%%%%%%%%%%%%%%%%%%

Just as the singlet extension, the 2HDM predicts two types of LHC
signatures: first, scalar and VEV mixing lead to modified light Higgs
couplings. Unlike for the singlet extension, these coupling
modifications are not universal and reflect the more flexible flavour
structure as well as the multiple scales of the model. Second, there
are three new heavy resonances $H^0$, $A^0$, and $H^\pm$, which should
have near-degenerate masses to avoid custodial symmetry breaking.

The light Higgs coupling to weak bosons $V=W,Z$ always scales like
%
\begin{equation}
  \Delta_V = \sin (\beta - \alpha) - 1
  = - \frac{\cos^2(\beta - \alpha)}{2} + \ord{\cos^4(\beta - \alpha)} \,.
  \label{eq:validity_2hdm_higgs_vector_coupling}
\end{equation}
%
We can insert the leading contribution of a mass-degenerate heavy
Higgs sector and find
%
\begin{equation}
  \Delta_V \approx \frac{\sin^2 (2\beta)}{8} \, \left(\frac{v}{m_{A^0}} \right)^4 \,.
  \label{eq:validity_2hdm_decoup}
\end{equation}
%
While in the singlet model the light Higgs coupling to gauge bosons is
shifted at $\ord {v^2/m_H^2}$, see
\autoref{eq:validity_singlet_decoup}, the same coupling is now
affected at $\ord{v^4/m_{A^0}^4}$, corresponding to a dimension-eight
effect.

The couplings to the fermions on the other hand are modified at
$\ord{v^2 / m_{A_0}^2}$. For up-type quarks, we find
%
\begin{equation}
  1 + \Delta_t = \dfrac {\cos \alpha} {\sin \beta} \,.
\end{equation}
%
The couplings to down-type quarks and leptons are
%
\begin{equation}
  1 + \Delta_b = 1 + \Delta_\tau = \frac {\cos \alpha} {\sin \beta}
\end{equation}
%
in a type-I model and
%
\begin{equation}
  1 + \Delta_b = 1 + \Delta_\tau = - \frac {\sin \alpha} {\cos \beta}
  \label{eq:validity_2hdm_last_coupling}
\end{equation}
%
in a type-II 2HDM.

Finally, a $H^\pm$ loop contributes to the Higgs-photon coupling. The
expression for this coupling shift is given in
\autoref{eq:appendix_models_2hdm_haa} in
Appendix~\ref{sec:appendix_models_2hdm}.

\newparagraph
%
Two aspects turn the decoupling in the general 2HDM into a challenge:
first, delayed decoupling effects appear after electroweak symmetry
breaking~\cite{Haber:2000kq}. For example, in type-II models we
find~\cite{Lopez-Val:2013yba}
%
\begin{equation}
  \Delta_b
  %
  % = - \tan \beta \, \sqrt{|2 \Delta_V|} + \Delta_V + \ord{\Delta_V^{3/2}}
  %
  \approx - \tan \beta \, \frac{\sin (2\beta)} 2 \, \left( \frac{v}{m_{A^0}} \right)^2 \,.
  \label{eq:validity_2hdm_delayed}
\end{equation}
%
This correction to the bottom Yukawa coupling corresponds to a
dimension-six effect, and already moderate values of $\tan \beta$
significantly delay the decoupling of the heavy 2HDM states in the
Yukawa sector.

Second, unlike in the MSSM, the Higgs self-couplings
$\lambda_1 \dots \lambda_5$ and $m_{12}$ are not bounded from
above. In combinations like $\lambda_j v^2$, potentially enhanced with
factors of $\tan 'beta$, they contribute to the masses of the heavy
Higgs bosons and to the interactions of the SM-like Higgs state,
effectively inducing new energy scales.

Such additional mass scales driven by $v$ leads to
problems with any EFT derived from and matched to the full theory
assuming only one new physics scale. While this should not be viewed
as a problem of the EFT approach in general, it will require a
$v$-improved matching procedure.



%%%%%%%%%%%%%%%%%%%%%%%%%%%%%%%%%%%%%%%%%%%%%%%%%%%%%%%%%%%%
\subsubsection{Dimension-six description}
%%%%%%%%%%%%%%%%%%%%%%%%%%%%%%%%%%%%%%%%%%%%%%%%%%%%%%%%%%%%

We first follow our default procedure and match the effective theory
to the 2HDM in the unbroken electroweak phase. To this end, we first
rotate $\phi_1$ and $\phi_2$ into the so-called Higgs basis, where
only one Higgs doublet obtains a vacuum expectation value,
$\langle \phi_l \rangle = v/\sqrt{2}$,
$\langle \phi_h \rangle = 0$~\cite{Glashow:1976nt, Davidson:2005cw}.
This doublet $\phi_l$ is then identified with the SM-like Higgs
doublet, while the other doublet $\phi_h$ is integrated out. This
doublet $\phi_l$ is then identified with the SM-like Higgs doublet,
while the other doublet $\phi_h$ is integrated out.  Its decoupling is
described by the mass scale
%
\begin{align}
  \Lambda_{\text{default}}^2 = m^2_{11}\sin^2\beta + m^2_{22}\cos^2\beta + m^2_{12} \sin (2\beta) \,.
\end{align}

The 2HDM generates a number of dimension-six operators at tree level,
for which the Wilson coefficients depend on the flavour
structure. While the up-type Yukawa coupling is always modified the
same way, the down-type and lepton couplings are different for type-I
and type-II. With the definition
%
\begin{equation}
  \overbar{\lambda} \equiv \frac{\sin (2\beta) } 2
    \left[\frac{\lambda_1} 2 - \frac{\lambda_2} 2 
    + \left(\frac{\lambda_1} 2 + \frac{\lambda_2} 2
    - \lambda_3 - \lambda_4 - \lambda_5 \right) \cos (2\beta) \right]
\end{equation}
%
we find
%
\begingroup%
\allowdisplaybreaks%
\begin{align}
  f_t &= - \frac {\overbar{\lambda} y_t} {\tan \beta} \,, \notag \\
  %
  f_b &=
                 \begin{cases}
                   - \dfrac {\overbar{\lambda} y_b} {\tan \beta}  & \quad \text{type I} \,, \\
                   \overbar{\lambda} y_b \tan \beta  & \quad \text{type II} \,,
                 \end{cases}
                                                       \notag \\
  %
  f_\tau &=
                 \begin{cases}
                   - \dfrac {\overbar{\lambda} y_\tau} {\tan \beta}  & \quad \text{type I} \,, \\
                   \overbar{\lambda} y_\tau \tan \beta  & \quad \text{type II} \,.
                 \end{cases}
  \label{eq:validity_2hdm_matching1}
\end{align}%
\endgroup
%
Here $y_t$, $y_b$, $y_\tau$ refer to the SM values of the respective
Yukawa couplings, $y_f = \sqrt{2} m_f / v$.

The contribution of the $H^\pm$ loop to the Higgs-photon coupling is
mapped onto $\ope{BB}$ with a Wilson coefficient
%
\begin{multline}
  f_{BB} = \frac {\left(\tan \beta + \cot \beta \right) } {3072 \,
    \pi^2} \, \Biggl [\left(\lambda_1 + \lambda_2 - 2 \lambda_3 + 6
    \lambda_4 + 6 \lambda_5 - 8 \frac
    {m_{h^0}^2} {v^2} \right) \sin (2 \beta) \\
%
  + 2 ( \lambda_1 - \lambda_2) \sin (4 \beta) + (\lambda_1 + \lambda_2
  - 2 \lambda_3 - 2 \lambda_4 - 2 \lambda_5 ) \sin (6 \beta) \Biggr]
  \,.
  \label{eq:validity_2hdm_matching2}
\end{multline}
%
In the effective Lagrangian there are no non-decoupling term of
$\ord{\Lambda^0}$ because the charged Higgs loop decouples in the
limit $m_{A^0} \to \infty$ with finite $\lambda_i$. If instead we keep
$m_{12}$ fixed and let one of the couplings $\lambda_i$ grow with
$m_{A^0}$, the charged Higgs does not decouple.
%
% Interestingly,
% Equations\,\eqref{eq:validity_2hdm_matching1} to
% \eqref{eq:validity_2hdm_matching2} show that in this model it is
% possible to realise alignment without decoupling
% scenarios~\cite{Gunion:2002zf, Craig:2013hca, Carena:2013ooa,
%   Delgado:2013zfa}, where the limit of SM-like couplings is achieved
% via very small prefactors of $(v/m_{A^0})^2$, while the additional
% Higgs states can remain moderately light\,---\,and hence potentially
% within LHC reach.
%
For a derivation of these results see
Appendix~\ref{sec:appendix_models_2hdm}.

Upon electroweak symmetry breaking, the physical heavy Higgs masses
$m_{H^0}$, $m_{A^0}$, and $m_{H^{\pm}}$ acquire contributions
$\sim \lambda_i v^2$ from the electroweak VEV in addition to the heavy
scale $\Lambda_{\text{default}}$ defined above. We therefore again
consider an alternative $v$-improved matching where the matching scale
is
%
\begin{equation}
  \Lambda_{\text{$v$-improved}} = m_{A^0} \,.
\end{equation}
%
In this setup, the Wilson coefficients in
Equations~\eqref{eq:validity_2hdm_matching1} and
\eqref{eq:validity_2hdm_matching2} remain unchanged. The two matching
schemes can exhibit significant differences in the 2HDM since the
pseudoscalar mass
$m^2_{A^0} = m_{12}^2/(\sin\beta\cos\beta) -\lambda_5\,v^2$ does not
coincide with $\Lambda_{\text{default}}$ over wide ranges of the
parameter space.



%%%%%%%%%%%%%%%%%%%%%%%%%%%%%%%%%%%%%%%%%%%%%%%%%%%%%%%%%%%%
\subsubsection{Benchmark points}
%%%%%%%%%%%%%%%%%%%%%%%%%%%%%%%%%%%%%%%%%%%%%%%%%%%%%%%%%%%%

\begin{table} 
  \begin{tabular}{c c rrrrrrr }
    \toprule
    %
    \multirow{2}{*}{} && \multicolumn{7}{c}{2HDM} \\
    %
    \cmidrule{3-9}
    %
    && Type & $\tan\beta$ & $\alpha/\pi$
    & $m_{12} $ & $m_{H^0} $ & $m_{A^0} $ & $m_{H^\pm}$ \\
    %
    \midrule
    %
    D1 && I & $1.5$ & $-0.086$ & $45$ & $230$ & $300$ & $350$ \\
    D2 && II & $15$ & $-0.023$ & $116$ & $449$ & $450$ & $457$ \\
    D3 && II & $10$ & $0.032$ & $157$ & $500$ & $500$ & $500$ \\
    D4 && I & $20$ & $0$ & $45$ & $200$ & $500$ & $500$ \\
    %
    \bottomrule
  \end{tabular}
 \caption[Benchmarks for the 2HDM model]{Benchmarks
   for the 2HDM extension. We show the model
   parameters and the heavy Higgs masses. All masses are in GeV.}
 \label{tbl:validity_2hdm_benchmarks}
\end{table}

In \autoref{tbl:validity_2hdm_benchmarks} we define four benchmark
points for the 2HDM. They are in agreement with all constraints at the
time of publication of Reference~\cite{Brehmer:2015rna}, implemented with
the help of \toolfont{2HDMC}~\cite{Eriksson:2009ws},
\toolfont{HiggsBounds}~\cite{Bechtle:2008jh, Bechtle:2011sb},
\toolfont{SuperIso}~\cite{Mahmoudi:2008tp}, and
\toolfont{HiggsSignals}~\cite{Bechtle:2013xfa}. To better illustrate
certain model features, in some scenarios we tolerate deviations
between $1\,\sigma$ and $2\,\sigma$ in the Higgs couplings
measurements.

The key physics properties of the different 2HDM scenarios can be
summarised as:
%
\begin{enumerate}
\item[D1] \emph{Moderate decoupling}: with Higgs couplings shifts of
  up to $2\sigma$ in terms of the LHC constraints.  This generates
  $\Delta_{\tau,b,t} \approx \mathcal{O}(15\%)$ as well as a large
  $h^0 H^+ H^-$ coupling. Additional Higgs masses around
  $250\dots350$~GeV can leave visible imprints.
%
\item[D2] \emph{Supersymmetric}: reproducing the characteristic mass
  splittings and Higgs self-couplings of the MSSM with light
  stops~\cite{Carena:2013ytb}.
%
\item[D3] \emph{Sign-flipped bottom Yukawa}: this is possible in
  type-II models at large $\tan\beta$, as shown in
  Equation\,\eqref{eq:validity_2hdm_delayed}~\cite{Ferreira:2014naa}. This
  can be viewed as a manifestation of a delayed
  decoupling~\cite{Haber:2000kq}.
%
\item[D4] \emph{Fermiophobic heavy Higgs}: possible only in type-I
  models for $\sin\alpha =0$. The heavy Higgs $H^0$ is relatively
  light, but essentially impossible to observe at the
  LHC~\cite{Hespel:2014sla}.
\end{enumerate}

In Tables~\ref{tbl:validity_2hdm_eft_default} and
\ref{tbl:validity_2hdm_eft_v-improved} we show the heavy scales
$\Lambda$ and the Wilson coefficients for the EFT in the two matching
schemes. With the exception of benchmark D2, the suppression scales
are drastically different. The matching in the unbroken phase is
particular pathological in benchmark D1, where
$\Lambda_{\text{default}}^2$ is negative and the signs of the Wilson
coefficients are switched compared to the $v$-improved matching.

\begin{table}
  \begin{tabular}{c c rrrrr }
    \toprule
    %
    \multirow{2}{*}{}
    && \multicolumn{5}{c}{Default EFT} \\
    %
    \cmidrule{3-7} 
    %
    && $|\Lambda|$~[GeV] & $f_t$ & $f_{b}$  & $f_{\tau}$ & $f_{BB}$ \\
    %
    \midrule
    %
    D1 && $100$ & $0.12$ & $0.003$ & $0.001$ & $0.009$ \\
    D2 && $448$ & $0.00$ & $-0.006$ & $-0.002$  & $-0.001$ \\
    D3 && $99$ & $-0.07$ & $0.206$ & $0.077$ &  $-0.016$\\
    D4 && $142$ & $0.00$ & $0.000$ & $0.000$ &  $-0.023$\\
    %
    \bottomrule
  \end{tabular}
  \caption[Default EFT description for the 2HDM benchmarks]{Matching scales and Wilson coefficients for the effective
    theory matched to the 2HDM, based on the default matching in the unbroken phase.}
 \label{tbl:validity_2hdm_eft_default}
\end{table}

\begin{table}
  \begin{tabular}{c c rrrrr}
    \toprule
    %
    \multirow{2}{*}{}
    && \multicolumn{5}{c}{$v$-improved EFT} \\
    %
    \cmidrule{3-7} 
    %
    && $\Lambda$~[GeV] & $f_t$ & $f_{b}$  & $f_{\tau}$ & $f_{BB}$ \\
    %
    \midrule
    %
    D1 && $300$ & $-0.12$ & $-0.003$ & $-0.001$ & $-0.009$ \\
    D2 && $450$ & $0.00$ & $-0.006$ & $-0.002$ & $-0.001$ \\
    D3 && $500$ & $-0.07$ & $0.206$ & $0.077$ & $-0.016$ \\
    D4 && $500$ & $0.00$ & $0.000$ & $0.000$ & $-0.023$ \\
    %
    \bottomrule
  \end{tabular}
  \caption[$v$-improved EFT description for the 2HDM benchmarks]{Matching scales and Wilson coefficients for the effective
    theory matched to the 2HDM, based on the $v$-improved matching in the unbroken phase
    with $\Lambda = m_{A^0}$.}
 \label{tbl:validity_2hdm_eft_v-improved}
\end{table}



%%%%%%%%%%%%%%%%%%%%%%%%%%%%%%%%%%%%%%%%%%%%%%%%%%%%%%%%%%%%
\subsubsection{Higgs couplings and total production rates}
%%%%%%%%%%%%%%%%%%%%%%%%%%%%%%%%%%%%%%%%%%%%%%%%%%%%%%%%%%%%

\autoref{tbl:validity_2hdm_couplings_tree} shows the tree-level
coupling shifts of the light Higgs in the three models. The results
confirm that the default matching defined in the unbroken phase does
not reproduce the coupling patterns of the full model at all. We
conclude that an EFT matched to the 2HDM in the unbroken electroweak
phase is essentially useless, and we have to rely on $v$-improved
matching. For simplicity, we will from now on leave out the results
based on the default matching, which only confirm these initial
results.

The $v$-improved matching, on the other hand, essentially captures
most of the coupling shifts. It still fails to describe shifts in the
couplings to weak bosons, which correspond to a dimension-eight
operator as discussed above.\footnote{Note that the operator
  $\ope{BB}$ does contribute to the $h^0 VV$ coupling, representing
  the effect of a charged Higgs loop. But as our results show, this
  effect is neglible.} Unlike in the singlet model, the $v$-improved
EFT also struggles with scenarios of very light new physics such as
D1. But all in all, it performs well in situations with a modest scale
hierarchy such as benchmarks D2.

A particularly interesting scenario is described by benchmark D3. In
the full model, the bottom Yukawa is exactly sign-flipped, a signature
hardly visible at the LHC.  Generating such a signature from
higher-dimensional operators requires their contributions to be twice
as large as the SM Yukawa coupling due to the enhancement of
$v/\Lambda$ by a large coupling. The EFT fails to capture this
coupling shift fully, leading to a significantly different prediction
for the Higgs decay into bottom quarks.

\begin{table}
  \begin{tabular}{c c rr c rrr c rrr}
    \toprule
    %
    \multirow{2}{*}{}
    && \multicolumn{2}{c}{$\Delta_V$} && \multicolumn{3}{c}{$\Delta_t$}
    && \multicolumn{3}{c}{$\Delta_b=\Delta_\tau$} \\
    %
    \cmidrule{3-4} \cmidrule{6-8} \cmidrule{10-12}
    %
    && 2HDM & EFT
    && 2HDM & dEFT & $v$EFT
    && 2HDM & dEFT & $v$EFT \\
    %
    \midrule
    %
    D1 && $-0.05$ & $0.00$ && $0.16$ & $-0.74$ & $0.08$ && $0.16$ & $-0.74$ & $0.08$ \\
    D2 && $0.00$ & $0.00$ && $0.00$ & $0.00$ & $0.00$ && $0.07$ & $0.07$ & $0.07$ \\
    D3 && $-0.02$ & $0.00$ && $0.00$ & $0.46$ & $0.02$ && $-2.02$ & $-46.5$ & $-1.84$ \\
    D4 && $0.00$ & $0.00$ && $0.00$ & $0.00$ & $0.00$ && $0.00$ & $0.00$ & $0.00$ \\
    %
\bottomrule
  \end{tabular}
  \caption[Tree-level couplings in the 2HDM]{Normalised
    tree-level couplings of the light Higgs in our
    2HDM benchmarks, comparing the full 2HDM model to the EFT based on the default
    matching (``dEFT'') and that based on $v$-improved matching (``$v$EFT'').}
  \label{tbl:validity_2hdm_couplings_tree}
\end{table}

In \autoref{tbl:validity_2hdm_couplings_loop} we repeat this
analysis for the loop-induced couplings of an on-shell Higgs to gluons
and photons. A large part of these coupling shifts stems from the
modified Higgs-top and Higgs-$W$ couplings, leading to good agreement
in the same scenarios where the tree-level couplings were described
accurately. Separating the $H^\pm$ contribution to the Higgs-photon
coupling, we find that $\ope{BB}$ captures its effect very well.

\begin{table}
  \begin{tabular}{c c rr c rrrr}
    \toprule
    %
    \multirow{2}{*}{}
    && \multicolumn{2}{c}{$\Delta_g$} && \multicolumn{4}{c}{$\Delta_\gamma$} \\
    %
    \cmidrule{3-4} \cmidrule{6-9}
    %
    && 2HDM & $v$-improved EFT
    && \multicolumn{2}{c}{2HDM} & \multicolumn{2}{c}{$v$-improved EFT} \\
    %
    \midrule
    %
    D1 && $0.16 + 0.00 \,\im$ & $0.08 + 0.00 \,\im$ && $-0.16$ & ($-0.05$) & $-0.10$ & ($-0.07$) \\
    D2 && $0.00 + 0.00 \,\im$ & $0.00 + 0.00 \,\im$ && $0.00$ & ($0.00$) & $0.00$ & ($0.00$) \\
    D3 && $0.07 - 0.09 \,\im$ & $0.02 + 0.00 \,\im$ && $-0.08$ & ($-0.05$) & $-0.05$ & ($-0.05$) \\
    D4 && $0.00 + 0.00 \,\im$ & $0.00 + 0.00 \,\im$ && $-0.05$ & ($-0.05$) & $-0.05$ & ($-0.05$) \\
    %
    \bottomrule
  \end{tabular}
  \caption[Loop-induced couplings in the 2HDM]{Normalised
    couplings of the light Higgs to gluons and
    photons in our 2HDM benchmarks.  The bottom loop leads to small
    imaginary parts of $\Delta_g$ and $\Delta_\gamma$.  For the
    Higgs-photon coupling, these imaginary parts are always smaller than
    $1\%$ of the real part of the amplitude and neglected here.  The
    numbers in parentheses ignore the modification of the Higgs-fermion
    couplings, allowing us to separately analyse how well the $H^\pm$ loop
    is captured by $\ope{BB}$.}
  \label{tbl:validity_2hdm_couplings_loop}
\end{table}

\autoref{tbl:validity_2HDM_rates} compares total production rates at
the LHC. Depending on the benchmark, the dimension-six truncation
leads to up to $10 \%$ departures, in agreement with the coupling
deviations.

\begin{table}
    \begin{tabular}{c c rrr}
      \toprule
      \multirow{2}{*}{}
      && \multicolumn{3}{c}{$\sigma_\text{$v$-improved EFT} / \sigma_\text{2HDM}$} \\
      %
      \cmidrule{3-5}
      %
      && ggF & WBF & $Vh$ \\
      \midrule
      D1 && $0.872$ & $1.109$ & $1.108$ \\
      D2 && $1.001$ & $1.000$ & $1.000$ \\
      D3 && $1.022$ & $1.042$ & $1.042$ \\
      D4 && $1.001$ & $1.001$ & $1.003$ \\
      %
      \bottomrule
    \end{tabular}
    \caption[Total Higgs production rates in the 2HDM]{Cross
      section ratios of the $v$-improved dimension-six
      approximation to the full 2HDM at the LHC. The statistical
      uncertainties on these ratios are below 0.4\%.}
  \label{tbl:validity_2HDM_rates}
\end{table}



%%%%%%%%%%%%%%%%%%%%%%%%%%%%%%%%%%%%%%%%%%%%%%%%%%%%%%%%%%%%
\subsubsection{Distributions}
%%%%%%%%%%%%%%%%%%%%%%%%%%%%%%%%%%%%%%%%%%%%%%%%%%%%%%%%%%%%

As for the singlet model, the phenomenology of the 2HDM is mostly
reflected in the coupling patterns discussed above, with little new
kinematic effects.  In the left panel of
\autoref{fig:validity_2hdm_distributions} we illustrate the coupling
deviations in gluon-fusion Higgs production with a decay
$h\to \tau^+ \tau^-$, showing how the full 2HDM and the ($v$-improved)
EFT give substantially different predictions for the size of the Higgs
signal.

While on-shell Higgs decays to photons are generally well described by
the EFT, this changes for off-shell Higgs production. At
$m_{\gamma \gamma} \gtrsim 2 m_{H^\pm}$, the $H^\pm$ in the loop can
resolve the charged Higgs, enhancing the size of its contribution
significantly. This effect is not captured by the effective operator
and leads to a different behaviour of the amplitude
$g g \to h^0 \to \gamma \gamma$ between the full and effective model,
as shown in the right panel of
\autoref{fig:validity_2hdm_distributions}. However, the tiny rate
and the large combinatorial background mean that this discrepancy will
be irrelevant for LHC phenomenology. Similar threshold effects have
been computed for the top-induced Higgs-gluon coupling and appear to
be similarly irrelevant in practice~\cite{Buschmann:2014twa}.

\begin{figure}
  \includegraphicsdummy[width=0.49\textwidth]{fig/validity/2HDM_tautau.pdf}%
  \includegraphicsdummy[width=0.49\textwidth]{fig/validity/2HDM_AA.pdf}%
  \caption[Kinematic distributions in the 2HDM]{Left: $m_{\tau \tau}$ distribution in the
    $gg \to \tau^+ \tau^-$ channel. Right: off-shell behaviour of the
    process $gg \to h^0 \to \gamma \gamma$ in 2HDM benchmark D1, only
    taking into account the Higgs diagrams. At
    $m_{\gamma \gamma} \gtrsim 2 m_{H^\pm} = 700\ \gev$, the charged
    Higgs threshold is visible.}
  \label{fig:validity_2hdm_distributions}
\end{figure}

The situation in Higgs pair production resembles the observations in
the singlet model~\cite{Baur:2003gp, Hespel:2014sla,
  Baglio:2014nea}. The agreement can be worse already at threshold if
Higgs-top coupling shifts are not correctly captured by the effective
model.



%%%%%%%%%%%%%%%%%%%%%%%%%%%%%%%%%%%%%%%%%%%%%%%%%%%%%%%%%%%%
\subsubsection{Summary}
%%%%%%%%%%%%%%%%%%%%%%%%%%%%%%%%%%%%%%%%%%%%%%%%%%%%%%%%%%%%

Eventually, the 2HDM discussion leads us to a similar conclusion as
the singlet model: as long as the mixing is small and Higgs-gauge
coupling shifts are neglible, all the LHC probes in single Higgs
production is a set of three coupling modifications $\Delta_t$,
$\Delta_b$, $\Delta_\tau$, and the charged Higgs loop contribution to
the Higgs-photon coupling. These aspects of Higgs phenomenology are
generally well captured by an appropriately defined EFT. Problems
arise in scenarios with very light new Higgs bosons; when the
Higgs-$W$ and Higgs-$Z$ couplings are modified, which requires
dimension-eight operators; or in the special case of Higgs pair
production.

A naive construction of the EFT by matching the effective
dimension-six Lagrangian to the 2HDM in the gauge symmetric phase
fails to correctly describe the modified Higgs boson dynamics in
typical 2HDM scenarios, since terms formally suppressed by powers of
$v^2/\Lambda^2$ can be crucial for phenomenologically relevant
scenarios.



%%%%%%%%%%%%%%%%%%%%%%%%%%%%%%%%%%%%%%%%%%%%%%%%%%%%%%%%%%%%
\subsection{Scalar top partners}
\label{sec:validity_stops}
%%%%%%%%%%%%%%%%%%%%%%%%%%%%%%%%%%%%%%%%%%%%%%%%%%%%%%%%%%%%

%%%%%%%%%%%%%%%%%%%%%%%%%%%%%%%%%%%%%%%%%%%%%%%%%%%%%%%%%%%%
\subsubsection{Model setup}
%%%%%%%%%%%%%%%%%%%%%%%%%%%%%%%%%%%%%%%%%%%%%%%%%%%%%%%%%%%%

New coloured scalar particles are, strictly speaking, not an extension
of the SM Higgs sector, but they can lead to interesting modifications
of the LHC observables. We consider a scalar top-partner sector
mimicking the stop and sbottom sector of the MSSM. Its Lagrangian has
the form
%
\begin{multline}
  \lgr{top partners} \supset
  %
  (D_{\mu} \tilde{Q})^\dagger  D^\mu \tilde{Q}
  + (D_\mu \tilde{t}_R )^* D^\mu \tilde{t}_R
  - M^2 \, \tilde{Q}^\dagger \tilde{Q}
  - M^2\, {\tilde{t}_R}^* \tilde{t}_R \\
  %
  - \kappa_{LL} \, (\phi \cdot \tilde{Q})^\dagger \, (\phi \cdot\tilde{Q})
  - \kappa_{RR} \, ({\tilde{t}_R}^*{\tilde{t}_R}) \, (\phi^\dagger\,\phi)
  - \left[ \kappa_{LR} \, M \, {\tilde{t}_R}^*\,(\phi \cdot \tilde{Q}) + \hc \right] \,.
  \label{eq:validity_stop_lagrangian}
\end{multline}
%
Here, $\tilde{Q}$ and ${\tilde{t}_R}$ are the additional isospin
doublet and singlet in the fundamental representation of
$SU(3)_C$. Their mass terms can be different, but for the sake of
simplicity we unify them to a single heavy mass scale $M$. The singlet
state ${\tilde{b}_{R}}$ is assumed to be heavier and integrated
out.

This leaves us with three physical degrees of freedom, the scalars
$\tilde{t}_1$, $\tilde{t}_2$ and $\tilde{b}_2= \tilde{b}_L$. The
eigenvalues of the stop mass matrix
%
\begin{equation}
  \mathcal{M}_{\tilde{t}}
  = \twomat {M^2 + \kappa_{LL} \, \dfrac{v^2}{2}} {\kappa_{LR} \, \dfrac{Mv}{\sqrt{2}} }
  {\kappa_{LR} \, \dfrac{Mv}{\sqrt{2}}} {M^2 + \kappa_{RR}\,\dfrac{v^2}{2}}
  \label{eq:validity_stops_mass}
\end{equation}
%
define two masses $m_{\tilde{t}_1} < m_{\tilde{t}_2}$ and a mixing
angle $\theta_{\tilde{t}}$. Again, we provide a detailed description
of the model setup in Appendix~\ref{sec:appendix_models_stops}.



%%%%%%%%%%%%%%%%%%%%%%%%%%%%%%%%%%%%%%%%%%%%%%%%%%%%%%%%%%%%
\subsubsection{Signatures and decoupling patterns}
%%%%%%%%%%%%%%%%%%%%%%%%%%%%%%%%%%%%%%%%%%%%%%%%%%%%%%%%%%%%

The main new physics effects in the Higgs sector are loop-induced
modifications of the Higgs interactions, most notably to $\Delta_g$,
$\Delta_\gamma$, $\Delta_W$, and $\Delta_Z$, possibly including new
Lorentz structures. The Yukawa couplings do not change at one loop
because we do not include gauge boson partners. As a side remark, the
2HDM described in \autoref{sec:validity_2hdm} combined with the
scalar top partners given here corresponds to an effective
description of the Minimal Supersymmetric Standard Model in the limit
of infinitely heavy gauginos, sleptons, and light-flavour squarks.

In the limit of small $\theta_{\tilde{t}}$, the leading correction to
the $hVV$ coupling scales like
%
\begin{equation}
  \Delta_V
  \approx
  \frac{\kappa_{LL}^2}{16 \pi^2} \, \left( \frac{v}{m_{\tilde{t}_{1}}} \right)^2 \,.
  \label{eq:validity_stops_couplings}
\end{equation}
%
This shift can be sizeable for relatively low stop and sbottom masses
combined with large couplings $\kappa_{ij}$ to the Higgs sector.

As already noted for the 2HDM, the decoupling of the heavy scalars
becomes non-trivial in the presence of a Higgs VEV. Following
\autoref{eq:validity_stops_mass}, the masses of the heavy scalars
$m_{\tilde{t}_1}$, $m_{\tilde{t}_2}$ are not only controlled by the
mass scale in the symmetric phase of the electroweak symmetry $M$, but
they receive additional contributions of the type $\kappa_{LR} \, vM$,
$\kappa_{LL} v^2$, or $\kappa_{RR} v^2$ after electroweak symmetry
breaking. This leads to a mass splitting of order $v$ between masses
of order $M$, which is increased by large values of the couplings
$\kappa_{i}$.
%
% This means that in the full model the
% decoupling is best described in terms of $m_{\tilde{t}_{1}} < M$.



%%%%%%%%%%%%%%%%%%%%%%%%%%%%%%%%%%%%%%%%%%%%%%%%%%%%%%%%%%%%
\subsubsection{Dimension-six description}
%%%%%%%%%%%%%%%%%%%%%%%%%%%%%%%%%%%%%%%%%%%%%%%%%%%%%%%%%%%%

This motivates us to again define two different matching
schemes. First, we stick to our default prescription and carry out the
matching of the linear EFT Lagrangian to the full model in the
unbroken phase. The matching scale it then dictated by the intrinsic
mass scale of the heavy fields,
%
\begin{equation}
  \Lambda_{\text{default}} = M \,,
\end{equation}
%
completely oblivious to contributions to the masses from the
electroweak VEV.The suppression scale of loop effects in the complete
model and this matching scale in the EFT only agree in the limit
$M - m_{\tilde{t}_{1}} \sim v \ll M$.

In this dimension-six approach the stop loops generate a number of
operators~\cite{Henning:2014wua, Drozd:2015kva, Drozd:2015rsp},
%
\begingroup%
\allowdisplaybreaks%
\begin{align}
  f_{\phi,1}
  &= - \frac{1}{2(4\pi)^2} \,
    \left[\kappa_{LL}^2 - \frac{\kappa_{LL}\,\kappa_{LR}^2}{2}
    + \frac{\kappa_{LR}^4}{10} \right] \,, \notag \\
  %
  f_{\phi,2}
  &= \frac{1}{4(4\pi)^2}\,
    \left[ 2\kappa_{RR}^2
    -  \kappa_{RR} \kappa_{LR}^2
    + \frac{\kappa_{LR}^4}{5} \right] \,, \notag \\
  %
  f_{GG}
  &= \frac{g_s^2}{24\,(4\pi)^2} \,
    \left[ \kappa_{LL} + \kappa_{RR} - \kappa_{LR}^2\right] \,, \notag \\
  %
  f_{BB}
  &= - \frac{1}{36\,(4\pi)^2} \,
    \left[ \kappa_{LL} + 16 \kappa_{RR} - \frac {67} {10} \kappa_{LR}^2 \right]\,, \notag \\
  %
  f_{WW}
  &= - \frac{1}{4\,(4\pi)^2} \,
    \left[\kappa_{LL} - \frac{3}{10} \kappa_{LR}^2 \right] \,, \notag \\
  %
  f_{BW}
  &= \frac{1}{12\,(4\pi)^2} \,
    \left[2 \kappa_{LL} - \frac{11}{5} \kappa_{LR}^2 \right] \,, \notag \\
  %
  f_B
  &= \frac{1}{20\,(4\pi)^2} \,
    \kappa_{LR}^2 \,, \notag \\
  %
  f_{W}
  &= \frac{1}{20\,(4\pi)^2} \,
    \kappa_{LR}^2 \,.
    \label{eq:validity_stops_wilson_coefficients}
\end{align}%
\endgroup
%
Unlike the tree-level effects in the previous two models, the top
partner loops do not only induce modifications to the SM Higgs
couplings, but induce new Lorentz structures. Note that some of these
operators are tightly constrained from electroweak precision data, see
\autoref{sec:foundations_heft_operators}. We will ignore these
constraints for our discussion of Higgs physics.

In addition, we define a $v$-improved matching. Like in the 2HDM we
pick the matching scale as a physical mass in the broken phase,
%
\begin{equation}
  \Lambda = m_{\tilde{t}_{1}} \,.
\end{equation}
%
The Wilson are the same as in
\autoref{eq:validity_stops_wilson_coefficients}.



%%%%%%%%%%%%%%%%%%%%%%%%%%%%%%%%%%%%%%%%%%%%%%%%%%%%%%%%%%%%
\subsubsection{Benchmark points}
%%%%%%%%%%%%%%%%%%%%%%%%%%%%%%%%%%%%%%%%%%%%%%%%%%%%%%%%%%%%

\begin{table}
\begin{tabular}{c c rrrr c rrr}
  \toprule
  %
  \multirow{2}{*}{}
  && \multicolumn{8}{c}{Scalar top-partner model} \\
  %
  \cmidrule{3-10}
  %
  && $M$ & $\kappa_{LL}$ & $\kappa_{RR}$ & $\kappa_{LR}$
  && $m_{\tilde{t}_{1}}$ & $m_{{\tilde{t}_{2}}}$ & $\theta_{\tilde{t}}$ \\
  %
  \midrule
  %
  P1 && $500$ & $-1.16$ & $2.85$ & $0.147$ && $500$ & $580$ & $-0.15$ \\
  P2 && $350$ & $-3.16$ & $-2.82$ & $0.017$ && $173$ & $200$ & $-0.10$ \\
  P3 && $500$ & $-7.51$ & $-7.17$ & $0.012$ && $173$ & $200$ & $-0.10$ \\
  %
  \bottomrule
 \end{tabular}
 \caption[Benchmarks for the top partners]{Benchmarks for the
   scalar top-partner scenario. We show Lagrangian parameters (left)
   and physical parameters (right). All masses are given in GeV.}
  \label{tbl:validity_stops_benchmarks}
\end{table}

As \autoref{eq:validity_stops_couplings} suggests, sizeable loop
corrections to the $hVV$ coupling require light top partners with
unrealistically strong couplings to the Higgs
sector~\cite{Hollik:2008xn}. In
\autoref{tbl:validity_stops_benchmarks} we define a set of benchmark
points with this aim in mind. The corresponding Wilson coefficients in
our two matching schemes are given in
\autoref{tbl:validity_stops_eft}.

\begin{table}
  \begin{tabular}{c c rrrr c rrrr}
    \toprule
    %
    \multirow{2}{*}{}
    %
    && \multicolumn{4}{c}{EFT} && \multicolumn{4}{c}{EFT ($v$-improved)} \\
    %
    \cmidrule{3-6} \cmidrule{8-11}
    %
    && $\Lambda$ & $f_{\phi,2}$ & $f_{WW}$ & $f_W$
    && $\Lambda$ & $f_{\phi,2}$ & $f_{WW}$ & $f_W$ \\
    %
    \midrule
    %
    P1 && $500$ & $0.026$ & $0.000$ & $0.000$ && $500$ & $0.026$ & $0.000$ & $0.000$\\
    P2 && $350$ & $0.023$ & $0.005$ & $0.000$ && $173$ & $0.023$ & $0.005$ & $0.000$ \\
    P3 && $500$ & $0.152$ & $0.115$ & $-0.207$ && $173$ & $0.152$ & $0.115$ & $-0.207$ \\
    %
    \bottomrule
  \end{tabular}
  \caption[EFT description for the top-partner benchmarks]{Matching
    scales (in GeV) and selected Wilson coefficient
    for the top-partner benchmarks, both for default and $v$-improved
    matching.}
  \label{tbl:validity_stops_eft}
\end{table}

% For fixed masses and mixing, the Higgs couplings to the top partners
% depend on the interplay between $M^2$ and the coupling constants
% $\kappa$. For small mixing and large $M^2$, light top partner masses
% require large four-scalar couplings $\kappa_{ii}$.  Conversely, if
% $M^2$ is close to the physical masses, the Yukawa couplings can be
% small.  This illustrates the balance between the VEV-dependent
% (non-decoupling) and the explicit (decoupling) mass contributions. 



%%%%%%%%%%%%%%%%%%%%%%%%%%%%%%%%%%%%%%%%%%%%%%%%%%%%%%%%%%%%
\subsubsection{Higgs production rates and distributions}
%%%%%%%%%%%%%%%%%%%%%%%%%%%%%%%%%%%%%%%%%%%%%%%%%%%%%%%%%%%%

The contributions from scalar top partners to Higgs production in
gluon fusion are well known~\cite{Berger:2012ec, Dawson:2012di,
  Fajfer:2013wca, Ellis:2014dza, Chen:2014xwa} and the validity of the
EFT approach for this process has been thoroughly
scrutinised~\cite{Dawson:2015gka, Drozd:2015kva}. We therefore focus
on corrections to the $hVV$ coupling in WBF and Higgs-strahlung.

The total Higgs production rates in these channels are given in
\autoref{tbl:validity_partners_rates}.  In benchmark P1 the WBF cross
section is reduced by about $0.6 \%$ compared to the Standard Model,
with good agreement between effective and full description. Clearly,
such a scenario is not relevant for LHC measurements in the
foreseeable future. In more extreme corners of the parameter space the
loop effects in the full model grow, higher-dimensional terms in the
EFT become larger, the validity of the latter worsens, and
discrepancies between both increase.  In benchmarks P2 and P3 the WBF
rate is reduced by $9.1\%$ and $43.5\%$ with respect to the Standard
Model. By construction, the EFT based on the default matching captures
only the formally leading term at $\ord{v^2/\Lambda^2}$, only giving a
reduction of $0.5\%$ and $2.0\%$. The corresponding difference is
again independent for example of the tagging jet's transverse
momentum.  With the $v$-improved matching, the cross section is
reduced by $2.4 \%$ and $17.7 \%$, still far from the result of the
full model. 

\begin{table}
    \begin{tabular}{c c rr c rr}
      \toprule
      %
      \multirow{2}{*}{Benchmark}
      && \multicolumn{2}{c}{$\sigma_\text{EFT} / \sigma_\text{triplet}$}
      && \multicolumn{2}{c}{$\sigma_\text{$v$-improved EFT} / \sigma_\text{triplet}$} \\
      %
      \cmidrule{3-4}\cmidrule{6-7}
      %
      && WBF & $Vh$
      && WBF & $Vh$ \\
      %
      \midrule
      %
      P1 && $1.000$ & $0.999$ && $1.000$ & $0.999$ \\
      P2 && $1.095$ & $1.100$ && $1.074$ & $1.049$ \\
      P3 && $2.081$ & $1.904$ && $1.749$ & $1.363$ \\
      %
      \bottomrule
    \end{tabular}
    \caption[Total Higgs production cross sections in the top-partner model]{Cross
      section ratios of the matched dimension-six EFT
      approximation to the full scalar top-partner model at the LHC.  We
      give the results both for the default matching scheme with matching
      scale $\Lambda = M$ as well as for the $v$-improved matching at
      $\Lambda = m_{\tilde{t}_{1}}$. The statistical uncertainties on these
      ratios are below 0.4\%.}
  \label{tbl:validity_partners_rates}
\end{table}

The results for Higgs-strahlung look similar: in the moderate
benchmark P1 the predictions of the full model and the dimension-six
Lagrangian agree within $0.1 \%$, but in this scenario the overall
deviation from the Standard Model is negligible. In scenarios with
larger loop effects, the dimension-six predictions fails to capture
most of the full top partner loops, with numerical results similar to
those given for WBF Higgs production. Again switching to the
$v$-improved matching does not improve the EFT approximation
significantly.

In \autoref{fig:validity_partners_distributions} we finally show that
these changes in the total rates do not have a dramatic effect on the
kinematic distributions. This is not surprising, since the largest
Wilson coefficient generated in our benchmark points is consistently
that of $\ope{\phi,2}$, see \autoref{tbl:validity_stops_eft}, which
corresponds to a universal rescaling of the SM Higgs couplings.

\begin{figure}
  \includegraphicsdummy[width=0.49\textwidth]{fig/validity/TopPartners_WBF.pdf}%
  \includegraphicsdummy[width=0.49\textwidth]{fig/validity/TopPartners_VH.pdf}%
  \caption[Kinematic distributions in the top-partner
  model]{Kinematic distributions for the top partner model in
    benchmark P2.  Left: tagging jet properties in WBF Higgs
    production.  Right: $m_{Vh}$ distribution in Higgs-strahlung.}
  \label{fig:validity_partners_distributions}
\end{figure}



%%%%%%%%%%%%%%%%%%%%%%%%%%%%%%%%%%%%%%%%%%%%%%%%%%%%%%%%%%%%
\subsubsection{Summary}
%%%%%%%%%%%%%%%%%%%%%%%%%%%%%%%%%%%%%%%%%%%%%%%%%%%%%%%%%%%%

Scalar top partners generate a large set of dimension-six operators
through electroweak loops. However, in realistic scenarios with a
large scale separation the loop corrections for example to the $hVV$
vertex are tiny.  Pushing for loop effects that are large enough to
leave a visible imprint in WBF and Higgs-strahlung requires breaking
the scale separation between the observed Higgs scalar and the top
partners dramatically. In that case the EFT fails already for the
total rates, kinematic distributions hardly add to this discrepancy.



%%%%%%%%%%%%%%%%%%%%%%%%%%%%%%%%%%%%%%%%%%%%%%%%%%%%%%%%%%%%
\subsection{Vector triplet}
\label{sec:validity_triplet}
%%%%%%%%%%%%%%%%%%%%%%%%%%%%%%%%%%%%%%%%%%%%%%%%%%%%%%%%%%%%

%%%%%%%%%%%%%%%%%%%%%%%%%%%%%%%%%%%%%%%%%%%%%%%%%%%%%%%%%%%%
\subsubsection{Model setup}
%%%%%%%%%%%%%%%%%%%%%%%%%%%%%%%%%%%%%%%%%%%%%%%%%%%%%%%%%%%%

Heavy vector bosons appear in many new physics scenarios, for instance
when a larger gauge group is spontaneously broken down to the SM gauge
group at higher energies. Often such particles are connected to the
gauge-Higgs sector of the SM, predicting signatures in Higgs
measurements~\cite{Low:2009di, Biekoetter:2014jwa,
  Pappadopulo:2014qza}. As an example, we study a massive vector
field\footnote{Note that such a model is not UV-complete: as
  fundamental particles, massive vector bosons are not
  renormalisable. However, such a mass can easily be generated from a
  consistent gauge theory with a Higgs mechanism at a higher
  scale~\cite{Pappadopulo:2014qza}. The details of such an embedding
  do not matter for the discussion here.}  $V^a_\mu$ which is a
triplet under $SU(2)_L$ and uncharged under $SU(3)_c$ and
$U(1)_Y$. These charges allow it to mix with the $W$ bosons of the
Standard Model and to couple to Higgs and fermion
currents~\cite{Pappadopulo:2014qza, Biekoetter:2014jwa}. For
simplicity, we assume $CP$ invariance and a flavour-universal coupling
to the fermion current. Following the conventions of
Reference~\cite{Pappadopulo:2014qza}, the Lagrangian then reads
%
\begin{align}
  \lgr{vector triplet}
  &\supset
  - \frac 1 4 \, V_{\mu\nu}^a V^{\mu\nu\, a}
  +\frac {M_V^2} 2 \, V_\mu^a V^{\mu\,a}  \notag \\
  %
  &\phantom{=}
  + \im \frac {g_V} 2 \, c_H \, V_{\mu}^a
  \left[\phi^\dagger \sigma^a \overleftrightarrow{D}^\mu \phi \right]
  +\frac {g_W^2} {2{g_V}} \, V_{\mu}^a \,
  \sum_f c_F \, \bar{f}_L \gamma^\mu \sigma^a f_L \notag \\
 %
  &\phantom{=}
  + g_V^2 \, c_{VVHH} \, V_{\mu}^a V^{\mu a} \phisq \notag \\
  %
  &\phantom{=}
  + \frac {g_V} 2 \, c_{VVV} \,\varepsilon_{abc} \, V_\mu^a \, V_\nu^b \, D^{[\mu}V^{\nu]c}
  - \frac{g_W} 2 \, c_{VVW} \, \varepsilon_{abc} \, W^{\mu\nu} V_\mu^b V_\nu^c
 \label{eq:validity_vector_triplet_lagrangian}
\end{align}
%
with the field strength
$V_{\mu\nu}^a = D_\mu V_{\nu}^a - D_\nu V_{\mu}^a$ and where
$D_\mu V_\nu^a = \partial_\mu V_\nu^a+ g_V \varepsilon^{abc} V^b_\mu
V_\nu^c$.

The coupling $g_V$ characterises the interactions of the heavy
vector, while $g_W$ is the $SU(2)_L$ coupling constant. After mixing,
the original fields $V^a$ and $W^a$ combine to the mass eigenstates
$W^\pm$, $\xi^\pm$, and $\xi^0$, while a combination of the couplings
$g_w$ and $g_V$ becomes the observed weak gauge coupling
$g$. $c_{VVW}$ and $c_{VVV}$ are irrelevant for Higgs phenomenology at
the LHC. For more details, see Appendix~\ref{sec:appendix_models_triplet}.



%%%%%%%%%%%%%%%%%%%%%%%%%%%%%%%%%%%%%%%%%%%%%%%%%%%%%%%%%%%%
\subsubsection{Signatures and decoupling patterns}
%%%%%%%%%%%%%%%%%%%%%%%%%%%%%%%%%%%%%%%%%%%%%%%%%%%%%%%%%%%%

In addition to the new heavy resonances $\xi^0$ and $\xi^\pm$, the
signature feature of the vector triplet is that the mixing of the new
states with the $W$ and $Z$ bosons affects the properties of the
electroweak gauge bosons at tree level. In particular, the shift from
the Lagrangian parameter $g_w$ to the observable weak coupling $g$
combined with the direct heavy vector coupling to the Higgs doublet
modify the Higgs couplings as
%
\begin{align}
  \Delta_V
  &\approx
    \frac{g^2 c_F c_H}{4} \left(\frac{v}{M_V} \right)^2
    - \frac{3 g_V^2 c_H^2}{8} \left(\frac{v}{M_V} \right)^2 \notag \\
  %
  \Delta_f
  &\approx \frac{g^2 c_F c_H}{4} \left( \frac{v}{M_V} \right)^2
    - \frac{g_V^2 c_H^2}{8} \left( \frac{v}{M_V} \right)^2 \,.
    \label{eq:validity_vector_triplet_coupling_shifts}
\end{align}
%
In addition, contributions from virtual heavy states $\xi$ modify the
phase-space behaviour of Higgs signals in many ways.

Just as for the 2HDM and the top partners, the mass matrix for the
massive vectors contains both the intrinsic mass scale $M_V$ and terms
proportional to some power of $v$ multiplied by a combination of
couplings. The new vector states have roughly degenerate masses
%
\begin{equation}
  m_\xi^2
  \approx
  M_V^2
  \left( 1 + g_V^2 c_{VVHH} \frac{v^2}{M_V^2}
    + \frac{g_V^2 c_H^2}{4} \frac{v^2}{M_V^2}
    + \ord{\frac {v^4} {M_V^4} } \right) \,.
  \label{eq:validity_vector_triplet_masses}
\end{equation}
%
Even if there appears to be a clear scale separation $M_V \gg v$,
large values of $g_V$, $c_{VVHH}$, or $c_H$ can change $m_\xi$
significantly and effectively induce a second mass scale.

% Just as for
% the top partners, a problem for the dimension-six approach arises from
% virtual $\xi$ diagrams contributing for example to WBF Higgs
% production. If $m_\xi < M_V \equiv \Lambda$ the lightest new particles
% appearing in Higgs production processes have masses below the matching
% scale of the linear representation.  The way out of a poor agreement
% between the full model and its dimension-six description is again
% switching to a $v$-improved matching in the broken phase with matching
% scale $\Lambda = m_\xi$.


%%%%%%%%%%%%%%%%%%%%%%%%%%%%%%%%%%%%%%%%%%%%%%%%%%%%%%%%%%%%
\subsubsection{Dimension-six description}
%%%%%%%%%%%%%%%%%%%%%%%%%%%%%%%%%%%%%%%%%%%%%%%%%%%%%%%%%%%%

The obvious choice of the matching scale in the unbroken electroweak
phase is the heavy mass scale in the Lagrangian
%
\begin{equation}
  \Lambda_{\text{default}} = M_V \,,
\end{equation}
%
which we use for our default matching scheme.

Integrating out the heavy vector triplet at tree level generates a
number of operators with Wilson coefficients
%
\begin{align}
  f_{WW} &= c_F \, c_H \,, &
  f_{\phi 2} &= \frac 3 4 \, c_H \left( c_H \, g_V^2 - 2 \,c_F \, g^2 \right) \,, \notag \\
  f_{BB} &= c_F \, c_H  \,,  &
  f_{\phi 3} &= -3 \, \lambda \, c_H \left( c_H \, g_V^2 - 2 \, c_F \,  g^2 \right) \,, \notag  \\
  f_{BW} &= c_F \, c_H   \,, &
  f_{f} &= - \frac 1 4 \, y_f \, c_H \left( c_H \, g_V^2 - 2 \, c_F \, g^2 \right)  \,,\notag  \\
  f_W &= - 2 \, c_F \,c_H \,.
  \label{eq:validity_vector_triplet_coefficients}
\end{align}
%
Additional four-fermion contributions are irrelevant for Higgs
physics. Loop-induced contributions will be further suppressed and do
not add qualitatively new features, so we neglect them here.

Once again, we compare this default matching to an alternative
$v$-improved matching, where as a cutoff we now use the physical mass 
%
\begin{equation}
  \Lambda_{\text{$v$-improved}} = m_{\xi^0} \,.
\end{equation}
%
The coefficients in \autoref{eq:validity_vector_triplet_coefficients}
remain unchanged.\footnote{In the spirit of $v$-improvement we could
  alternatively parametrise the Wilson coefficients with the physical
  mixing angles between the $W$, $Z$ and $V$ bosons, but this does not
  significantly change the results.}

Unlike the previous models, the vector triplet generates $\ope{W}$,
$\ope{WW}$, and $\ope{BB}$ at tree level. As discussed in
Section~\ref{sec:foundations_heft_pheno}, these induce new kinematic
structures in the $hWW$ and $hZZ$ couplings.



%%%%%%%%%%%%%%%%%%%%%%%%%%%%%%%%%%%%%%%%%%%%%%%%%%%%%%%%%%%%
\subsubsection{Benchmark points}
%%%%%%%%%%%%%%%%%%%%%%%%%%%%%%%%%%%%%%%%%%%%%%%%%%%%%%%%%%%%

\begin{table}
    \begin{tabular}{c c rrrrrr}
      \toprule
      %
      \multirow{2}{*}{}
      && \multicolumn{6}{c}{Triplet model} \\
      %
      \cmidrule{3-8}
      %
      && $M_V$ & $g_V$ & $c_H$ & ${c}_{F}$ & ${c}_{VVHH}$ & $m_\xi$ \\
      %
      \midrule
      %
      T1 && $591$ & $3.0$ & $-0.47$ & $-5.0$ & $2.0$ & $1200$ \\
      T2 && $946$ & $3.0$ & $-0.47$ & $-5.0$ & $1.0$ & $1200$ \\
      T3 && $941$ & $3.0$ & $-0.28$ & $3.0$ & $1.0$ & $1200$ \\
      T4 && $1246$ & $3.0$ & $-0.50$ & $3.0$ & $-0.2$ & $1200$ \\
      T5 && $846$ & $1.0$ & $-0.56$ & $-1.32$ & $0.08$ & $849$ \\
      %
      \bottomrule
    \end{tabular}
    \caption[Benchmark points for the vector triplet model]{Benchmark points
      for the vector triplet model. The masses are given in GeV.}
  \label{tbl:validity_vector_triplet_benchmarks}
\end{table}

As for the other models we study a set of benchmark points, defined in
\autoref{tbl:validity_vector_triplet_benchmarks} and
\autoref{tbl:validity_vector_triplet_eft}. Unlike additional scalars,
light new vector triplets with masses just above the electroweak scale
are unrealistic given the constraints from electroweak precision
measurements and direct searches. We therefore focus on new vector
bosons around the TeV scale. The different setups are motivated
phenomenologically, from experimental constraints, or based on
specific UV completions:
%
\begin{itemize}
\item[T1-2] \emph{Higgs-gauge dynamics}: designed for large
  momentum-dependent effects in the $hVV$ couplings. $\ope{W}$ and
  $\ope{WW}$ receive large Wilson coefficients, while $\ope{\phi,2}$,
  $\ope{\phi,3}$, and $\ope{f}$ vanish along the line
  % 
  \begin{equation}
    \frac {c_H} {c_F} = 2 \frac {g^2} {g_V^2} \,.
  \end{equation}
  %
  The large couplings also imply a large difference between $M_V$ and
  $m_\xi$.
  % 
\item[T3] \emph{Interference patterns}: changes the sign of $c_W$ or
  the Wilson coefficients compared to T1 and T2. This allows us to
  compare constructive and destructive interference patterns between
  SM amplitudes and new physics contributions.
%
\item[T4] \emph{Realistic:} the vector triplet couplings and masses
  satisfy the leading constraints from direct collider searches at the
  time of publication of Reference~\cite{Brehmer:2015rna}. The most
  stringent bounds come from di-lepton and di-boson
  channels~\cite{Pappadopulo:2014qza, Kaminska:2015ora}.
%
\item[T5] \emph{UV completion:} typical coupling patterns from a
  weakly coupled UV completion based on the extended gauge group
  $SU(3) \times SU(2) \times SU(2) \times U(1)$~\cite{Barger:1980ti}.
  Such a scenario could for instance arise from deconstructed extra
  dimensions~\cite{ArkaniHamed:2001nc}. The vector triplet
  phenomenology is effectively described by the parameter
  $\alpha = g_V / \sqrt{g_V^2 - g_w^2}$ together with the symmetry
  breaking scale $f$~\cite{Pappadopulo:2014qza}, with the couplings
%
  \begin{align}
    M_V^2 &= \alpha^2 g_V^2 f^2 \,, &
    c_H &= - \alpha \frac {g_W^2}{g_V^2} \,, &
    c_{VVHH} &= \alpha^2 \left[ \dfrac {g_W^4} {4g_V^4} \right] \,, \notag \\
    c_F &= - \alpha \,, &
    c_{VVW} &= 1 \,, &
    c_{VVV} &= - \frac{\alpha^3}{g_V}
               \left[1 - \frac{3g_W^2}{g_V^2} + \dfrac {2g_W^2} {g_V^4} \right] \,.
 \end{align}
%
\end{itemize}

\begin{table}
  \begin{tabular}{c c rrrrr c rrrrr}
    \toprule
    %
    \multirow{2}{*}{} && \multicolumn{5}{c}{Default EFT}
    && \multicolumn{5}{c}{$v$-improved EFT} \\
    %
    \cmidrule{3-7} \cmidrule{9-13}
    %
    && $\Lambda$ & $f_{\phi 2}$ & $f_{WW}$ & $f_W$ & $f_{t}$
    && $\Lambda$ & $f_{\phi 2}$ & $f_{WW}$ & $f_W$ & $f_{t}$\\
    %
    \midrule
    %
    T1 && $591$ & $0.00$ & $2.45$ & $-4.90$ & $0.00$ 
        && $1200$ & $0.00$ & $2.45$ & $-4.90$ & $0.00$ \\
    T2 && $946$ & $0.00$ & $2.35$ & $-4.71$ & $0.00$
         && $1200$ & $0.00$ & $2.35$ & $-4.71$ & $0.00$ \\
    T3 && $941$ & $1.09$ & $-0.82$ & $1.64$ & $-0.36$ 
         && $1200$ & $1.09$ & $-0.82$ & $1.64$ & $-0.36$  \\
    T4 && $1246$ & $2.64$ & $-1.56$ & $3.12$ & $-0.87$
         && $1200$ & $2.64$ & $-1.56$ & $3.12$ & $-0.87$ \\
    T5 && $846$ & $-0.24$ & $0.78$ & $-1.55$ & $0.08$ 
         && $849$ & $-0.24$ & $0.78$ & $-1.55$ & $0.08$ \\
    %
    \bottomrule
    \end{tabular}
    \caption[EFT description for the vector triplet benchmarks]{Matching
      scales (in GeV) and selected Wilson coefficients for the effective
      theory matched to the vector triplet model. We give these results both
      for the EFT matching in the unbroken phase as well as for the
      $v$-improved matching with $\Lambda = m_{\xi^0}$.}
  \label{tbl:validity_vector_triplet_eft}
\end{table}

Benchmarks T1 to T3 are meant to emphasise the phenomenological
possibilities of the vector triplet model, ignoring experimental
constraints or parameter correlations from an underlying UV
completion.



%%%%%%%%%%%%%%%%%%%%%%%%%%%%%%%%%%%%%%%%%%%%%%%%%%%%%%%%%%%%
\subsubsection{Higgs production rates and distributions}
%%%%%%%%%%%%%%%%%%%%%%%%%%%%%%%%%%%%%%%%%%%%%%%%%%%%%%%%%%%%


\begin{figure}
  \centering
  \fmfframe(0,15)(15,15){ %(L,T) (R,B)
    \begin{fmfgraph*}(100,60)
      \feynmansetup
      \fmfleft{i2,i1}
      \fmfright{o4,o2,o1}
      \fmflabel{\small $q$}{i1}
      \fmflabel{\small $q$}{i2}
      \fmflabel{\small $q$}{o1}
      \fmflabel{\small $h$}{o2}
      \fmflabel{\small $q$}{o4}
      \fmf{fermion,tension=4}{i1,v3}
      \fmf{fermion,tension=4}{i2,v4}
      \fmf{fermion,tension=2.5}{v3,o1}
      \fmf{fermion,tension=2.5}{v4,o4}
      \fmf{wiggly,label=\small $W$,, $Z$,label.side=right}{v3,v5}
      \fmf{dbl_wiggly,label=\small $\xi^\pm$,, $\xi^0$,label.side=left}{v4,v5}
      \fmf{dashes,tension=0.5}{v5,o2}
      %\fmfv{decoration.shape=circle,foreground=(0.776,, 0.094,, 0.149),decoration.size=5}{v5}
    \end{fmfgraph*}
  }
  \hspace{1cm}
  \fmfframe(0,15)(15,15){ %(L,T) (R,B)
    \begin{fmfgraph*}(100,60)
      \feynmansetup
      \fmfleft{i2,i1}
      \fmfright{o4,o2,o1}
      \fmflabel{\small $q$}{i1}
      \fmflabel{\small $q$}{i2}
      \fmflabel{\small $q$}{o1}
      \fmflabel{\small $h$}{o2}
      \fmflabel{\small $q$}{o4}
      \fmf{fermion,tension=4}{i1,v3}
      \fmf{fermion,tension=4}{i2,v4}
      \fmf{fermion,tension=2.5}{v3,o1}
      \fmf{fermion,tension=2.5}{v4,o4}
      \fmf{dbl_wiggly,label=\small $\xi^\pm$,, $\xi^0$,label.side=right}{v3,v5}
      \fmf{dbl_wiggly,label=\small $\xi^\pm$,, $\xi^0$,label.side=left}{v4,v5}
      \fmf{dashes,tension=0.5}{v5,o2}
      %\fmfv{decoration.shape=circle,foreground=(0.776,, 0.094,, 0.149),decoration.size=5}{v5}
    \end{fmfgraph*}
  }
  \hspace{1cm}
  \fmfframe(0,15)(15,15){ %(L,T) (R,B)
    \begin{fmfgraph*}(80,60)
      \feynmansetup
      \fmfleft{i2,i1}
      \fmfright{o2,o1}
      \fmflabel{\small $q$}{i1}
      \fmflabel{\small $q$}{i2}
      \fmflabel{\small $h$}{o1}
      \fmflabel{\small $W$, $Z$}{o2}
      \fmf{fermion}{i1,v1,i2}
      \fmf{dbl_wiggly,label=\small $\xi^\pm$,, $\xi^0$}{v1,v2}
      \fmf{dashes}{v2,o1}
      \fmf{photon}{v2,o2}
    \end{fmfgraph*}
  }
  \caption[Feynman diagrams for vector triplet effects in Higgs production]{Example
    Feynman diagrams with contributions from virtual heavy vector bosons $\xi$ to
    Higgs production in weak boson fusion (left, middle) or Higgs-strahlung (right).}
  \label{fig:validity_vector_triplet_xi_diagrams}
\end{figure}

As shown in \autoref{fig:validity_vector_triplet_xi_diagrams}, virtual
heavy vector bosons contribute to as intermediate $t$-channel
mediators to WBF Higgs production and in the $s$-channel to
Higgs-strahlung, promising non-trivial kinematic features in these
channels. In the EFT these effects are mapped onto large Wilson
coefficients like $\ope{W}$ and $\ope{WW}$, adding a momentum
dependence to the $hVV$ couplings. Therefore our analysis focuses on
these electroweak Higgs production modes.

\begin{table}
  \begin{tabular}{c c rr c rr}
    \toprule
    %
    && \multicolumn{2}{c}{$\sigma_\text{default EFT} / \sigma_\text{triplet}$}
    && \multicolumn{2}{c}{$\sigma_\text{$v$-improved EFT} / \sigma_\text{triplet}$} \\
    %
    \cmidrule{3-4} \cmidrule{6-7}
    %
    && WBF & $Vh$ && WBF & $Vh$ \\
    %
    \midrule
    %
    T1 && $1.299$ & $0.299$ && $0.977$ & $0.794$ \\
    T2 && $1.045$ & $0.737$ && $0.992$ & $0.907$ \\
    T3 && $0.921$ & $1.066$ && $0.966$ & $1.024$ \\
    T4 && $1.026$ & $0.970$ && $1.012$ & $0.978$ \\
    T5 && $1.001$ & $1.043$ && $1.002$ & $1.043$ \\
    %
    \bottomrule
    \end{tabular}
    \caption[Total Higgs production rates in the vector triplet model]{Cross
      section ratios of the matched dimension-six EFT
      approximation to the full vector triplet at the LHC.  To avoid large
      contributions from the $\xi$ resonance in the $Vh$ channel, we only
      take into account the region $m_{Vh} < 600$~GeV.  The statistical
      uncertainties on these ratios are below 0.4\%.}
  \label{tbl:validity_vector_triplet_rates}
\end{table}

\autoref{tbl:validity_vector_triplet_rates} shows the agreement
between EFT and full model for the total Higgs production rates in WBF
Higgs production and Higgs-strahlung. The default dimension-six model
matched in the unbroken phase, oblivious to the difference between the
Lagrangian mass term $M_V$ and the actual physical mass $m_\xi$,
struggles with the first three benchmark points, in which this
splitting is large. The discrepancies to the full model are
particularly evident in $Vh$ production.

The $v$-improved EFT, on the other hand, performs better and describes
the rate accurately in most of the scenarios. Only in Higgs-strahlung
and only in the extreme scenarios T1 and T2 we find significant
deviations. 

\begin{figure}
  \includegraphicsdummy[width=0.49\textwidth]{fig/validity/Triplet_WBF.pdf}%
  \includegraphicsdummy[width=0.49\textwidth]{fig/validity/Triplet_WBF_log.pdf}\\%
  \includegraphicsdummy[width=0.49\textwidth]{fig/validity/Triplet_WBF_deltaphi.pdf}%
  \includegraphicsdummy[width=0.49\textwidth]{fig/validity/Triplet_WBF_realistic.pdf}%
  \caption{Tagging jet distributions in WBF Higgs production in the
    vector triplet model.  Top: $p_{T,j1}$ distribution in benchmark
    T1, focusing on the low (left) and high (right) transverse
    momentum regions.  Bottom left: $\Delta \phi_{jj}$ distribution
    above a certain $p_{T,j1}$ threshold for T1.  Bottom right:
    $p_{T,j1}$ distribution for scenario T5.}
  \label{fig:validity_vector_triplet_wbf}
\end{figure}

To better understand these differences, we have to look at kinematic
distributions. \autoref{fig:validity_vector_triplet_wbf} shows
different properties of the tagging jets in WBF Higgs production. In
addition to the predictions of the full vector triplet model and the
default and $v$-improved EFT, we show distributions of the vector
triplet model where we have artificially removed all contributions
from virtual $\xi$ propagators.

We find that the vector triplet significantly modifies the WBF rate
with respect to the Standard Model. Its effect increases with momentum
transfer, measured as transverse jet momentum. This modification can
be traced to contributions from $\xi$ fusion and mixed $W$-$\xi$
fusion diagrams as given in
\autoref{fig:validity_vector_triplet_xi_diagrams}. These contributions
from $\xi$ propagators can become relevant already at energy scales
well below $m_\xi$ and further increase with the energy flow. In
addition to the high-energy tails of the transverse momenta, large
effects are visible in the azimuthal angle between the tagging jets,
as shown in the bottom left panel of
\autoref{fig:validity_vector_triplet_wbf}. This angular correlation is
well known to be sensitive to the modified Lorentz structure of the
$hWW$ vertex~\cite{Eboli:2000ze, Plehn:2001nj, Hankele:2006ma,
  Hagiwara:2009wt, Englert:2012xt, Buckley:2014fqa, Brehmer:2014pka}.

The EFT approach qualitatively captures these features of the full
model, now parametrised by momentum-dependent operators such as
$\ope{W}$ and $\ope{WW}$. The signs of the Wilson coefficients in
benchmarks T1 and T2 yields a non-linear increase of the cross section
with energy. Conversely, the switched signs in T3 reduces the rate
with energy, eventually driving the combined amplitude through zero.

Comparing full and effective model for the more realistic benchmark
points T4 and T5, we find good agreement in the bulk of the
distribution. The deviations from the Standard Model are entirely
captured by the dimension-six operators, including the momentum
dependence coming from the $\xi$ diagrams. Only at very large momentum
transfer, likely beyond the sensitivity of the LHC, the validity of
the EFT breaks down.

In the more strongly coupled benchmark points T1 to T3, the full model
predicts shifts in the jet distributions that are large enough to be
relevant for the upcoming LHC run. We find good agreement between the
full model and the default EFT only at low momentum transfer, where
the effects of new physics are small. This naive dimension-six model
fails to reproduce the full model results already at energy scales
$p_{T,j} \gtrsim 80$~GeV, a phase space region highly relevant for
constraints on new physics~\cite{Corbett:2015ksa}.  Perhaps
counter-intuitively, this discrepancy does not signal a breakdown of
the $E / \Lambda$ expansion, but is linked to the difference between
the physical mass $m_\xi$, which suppresses the $\xi$ fusion diagrams,
and the matching scale $\Lambda_{\text{default}} = M_V$, which
suppresses the dimension-six operators. This can be seen by comparing
the results to those based on $v$-improved matching, where the EFT
cutoff scale matches the physical mass. Here the agreement is
significantly better, and the dimension-six description succesfully
describes the momentum dependence up to large momentum transfer. Only
at very high energies, $p_{T,j1} \gtrsim 300$~GeV, even the
$v$-improved EFT breaks down.

\begin{figure}
  \includegraphicsdummy[width=0.49\textwidth]{fig/validity/Triplet_VH.pdf}%
  \includegraphicsdummy[width=0.49\textwidth]{fig/validity/Triplet_VH_log.pdf}\\%
  \includegraphicsdummy[width=0.49\textwidth]{fig/validity/Triplet_VH_pT.pdf}%
  \includegraphicsdummy[width=0.49\textwidth]{fig/validity/Triplet_VH_realistic.pdf}%
  \caption{Higgs-strahlung distributions in the vector triplet model.
    Top: $m_{Vh}$ distribution for benchmark T2, focusing on the low
    (left) and high (right) invariant mass regions. Bottom left:
    $p_{T,V}$ distribution for the same benchmark. Bottom right:
    $m_{Vh}$ distribution for T4.}
  \label{fig:validity_vector_triplet_vh}
\end{figure}

The situation is similar in Higgs-strahlung, shown in
\autoref{fig:validity_vector_triplet_vh}. Again, the dominant new
physics effect is the interference with $\xi$-mediated diagrams rather
than the modified $hWW$ interaction. Not only does this lead to a
significant change of the rate, it also introduces a strong dependence
on the momentum transfer, probed by either the invariant mass of the
gauge-Higgs system or the transverse momentum of the final vector
boson. The relative sign of the interference between $\xi$ amplitudes
and SM-like diagrams is opposite to that in WBF: in T3 and T4 we find
a non-linear increase of the cross section with the energy scale. The
other benchmarks predict a decrease of the amplitude with energy,
eventually including a sign flip when the amplitude is driven through
zero.

In the more weakly coupled benchmarks T4 and T5, the full and
effective models agree well over most of the phase space, and the
dimension-six operators succesfully capture how the $\xi$
contributions affect the Higgs-strahlung kinematics. At larger
momentum transfer, higher-order terms in the EFT expansion become
important, and dimension-six operators alone cannot describe the
kinematics accurately any more. Ultimately, the $\xi$ resonance in the
full model marks the obvious failure of the effective theory.

For benchmarks T1 to T3, the default EFT has a more limited validity
range. The large couplings lead to a failure of this dimension-six
model already at low energies $m_{Vh} \gtrsim 220~\gev $, even though
the actual $\xi$ resonances only appear at $m_\xi = 1.2$~TeV.  The EFT
approximation can again be significantly improved by switching to the
$v$-improved matching. But even then, there is a pronounced mismatch
between full and effective model. This EFT error is larger in
Higgs-strahlung than in WBF, showing how $\xi$ contributions play a
larger role in this $s$-channel process than in the $t$-channel WBF
diagrams.



%%%%%%%%%%%%%%%%%%%%%%%%%%%%%%%%%%%%%%%%%%%%%%%%%%%%%%%%%%%%
\subsubsection{Summary}
%%%%%%%%%%%%%%%%%%%%%%%%%%%%%%%%%%%%%%%%%%%%%%%%%%%%%%%%%%%%

Heavy vector bosons can induce large kinematic effects in Higgs-gauge
interactions, providing a perfect test case for the EFT approach. For
realistic scenarios, the EFT works up to large momentum
transfer. Operators such as $\ope{W}$ and $\ope{WW}$ successfully
capture the effects from virtual $\xi$ contributions to WBF Higgs
production, including non-trivial momentum dependencies. In
Higgs-strahlung, $s$-channel $\xi$ contributions are more
difficult to map onto effective operators, but the agreement is still
good realistic parameter choices.

Again, this good performance of the dimension-six model requires
particular care in the matching to the full theory. When the mixing
with the SM gauge bosons is large, a naive matching procedure, defined
in the unbroken electroweak phase, can lead to substantial errors
already in the bulk of the WBF distributions. A $v$-improved
dimension-six description, however, improves the EFT accuracy such
that large deviations only occur in the high-energy tails of
distributions.
%
\comment{To here}




%%%%%%%%%%%%%%%%%%%%%%%%%%%%%%%%%%%%%%%%%%%%%%%%%%%%%%%%%%%%
\section{Practical questions}
\label{sec:validity_practical_questions}
%%%%%%%%%%%%%%%%%%%%%%%%%%%%%%%%%%%%%%%%%%%%%%%%%%%%%%%%%%%%




%%%%%%%%%%%%%%%%%%%%%%%%%%%%%%%%%%%%%%%%%%%%%%%%%%%%%%%%%%%%
\subsection{To square or not to square}
%%%%%%%%%%%%%%%%%%%%%%%%%%%%%%%%%%%%%%%%%%%%%%%%%%%%%%%%%%%%

If we accept that a dimension-six Lagrangian describing Higgs signatures
at the LHC is not necessarily part of a consistent effective field
theory, but rather a successful and reproducible parametrisation of
weakly interacting new physics, there exists no fundamental
motivation~\cite{legacy,too_long,mvh,gino,spanno} to include or to not
include the dimension-six squared term in
%
\begin{align}
|\mat_{4+6}|^2 = |\mat_4|^2 + 2 \, \text{Re} \mat_4^* \mat_6 \stackrel{?}{+} |\mat_6|^2 \; .
\end{align}
%
A dimension-six squared term of comparable or larger size than
the interference term can appear in phase-space regions with
a suppressed dimension-4 prediction, even when the EFT expansion in
$E/\Lambda$ holds and dimension-eight effects are negligible.
In the absence of any first-principle reason how to treat this term,
we need to test the different possibilities from a practical
perspective.


%%%%%%%%%%%%%%%%%%%%%%%%%%%%%%%%%%%%%%%%%%%%%%%%%%%%%%%%%%%%
\subsubsection*{Higgs-strahlung}
%%%%%%%%%%%%%%%%%%%%%%%%%%%%%%%%%%%%%%%%%%%%%%%%%%%%%%%%%%%%

We first analyse associated $Vh$ production with $V =
W^\pm, Z$ at 13~TeV LHC energy. To retain as much phase space as
possible we only consider the parton-level process
%
\begin{align}
  p p \to V h
\end{align}
%
simulated in \toolfont{MadGraph}~\cite{madgraph} without cuts or
decays. It is easy to see where in phase space the effective theory
breaks down: for on-shell outgoing Higgs and gauge bosons a large
momentum flow through the Higgs operator can only be generated through
the virtual $s$-channel propagator. We can directly test this in the
observable $m_{Vh}$ distribution, comparing the full model
with the dimension-six approach at large momentum flow.

We show the $m_{Vh}$ distributions in the left panels of
\autoref{fig:validity_squared_VH}.  While theoretically the $m_{Vh}$
distribution is cleaner, for example when we include initial state
radiation, we can see the same effects in the highly correlated
$p_{T,V}$ distribution (right panels), due to the simple $2 \to 2$
signal kinematics~\cite{mvh,gino}.  The T1 benchmark point is constructed with a low
new physics scale and a destructive interference between Standard
Model and dimension-six term. We see that the squared dimension-six terms
are clearly needed to avoid negative cross sections in the high-energy
tails of the distributions. Driven by the light new particles,
inconsistencies otherwise occur around
%
\begin{align} 
m_{Vh} > 600~\gev \approx \frac{m_\xi^\text{(T1)} }2
\qquad \text{or} \qquad 
p_{T,V} > 300~\gev \approx \frac{m_\xi^\text{(T1)} }{4} \; ,
\label{eq:validity_breakdown_vh}
\end{align}
%
clearly within reach of Run~II. The reason why differences appear much
below $m_{Vh} = m_\xi$ is that the new states are wide and their pole
contribution extends through a large interference effect. Because for
this benchmark point the discrepancies signal the onset of a new
$s$-channel propagator pole, the agreement between full model and
dimension-six operators is limited and will hardly improve once we
include for example dimension-eight terms~\cite{kilian}.\footnote{We
  should note that if the LHC experiments should observe such a new
  resonance, the justification of a dimension-six description will most
  likely not be of experimental or theoretical concern.}

For the constructively interfering benchmark point T4 we observe no
dramatic effects in the tails, but the agreement between the full
model and the dimension-six approximation is improved when we include
these terms. Both benchmark points therefore suggest to include the
dimension-six squared terms in the LHC analysis, to improve the
agreement between the model and the dimension-six Lagrangian.

%------------------------------------------------------------
\begin{figure}[t]
  \includegraphics[width=0.43\textwidth]{fig/validity/VH_T1_mVH.pdf}
  \hspace*{0.05\textwidth}
  \includegraphics[width=0.43\textwidth]{fig/validity/VH_T1_Vpt.pdf} \\
  \includegraphics[width=0.43\textwidth]{fig/validity/VH_T1_mVH_zoom.pdf} 
  \hspace*{0.05\textwidth}
  \includegraphics[width=0.43\textwidth]{fig/validity/VH_T1_Vpt_zoom} \\
  \includegraphics[width=0.43\textwidth]{fig/validity/VH_T4_mVH}
  \hspace*{0.05\textwidth}
  \includegraphics[width=0.43\textwidth]{fig/validity/VH_T4_Vpt}
  \caption{$Vh$ distributions with (``D6$^{2}$'') and without (``D6'')
    the dimension-six squared
    term. The left panels show $m_{VH}$, the right panels
    $p_{T,V}$. The central panels show the region where leaving out
    the squared dimension-six terms leads to a negative cross section.}
  \label{fig:validity_squared_VH}
\end{figure}
%------------------------------------------------------------


%%%%%%%%%%%%%%%%%%%%%%%%%%%%%%%%%%%%%%%%%%%%%%%%%%%%%%%%%%%%
\subsubsection*{WBF Higgs production}
%%%%%%%%%%%%%%%%%%%%%%%%%%%%%%%%%%%%%%%%%%%%%%%%%%%%%%%%%%%%

Weak-boson-fusion Higgs production is a $2 \to 3$ process with two
$t$-channel gauge bosons carrying the momentum to the Higgs vertex.
The relevant kinematic variables are the two virtualities of the weak
bosons. Following many studies in the framework of the effective $W$
approximation~\cite{effective_w,polarized_ww} it is straightforward to
link them to the $p_T$ of the tagging jets, which even for multiple
jet radiation can be linked to the transverse momentum of the
Higgs~\cite{Buschmann:2014twa} (even though it is not clear if this
distribution is theoretically or experimentally favoured).  Again, we
start with the parton-level signal process
%
\begin{align}
u d \to u' d' h
\label{eq:validity_def_wbf}
\end{align}
%
with only one minimal cut $p_{T,j} > 20$~GeV for the two tagging jets.  We
show the results for the now constructively interfering benchmark
point T1 and the now destructively interfering benchmark point T4 in
\autoref{fig:validity_squared_WBF}. Negative event rates for T4 appear around
%
\begin{align}
p_{T,j_1} > 600~\gev \approx \frac{m_\xi^\text{(T4)}}{2} \; , 
\label{eq:validity_breakdown_wbf}
\end{align}
%
forcing us to either disregard the corresponding model hypothesis or
to add the dimension-six squared term.  For the less critical point T1
the agreement between the vector triplet model and its dimension-six
approximation including the squared terms extends well into the range
where deviations from the Standard Model become visible.

%------------------------------------------------------------
\begin{figure}
  \includegraphics[width=0.43\textwidth]{fig/validity/WBF_T1_j1pt.pdf} 
  \hspace*{0.05\textwidth}
  \includegraphics[width=0.43\textwidth]{fig/validity/WBF_T4_j1pt_zoom.pdf}\\
  \includegraphics[width=0.43\textwidth]{fig/validity/WBF_T1_Hpt.pdf} 
  \hspace*{0.05\textwidth}
  \includegraphics[width=0.43\textwidth]{fig/validity/WBF_T4_Hpt.pdf}\\
  \includegraphics[width=0.43\textwidth]{fig/validity/WBF_T1_q.pdf} 
  \hspace*{0.05\textwidth}
  \includegraphics[width=0.43\textwidth]{fig/validity/WBF_T4_q.pdf}
  \caption{WBF distributions with (``D6$^{2}$'') and without (``D6'') the
    dimension-six squared term. From top to bottom: $p_{T,j1}$, $p_{T,h}$, and
    virtuality $q$ defined in Equation\;\eqref{eq:validity_virt}. The right panels show the region where
    leaving out the squared dimension-six terms leads to a negative cross
    section.}
  \label{fig:validity_squared_WBF}
\end{figure}
%------------------------------------------------------------


%------------------------------------------------------------
\begin{figure}[t]
  \includegraphics[width=0.43\textwidth]{fig/validity/WBF_correl_q_j1pt.pdf}
  \hspace*{0.05\textwidth}
  \includegraphics[width=0.43\textwidth]{fig/validity/WBF_correl_q_Hpt.pdf} 
  \caption{WBF correlations between the virtuality $q$ and
    $p_{T,j_1}$ (left) or $p_{T,h}$ (right).}
  \label{fig:validity_virt_corr}
\end{figure}
%------------------------------------------------------------

%------------------------------------------------------------
\begin{figure}[b!]
  \includegraphics[width=0.43\textwidth]{fig/validity/WBF_limits_150.pdf}
  \hspace*{0.05\textwidth}
  \includegraphics[width=0.43\textwidth]{fig/validity/WBF_limits_300.pdf} 
  \caption{Expected limits on a two-dimensional slice of the vector
    triplet parameter space. We show the analysis based on the event
    numbers in $150~\gev < p_{T,j_1} < 300~\gev$ (left) and based on
    the tail $p_{T,j_1} > 300$~GeV.}
  \label{fig:validity_limits}
\end{figure}
%------------------------------------------------------------


In the middle panels of \autoref{fig:validity_squared_WBF} we see that indeed
the $p_{T,h}$ distribution looks almost identical to $p_{T,j_1}$. Both
of them can be traced back to the unobservable virtualities of the
weak bosons. Due to the preferred collinear direction of the
quark-vector splittings, the $W$-mediated and $Z$-mediated diagrams
populate very different parton-level phase-space regions, with
basically no interference between them.  We can thus define the
virtuality variable~\cite{gino,polarized_ww}
%
\begin{align}
  q =
  \begin{cases}
    \max\left(   \sqrt{ | (p_{u'} - p_{d})^2 | } \, , \, \sqrt{ | (p_{d'} - p_{u})^2 | }  \right) & \text{for $W$-like phase-space points,} \\
    \max\left(   \sqrt{ | (p_{u'} - p_{u})^2 | } \, , \, \sqrt{ | (p_{d'} - p_{d})^2 | }  \right)  & \text{for $Z$-like phase-space points,}
  \end{cases}
  \label{eq:validity_virt}
\end{align}
%
with the distribution shown in the bottom panels of
\autoref{fig:validity_squared_WBF}. Comparing it to $p_{T,h}$ and $p_{T,j_1}$ we see
essentially the same behaviour.  The strong correlation of $q$ with
the observable transverse momenta of the leading tagging jet and the
Higgs is explicitly shown in \autoref{fig:validity_virt_corr}.

Finally, we compare expected exclusion limits on the vector triplet in
the absence of a signal, based on the full model vs the dimension-six
approach.  For the process shown in Equation\;\eqref{eq:validity_def_wbf} we
multiply the cross sections with a branching ratio
$\br(h \to 2\ell 2\nu) \approx 0.01$.  We disregard non-Higgs
backgrounds as well as parton-shower or detector effects.  We then
count events in two high-energy bins of the $p_{T,j_1}$ distributions,
defining a parameter point to be excluded if $S/\sqrt{S+B} > 2$.
While this statistical analysis is not designed to be realistic, it
illustrates how the validity of our dimension-six approach affects
possible limits.  For our limit setting procedure we choose a
two-dimensional plane defined by $m_\xi$ versus a universal coupling
rescaling $c$,
%
\begin{align}
  g_V = 1 \; , \qqquad 
  c_H = c \; , \qqquad 
  c_F = \frac {g_V^2}{2g^2} \, c \; , \qqquad 
  c_{HHVV} = c^2 \; .
\end{align}
%
This reduces the list of generated dimension-six operators to
%
\begin{align}
  f_{WW} = f_{BW} = \frac {c^2} {2g^2} \qquad \text{and} \qquad  f_W = - \frac {c^2} {g^2} \,,
\end{align}
%
and all dimension-six deviations scale like $c^2/m_\xi^2$. To avoid
effects from strongly interacting theories we limit our analysis to
$\Gamma_{\xi}/m_{\xi} < 1/4$.

In the left panel of \autoref{fig:validity_limits} we see that based on event
numbers in the range $150~\text{GeV} < p_{T,j_1} < 300~\text{GeV}$,
the dimension-six approximation with the squared terms gives the
same limits as the full model, as long as we ensure that the new
resonance remains narrow.  In the high-energy tail
$p_{T,j_1} > 300$~GeV including the squared terms also improves the validity of
the dimension-six approach, but it only leads to identical limits for
large $m_\xi$, combined with strong couplings. Indeed, limiting the
momentum transfer of events for example through an upper limit on
$p_{T,j}$ is well known to reduce the dependence on model
assumptions~\cite{spins1,spins2}.

Just as for the $Vh$ production process, at least as long as the event
numbers remain small the square of the dimension-six operators always
improves the agreement with the full theory in weak boson fusion. With
improved statistics the differences become smaller and ultimately
negligible, and the question of whether the squared dimension-six
amplitudes should be taken into account is rendered irrelevant.





%%%%%%%%%%%%%%%%%%%%%%%%%%%%%%%%%%%%%%%%%%%%%%%%%%%%%%%%%%%%
\subsection{Realistic tagging jets}
%%%%%%%%%%%%%%%%%%%%%%%%%%%%%%%%%%%%%%%%%%%%%%%%%%%%%%%%%%%%

%------------------------------------------------------------
\begin{figure}[t]
  \includegraphics[width=0.43\textwidth]{fig/validity/WBF_realistic_T1_j1pt.pdf}
  \hspace*{0.05\textwidth}
  \includegraphics[width=0.43\textwidth]{fig/validity/WBF_realistic_T1_deltaEtaJJ.pdf}
  \caption{WBF distribution at hadron level. Left: $p_{T,j_1}$
    distribution based on the full process, the dashed lines show the
    distributions based on WBF diagrams only and without a
    $\Delta \eta_{jj}$ cut. Right: $\Delta \eta_{jj}$ based on WBF
    diagrams only, the vertical line marks the standard WBF cut
    following Equation\;\eqref{eq:validity_wbf_cuts}.}
  \label{fig:validity_realistic_jets}
\end{figure}
%------------------------------------------------------------

Before we attempt to further improve the description of the full vector
triplet model for example in the benchmark point T1, we briefly test
if the parton-level effects described above survive a realistic environment.
We add a parton shower and jet reconstruction now for the full
process
%
\begin{align}
  p p \to h \; j j \, (+j) \; ,
\end{align}
%
simulated in \toolfont{MadGraph}~\cite{madgraph}.  Parton showering is
performed by \toolfont{PYTHIA6}~\cite{pythia} using the $k_T$-jet MLM
matching scheme~\cite{mlm} with a minimum $k_T$ jet measure between
partons of \toolfont{xqcut}=20~GeV. \toolfont{Fastjet}~\cite{fastjet}
is used to construct jets based on the $k_T$ algorithm with $R = 0.4$. We do not
include a Higgs decay because we are only interested in
production-side kinematics.  The standard WBF cuts then are
%
\begin{align}
  p_{T,j} > 20~\gev \,, \qqquad 
  m_{jj} > 500~\gev \,, \qqquad 
  \Delta \eta_{jj}~>~3.6
\label{eq:validity_wbf_cuts}
\end{align}
%
on the two hardest jets. We veto additional jets with
$p_{T,j} > 20$~GeV between these two tagging jets.  To analyse the
effects of the $\Delta \eta_{jj}$ cut~\cite{spins2}, we generate
additional samples explicitly excluding Higgs-strahlung diagrams, in
spite of the fact that it might break gauge invariance.

In \autoref{fig:validity_realistic_jets} we show that the distributions are
generally robust under parton shower and jet reconstruction, but two
complications arise.  First, on-shell $\xi$ production contributes to
this process and is not entirely removed by the WBF cuts in
Equation\;\eqref{eq:validity_wbf_cuts}, leading to visible differences between the
full and effective model already at low momenta. Such a resonance peak
would be easy to identify experimentally and does not present a major
problem for the dimension-six approximation.

Second, the tension between the full model and the dimension-six
approximation at large momenta now remains below $10~\%$.  This means
that the $\Delta \eta_{jj}$ cut not only removes large contributions
from Higgs-strahlung--like diagrams, it also gets rid of phase-space
regions where the full model and the dimension-six description differ
the most.  At the same time, the $\Delta \eta_{jj}$ removes some of
its well-known discrimination power for new physics effects versus the
Standard Model~\cite{spins2}.




%%%%%%%%%%%%%%%%%%%%%%%%%%%%%%%%%%%%%%%%%%%%%%%%%%%%%%%%%%%%
\subsection{Towards a simplified model}
\label{sec:validity_simplified}
%%%%%%%%%%%%%%%%%%%%%%%%%%%%%%%%%%%%%%%%%%%%%%%%%%%%%%%%%%%%

%------------------------------------------------------------
\begin{figure}[t]
  \includegraphics[width=0.43\textwidth]{fig/validity/WBF_separate_T1_j1pt.pdf} 
  \hspace*{0.05\textwidth}
  \includegraphics[width=0.43\textwidth]{fig/validity/WBF_separate_T1_Hpt.pdf}\\
  \includegraphics[width=0.43\textwidth]{fig/validity/WBF_separate_T1_deltaEtaJJ.pdf} 
  \hspace*{0.05\textwidth}
  \includegraphics[width=0.43\textwidth]{fig/validity/WBF_separate_T1_deltaPhiJJ.pdf}
  \caption{Normalised WBF distributions of the tagging jets. We separate
    the squared new-physics amplitudes, shown as solid lines, from the
    interference with the SM-like diagrams (dashed).}
  \label{fig:validity_squared_separate}
\end{figure}
%------------------------------------------------------------

In the first part of the paper we have shown where in phase space a
dimension-six description of LHC observables breaks down, both for $Vh$
production and for weak boson fusion. For $Vh$ production with its
simple $2 \to 2$ kinematics problems are clearly linked to a possible
$s$-channel resonance, as seen in Equation\;\eqref{eq:validity_breakdown_vh}.  For
weak boson fusion there appears no resonance, but the result of
Equation\;\eqref{eq:validity_breakdown_wbf} suggests that the new states in the
$t$-channel have a similar effect.  In \autoref{fig:validity_squared_separate}
we show different tagging jet distributions, separating the Feynman
diagrams including the heavy $\xi$ states. In particular for the
critical $p_{T,j_1}$ distribution, the $\Delta \eta_{jj}$
distribution, and the $\Delta \phi_{jj}$ distribution these diagrams
are only very poorly described by the dimension-six approach. In
practice this is not a problem because these contributions are
strongly suppressed by the heavy mass $m_\xi$, but it poses the
question how we can improve the agreement. The obvious solution to
these problems in the $s$-channel of $Vh$ production and in the
$t$-channel of weak boson fusion is a simplified
model~\cite{simp,simp_higgs}. A new vector field mixing with the weak
bosons as described by the Lagrangian shown in
Equation\;\eqref{eq:validity_lag-vectortriplet} is such a simplified model, but its
structure is still relatively complex. Obviously, an additional heavy
scalar with mass around $m_\xi$ and the appropriate couplings will
improve the $2 \to 2$ kinematics for $Vh$ production. The question we
want to study in this section is if such a scalar can also improve the
weak boson fusion kinematics.



%%%%%%%%%%%%%%%%%%%%%%%%%%%%%%%%%%%%%%%%%%%%%%%%%%%%%%%%%%%%
\subsubsection*{A pseudo-scalar as a simplified vector}
%%%%%%%%%%%%%%%%%%%%%%%%%%%%%%%%%%%%%%%%%%%%%%%%%%%%%%%%%%%%

The simplest simplified model we can write down includes one new
massive scalar $S$ with a Higgs portal and a Yukawa coupling. 
However, a scalar state will not interfere with the Standard Model
diagrams. In analogy to the CP properties of the Goldstone mode
contributing to the massive $Z$ boson we define our simplified model
with a pseudo-scalar state as
%
\begin{align}
\mathcal{L} \supset 
  \frac{1}{2} (\partial_\mu S)^2 
- \frac{m_S}{2} S^2 
+ \sum_\text{fermions} g_F \; S \overline{F} \gamma_5 F 
+ g_S \; S^2 \phi^\dagger \phi \,.
  \label{eq:validity_simplified_model}
\end{align}

%------------------------------------------------------------
\begin{figure}[t]
  \includegraphics[width=0.43\textwidth]{fig/validity/WBF_simplified_j1pt.pdf}
  \hspace*{0.05\textwidth}
  \includegraphics[width=0.43\textwidth]{fig/validity/WBF_simplified_Hpt.pdf} \\
  \includegraphics[width=0.43\textwidth]{fig/validity/WBF_simplified_deltaEtaJJ.pdf}
  \hspace*{0.05\textwidth}
  \includegraphics[width=0.43\textwidth]{fig/validity/WBF_simplified_deltaPhiJJ.pdf}
  \caption{Normalised WBF distributions for a scalar simplified model
    defined in Equation\;\eqref{eq:validity_simplified_model} vs the vector triplet
    benchmark.}
  \label{fig:validity_simplified}
\end{figure}
%------------------------------------------------------------

In \autoref{fig:validity_simplified} we show the same WBF distributions as in
\autoref{fig:validity_squared_separate}, but including the simplified scalar
model. For the $p_{T,j}$ distribution the squared new-physics
amplitudes in the full vector model and the simplified scalar model
indeed agree well, improving upon the dimension-six description which
breaks down in this distribution.  However, the interference term with
the Standard Model, which is numerically dominant for most of the
distribution and well described in the dimension-six model, poses a
problem.  The $\Delta \eta_{jj}$ distributions show even poorer
agreement: the spin-1 amplitudes of the Standard Model and the vector
triplet have similar phase-space distributions and give two forward
tagging jets, while the scalar mediator favours central
jets~\cite{spins2}.  The $\Delta \phi_{jj}$ distribution, known to be
sensitive to the tensor structure of the hard $VVh$
interaction~\cite{delta_phi}, exposes similar differences between the
full and simplified model.  Altogether, our simplified scalar model
with its very different $VVh$ interaction structure does improve the
description in the region where the dimension-six approach breaks down,
but it fails to describe interference patterns and angular
correlations of the tagging jets.



%%%%%%%%%%%%%%%%%%%%%%%%%%%%%%%%%%%%%%%%%%%%%%%%%%%%%%%%%%%%
\subsubsection*{Splitting functions and equivalence theorem}
%%%%%%%%%%%%%%%%%%%%%%%%%%%%%%%%%%%%%%%%%%%%%%%%%%%%%%%%%%%%

We can understand this very different behaviour of the scalar
$t$-channel mediator as compared to the vector from the splitting
kernels in the collinear limit.  The matrix element squared for the
weak boson fusion process mediated by pseudo-scalars $S$ has the form
%
\begin{align}
 | \mathcal{M}(qq \to q'q'h) |^2 \propto 
  \frac{g_F^4 \;  t_1 t_2}{(t_1 - m_S^2)^2 \; (t_2 - m_S^2)^2} 
\stackrel{m_S \to 0}{\longrightarrow} \frac{\text{const}}{t_1 t_2} \; ,
\end{align}
%
where $t_1$ and $t_2$ denote the respective momentum flow through each
scalar propagator. For $m_S \to 0$ the Jacobians from the phase-space
integration cancel a possible collinear divergence, while for a light
vector boson a soft and a collinear divergence remains. Unlike in the
usual WBF process, the tagging jets in our simplified scalar model
will not be forward.  The reason for this difference in the
infrared is the (pseudo-)scalar coupling to quarks: since the scalar
carries no Lorentz index, a $q \to q S$ splitting will be expressed in
terms of the momentum combinations $(p_q p_q')$, $p_q^2 = m_q^2$, and
$p_q'^2 = m_q^2$. In the limit of massless quarks only the first term
remains as $t = 2 (p_q p_q')$.  This factor in the numerator cancels
the apparent divergence of the $t$-channel propagator.

Adding higher-dimensional couplings of the (pseudo-)scalar to
fermions, such as
%
\begin{align}
  \mathcal{L} \supset 
\sum_\text{fermions} \Biggl[  
  g_{F,2} S \overline{F} F 
+ g_{F,3} (\partial_\mu S) \overline{F} \gamma^\mu F
+ g_{F,4} S (\partial_\mu S) \overline{F} \gamma^\mu \gamma_5 F 
+ g_{F,5} S (\partial_\mu \partial_\nu S) \overline{F} [\gamma^\mu,\gamma^\nu] F
\Biggr] \; ,
\label{eq:validity_simplified_model_extended}
\end{align}
%
does not change this result qualitatively. After partial integration
and using the Dirac equation for the on-shell quarks the coupling
$g_{F,3}$ is equivalent to the simple scalar coupling, $g_{F,2} = m_q^2
g_{F,3}$. In the limit of massless quarks, only two of the new
structures listed in Equation\;\eqref{eq:validity_simplified_model_extended}
contribute at all: $g_{F,2}$ gives exactly the same result as $g_F$,
while $g_{F,5}$ leads to even higher powers of $t$ in the numerator,
%
\begin{align}
  | \mathcal{M}(qq \to q'q'h) |^2 \propto 
  \frac{g_{F,5}^4 \; t_1^3 t_2^3}{(t_1 - m_s^2)^2 \; (t_2 - m_s^2)^2} \; . 
\end{align}
%
No matter how we couple the (pseudo-)scalar of the simplified model to
the external quarks, it never reproduces the collinear splitting
kernel of a vector boson.

To be a little more precise, we can write out the spin-averaged matrix
element squared for the $q \to q' S$ splitting in terms of the energy
of the initial quark $E$, the longitudinal momentum fraction $x$, and
the transverse momentum $p_T$, both carried by $S$,
%
\begin{align}
 | \mathcal{M}(q \to q'S) |^2 &= - 2 g_F^2 x m_q^2
                     + 2 g_F^2 E^2 (1-x)
                     \Biggl[ \sqrt{1 + \frac {p_T^2} {E^2 (1-x)^2} + \frac {m_q^2 (1 - (1-x)^2)} {E^2 (1-x)^2} } - 1 \Biggr] \notag \\
                   &= g_F^2 \, \frac {x^2 \, m_q^2} {1-x} 
                     + g_F^2 \,  \frac {p_T^2} {1-x} 
                     + \ord { \frac{m_q^2 p_T^2}{E^2}, \frac{m_q^4}{E^2}, \frac{p_T^4}{E^2} } \;.
\label{eq:validity_splitting_s}
\end{align}
%
From Equation\;\eqref{eq:validity_splitting_s} one can derive an effective Higgs
approximation or \emph{effective scalar
  approximation}~\cite{effective_scalar}: in the collinear and
high-energy limit, a process $q X \to q' Y$ mediated by a
(pseudo-)scalar $S$ is described by
%
\begin{align}
  \sigma (qX \to q'Y) = \int \mathrm{d}x \, \mathrm{d} p_T \, F_S(x,p_T)
  \, \sigma (SX \to Y)
\label{eq:validity_def_splitting}
\end{align}
%
with the splitting function
%
\begin{align}
  F_S(x,p_T) &= \frac {g_F^2} {16 \pi^2} \, 
               \frac {x \, p_T^3} {\left( m_S^2 (1-x) + p_T^2 \right)^2} \,.
\label{eq:validity_kernel_s}
\end{align}
%
Unlike for vector emission, there is no soft divergence for $x \to 0$.
The $p_T$ dependence is the same as for transverse vector
bosons~\cite{effective_w,polarized_ww}, as we discuss in some detail in the
appendix. 

It might seem surprising that our pseudo-scalar is emitted with a
fundamentally different phase-space dependence than longitudinal $W$
and $Z$ bosons, in apparent contradiction of the Goldstone boson
equivalence theorem.  However, the latter only makes a statement about
the leading term in an expansion in $m_W / E$, where 
$\varepsilon^\mu_L \sim p^\mu / m_W$. At this order the squared matrix
element for the splitting $q \to q' W_L$ agrees with the pseudo-scalar
result, but is suppressed by a factor of $m_q^2 / E^2$. Higher orders
in the $m_W/E$ expansion, outside the validity range of the
equivalence theorem, are not suppressed by quark masses.  The
equivalence theorem is therefore of very limited use in describing the
$W$ or $Z$ couplings to quarks except the top.

%%%%%%%%%%%%%%%%%%%%%%%%%%%%%%%%%%%%%%%%

In Sec.~\ref{sec:validity_simplified} we have introduced a pseudo-scalar in the
$t$-channel of weak boson fusion to describe some of the features
which we find in the full vector triplet model and which our
dimension-six description does not describe well. In this appendix we
collect some of the main formulas and compare the kinematics of
fermions radiating scalars, transverse, or longitudinal gauge
bosons. Our formalism follows the effective
$W$ approximation~\cite{effective_w} as well as the effective Higgs
approximation~\cite{effective_scalar} and allows us to analytically
describe the soft and collinear behaviour. If we do not need to
describe interference terms with SM gauge bosons we can start with a
CP-even scalar splitting $q \to qS$, in terms of the energy of the
initial quark $E$, the longitudinal momentum fraction $x$, carried by $S$, and the
scalar's transverse momentum $p_T$:
%
\begin{align}
 | \mathcal{M}(q \to q'S)  |^2 &= 2 g_F^2 (2-x) m_q^2
                     + 2 g_F^2 E^2 (1-x)
                     \Biggl[ \sqrt{1 + \frac {p_T^2} {E^2 (1-x)^2} + \frac {m_q^2 (1 - (1-x)^2)} {E^2 (1-x)^2} } 
                       - 1 \Biggr] \notag \\
                   &= g_F^2 \left( 4  + \frac {x^2} {1-x} \right) m_q^2
                     + g_F^2 \, \frac {p_T^2} {1-x} 
                     + \ord {\frac{m_q^2 p_T^2}{E^2}, \frac{m_q^4}{E^2}, \frac{p_T^4}{E^2} } \; .
\end{align}
%
The main feature of this splitting is that the infrared behaviour is
different for the term proportional to the quark mass and for the
surviving term in the realistic limit $m_q \to 0$: in the absence of a
fermion mass the collinear divergence from a $t$-channel propagator is
cancelled by the coupling structure. If the term proportional to $m_q$
dominates there will be the usual collinear divergence once we include
a scalar propagator. For a pseudo-scalar the structure shown in
Equation\;\eqref{eq:validity_splitting_s} is very similar,
%
\begin{align}
 |\mathcal{M}(q \to q'S)  |^2 &= - 2 g_F^2 x m_q^2
                     + 2 g_F^2 E^2 (1-x)
                     \Biggl[ \sqrt{1 + \frac {p_T^2} {E^2 (1-x)^2} + \frac {m_q^2 (1 - (1-x)^2)} {E^2 (1-x)^2} } 
                                - 1 \Biggr] \notag \\
                   &= g_F^2 \, \frac {x^2 \, m_q^2} {1-x} 
                     + g_F^2 \,  \frac {p_T^2} {1-x} 
                     + \ord {\frac{m_q^2 p_T^2}{E^2}, \frac{m_q^4}{E^2}, \frac{p_T^4}{E^2} } \;.
\end{align}

In the limit $m_q \to 0$ we can compute universal splitting kernels
including only the leading term in $p_T$, as defined in
Equation\;\eqref{eq:validity_def_splitting}.  Obviously, the scalar and pseudoscalar
case given in Equation\;\eqref{eq:validity_kernel_s} are identical, and we can compare
them with the splitting kernels for longitudinal or transverse
$W$ bosons~\cite{effective_w},
%
\begin{align}
  F_S(x,p_T) &= \frac {g_F^2} {16 \pi^2} \, x \,
               \frac {p_T^3} {\left( m_S^2 (1-x) + p_T^2 \right)^2} \,,\notag \\
  F_T(x,p_T) &= \frac {g^2} {16 \pi^2} \, \frac {1+(1-x)^2} x \, \frac {p_T^3} {\left( m_W^2 (1-x) + p_T^2 \right)^2} \,, \notag \\
  F_L(x,p_T) &= \frac {g^2} {16 \pi^2} \, \frac {(1-x)^2} x \, \frac {2 m_W^2 \, p_T} {\left( m_W^2 (1-x) + p_T^2 \right)^2} \,.
  \label{eq:validity_splittings}
\end{align}

In \autoref{fig:validity_effective_scalar} we show how these different
splittings translate into WBF distributions and compare full simulations
in \toolfont{MadGraph} to the predictions of Equation\;\eqref{eq:validity_splittings}.
A heavy Higgs, $m_h = 1$~TeV, is needed to guarantee a large energy scale
$E \sim m_h \gg p_T \sim m_W, m_S$. In this case we find that the
effective scalar approximation quite accurately describes the transverse
momentum distribution of the tagging jets. For $m_h = 125$~GeV the
assumption of on-shell $W$ bosons or scalars breaks down and the
effective descriptions lose their validity.

%------------------------------------------------------------
\begin{figure}[t]
  \centering
  \includegraphics[width=0.43\textwidth]{fig/validity/WBF_ESA.pdf}
  \caption{Normalised WBF distributions of the tagging jets in the SM with
  a heavy Higgs, $m_h = 1$~TeV. Scalar mediators are compared to
  longitudinal and transverse $W$ bosons following
  Reference~\cite{polarized_ww}.
  The dotted lines give the corresponding predictions of the effective
  $W$ and scalar approximations, Equation\;\eqref{eq:validity_splittings}.}
  \label{fig:validity_effective_scalar}
\end{figure}
%------------------------------------------------------------



%%%%%%%%%%%%%%%%%%%%%%%%%%%%%%%%%%%%%%%%%%%%%%%%%%%%%%%%%%%%
\subsection{Which observables to study}
%%%%%%%%%%%%%%%%%%%%%%%%%%%%%%%%%%%%%%%%%%%%%%%%%%%%%%%%%%%%

%------------------------------------------------------------
\begin{figure}[t]
%  \centering
  \includegraphics[width=0.43\textwidth]{fig/validity/WBF_cuts_T1_SB.pdf}
  \hspace*{0.05\textwidth}
  \includegraphics[width=0.43\textwidth]{fig/validity/WBF_cuts_T1_SsqrtB.pdf}
  \caption{Experimental reach in systematics-driven and
    statistics-driven channels vs theoretical uncertainties of the
    dimension-six description. Each point corresponds to a window
    $x_\text{min,max}$ in one of the four momentum observables that leaves a signal
    cross section of at least 20~fb.}
  \label{fig:validity_cuts}
\end{figure}
%------------------------------------------------------------

Now that it is clear that we cannot further improve the agreement
between the vector triplet and its dimension-six approximation by adding
a heavy scalar as a simplified model, we go back to the original
problem: how can we best use the dimension-six approximation for limit
setting, and do the shortcomings shown in
\autoref{fig:validity_squared_separate} harm this approach?

We know that in our LHC analysis we should avoid angular correlations
of the tagging jets, like $\Delta \eta_{jj}$ or $\Delta
\phi_{jj}$. Instead, we can use momentum-related kinematic variables
like
%
\begin{align}
 x \in \left\{  q, \, p_{T,j_1}, \, p_{T,j_2}, \, p_{T,h} \right\} \; .
\label{eq:validity_variables}
\end{align}
%
An acceptance cut $x > x_\text{min}$ on any of those variables
projects out the interesting phase-space regions, while the cut $x <
x_\text{max}$ ensures the validity of an effective theory
description. If $x_\text{min} > x_\text{max}$ the dimension-six
description is not useful. For each window $x_\text{min,max}$ we can
compute the contribution to the theoretical uncertainty
%
\begin{align}
  \Delta_\text{theo} (x_\text{min,max}) 
= \left| \frac {\sigma_\text{D6} - \sigma_\text{full}} {\sigma_\text{full}} \right| \; ,
\label{eq:validity_err_th}
\end{align} 
%
as well as the statistics-driven and systematics-driven significances
%
\begin{align}
  \frac{S}{B} (x_\text{min,max}) 
= \left| \frac {\sigma_\text{full} - \sigma_\text{SM}} {\sigma_\text{SM}} \right| 
\qquad \text{and} \qquad 
  \frac{S}{\sqrt{B}} (x_\text{min,max}) 
= \sqrt{L} \, \left| \frac {\sigma_\text{full} - \sigma_\text{SM}} {\sqrt{\sigma_\text{SM}}} \right| \; ,
\label{eq:validity_err_ex}
\end{align}
%
where $L = 30~\ifb$ is used as a toy number.

The question is for which
observable $x$ we find the largest $S/B$ and $S/\sqrt{B}$ values while
keeping $\Delta_\text{theo}$ small.  In \autoref{fig:validity_cuts} we show
the correlations between theoretical uncertainty and experimental
reach for the variables defined in Equation\;\eqref{eq:validity_variables} for a
parton-level analysis as defined in Equation\;\eqref{eq:validity_def_wbf}. We see
that the momentum transfer $q$ or the leading tagging jet's
$p_{T,j_1}$ lead to the envelopes with the highest significance for a
given theoretical uncertainty $\Delta_\text{theo}$. This indicates
that the leading tagging jet's transverse momentum is the best way of
experimentally accessing the momentum flow through the hard process,
at least for the hard parton-level process with only two tagging
jets.





%%%%%%%%%%%%%%%%%%%%%%%%%%%%%%%%%%%%%%%%%%%%%%%%%%%%%%%%%%%%
\section{Conclusions}
\label{sec:validity_conclusions}
%%%%%%%%%%%%%%%%%%%%%%%%%%%%%%%%%%%%%%%%%%%%%%%%%%%%%%%%%%%%

An effective field theory for the Higgs sector offers a theoretically
well-defined, efficient, and largely model-independent language to
analyse extensions of the Standard Model in both rate measurements and
kinematic distributions. A fit of dimension-six operators to LHC Higgs
measurements works fine~\cite{Corbett:2015ksa} and constitutes the
natural extension of the Higgs couplings analyses of Run~I.  Most of
the relevant higher-dimensional operators correspond to simple
coupling modifications, supplemented by four operators describing new
Lorentz structures in the Higgs coupling to weak
bosons~\cite{Corbett:2015ksa}.

In this paper we have studied the validity of this approach from the
theoretical side.  We know that at the LHC a clear hierarchy of
electroweak and new physics scales cannot be guaranteed, the question
is whether dimension-six operators nevertheless capture the
phenomenology of specific UV-complete theories with sufficient
accuracy.  We have systematically compared a singlet Higgs portal
model, a two-Higgs doublet model, scalar top partners, and a heavy
vector triplet to their dimension-six EFT descriptions, based on the
linear realisation of electroweak symmetry breaking with a Higgs
doublet.  We have analysed the main Higgs production and decay
signatures, covering rates as well as kinematic distributions.

We have found that the dimension-six operators provide an adequate
description in almost all realistic weakly coupled scenarios. Shifts
in the total rates are well described by effective operators.
Kinematic distributions typically do not probe weakly interacting new
physics with sufficient precision in the high-energy tails to
challenge the effective operator ansatz.  This is obvious for the
extended scalar models, where new Lorentz structures and
momentum-dependent couplings with dramatic effects in LHC
distributions only appear at the loop level.  A loop-suppressed
effective scale suppression $E^2/(4 \pi \Lambda)^2$ has to be compared
with on-shell couplings modifications proportional to $v^2/\Lambda^2$.
Only phase space regions probing energies around $4 \pi v \approx
3$~TeV significantly constrain loop contributions in the Higgs sector
and eventually lead to breakdown of the effective field theory. In
turn, a simple dimension-six descriptions will capture all effects that
are expected to be measurable with sufficient statistics at the LHC
Run II.  On the other hand, the vector triplet model shows that
modifications of the gauge sector can generate effects in LHC
kinematics at tree level. However, we again find that for weakly
interacting models and phenomenologically viable benchmark points they
are described well by an appropriate set of dimension-six
operators.

%-----------------------------------------------------------
\begin{table}[t] \renewcommand{\arraystretch}{1.2} \centering
\begin{tabular}{ll c ccc} \toprule Model & Process &\hspace*{1em}&
\multicolumn{3}{c}{EFT failure} \\ \cmidrule{4-6} & && resonance &
kinematics & matching \\ \midrule singlet & on-shell $h \to 4 \ell$,
WBF, $Vh$, \dots && & & \largex \\ & off-shell WBF, \dots && &
\brlargex & \largex \\ & $hh$ && \largex & \largex & \largex \\ 2HDM &
on-shell $h \to 4 \ell$, WBF, $Vh$, \dots && & & \largex \\ &
off-shell $h \to \gamma \gamma$, \dots && & \brlargex & \largex \\ &
$hh$ && \largex & \largex & \largex \\ top partner & WBF, $Vh$ && & &
\largex \\ vector triplet & WBF && & \brlargex & \largex \\ & $Vh$ &&
\largex & \brlargex & \largex \\ \bottomrule
\end{tabular}
 \caption{Possible sources of failure of dimension-six Lagrangian at the
LHC.  We use parentheses where deviations in kinematic distributions
appear, but are unlikely to be observed in realistic scenarios.}
 \label{tbl:validity_differences}
\end{table}
%-----------------------------------------------------------


Three sources for a possible breakdown of the dimension-six description
are illustrated in \autoref{tbl:validity_differences}\footnote{Forcing the EFT
approach into a spectacular breakdown was the original aim of this
paper, but to our surprise this did not happen.}: First, the EFT
cannot describe light new resonances. Such a signature at the LHC
would be an obvious signal to stop using the EFT and switch to
appropriate simplified models.  Second, selected kinematic
distributions fail to be described by the dimension-six Lagrangian, in
particular for Higgs pair production.  Deviations in the high-energy
tails of WBF and Higgs-strahlung distributions on the other hand are
too small to be relevant in realistic weakly coupled scenarios. These
two cases do not threaten LHC analyses in practice.

The third issue with the dimension-six EFT description is linked to
matching in the absence of a well-defined scale hierarchy.  Even with
only one heavy mass scale in the Lagrangian, the electroweak VEV
together with large couplings can generate several new physics scales,
defined by the masses of the new particles.  A linear EFT description,
which is justified by the SM-like properties of the newly discovered
Higgs boson, should in principle be matched in the phase where the
electroweak symmetry is unbroken. Such a procedure is blind to
additional scales induced by the electroweak VEV, potentially leading
to large errors in the dimension-six approximation.  Including
$v$-dependent terms in the Wilson coefficients, which corresponds to
matching in the broken phase, can significantly improve the EFT
performance. We have explicitly demonstrated this for all the models
considered in this paper.

None of these complications with the dimension-six description presents
a problem in using effective operators to fit LHC Higgs data.  They
are purely theoretical issues that need to be considered for the
interpretation of the results.


%%%%%%%%%%%%%%%%%%%%%%%%%%%%%%%%%%%%%%%%

While a dimension-six Higgs analysis at the LHC cannot be considered the
leading part of a consistent effective theory, it describes the
effects of weakly interacting extensions of the Higgs-gauge sector very
well~\cite{too_long}. In this brief study we have answered two practical
question concerning such a dimension-six analysis for Run~II.

First, a priori it is not clear if squared dimension-six terms should be
included in calculations. We have studied two particularly challenging
parameter points of a vector triplet model for $Vh$ production and for
weak-boson-fusion Higgs production. For both processes we find that
the dimension-six squared term avoids negative rate predictions in the
$m_{Vh}$ or $p_{T,V}$ distributions of $Vh$ production and in the
$p_{T,j_1}$ distribution of weak boson fusion. Even for cases with a
constructive interference between the dimension-six and the Standard
Model contributions, it turns out that including the dimension-six
squared term improves the agreement of kinematic distributions between
the full model and the dimension-six approximation. Ultimately, this
translates into a better agreement in the expected exclusion
limits\footnote{Similar conclusions in a different framework have recently
been published in Reference~\cite{gino}.}.

Second, we have attempted to improve the agreement between the full
model and our approximation by using a simplified model. The only
significantly simpler model than a mixing gauge extension is an
extended scalar sector. While the corresponding deviations between the
full model and the dimension-six approximation are phenomenologically
hardly relevant, we find that such an additional scalar improves the
modelling of kinematic distributions of the kind $m_{Vh}$ and
$p_{T,j_1}$ where the dimension-six description breaks down. However,
this comes at the cost of significant deviations in the dominant
interference terms. Moreover, once we include angular correlations
like $\Delta \eta_{jj}$ or $\Delta \phi_{jj}$ in weak boson fusion,
the simplified model fails badly. The difference can be traced to the
divergence structure of the corresponding splittings.

Seeing that the dimension-six approach is still the better simple model
to describe new physics in WBF distributions, we have finally analysed
which phase-space regions provide an interesting window to new physics
while being well described by the dimension-six approximation.  We have
demonstrated that the leading tagging jet's $p_T$ distribution is
particularly suited for such a search for new physics.
