

\selectlanguage{ngerman}
\KOMAoptions{open=left}

%%%%%%%%%%%%%%%%%%%%%%%%%%%%%%%%%%%%%%%%%%%%%%%%%%%%%%%%%%%%
\chapter*{Abstract in deutscher \"{U}bersetzung}
%%%%%%%%%%%%%%%%%%%%%%%%%%%%%%%%%%%%%%%%%%%%%%%%%%%%%%%%%%%%

% Maximum: 200 words 

% Higgs effektive Feldtheorie bietet eine modellunabhängige und
% Phänomenologisch kraftvoller Rahmen für Messungen der Higgs
% Boson Eigenschaften. Wir analysieren zwei Aspekte solcher Messungen
% Für Run ~ 2 des LHC. Erstens kann die begrenzte Präzision des LHC nicht
% Garantieren eine klare Skalenhierarchie zwischen dem experimentellen Momentum
% Transfer und die untersuchte neue Physik Skalen, Zweifel an der
% Gültigkeit des effektiven Modells. Durch den Vergleich einer Reihe neuer Physik
% Szenarien zu ihrer effektiven Approximation, analysieren wir, ob ein
% Beschreibung des Higgs-Sektors mit Dimension-sechs Operatoren ist
% Nützlich, wo die effektive Theorie zerbricht und wie ihre Gültigkeit
% Kann verbessert werden.

% Zweitens verwenden wir Informationsgeometrie, um Higgs zu verstehen und zu optimieren
% Messungen am LHC. Unser neuartiger Ansatz basiert auf dem Fisher
% Informationen, die das maximale Wissen kodieren, auf das man sich beziehen kann
% Parameter in einem gegebenen Experiment. Wir entwickeln einen Algorithmus zu berechnen
% Die Fisher-Information in Teilchenphysikprozessen und berechnen die
% Informationen über Dimension-sechs Operatoren in verschiedenen Higgs-Signaturen.
% Wir zeigen, wie sich Informationsgeometrie ein optimales Ereignis definieren lässt
% Selektionen, bestimmen die mächtigsten kinematischen Observablen und
% Vergleichen die Macht der modernen multivariaten Techniken mit traditionellen
% Histogramm-basierte Analysemethoden.

% \newparagraph
% %

Mit einer effektiven Feldtheorie k\"onnen die Signaturen neuer Physik
im Higgs-Sektor ohne starke Theorieannahmen parametrisiert werden. In
dieser Arbeit untersuchen wir zwei Aspekte dieses Ansatzes, die f\"ur
Messungen der Eigenschaften des Higgs-Bosons w\"ahrend Run~2 des LHC
relevant sind.

Aufgrund der beschr\"ankten Pr\"azision kann neue Physik im
Higgs-Sektor nur dann entdeckt werden, wenn deren typische
Energieskala nah am experimentellen Impuls\"ubertrag liegt. Die
N\"aherungen der effektiven Theorie sind daher m\"oglicherweise
ung\"ultig. Im ersten Teil dieser Arbeit vergleichen wir die
Signaturen mehrerer Modelle von Physik jenseits des Standardmodells
mit den entsprechenden Beschreibungen in der effektiven Theorie,
untersuchen die N\"utzlichkeit des effektiven Modells und
zeigen, wie dessen G\"ultigkeitsbereich vergr\"o\ss{}ert werden
kann.

Im zweiten Teil verwenden wir Methoden aus der Informationsgeometrie,
um Messungen von Higgs-Eigenschaften zu optimieren. Unser Ansatz
basiert auf der Fisher-Information, die die maximale Pr\"azision
angibt, mit der Parameter in einem Experiment gemessen werden
k\"onnen. Wir entwickeln Methoden zur Berechnung der
Fisher-Information in der Teilchenphysik und wenden sie auf
verschiedenen Higgs-Prozesse an. Dabei zeigen wir, wie mit
Informationsgeometrie Selektionsschnitte optimiert, die wichtigsten
kinematischen Observablen definiert und das Potential von modernen
multivariaten Methoden mit dem von Histogramm-basierten Analysen
verglichen werden kann.



\selectlanguage{british}
\KOMAoptions{open=right}

%%%%%%%%%%%%%%%%%%%%%%%%%%%%%%%%%%%%%%%%%%%%%%%%%%%%%%%%%%%%
\chapter*{Abstract}
%%%%%%%%%%%%%%%%%%%%%%%%%%%%%%%%%%%%%%%%%%%%%%%%%%%%%%%%%%%%

% Maximum: 200 words

An effective field theory provides a model-independent and
phenomenologically powerful parametrisation of new physics in the
Higgs sector. We analyse two aspects of this framework that are
relevant for measurements of the Higgs properties during Run~2 of the
LHC.

First, the limited precision of the LHC analyses cannot guarantee a
clear scale hierarchy between the experimental momentum transfer and
the probed new physics scales, casting doubt on the validity of the
effective model. By comparing a range of new physics scenarios to
their dimension-six approximation, we analyse if an effective
description of the Higgs sector is useful, where it breaks down, and
how its validity can be improved.

Second, we use information geometry to understand and optimise Higgs
measurements at the LHC. Our novel approach is based on the Fisher
information, which encodes the maximum
precision with which theory parameters can be measured in an
experiment. We develop an algorithm to calculate the Fisher
information in LHC processes, and compute the information on
dimension-six operators in different Higgs signatures.  We demonstrate
how information geometry lets us improve event selections, determine
the most powerful observables, and compare the power of modern
multivariate techniques to that of traditional histogram-based
analyses.
