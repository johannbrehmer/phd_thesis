

\selectlanguage{ngerman}
\KOMAoptions{open=left}

%%%%%%%%%%%%%%%%%%%%%%%%%%%%%%%%%%%%%%%%%%%%%%%%%%%%%%%%%%%%
\chapter*{Abstract in deutscher \"{U}bersetzung}
%%%%%%%%%%%%%%%%%%%%%%%%%%%%%%%%%%%%%%%%%%%%%%%%%%%%%%%%%%%%

% Maximum: 200 words 

% Higgs effektive Feldtheorie bietet eine modellunabhängige und
% Phänomenologisch kraftvoller Rahmen für Messungen der Higgs
% Boson Eigenschaften. Wir analysieren zwei Aspekte solcher Messungen
% Für Run ~ 2 des LHC. Erstens kann die begrenzte Präzision des LHC nicht
% Garantieren eine klare Skalenhierarchie zwischen dem experimentellen Momentum
% Transfer und die untersuchte neue Physik Skalen, Zweifel an der
% Gültigkeit des effektiven Modells. Durch den Vergleich einer Reihe neuer Physik
% Szenarien zu ihrer effektiven Approximation, analysieren wir, ob ein
% Beschreibung des Higgs-Sektors mit Dimension-sechs Operatoren ist
% Nützlich, wo die effektive Theorie zerbricht und wie ihre Gültigkeit
% Kann verbessert werden.

% Zweitens verwenden wir Informationsgeometrie, um Higgs zu verstehen und zu optimieren
% Messungen am LHC. Unser neuartiger Ansatz basiert auf dem Fisher
% Informationen, die das maximale Wissen kodieren, auf das man sich beziehen kann
% Parameter in einem gegebenen Experiment. Wir entwickeln einen Algorithmus zu berechnen
% Die Fisher-Information in Teilchenphysikprozessen und berechnen die
% Informationen über Dimension-sechs Operatoren in verschiedenen Higgs-Signaturen.
% Wir zeigen, wie sich Informationsgeometrie ein optimales Ereignis definieren lässt
% Selektionen, bestimmen die mächtigsten kinematischen Observablen und
% Vergleichen die Macht der modernen multivariaten Techniken mit traditionellen
% Histogramm-basierte Analysemethoden.

% \newparagraph
% %

Mit einer effektiven Feldtheorie k\"onnen die Eigenschaften des
Higgs-Bosons ohne starke Theorieannahmen parametrisiert und gemessen
werden. In dieser Arbeit untersuchen wir zwei Aspekte solcher
Messungen, die f\"ur Run~2 des LHC relevant sind. Die beschr\"ankte
Pr\"azision dieser Analysen kann eine klare Hierarchie zwischen dem
Impuls\"ubertrag im Experiment und den typischen Energieskalen neuer
Physik nicht garantieren, und die N\"aherungen der effektiven Theorie
sind m\"oglicherweise ung\"ultig. Im ersten Teil dieser Arbeit
vergleichen wir deshalb die Signaturen mehrerer Modelle von Physik
jenseits des Standardmodells mit den entsprechenden Beschreibungen in
der effektiven Theorie, untersuchen die N\"utzlichkeit des effektiven
Modells, und demonstrieren, wie dessen G\"ultigkeitsbereich
vergr\"o\ss{}ert werden kann.

Im zweiten Teil verwenden wir Methoden aus der Informationsgeometrie,
um Strategien zur Messung von Higgs-Eigenschaften zu optimieren. Unser
Ansatz basiert auf der Fisher-Information, die die maximale
Pr\"azision angibt, mit der Parameter in einem Experiment gemessen
werden k\"onnen. Wir entwickeln Methoden zur Berechnung der
Fisher-Information in der Teilchenphysik, und ermitteln die
Information in verschiedenen Higgs-Prozessen. Anschlie\ss{}end zeigen
wir, wie Informationsgeometrie optimale Selektionen und Observablen
definiert und uns das Potential von modernen multivariaten Methoden
mit traditionellen Histogramm-basierten Analysen vergleichen l\"asst.


\selectlanguage{british}
\KOMAoptions{open=right}

%%%%%%%%%%%%%%%%%%%%%%%%%%%%%%%%%%%%%%%%%%%%%%%%%%%%%%%%%%%%
\chapter*{Abstract}
%%%%%%%%%%%%%%%%%%%%%%%%%%%%%%%%%%%%%%%%%%%%%%%%%%%%%%%%%%%%

% Maximum: 200 words

Higgs effective field theory provides a model-independent and
phenomenologically powerful framework to measure the properties of the
Higgs boson. We analyse two aspects of such measurements relevant for
Run~2 of the LHC. First, the limited precision of the LHC cannot
guarantee a clear scale hierarchy between the experimental momentum
transfer and the probed new physics scales, casting doubt on the
validity of the effective model. By comparing a range of new physics
scenarios to their effective approximation, we analyse whether a
description of the Higgs sector with dimension-six operators is
useful, where the effective theory breaks down, and how its validity
can be improved.

Second, we use information geometry to understand and optimise Higgs
measurements at the LHC. Our novel approach is based on the Fisher
information, which encodes the maximum knowledge on theory parameters
one can derive in a given experiment. We develop an algorithm to
calculate the Fisher information in particle physics processes, and
compute the information on dimension-six operators in different Higgs
signatures.  We demonstrate how information geometry lets us define
optimal event selections, determine the most powerful kinematic
observables, and compare the power of modern multivariate techniques
to that of traditional histogram-based analysis methods.
