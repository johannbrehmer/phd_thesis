

\selectlanguage{ngerman}
\KOMAoptions{open=left}

%%%%%%%%%%%%%%%%%%%%%%%%%%%%%%%%%%%%%%%%%%%%%%%%%%%%%%%%%%%%
\chapter*{Abstract in deutscher \"{U}bersetzung}
%%%%%%%%%%%%%%%%%%%%%%%%%%%%%%%%%%%%%%%%%%%%%%%%%%%%%%%%%%%%

% Maximum: 200 words

Mit einer effektiven Feldtheorie k\"onnen die Eigenschaften des
Higgs-Bosons ohne starke Theorieannahmen parametrisiert und gemessen
werden. In dieser Arbeit untersuchen wir zwei Aspekte solcher
Messungen, die f\"ur Run~2 des LHC relevant sind. Aufgrund der
beschr\"ankten Pr\"azision ist eine klare Skalenhierarchie hier nicht
garantiert, und die N\"aherungen der effektiven Theorie
m\"oglicherweise ung\"ultig. Im ersten Teil dieser Arbeit vergleichen
wir deshalb die Signaturen mehrerer spezifischer Theorien mit den
entsprechenden effektiven Beschreibungen, untersuchen die
N\"utzlichkeit des effektiven Modells, und demonstrieren, wie dessen
G\"ultigkeitsbereich vergr\"o\ss{}ert werden kann.

Im zweiten Teil optimieren wir Messungen von Higgs-Eigenschaften mit
Methoden der Informationsgeometrie. Unser Ansatz basiert auf der
Fisher-Information, die die maximale Pr\"azision angibt, mit der
Parameter in einem Experiment gemessen werden k\"onnen. Wir entwickeln
Methoden zur Berechnung der Fisher-Information in der Teilchenphysik,
und ermitteln die Information in verschiedenen
Higgs-Kan\"alen. Anschlie\ss{}end zeigen wir, wie
Informationsgeometrie optimale Selektionen und Observablen definiert
und uns das Potential von modernen multivariaten Methoden mit
traditionellen Histogramm-basierten Analysen vergleichen l\"asst.



\selectlanguage{british}
\KOMAoptions{open=right}

%%%%%%%%%%%%%%%%%%%%%%%%%%%%%%%%%%%%%%%%%%%%%%%%%%%%%%%%%%%%
\chapter*{Abstract}
%%%%%%%%%%%%%%%%%%%%%%%%%%%%%%%%%%%%%%%%%%%%%%%%%%%%%%%%%%%%

% Maximum: 200 words

Higgs effective field theory provides a model-independent and
phenomenologically powerful framework for measurements of the Higgs
boson properties. We analyse two aspects of such measurements relevant
for Run~2 of the LHC. First, the limited precision of the LHC cannot
guarantee a clear scale hierarchy between the experimental momentum
transfer and the probed new physics scales, casting doubt on the
validity of the effective model. By comparing a range of new physics
scenarios to their effective approximation, we analyse whether an
description of the Higgs sector with dimension-six operators is
useful, where the effective theory breaks down, and how its validity
can be improved.

Second, we use information geometry to understand and optimise Higgs
measurements at the LHC. Our novel approach is based on the Fisher
information, which encodes the maximum knowledge one can derive on
parameters in a given experiment. We develop an algorithm to calculate
the Fisher information in particle physics processes, and compute the
information on dimension-six operators in different Higgs signatures.
We demonstrate how information geometry lets us define optimal event
selections, determine the most powerful kinematic observables, and
compare the power of modern multivariate techniques to traditional
histogram-based analysis methods.
